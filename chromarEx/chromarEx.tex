\documentclass[phd]{infthesis}

\usepackage[utf8]{inputenc}
\usepackage[T1]{fontenc}
\usepackage[british]{babel}
\usepackage{microtype}
\usepackage[usenames,dvipsnames,svgnames,table]{xcolor}
\usepackage[english=british,autopunct=false]{csquotes}
\usepackage[natbib=true,style=authoryear-comp,maxbibnames=6]{biblatex}
\usepackage{graphicx}
\usepackage{textcomp}
\usepackage{wrapfig}
\usepackage{xfrac}
\usepackage{xspace}
\usepackage{mathcommon}
\usepackage[sc]{mathpazo}
\usepackage{hyperref}
\usepackage{expl3}
\usepackage{enumitem}

\usepackage{listings}% http://ctan.org/pkg/listings
\lstset{
  basicstyle=\ttfamily,
  mathescape
}

\frenchspacing

% Bibliography
\addbibresource{chromarEx.bib}
\bibliography{chromarEx}

% Text
\newcommand{\ie}{i.e.\xspace}
\newcommand{\eg}{e.g.\xspace}

% Referencing
\newcommand{\chp}[1]{\S\ref{chp:#1}}
\newcommand{\sct}[1]{\S\ref{sec:#1}}
\newcommand{\ssec}[1]{\S\ref{subsec:#1}}
\newcommand{\eqn}[1]{Eq.~\ref{eq:#1}}
\newcommand{\eqns}[2]{Eq. \ref{eq:#1} and \ref{eq:#2}}
\newcommand{\lem}[1]{Lemma~\ref{lemma:#1}}
\newcommand{\lems}[2]{Lemmas \ref{lemma:#1} and \ref{lemma:#2}}
\newcommand{\thm}[1]{Th.~\ref{thm:#1}}
\newcommand{\fig}[1]{Fig.~\ref{fig:#1}}
\newcommand{\diagram}[1]{diagram~\ref{eq:#1}}
\newcommand{\app}[1]{Appendix~\ref{app:#1}}
\newcommand{\mcite}[1]{\textcolor{gray}{#1}} % missing cite
\newcommand{\defn}[1]{Def.~\ref{def:#1}}
\newcommand{\prop}[1]{Prop.~\ref{prop:#1}}

% Math
\renewcommand{\tuple}[1]{\left(#1\right)}
\DeclareMathOperator*{\expn}{exp}
\renewcommand*{\exp}[1]{e^{\,#1}} % \mathrm{e}^{#1}}
\renewcommand{\qedsymbol}{\ensuremath{\blacksquare}}
\newcommand{\partialto}{\rightharpoonup}
\newcommand{\id}{\vec{1}} % identity function
\newcommand{\mr}[1]{\mathrm{#1}}
\newcommand{\m}[1]{\{\!| #1 |\!\}}

% Other stuff
\newcommand{\maybe}[1]{\textcolor{gray}{#1}}
\newcommand{\todo}[1]{\textcolor{red}{TODO: #1}}

% rules
\newcommand{\ar}[2]{\mr{#1} \! = \! {#2}}

% Abstract
\begin{document}

\chapter{Chromar by example}
\label{chp:chromarEx}
In this chapter we will introduce Chromar (formally introduced in the next
chapter) by means of examples. The example serve two purposes: to introduce the
features of the languages but also as a means of comparing Chromar to the related
languages in the previous chapter by seeing how Chromar features handle the
modelling requirements that they present.

Chromar has the following properties, which we will explore in this chapter
through the examples:
\begin{itemize}
\item \emph{basic Chromar} is a rule-based notation with stochastic semantics
  yielding a Continuous Time Markov Chain (CTMC). It follows the object-based
  tradition (see previous section) and makes use of discrete objects called
  \emph{agents}. These have attributes that are defined at the type level, so
  that every agent of that type has these attributes. For a version of our auxin
  example a $\mr{Cell}$ agent type could be, similarly to other parametric
  languages,
  $\mathrm{Cell}(\mathrm{pos}:\mathrm{int}, \mathrm{a}: \mathrm{real})$ with
  attribute $\mathrm{pos}$ for positional information and attribute $\mathrm{a}$
  to keep its auxin concentration. Agents are instantiations of this type with
  concrete values for the attributes like
  $\mathrm{Cell}(\mathrm{pos}=1, \mathrm{a}=a_1)$ for the first cell,
  $\mathrm{Cell}(\mathrm{pos}=2, \mathrm{a}=a_2)$ for the second cell and so
  on. The states of the CTMC are multisets of agents. The rules describe how
  agents are added to or removed from states at a more abstract level than
  individual agents. This is done by using patterns on the rule left-hand sides
  specifying the group of agents for which the dynamics are defined. As we have
  seen, having parameters help us overcome some of the limitations of object-only
  languages.
\item \emph{extended Chromar} extends basic Chromar with two new features:
  \begin{itemize}
  \item[(i)] \textit{Fluents} --- the incorporation of deterministically
    changing time-dependent values. These are important for modelling dynamics
    in a changing environment, and
  \item[(ii)] \textit{Observables} --- values calculated from a global view of
    the system. These are important in cases where a coarse-grained view of the
    system is needed. This may be because we cannot acquire atomistic data, or
    because we do not wish to model everything at the same level of
    detail. Observables also give us a flexible way to observe the state of the
    system that can be used to report the results of model simulations, as we
    often need time series of some observable on the state of the system rather
    than time series of the state itself.
\end{itemize}
\item A concrete realisation of extended Chromar as an embedded Domain Specific
  Language (DSL) inside Haskell, a functional programming language
  \citep{gibbons_functional_2015}. The embedding means that we can use any valid
  Haskell expression where expressions are expected, for example in the rate
  expressions and in the right-hand sides of rules. Agent types and the outside
  environment (defining rate, condition function etc.) are defined directly as
  Haskell expressions but rules, fluents, and observables are defined using
  abstract Chromar syntax via quotation \citep{mainland_why_2007}.
\end{itemize}

The examples from the previous section are extensions to the examples that
appear in the papers describing Chromar \citet{honorato-zimmer_chromar_2017,
  honorato-zimmer_chromar_2018}. Therefore the following text is also in part
comes from the same sources.

\section{Plant development in a field}

\subsection{Single plant}
Since agents are the main entities in our language, let us
consider what types of agents we should have to model the above
system, given the above discussion. We will need:
\begin{itemize}
\item a $\mathrm{Leaf}$ type with attributes for 
appearance: $\mathrm{Leaf}(\mathrm{age}:\mathrm{int},
\mathrm{mass}:\mathrm{real})$,
\item a $\mathrm{Cell}$ type that represents our main plant `cell' with an
attribute, $\mathrm{carbon}$, to keep the current carbon level:
$\mathrm{Cell}(\mathrm{carbon}:\mathrm{real})$ (there will only be one agent
\item a $\mathrm{Ros}$ type that represents the entire Rosette with an
attribute, $\mathrm{n}$, to keep the current number of leaves:
$\mathrm{Ros}(\mathrm{n}:\mathrm{int})$ (there will also only be one agent
\end{itemize}
We will use the current number of leaves relative to the index of
a particular leaf as a proxy for the leaf's age: the larger the difference
between the index of a leaf and the current number of leaves, the older that
leaf is.

For the \textit{carbon assimilation} from one particular leaf we need to
increase the carbon concentration of the central Cell. The bigger the leaf the
faster it contributes to the production of carbon:
\[\mathrm{Leaf}(\mathrm{mass}=m), \: \mathrm{Cell}(\mathrm{carbon}=c) \:
\xrightarrow{f(m)} \: \mathrm{Leaf}(\mathrm{mass}=m), \:
\mathrm{Cell}(\mathrm{carbon}=c+1)\]
We can read this rule as saying that for
any two Leaf and Cell agents, the Leaf agent remains the same and the Cell agent
increases its carbon content by one. Note that we assign the values of the
attributes 
the left-hand side of the rule to the variables $m$ and $c$, so that we can
refer to them in the right-hand side of the rule and the rate expression. Since
the $\mathrm{age}$ is not used in the rest of the rule, it can be omitted (see
\textit{Missing attributes} syntactic extension, Section~\ref{}). If
we were to write this in a traditional reaction notation we would have to write
a reaction for every possible Leaf agent which leads to the compactness problem
we have noted earlier (indeed, there would be infinitely many such
possibilities). The implicit `for-all' in the pattern on the left-hand side
allows the rule to be applied to new leaves when they are
created. For example if the state of the system has two leaves, the central
cell, and the rosette agent:
\begin{align*}
\m{ \mathrm{Leaf}(\mathrm{age}=1, \mathrm{mass}=10), \:
\mathrm{Leaf}(\mathrm{age}=2, \mathrm{mass}=5),
\mathrm{Cell}(\mathrm{carbon}=20), \mathrm{Ros}(\mathrm{n} =2)}
\end{align*}
then the rule is applicable to the two substates
\begin{align*}
& \m{\mathrm{Leaf}(\mathrm{age}=1, \mathrm{mass}=10),
                 \mathrm{Cell}(\mathrm{carbon}=20)} \: \text{ and} \\
&\m{\mathrm{Leaf}(\mathrm{age}=2,
  \mathrm{mass}=5), \mathrm{Cell}(\mathrm{carbon}=20)}
\end{align*}

For \textit{maintenance respiration}, the central $\mathrm{Cell}$ agent gives
some carbon to a $\mathrm{Leaf}$ agent, with the amount of carbon needed for
maintenance depending on the size of the Leaf:
%
\begin{align*}
&\mathrm{Leaf}(\mathrm{mass}=m),\: \mathrm{Cell}(\mathrm{carbon}=c) \:
\xrightarrow{h(m)} \\ &\mathrm{Leaf}(\mathrm{mass}=m),\:
\mathrm{Cell}(\mathrm{carbon}=c-g(m)) \; [c \geq g(m)]
\end{align*}
Note the use of the condition $c \geq g(m)$ to make sure that the
carbon levels do not become negative. Also note that here we use both the rate
($h(m)$) and the increment size ($g(m)$) to control the amount by which carbon
is updated in a time unit. Since the rate controls the number of times the
update will happen, the product of the rate and update functions is the expected
amount that will be removed from the carbon pool in a time unit.

Next, \textit{leaf growth} depends on the mass of the leaf, its age (there is a
limit on how much a leaf can grow so older leaves stop growing at some point),
and the amount of carbon available:
%
\begin{align*}
  & \mathrm{Ros}(\mathrm{n}=n), \mathrm{Leaf}(\mathrm{age}=i,
\mathrm{mass}=m),\: \mathrm{Cell}(\mathrm{carbon}=c) \: \xrightarrow{h(n-i, m,
    c)}\: \\
  & \mathrm{Ros}(\mathrm{n} =n), \mathrm{Leaf}(\mathrm{age}=i,
\mathrm{mass}=m+1),\: \mathrm{Cell}(\mathrm{carbon}=c-1) \; [c \geq 1]
\end{align*} The mass of the leaf is also needed in the calculation to make sure
that the growth does not exceed observed physical constraints. The growth is
capped to a fraction of the current mass of the leaf.

Finally, for \textit{leaf creation} we have:
\begin{equation*} \mathrm{Ros}(\mathrm{n}=n) \: \xrightarrow{k} \:
\mathrm{Ros}(\mathrm{n}=n+1), \: \mathrm{Leaf}(\mathrm{age}=n+1,
\mathrm{mass}=m_0)
\end{equation*}

\subsubsection*{Fluents and Observables}
There are two problems with the above definition: 
\begin{itemize}
\item[(i)] There is no way to include the environment in the model, and so it is
  assumed constant. This makes the model very detached from reality.  For
  example, our plants photosynthesise all the time, whereas in reality they only
  do so during daylight, but we have no direct way to switch between day and
  night.
\item[(ii)] We have two representations of the same process at two different
  levels of abstraction that have to be kept consistent with each other by the
  user.  Specifically, the $\mathrm{n}$ attribute in the $\mathrm{Ros}$ agent
  keeps track of the total number of leaves in the plant. However, there is no
  way in the language to specify the global information the attribute is
  tracking, and check that its value is indeed what it should be.
\end{itemize}

We next introduce two new notational features
to solve these problems.

\textit{Fluents}

To solve the first problem time-dependent values we call \textit{Fluents}. These
can be constructed through a combination of a small set of primitives and
general expressions, taken from Reactive Programming (FRP) constructs in
\citep{wan_functional_2000} (our Fluents are usually called Behaviours in FRP). For
example we could write a fluent for light, which is a function from time to the
Booleans, and another one for temperature that depends on light:
\begin{align*}
& light = \mathbf{repeatEvery} \; 24.0 \; (\mathrm{True} \; \mathbf{when} \; (6 < \mathbf{time} < 18) \; \mathbf{else} \; \mathrm{False}) \\
& temp = 21.0 \; \mathbf{when} \; light \; \mathbf{else} \; 16.0
\end{align*}
where the $\mathbf{when} \dotso \mathbf{else}$ construct denotes conditional
behaviour and $\mathbf{repeatEvery}$ cycling behaviour. Cycling behaviour is
achieved with the modulo operation so the value of $\mathbf{repeatEvery} \; t_o
\; f$ at $t$ is the value of $f$ at $t \; mod \; t_0$.

The assimilation rule can then be written as follows:
\begin{align*}
& \mathrm{Leaf}(\mathrm{age} \!= \!i, \mathrm{mass} \!= \!m), \: \mathrm{Cell}(\mathrm{carbon} \!= \!c) \: \xrightarrow{f(m, temp)}  \\
&\mathrm{Leaf}(\mathrm{age} \!= \!i, \mathrm{mass} \!= \!m), \:
  \mathrm{Cell}(\mathrm{carbon} \!= \!c+1) \; [light ]
\end{align*}

Very often in biology we have empirical relationships of various quantities with
time. Fluents can be used to encode these as well. Such empirical relationships
do not provide any mechanistic insight but they can be useful when we either do
not have data or do not want to model more mechanistically using rules.

\textit{Observables}

To solve the second problem, we introduce functions of the state of the system
called \textit{observables}. Observables are constructed using a combination of
database-inspired $\mathbf{select}$ and $\mathbf{aggregate}$ operations: % we
think of our state as a kind of database where we have a collection of agents
rather than the more usual (and very similar) collection of records. The $
\mathbf{select}$ operation selects agents of a particular type from %specifies a
part of the state, producing a binding of each such agent's attribute values;
the $\mathbf{aggregate}$ operation then folds the resulting collection of
attribute value bindings into a single value, using a specified combining
function and initial value.

 For example, to calculate the total mass  of the leafs in a state, we could write:
 \begin{align*}
lm = \mathbf{select} \, \mathrm{Leaf}(\mathrm{age} = i, \mathrm{mass} = m) \mathbf{;} \; \mathbf{aggregate} \;
 (total: \mathrm{int}.\, total + m) \; 0
\end{align*}
%
where we first select every agent that matches a $\mathrm{Leaf}$ pattern,
producing a binding of its $\mathrm{age}$ and $\mathrm{mass}$ attributes to $i$
and $m$, respectively, and then, starting from $0$, calculate the result using a
combining function that adds the value of $m$ in every such binding to the
running total.
 %
If we further wished to calculate the average mass of all the leafs in a state,
we could use the observable
\begin{align*}
nl = \mathbf{select} \, \mathrm{Leaf}(\mathrm{age} = i, \mathrm{mass} = m) \mathbf{;} \; \mathbf{aggregate} \;
 (count: \mathrm{int}.\, count + 1) \; 0
\end{align*}
%
to count the number of leafs in a state, and then divide $lm$ by $nl$.

Whenever an observable is used, one obtains its `fresh' value, even when the
underlying state has changed (\eg, by creating new leaves, in this case). For
example our \textit{leaf growth} rule now becomes:
%
\begin{align*}
&\mathrm{Leaf}(\mathrm{age} \!= \!i, \mathrm{mass} \!= \!m),\:
  \mathrm{Cell}(\mathrm{carbon} \!= \!c) \xrightarrow{h(nl-i, m, c)}\:   \\
  &
 \mathrm{Leaf}(\mathrm{age} \!= \!i, \mathrm{mass} \!= \!m+1),\:
    \mathrm{Cell}(\mathrm{carbon} \!= \!c-1) \; [c \geq  1]
\end{align*}
where we use our observable $nl$ directly in the rule rate. There is no need to
use the $\mathrm{Ros}$ agent any more as its only purpose was to keep track of
the number of leaves.

Note that observables, like fluents, can be used to model parts of the system in
cases where there is either not enough knowledge about a process, or else no
desire to model at a more mechanistic level. In this case, for example, the
leaves must have some mechanism for knowing their age. However this is not
central to our model, so we instead use global knowledge via observables to
abstract away from the details that would be involved in modelling that
mechanism.

The two features, fluents and observables, can be mixed arbitrarily along with
ordinary expressions. We might for example have a fluent inside an observable or
a fluent inside an observable and so on. For example we could write:
\begin{equation*}
f(m) \; \mathbf{when} \; nl > 10 \; \mathbf{else} \; f'(m)
\end{equation*}
where $nl$ is an observable for the number of leaves. This might be used to
introduce the crowding effect on rosette leaves; crowding affects the
assimilation rate as it reduces the photosynthetically active area.

\subsection{Field}
Going to the field where we have multiple plants we need some way of knowing
first which agents belong to the same plant and second the location of the
plant. We do not have an explicit way of representing these relations but we can
again use our attributes to encode them. We will use one identifier for the plant
and one for the patch. Every agent now has a further attribute to store this
information. For example, the $\mr{Leaf}$ agent type becomes:
$\mathrm{Leaf}(\mr{pos}:(\mr{int}, \mr{int}), \mathrm{age}:\mathrm{int},
\mathrm{mass}:\mathrm{real})$. The first element of the $\mr{pos}$ tuple is the
plant identifier and the second is the identifier of the patch it belongs to. We
further assume that we are in a suitably powerful programming language (\eg our
Haskell embedding) that we have (or can define) the functions $\mr{pid}$ to
select the first element of the tuple (plant id) and $\mr{cid}$ to select the second
element of the tuple (patch id).

The metabolic rules stay the same but add a further condition that the
agents on the lhs belong to the same plant. For example, maintenance respiration
now becomes:
\begin{align*}
  &\mathrm{Leaf}(\mathrm{pos}=p, \mathrm{mass}=m),\:
    \mathrm{Cell}(\mathrm{pos}=p', \mathrm{carbon}=c) \:
    \xrightarrow{h(m)} \\ &\mathrm{Leaf}(\mathrm{mass}=m),\:
                            \mathrm{Cell}(\mathrm{carbon}=c-g(m)) \; [c \geq g(m)
                            \, \land pid \, p = pid \, p']
\end{align*}

We assume that competition between plant affects the amount of light they
receive, which in turn affects the photosynthesis rate. The assimilation rule
will therefore change so that the light fluent is modulated by a competition
index, which will be an observable on the global state of an entire patch. Here
we assume that we are in our Haskell embedding and we have access to the full
power of the language. We define the following Haskell functions:
\begin{align*}
& \mr{comp} \: p = f (\mathbf{select} \; \mathrm{Leaf}(\mathrm{pos}=p' ,
  \mathrm{mass} = m) \; ; \; \mathbf{aggregate} \; (c: \mathrm{real}. \,
  \mr{sumComp}) \; 0 \\
& \mr{light} \: p = \mathbf{repeatEvery} \; 24.0 \; (\mathrm{light_{max}
                          } \cdot \mr{comp} \, p \; \mathbf{when} \; (6 < \mathbf{time} < 18) \; \mathbf{else} \; \mathrm{0})
\end{align*}
that take as input a $\mr{pos}$ tuple and return a relevant expression for that
particular plant in that patch. The combining function $\mr{sumComp}$ is defined
as: $\mr{sumComp} = \mathbf{if}\; cid \, p' = cid \, p \land pid \, p' \neq pid \, p
\;\mathbf{then} \; c + m; \; \mathbf{else} \; c$ that sums the masses of all
leaves in a patch except from the given plant. Then with the modulated light
definition the assimilation rules will change to:
\begin{align*}
& \mathrm{Leaf}(\mathrm{pos} \!= \!p, \mathrm{age} \!= \!i, \mathrm{mass} \!= \!m), \:
                 \mathrm{Cell}(\mathrm{carbon} \!= \!c) \: \xrightarrow{f(m,
                 temp, light \, p)}  \\
&\mathrm{Leaf}(\mathrm{age} \!= \!i, \mathrm{mass} \!= \!m), \:
  \mathrm{Cell}(\mathrm{carbon} \!= \!c+1) \; [light \, p > 0]
\end{align*}

\section{Root development}
For the root example, we again assume we are in our Haskell implementation and
we therefore have Haskell types and expressions in our disposal for the
definition of the model. We define the following agent types:
\begin{itemize}
\item a $\mr{Cell}$ type to represent the root cells in the array defined as:
  \begin{align*}
    \mr{Cell}(&\mr{id}: \mr{int}, \\
              &\mr{l, r}: \mr{link}, \\
              &\mr{s, a, d}: \mr{real}, \\
              &\mr{m}:\mr{mode})
    \end{align*}
    The type label $\mr{link}$ is interpreted as the set
    $\mathbb{N} \cup \{\epsilon\}$ and the relevant attributes ($\mr{l}$ for the
    left and $\mr{r}$ for the right neighbour) represent either a bound state
    (via an identifier) or a non-bound state ($\epsilon$). The type label
    $\mr{mode}$ is intepreted as the set $\{ 1, 2 \}$ and it represents the
    state of the cell, either growing (1) or idle (2). The other attributes
    represent the size of the cell ($\mr{s}$) and its auxin ($\mr{a}$) and
    division-factor ($\mr{d}$) concentrations (Figure~\ref{fig:rootDevChromar}).
  \item a $\mr{Shoot}$ agent type to represent the entire shoot of the plant
    defined as $\mr{Shoot}(\mr{s}: \mr{real})$ where the attribute $\mr{s}$
    represents the size of the shoot.
  \end{itemize}
  
\begin{figure}
    \centering
    \includegraphics[width=0.7\linewidth]{figures/rootChromar.eps}
    \caption{Scheme for capturing the `next-to' relation between cells and the
      division rule.}
    \label{fig:rootDevChromar}
  \end{figure}

In order to be able to give fresh identifiers to cells created by division we
also need to know the current number of cells in the array. We do this using
an observable:
\begin{align*}
nc = \mathbf{select} \; \mathrm{Cell}()\mathbf{;} \;\mathbf{aggregate} \; (count: \mathrm{int}.\, count + 1) \; 0
\end{align*}
As with our plant example in the previous section, we use a $\mathrm{count}$
expression that increases the running count by $1$ for each
$\mathrm{Cell}$. Note that like rules we can omit attribute bindings when the
values are not used in the subsequent definitions of the expression.

Turning to the dynamics of the system, for the auxin flow from the shoot we can
have the following rule:
\begin{align*}
\mr{Cell}(\mr{l} \!=\! \epsilon, a=a) \xrightarrow{} \mr{Cell}(\mr{a}=a + (a_{init} + \frac{0.17 \cdot \mathbf{time}}{t_{cc}}))
\end{align*}
where $t_{cc}$ is the time duration of the cell cycle. We represent auxin flow
from the shoot as auxin being produced in the first cell in the array. We
distinguish the first cell by requiring it to be unbound to the left. 

For the auxin diffusion we can write the following:
\begin{align*}
\mr{Cell}(\ar{r}{i}, \ar{a}{a}), \mr{Cell}(\ar{id}{i}, \ar{a}{a'})
  \xrightarrow{} \mr{Cell}(\ar{a}{a + D(a'-a)}), \mr{Cell}(\ar{a}{a' + D(a - a')})
\end{align*}

For the active transport of auxin in the cell by PIN proteins we have,
\begin{align*}
& \mr{Cell}(\ar{r}{i}, \ar{a}{a}), \mr{Cell}(\ar{id}{i}, \ar{a}{a'})
                 \xrightarrow{} \\
  &\mr{Cell}(\ar{a}{a - K_0 a PIN(a)}), \mr{Cell}(\ar{a}{a' + K_0
  a PIN(a)})
\end{align*}
Note that instead of modelling $\mr{PIN}$ concentration and its effect on auxin
directly, we do so implicitly through a function that modulates the
amount of auxin being transported. Unlike diffusion, which is bidirectional,
active transport takes place in the direction from root shoot to cap.

The production of the division factor is affected by the difference in auxin
concentration between neighbouring cells,
\begin{align*}
\mr{Cell}(\ar{id}{i}, \ar{a}{a'}), \mr{Cell}(\ar{l}{i}, \ar{a}{a}, \ar{d}{d})
  \xrightarrow{} \mr{Cell}(), \mr{Cell}(\ar{d}{d+\sigma(a'-a)})
\end{align*}

For the growth of the cell and degradation of the concentrations of the division
factor, $\mr{d}$, and auxin concentration, $\mr{a}$, we write:
\begin{align*}
\mr{Cell}(\ar{s}{s}, \ar{a}{a}, \ar{d}{d}) \xrightarrow{} \mr{Cell}(\ar{a}{a (1 - d +
  \frac{v(s)}{s})}, d=d (1 - k(a) + \frac{v(s)}{s}), \ar{s}{v(s)})
\end{align*}
The degradation of the division factor, again, depends on the auxin
concnetration in the cell.

For the cell cycle we need to switch the mode of the cell, $\mr{m}$, from
growing to idle.
\begin{align*}
& \mr{Cell}(\ar{m}{1}, \ar{s}{s}) \xrightarrow{\rho(s)} \mr{Cell}(\ar{m}{2}) \,
  \text{with} \\
&  \rho(s) = \frac{1}{1 + e^{- \frac{s-s_{\mr{min}}}{\mathbf{time}}}}
\end{align*}

For cell division, we create a new cell on the right of the dividing cell and
split the concentrations of auxin and the dividing factor between the two cells:
\begin{align*}
  &\mr{Cell}(\ar{id}{i}, \ar{r}{k}, \ar{a}{a}, \ar{d}{d}, \ar{s}{s}, \ar{m}{2}),
  \mr{Cell}(\ar{id}{k}, \ar{l}{i}) \xrightarrow{\rho(d)} \\
  & \mr{Cell}(\ar{id}{i}, \ar{id}{nc}, \ar{a}{a/2}, \ar{d}{d/2}, \ar{s}{s/2},
    m=1), \\
  &  \mr{Cell}(\ar{id}{nc}, \ar{l}{i}, \ar{r}{k}, \ar{a}{a/2}, \ar{d}{d/2}, \ar{s}{s/2},
    m=1), \\
  &  \mr{Cell}(\ar{id}{k}, \ar{l}{nc})
\end{align*}
The cell divides only in the idle phase and the division sets the the mode back
to growing while the rate of division depends on the concentration of the
division factor. The dividing cell gets pushed to the left, keeping its id and
changing its right-neighbour identifier to the identifier of the new cell. The
new cell gets a fresh identifier from our cell counter observable $nc$ and a
right-neighbour identifier from the old neighbour of its mother cell
(Figure~\ref{fig:rootDevChromar}).

Finally, cells die at the root cap:
\begin{align*}
\mr{Cell}(\ar{r}{\epsilon}) \xrightarrow{\rho_{\mr{death}}} \emptyset
\end{align*}
We distinguish the end of the root by requiring the cell to be unbound on the
right.

Finally, the growth of the shoot follows an abstract functional form:
\begin{align*}
\mr{Shoot}(\ar{s}{s}) \xrightarrow{} \mr{Shoot}(\ar{s}{f(w, \mathbf{time})})
\end{align*}
The growth function is a phenomenological function of time and more
interestingly water uptake. Assuming that the water uptake depends on the root
size, we model it as follows:
\begin{align*}
& rs  = \mathbf{select} \; \mathrm{Cell}(\ar{s}{s}) \mathbf{;} \; \mathbf{aggregate} \;
                                                              (total:
                                                              \mathrm{real}.\,
                                                              total + s) \; 0 \\
& w = f(\mathbf{time}) \; \mathbf{when} \; rs > s_0 \; \mathbf{else} \;
f(\mathbf{time}) \cdot d                                       
\end{align*}
where $rs$ is an observable abstracting the size of the entire root from the
size of the constituent cells and $w$ is a fluent that uses that size to compute
the water uptake. The bigger the root the more water the plant takes. We use the
fluent here to model a process that while relevant for our process we do not
wish to model in detail and mechanistically via rules.

\section{Comparison to related work}
Havign seen the properties of Chromar in action in this chapter and of other
languages in the same design space in the previous chapter, we can compare their
features. Here we will discuss properties relating to general usability features
that can be understood without the formal definitions. Comparison and discussion
related to more technical parts is done at the point where these are introduced
(see next Chapters for formal definition of Chromar and its embedding in
Haskell).

Table~\ref{tab:comp} shows a summary of the discussion points and a comparison
of the main features of Chromar and the languages from the previous chapter.

\begin{center}
  \begin{tabular}{p{0.15\linewidth}p{0.15\linewidth}p{0.15\linewidth}p{0.15\linewidth}p{0.15\linewidth}p{0.15\linewidth}}
    \toprule
    & PN & SMMR & CPN & DG & Simile\\
    \midrule
Objects \newline (dynamics) & Yes \newline (rule) & Yes \newline (rule) & Yes
                                                                          \newline
                                                                          (rule)
                      & Yes \newline(rule) & Yes \newline (population) \\
    \addlinespace[0.2cm]
Relations \newline (explicit) & No & Yes \newline (hierarchy) & No \newline
                                                                (implicit via
                                                                attrs) & No \newline (implicit via attrs & No \newline (implicit via attrs) \\
    \addlinespace[0.2cm]
    Properties \newline (dynamics) & No & No & Yes \newline
(transitions) & Yes \newline (transitions or ODEs) & Yes \newline (ODEs) \\
    \addlinespace[0.2cm]
    State \newline abstraction & No & No & No & No & Yes \\
    \addlinespace[0.2cm]
    Time & No & No & No & Yes \newline (via ODEs) & Yes \\
    Notes & & & & & \\
    \bottomrule
\end{tabular}
\end{center}

\subsection{Basic Chromar}



\subsection{Extended Chromar}

\subsection{Haskell}


\singlespace


\printbibliography[heading=bibintoc]

%% ... that's all, folks!
\end{document}

%%% Local Variables:
%%% mode: latex
%%% TeX-master: t
%%% End: