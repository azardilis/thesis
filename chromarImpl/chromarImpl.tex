\documentclass[phd]{infthesis}
\usepackage[utf8]{inputenc}
\usepackage[T1]{fontenc}
\usepackage[british]{babel}
\usepackage{microtype}
\usepackage[usenames,dvipsnames,svgnames,table]{xcolor}
\usepackage[english=british,autopunct=false]{csquotes}
\usepackage[natbib=true,style=authoryear-comp,maxbibnames=6]{biblatex}
\usepackage{graphicx}
\usepackage{textcomp}
\usepackage{wrapfig}
\usepackage{xfrac}
\usepackage{xspace}
\usepackage{mathcommon}
\usepackage[sc]{mathpazo}
\usepackage{hyperref}
\usepackage{expl3}
\usepackage{enumitem}
\usepackage{booktabs}
\usepackage{tabularx}
\usepackage{mathpartir}
\usepackage{fancyvrb}
\usepackage{inconsolata}

\frenchspacing

% Bibliography
\addbibresource{chromarImpl.bib}
\bibliography{chromarImpl}

% Text
\newcommand{\ie}{i.e.\xspace}
\newcommand{\eg}{e.g.\xspace}

% Referencing
\newcommand{\chp}[1]{\S\ref{chp:#1}}
\newcommand{\sct}[1]{\S\ref{sec:#1}}
\newcommand{\ssec}[1]{\S\ref{subsec:#1}}
\newcommand{\eqn}[1]{Eq.~\ref{eq:#1}}
\newcommand{\eqns}[2]{Eq. \ref{eq:#1} and \ref{eq:#2}}
\newcommand{\lem}[1]{Lemma~\ref{lemma:#1}}
\newcommand{\lems}[2]{Lemmas \ref{lemma:#1} and \ref{lemma:#2}}
\newcommand{\thm}[1]{Th.~\ref{thm:#1}}
\newcommand{\fig}[1]{Fig.~\ref{fig:#1}}
\newcommand{\diagram}[1]{diagram~\ref{eq:#1}}
\newcommand{\app}[1]{Appendix~\ref{app:#1}}
\newcommand{\mcite}[1]{\textcolor{gray}{#1}} % missing cite
\newcommand{\defn}[1]{Def.~\ref{def:#1}}
\newcommand{\prop}[1]{Prop.~\ref{prop:#1}}

% Math
\renewcommand{\tuple}[1]{\left(#1\right)}
\DeclareMathOperator*{\expn}{exp}
\renewcommand*{\exp}[1]{e^{\,#1}} % \mathrm{e}^{#1}}
\renewcommand{\qedsymbol}{\ensuremath{\blacksquare}}
\newcommand{\partialto}{\rightharpoonup}
\newcommand{\id}{\vec{1}} % identity function
\newcommand{\mr}[1]{\mathrm{#1}}

\newcommand{\den}[1]{\llbracket #1 \rrbracket}
\newcommand{\m}[1]{\{\!| #1 |\!\}}
\newcommand{\M}[1]{\mathcal{#1}}
\newcommand{\MS}[0]{\mathrm{M}}
\newcommand{\SQ}[0]{\mathrm{S}}
\newcommand{\s}[1]{\underline{#1}}
\newcommand{\G}[0]{\Gamma}
\newcommand{\D}[0]{\Delta}
\newcommand{\mytt}{t\!t}
\newcommand{\myff}{f\!\!f}

\newcommand{\V}{\mathrm{V}}

\newcommand{\sel}{\mathrm{sel}}
\newcommand{\fold}{\mathrm{fold}}

\newcommand{\ms}{\mathrm{ms}}


\newtheorem{mydef}{Definition}
\def\dotminus{\mathbin{\ooalign{\hss\raise1ex\hbox{.}\hss\cr
  \mathsurround=0pt$-$}}}
\setlength{\tabcolsep}{8pt}
\renewcommand{\arraystretch}{1.0}

\newcommand{\match}{m}
\newcommand{\up}[1]{\uparrow\! #1}

\newcommand{\n}{\mathrm{n}}

% Other stuff
\newcommand{\maybe}[1]{\textcolor{gray}{#1}}
\newcommand{\todo}[1]{\textcolor{red}{TODO: #1}}

% rules
\newcommand{\ar}[2]{\mr{#1} \! = \! {#2}}

\setlength{\tabcolsep}{8pt}
\renewcommand{\arraystretch}{1.2}

\begin{document}


\chapter{Implemented Chromar}
\section{Haskell embedding}
\label{sec:impl}
As we have seen, extended Chromar is abstract in that no choice is made of the
expression language or its types. In this section we describe the implementation
via embedding in Haskell of one concrete realisation of Chromar where we fix the
expression language $E$ and set of types $T$ to Haskell expressions and types
respectively.

Embedding a domain-specific language inside a host general purpose programming
languages has the advantage that it allows the reuse of the functionality of the
host language for the implementation therefore relieving some of the effort
involved in creating the infrastructure for a new language. From the
programmer's point of view it also allows the possibility of mixing features of
the host language with features of the hosted language thus getting a more
expressive result. There are two basic ways of embedding a language: (i) shallow
embedding where the domain-specific constructs are directly expressed in host
language terms (usually through a library) and (ii) deep embedding where
embedded language terms are built in some representation in the host language
and are then given meaning via interpretation \citep{hudak_modular_1998}. There
are advantages and disadvantages to both approaches. Here we choose a middle
ground where we use shallow embedding for expressing agent types, initial state,
and outside value environment (for the definitions of constants, functions and
so on) and deep embedding (via quasiquotations) using domain-specific syntax for
rules and enriched expressions. Such a mixed-level embedding in Haskell has been
explored before for another modelling language, Hydra, in the context of
non-causal, hybrid modelling \citep{giorgidze_mixed-level_2010}.

We next go through the different entities in extended Chromar and see how they
are represented in Haskell. We will use \texttt{typeface} font for Haskell code
to distinguish it from the abstract Chromar syntax.


\subsection{Agents}
Agent type declarations of the form $\mathbf{agent} \; n_a = (\s{n_f : b})$
become Haskell types with $n_a$ as the name of the type constructor and a
sequence of named, typed fields (records). Sequences of agent declarations,
$\s{a_d}$, can be collected in a union type. For example our $\mathrm{Leaf}$ and
$\mathrm{Cell}$ agent declarations from Section~\ref{sec:overview} are written
thus:
\begin{center}
\begin{BVerbatim}
data Agent
    = Leaf { age :: Int
           , mass :: Double }
    | Cell { carbon :: Double }
\end{BVerbatim}
\end{center}
where the \texttt{data} keyword introduces a new datatype.  Agents are values of
the \texttt{Agent} type, for example \texttt{Leaf\{age=1, mass=1.5\}} and
\texttt{Cell\{carbon=1.2\}}.  To construct multisets of Agents, we introduce the
parametric type \texttt{Multiset a = [(a, Int)]}.  Then the concrete type
\texttt{Multiset Agent} maps agents to their counts.
Given a function \texttt{mset} that constructs a multiset from a list (analogous to our function $\ms$ from Section~\ref{sec:semantics}),
our initial state definition becomes:
\begin{center}
\begin{BVerbatim}
init = mset [ Leaf{age=1,mass=0.2}, 
              Leaf{age=2, mass=0.1}, 
              Cell{carbon=0.5} ]
\end{BVerbatim}
\end{center}


\subsection{Rules (via Quasiquotation)}
\label{sec:ruleQQ}
We use a deeper embedding of rules where we allow the expression of rules
directly in domain-specific syntax using Quasi-quotes. Quasi-quotes is a Haskell
language extension that allows the use of special syntax (i.e., non standard
Haskell syntax) inside \texttt{[\$quoter| ... |]} brackets along with the name
of a quoter function (\texttt{\$quoter}) that is a function that takes the
string inside the brackets and produces Haskell abstract syntax. The produced
abstract syntax is injected by the Haskell compiler in place of the quotes.

The implementation here provides the \texttt{rule} quoter function that takes
the string inside the quotes and produces a reaction generation function (or
rather the abstract syntax of such a function) that given a state and time
produces a multiset of reactions. We therefore take the rule to be a reaction
generating function of type: \texttt{Time -> Multiset a -> [Reaction a]} where a
reaction is naturally defined as a record type parameterised on an agent type:
\begin{center}
\begin{BVerbatim}
data Reaction a = Reaction
    { lhs :: Multiset a
    , rhs :: Multiset a
    , rate :: Double
    }
\end{BVerbatim}
\end{center}
Our quoted rules have the following syntax that is almost the same as the one we presented earlier:
$$\mathrm{r_q} ::= \; \s{n_a\{\s{n_f \mathbf{=} id}\}} \; \rightarrow \; \s{n_a'\{\s{n'_f \mathbf{=} e}\}} \; @\,e_r \; {[} e_c {]} $$                         
For example our growth rule from the plant model example becomes:
\begin{center}
\begin{BVerbatim}
growth = [rule| Leaf{mass=m}, Cell{carbon=c} -->
                Leaf{mass=m}, Cell{carbon=c-1} @g m [c >= 1] |]
\end{BVerbatim}
\end{center}
This looks very close to our abstract syntax, but with some minor syntactic
differences such as the placement of the rate expression at the end of the rule
preceded by the \texttt{@} symbol. Crucially, being inside a programming
language means that we can use any valid Haskell expression in the places where
expressions are expected, i.e., in the values of fields in the right-hand side
of rules, rates, and conditions. A very wide range of expressions are therefore
supported, without further effort. We next look at the rule quoter function in
more detail.

\subsubsection*{Rule quoter function}
The \texttt{rule} quoter function takes a quoted rule expression, $r_q$, and
produces a Haskell function of type \texttt{Time -> Multiset a -> [Reaction a]}
where \texttt{[Reaction a]} is a list of reactions and \texttt{Time} is a
synonym for \texttt{Double}. The rule function needs to find all matches
(according to the rule lhs) and then generate a concrete reaction for each.
This leads to the two parts of the rule functions we need to construct: the
query part --- for finding the matches --- and the reaction generation part. For
the query part we use list comprehensions and pattern matching for binding the
variables in the rule lhs. This structure follows the definition of the
reactions denoted by a rule we have seen before (see $\mathcal{R}$,
Section~\ref{sec:ruleSem}).

The following example illustrates the translation from quoted rule expression to
rule functions. Consider the agent declaration
$\mathbf{agent} \; \mathrm{A}(\mathrm{a}:\mathrm{int})$ and rule
$\mathrm{A}(\mathrm{a}=x), \mathrm{A}(\mathrm{a}=y) \xrightarrow{f(x)}
\mathrm{A}(\mathrm{a}=x) \; [x > y]$. For the query part we make one
comprehension generator statement for each agent on the left-hand side. The
statement for the first agent becomes: \texttt{(A\{a=x\}, \_) <- s} for some state
$s$ of type \texttt{Multiset Agent}. Pattern matching ensures that we only look
at $A$'s in the state and that we also bind the \texttt{x} variable. Note that
we ignore the count of each agent-count pair since we later calculate the
multiplicity of the entire lhs inside the state in the reaction generation part
of the comprehension.  For each such successful match of the first statement we
continue to the second statement with \texttt{x} bound. The statement for the
second element of the rule lhs is similarly \texttt{(A\{a=y\}, \_) <- s}.  With
\texttt{x} and \texttt{y} bound we have our match $m$ (as defined in
Section~\ref{sec:ruleSem}) and we produce a reaction if the condition is true in
the environment extended with $m$. The full comprehension statement is thus:
\begin{center}
\begin{BVerbatim}
r :: Multiset Agent -> Time -> [Reaction Agent]
r s t =
     [ Reaction
      { lhs = mset [A{a = x}, A{a = y}]
      , rhs = mset [A{a = x}]
      , rate = rr
      }
     | (A {a = x}, _) <- s
     , (A {a = y}, _) <- s
     , x > y
     , let rr = (f x) * m (mset [A{a = x}, A{a = y}]) s
     , rr > 0 ]
\end{BVerbatim}
\end{center}
Each rule generates a multiset of reactions, as noted earlier, since the same
reaction can occur from two different matches. The multiset is given as a list
where the order is not important. Function \texttt{m} is the $\mu$ function that
calculates the multiplicity of the match (defined in
Section~\ref{sec:stoch}). The multiplicity also makes sure that the two patterns
do not match the same agent in the state. The comprehension allows that, which
means we create some spurious reactions, but the rate of those will be 0. The
rate function, \texttt{f}, is provided by the outside Haskell environment. This
is an example where we leverage the strength of the host language.


\subsection{Fluent and Observable features (enriched expressions)}
Enriched expressions are also embedded using Quasi-quotes that allows the direct
use of their syntax. The implementation provides a quoter function, \texttt{er},
for the translation of the surface syntax to its interpretation. The
interpretation of an enriched expression is a function from states and time to
values in $V$, exactly like our formal definition in
Section~\ref{sec:extSem}. In Haskell we use the following type:
\begin{center}
  \texttt{data Er a b = Er \{ at :: Multiset a -> Time -> b\}}
\end{center} 
parameterised by the agent type (\texttt{a}) and the return type (\texttt{b}).

The surface syntax for enriched expressions is the same as the abstract one with
the exception of the $\mathrm{where}$ construct. The most common use-case of the
$\mathrm{where}$ construct is to mix expressions from the provided language $E$
with fluents and observables. For example being able to write $nl + 1$ for some
observable $nl$. In the current implementation we only cover this particular
sub-case where variables referring to enriched expressions are embedded inside
Haskell expressions. Instead of being `tagged' with the $\mathrm{where}$ clause,
we surround them with \texttt{\$..\$} symbols. This resolves the problem of
mixing simple and functional values. The functional values (inside
\texttt{\$..\$}) will be interpreted to their simple value at current state and
time. Other local definitions of enriched expressions can be done using the
restricted scope declaration mechanisms of the host language (\texttt{let} and
\texttt{where} in Haskell). All other forms of enriched expressions remain the
same as in the grammar provided in Section~\ref{sec:extSyntax}.

We omit a formal translation from surface syntax to the Haskell \texttt{Er}'s
but go through an example to illustrate the \texttt{er} quoter
function. Consider for instance our enriched expression example from
Section~\ref{sec:extChromar}, giving alternative rate functions depending on the
current number of leaves in the plant:
$$
f \; \mathbf{when} \; nl  > 10 \; \mathbf{else} \; f'
$$
where $nl$ is an observable for the number of leaves. This will be written as:
\begin{center}
\begin{BVerbatim}
rateF = [er| f when $nl$ > 10 else f'|]
  where
    nl = [er| select Leaf{age=i,mass=m}; aggregate (count. count + 1) 0 |]
\end{BVerbatim}
\end{center}
Note that the \texttt{\$nl\$ > 10} to embed an enriched expression inside a
normal Haskell expression instead of the
$\mathrm{n} \; \mathbf{where} \; n \; \mathbf{is} \; \mathrm{nl}$ construct from
our abstract syntax. The number of leaves observable ($nl$) is given using
Haskell's \texttt{where} clause. The \texttt{f} and \texttt{f'} functions come
from the outside Haskell environment. The above \texttt{rateF} quote will be
translated to:
\begin{center}
\begin{BVerbatim}
[er| f when $nl$ > 10 else f'|] = Er {
  at = \s t -> if at [er|$nl$ > 10|] s t
                 then at [er|f|] s t
                 else at [er|f'|] s t }
               
[er|$nl$ > 10|] = Er { at = \s t -> at nl s t > 10 }
[er|f|] = Er { at = \s t -> f }
[er|f'|] = Er { at = \s t -> f' }
\end{BVerbatim}
\end{center}
The first equality is an implementation of the semantics of conditional
expressions in Section~\ref{sec:extSem}.  The rest follow the semantics of
ordinary expressions.  Note how we unfold the enriched expression,
$\mathrm{nl}$, to its evaluation (\texttt{at nl s t}) to achieve the same effect
as evaluating the expression in an extended environment (as defined in the
semantics of the $\mathrm{where}$ construct, Section~\ref{sec:extSem}).  The
interpretation of observables will depend on the ordering imposed by our list
representation of multisets.  While this is not consistent with the semantics,
we do not see any problems arising from that as the difference only occurs for
non-commutative fold functions, which, in any case, the programmer should not
employ (see \textit{Semantics of enriched expressions},
Section~\ref{sec:extSem}, for a discussion of the validity of the folding
function when using linearised multisets).  The translation for other types of
enriched expressions similarly follows the semantics given in
Section~\ref{sec:extSem}.


\singlespace


\printbibliography[heading=bibintoc]

%% ... that's all, folks!
\end{document}