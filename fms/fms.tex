\documentclass[phd]{infthesis}
\usepackage[utf8]{inputenc}
\usepackage[T1]{fontenc}
\usepackage[british]{babel}
\usepackage{microtype}
\usepackage[usenames,dvipsnames,svgnames,table]{xcolor}
\usepackage[english=british,autopunct=false]{csquotes}
\usepackage[natbib=true,style=authoryear-comp,maxbibnames=6]{biblatex}
\usepackage{graphicx}
\usepackage{textcomp}
\usepackage{wrapfig}
\usepackage{xfrac}
\usepackage{xspace}
\usepackage{mathcommon}
\usepackage[sc]{mathpazo}
\usepackage{hyperref}
\usepackage{expl3}
\usepackage{enumitem}
\usepackage{booktabs}
\usepackage{tabularx}
\usepackage{mathpartir}
\usepackage{fancyvrb}
\usepackage{inconsolata}

\frenchspacing

% Bibliography
\addbibresource{fms.bib}
\bibliography{fms}

% Text
\newcommand{\ie}{i.e.\xspace}
\newcommand{\eg}{e.g.\xspace}

% Referencing
\newcommand{\chp}[1]{\S\ref{chp:#1}}
\newcommand{\sct}[1]{\S\ref{sec:#1}}
\newcommand{\ssec}[1]{\S\ref{subsec:#1}}
\newcommand{\eqn}[1]{Eq.~\ref{eq:#1}}
\newcommand{\eqns}[2]{Eq. \ref{eq:#1} and \ref{eq:#2}}
\newcommand{\lem}[1]{Lemma~\ref{lemma:#1}}
\newcommand{\lems}[2]{Lemmas \ref{lemma:#1} and \ref{lemma:#2}}
\newcommand{\thm}[1]{Th.~\ref{thm:#1}}
\newcommand{\fig}[1]{Fig.~\ref{fig:#1}}
\newcommand{\diagram}[1]{diagram~\ref{eq:#1}}
\newcommand{\app}[1]{Appendix~\ref{app:#1}}
\newcommand{\mcite}[1]{\textcolor{gray}{#1}} % missing cite
\newcommand{\defn}[1]{Def.~\ref{def:#1}}
\newcommand{\prop}[1]{Prop.~\ref{prop:#1}}

% Math
\renewcommand{\tuple}[1]{\left(#1\right)}
\DeclareMathOperator*{\expn}{exp}
\renewcommand*{\exp}[1]{e^{\,#1}} % \mathrm{e}^{#1}}
\renewcommand{\qedsymbol}{\ensuremath{\blacksquare}}
\newcommand{\partialto}{\rightharpoonup}
\newcommand{\id}{\vec{1}} % identity function
\newcommand{\mr}[1]{\mathrm{#1}}

\newcommand{\den}[1]{\llbracket #1 \rrbracket}
\newcommand{\m}[1]{\{\!| #1 |\!\}}
\newcommand{\M}[1]{\mathcal{#1}}
\newcommand{\MS}[0]{\mathrm{M}}
\newcommand{\SQ}[0]{\mathrm{S}}
\newcommand{\s}[1]{\underline{#1}}
\newcommand{\G}[0]{\Gamma}
\newcommand{\D}[0]{\Delta}
\newcommand{\mytt}{t\!t}
\newcommand{\myff}{f\!\!f}

\newcommand{\V}{\mathrm{V}}

\newcommand{\sel}{\mathrm{sel}}
\newcommand{\fold}{\mathrm{fold}}

\newcommand{\ms}{\mathrm{ms}}


\newtheorem{mydef}{Definition}
\def\dotminus{\mathbin{\ooalign{\hss\raise1ex\hbox{.}\hss\cr
  \mathsurround=0pt$-$}}}
\setlength{\tabcolsep}{8pt}
\renewcommand{\arraystretch}{1.0}

\newcommand{\match}{m}
\newcommand{\up}[1]{\uparrow\! #1}

\newcommand{\n}{\mathrm{n}}

% Other stuff
\newcommand{\maybe}[1]{\textcolor{gray}{#1}}
\newcommand{\todo}[1]{\textcolor{red}{TODO: #1}}

% rules
\newcommand{\ar}[2]{\mr{#1} \! = \! {#2}}

\setlength{\tabcolsep}{8pt}
\renewcommand{\arraystretch}{1.2}

\begin{document}
\chapter{Framework Models}
Understanding the links between biological processes at multiple scales,
from molecular regulation to populations and evolution, is a major
challenge in understanding life. As systems become more complex we need
models to describe our understanding and help our thinking when trying
to explain and make predictions across scales. This approach has also
been proposed in attempts to engineer crop traits starting from genetics
or from genomes (Welch \emph{et al.}, 2005; Yin and Struik, 2008, 2010;
Parent and Tardieu, 2014; Wu \emph{et al.}, 2016; Chenu \emph{et al.},
2018), where simpler models have demonstrated both the potential of crop
modelling in general and the significant demands of detailed models for
empirical data that varies in availability (Hammer \emph{et al.}, 2006;
Asseng \emph{et al.}, 2013). For micro-organisms, comprehensive models
link the metabolic and molecular level with the cellular (Karr \emph{et
al.}, 2012) and population growth scales (Weiße \emph{et al.}, 2015),
whereas contemporary work in more complex organisms has necessarily
focused more narrowly (Buckley and Mott, 2013; Lynch, 2013; Zhu \emph{et
al.}, 2013; Klose \emph{et al.}, 2015; Le Novere, 2015; Hepworth
\emph{et al.}, 2018).

The concentration of plant science research on the laboratory model
species \emph{Arabidopsis thaliana} offers an opportunity for broad
understanding that includes mechanistic models (Chew et al., 2014a; Voss
et al., 2014, Urquiza et al., this volume). The Framework Model (FMv1)
represented vegetative growth of Arabidopsis in lab conditions (Chew
\emph{et al.}, 2014\emph{b}), starting from four independent models that
represent photosynthesis and carbon storage (Rasse and Tocquin, 2006),
plant structure and carbon partitioning among organs (Christophe
\emph{et al.}, 2008), flowering phenology (Chew \emph{et al.}, 2012) and
the circadian clock gene circuit and its output to photoperiodic
flowering (Salazar \emph{et al.}, 2009). Later updates focussed on plant
phenotypes controlled by the clock, such as tissue elongation and starch
metabolism (FMv2; Chew \emph{et al.}, 2017), or temperature and
organ-specific inputs to flowering (Kinmonth-Schultz \emph{et al.},
2018). The Framework Models align with community efforts to link
understanding of crop plant processes at multiple scales, for benefits
in agriculture (Wu \emph{et al.}, 2016; Zhu \emph{et al.}, 2016). Among
the limitations of the Framework Models, growth was limited to the
vegetative stage, ending upon flower induction. Without reproduction,
the models had no seed yield or link to evolutionary fitness. Without
seed dormancy, they lacked a major determinant of Arabidopsis life
history in the field. Their representation of the circadian clock was
also unnecessarily detailed for many studies outside chronobiology.

Other models have considered reproductive success through growth,
including for Arabidopsis. One simplified approach relates growth and
fitness only to the duration of the developmental period and not to its
timing in the year, ignoring environmental influences (Prusinkiewicz
\emph{et al.}, 2007). On the other hand, ecological phenology models of
the Arabidopsis life cycle consider natural environmental conditions but
ignore physical growth and development (Chuine and Beaubien, 2001;
Chuine, 2010; Burghardt \emph{et al.}, 2015). We apply a declarative
agent-based modelling approach (Urquiza et al., this volume) that
facilitates model composition, to develop FM-life, an extension of the
Framework Model to the whole Arabidopsis life cycle. FM-life includes a
simpler model of vegetative growth, FM-lite, without the clock circuit,
and a new model of inflorescence growth including reproduction. We
introduce a clustering approximation, in order to simulate FM-life
tractably at the population scale over decades. Testing the FM-life
model with contrasting environmental and genetic inputs shows that
ecological questions can increasingly be informed by mechanistic
understanding of growth processes (Millar, 2016; Doebeli \emph{et al.},
2017).

\subsection{Phenology models in
  Chromar}
\label{phenology-models-in-chromar}

In many phenology models, the simulated plant accumulates a conceptual
development indicator in every time unit as a function of the
contributing environmental factors, until a threshold is reached for
transition to the next developmental stage. For example, in a seed type
\(\text{Seed}(\text{dev}:\text{real})\), the dev attribute measures
development towards germination. A phenology rule for germination
affected by temperature and moisture, starting from dev value \emph{d},
could be:

\[\text{Seed}\left( \text{dev} = d \right){}\text{Seed}(\text{dev} = d + f\left( temp,\ moist \right))\]

On average once every time unit the \(\text{dev}\) attribute of a
particular seed will be increased from the present value, \(d\), by a
function of the contributing factors \(\text{temp}\) and
\(\text{moist}\). Further parameters might represent how sensitive the
seed is to the environmental factors. At the threshold
\emph{D\textsubscript{t}}, the seed germinates to a plant and resets the
development measure to 0:

\[\text{Seed}\left( \text{dev} = d \right)\ \text{Plant}\left( \text{dev} = 0 \right)\ \ \lbrack d > D_{t}\rbrack\]

where the expression inside the square brackets is used to indicate
conditional activity of the rule. The rule is active only when the
expression evaluates to true.

\subsection{The component models}
\label{the-component-models}

The models presented here represent the full life cycle in three stages:
seed dormancy (A, left panel, Figure 1), vegetative growth up to
flowering (B, left panel, Figure 1), and the reproductive stage up to
seed dispersal (C, left panel, Figure 1). Each model (A, B, C) includes
a phenology component that represents only timing
(Section~\protect\hyperlink{phenology-models-in-chromar}{2.2}). The
vegetative and reproductive stage models also represent biomass growth
at the organ level, based on the carbon budget of the plant. We varied
genetic parameters that affect only the timing components of A (seed
dormancy, \(\psi_{i}\)) and B (floral repression during vegetative
growth, \(f_{i}\)), for comparison to Burghardt \emph{et al.} (2015).
Each parameter value for an individual plant can be fixed or selected
probabilistically from a distribution as described (Burghardt \emph{et
al.}, 2015). The three models were integrated in a whole life-cycle
model of one plant (FM-life), and then extended to a population of such
plants.

\subsubsection{Seed dormancy model (A)}
\label{seed-dormancy-model-a}

The seed dormancy model is the Chromar version of the model of
(Burghardt \emph{et al.}, 2015), which is based in turn on (Alvarado and
Bradford, 2002) . It represents the development of a newly-dispersed
seed from \(dev = 0\) to a threshold value, \(D_{g}\), where the seed
germinates. Above baseline levels of temperature
\emph{T\textsubscript{b}} and of moisture (see below), increasing
moisture and temperature speed the progress towards germination. The
additional developmental units added (hydrothermal units,
\(\text{htu}\)) at every time unit are described by:

\[\text{htu}\left( t \right) = \ \left\{ \begin{matrix}
\left( \Psi\left( t \right) - \Psi_{b}\left( t \right) \right) \cdot \left( T\left( t \right) - T_{b} \right) \\
\ \left( \Psi\left( t \right) - \Psi_{b}\left( t \right) \right)\  \cdot \left( T_{o} - T_{b} \right) \\
0 \\
\end{matrix} \right.\ \ \begin{matrix}
\text{\ \ if\ \ }T_{b} < T\left( t \right) \leq T_{o}\ \text{and}\ \Psi_{b}\left( t \right) < \ \Psi(t) \\
\text{if\ \ T}\left( t \right) > T_{o}\ \text{and}\ \Psi_{b}\left( t \right) < \ \Psi\left( t \right) \\
\text{otherwise} \\
\end{matrix}\]

where \(\Psi(t)\) and \(T(t)\) give the moisture and temperature levels
at time \(t\) respectively. The definition distinguishes between
operating in suboptimal and supraoptimal temperatures (below or above
\(T_{o}\ \text{respectively})\ \). The baseline moisture is used to
represent the dormancy level of the seed. If \(\Psi_{b}\) is high, the
seed accumulates htu slowly for a given set of environment conditions,
whereas if \(\Psi_{b}\) is low, development is faster in the same
conditions. From an initial dormancy level, \(\psi_{i}\), seeds lose
dormancy (\(\Psi_{b}\) becomes smaller) over time at a rate \emph{r}
that is also a function of the environmental conditions, moisture and
temperature, and represents the observed process of after-ripening.
\(\psi_{i}\) is also used to represent the genetic effect on dormancy,
where high \(\psi_{i}\) represents stronger dormancy.

In Chromar, the \(\text{Seed}\) type captures information about the seed
development process:
\(\text{Seed}(\text{gntp}:(\text{real},\text{real}),\text{dev}:\text{real},r:real)\).
The \(\text{gntp}\) attribute stores the genotype of the organism,
\(\psi_{i}\) (seed dormancy level) and \(f_{i}\) (floral repression
level), which is passed on to the agents representing the later stages
of development and transmitted unchanged to the next generation.
\(\text{dev}\) stores the cumulative development indicator (sum of
\(\text{htu}\) up to the current timepoint), and \(r\) stores the
after-ripening up to the current timepoint. The development rule is the
following:

\[{\text{Seed}\left( \text{gntp} = a,\ \text{dev} = d,\ r = r \right)\ 
}{\text{Seed}(\text{dev} = d + htu\left( temp,\ moist,\ f(r,\ temp,\ moist),\ a.\psi_{i} \right),\ r = f\left( r,\ temp,\ moist \right))}\]

where \(\text{temp}\) and \(\text{moist}\) are fluents describing
temperature and moisture. We use the 'dot' (\(.\)) operator for
accessing the two genetic parameters of the \(\text{gntp}\) attribute.
The following rule represents germination, starting the vegetative
stage:

\[\text{Seed}\left( \text{gntp} = a,\text{\ dev} = d,\ r = r \right)\ \text{Plant}\left( \text{gntp} = a \right),\ \text{Root}\left( \ldots \right),\text{\ Leaf}\left( \ldots \right),\ \text{Leaf}\left( \ldots \right)\ \ \lbrack d > D_{g}\rbrack\]

The abstract \(\text{Plant}\) agent represents the plant at the
vegetative stage, along with agents for the root and the two cotyledon
leaves. The initial configuration of the organs at germination is as
introduced by Chew \emph{et al.} (2014\emph{b}) . Note that the genotype
attribute is passed from seed to emerged plant unchanged.

\subsubsection{Vegetative growth model (FM-lite) (B)}
\label{vegetative-growth-model-fm-lite-b}

For the vegetative stage we introduce a simplified version of FMv1 (Chew
\emph{et al.}, 2014\emph{b}) for use in studies that do not focus on
circadian timing. FM-lite has three constituent models represented in
Chromar with modifications to environmental responses (see below), and
without the fourth, circadian clock model of FMv1.

\paragraph{Timing}
\label{timing}

The timing component is the simpler flowering phenology model of
(Wilczek \emph{et al.}, 2009) rather than the augmented version in FMv1
(combination of Chew \emph{et al.} (2012) with Salazar \emph{et al.}
(2009) models). Vegetative development extends from \(dev = 0\) to a
threshold value, \(D_{f}\), where the plant flowers. The main
contributing environmental factors are photoperiod, ambient temperature
and vernalisation, giving the modified photothermal units,
\(\text{mptu}\), at a time \(t\) as:

\[\text{mptu}(t) = \text{photoperiod}(t) \cdot \text{thermal}(t) \cdot \text{vernalisation}(t)\]

The \(\text{vernalisation}\) term accounts for both the observed
requirement for a specific duration of exposure to cold and is also used
to represent the genetic effect on the progress towards flowering,
modelled as \(\text{vernalisation}(t) = f(wc,f_{i})\), where \emph{wc}
is the exposure to cold accumulated up to \(t\) and \(f_{i}\) is the
genetic parameter for the initial floral repression, as in Wilczek
\emph{et al.} (2009).

In Chromar, the plant type:
\(\text{Plant}(\text{gntp}:(\text{real},\text{real}),\text{dev}:\text{real},\text{wc}:\text{real})\)
includes the genotype attributes as noted above, the development so far
(\(dev)\), and finally the accumulated winter chilling (\(wc)\). The
development rule is then:

\[{\text{Plant}\left( \text{gntp} = a,\text{\ dev} = d,\ \text{wc} = w \right)\ 
}{\text{Plant}(\text{dev} = d + mptu\left( temp,\ dl,\ a.f_{i},\ w \right),\ \text{wc} = f\left( w \right))}\]

where \(\text{temp}\) and \(\text{dl}\) are fluents for temperature and
day length respectively, and \emph{w} is the present value of wc. The
transition to a flowering plant, \(\text{FPlant}\), follows:

\[\text{Plant}\left( \text{gntp} = a,\ \text{dev} = d,\ \text{wc} = w \right)\text{\ FPlant}\left( \text{gntp} = a \right)\text{\ \ }\left\lbrack d > D_{f} \right\rbrack\]

\paragraph{Growth}
\label{growth}

As in FMv1 (Chew \emph{et al.}, 2014\emph{b}), the growth component
includes a carbon budget for the plant from Rasse and Tocquin (2006),
which in turn includes photosynthesis rate equations based on the
Farquhar \emph{et al.} (1980) model. Growth at the organ level (rosette
leaves and root) is represented based on the Greenlab model (Christophe
\emph{et al.}, 2008). We will consider a sucrose carbon pool (\(c\)), a
starch carbon pool (\(s\)), and one pool for the biomass of the root and
each of the rosette leaves (left panel, ­­­Figure 2). In Chromar we have
the following agents to store the state (amount of carbon, or total
biomass) of these pools:

\begin{itemize}
\item
  \(\text{Cell}(c,s:\text{real})\) An agent that stores the amount of
  carbon in the sucrose (\(c\) attribute) and starch pools (\(s\)
  attribute). The amounts are carbon totals at the whole plant level.
\item
  \(\text{Leaf}(m:\text{real},i:\text{int})\) An agent that represents a
  rosette leaf. It has attributes for its mass (\(m\)) and its index of
  appearance (\(i\)).
\item
  \(\text{Root}(m:\text{real})\) An agent that represents the root with
  an attribute for its mass (\(m\)).
\end{itemize}

For each organ we have a growth flow from the sucrose carbon pool to the
mass of the organ (growth rule, ­­­Figure 2. The growth amount depends
on the demand function of the organ (\(d(i,t)\) rule rate function) and
its 'sink strength' (\(g(m)\)), which varies among organs. The value of
the demand function varies over time between 1 (maximum demand) and 0
(no demand) at the end of the expansion period of the organ. The amount
of carbon requested by an organ at every time unit is
\(g\left( m \right) \bullet d(i,t)\). Depending on the metabolic status
of the whole plant (level of \(c\) pool) and the requests from other
organs, an organ will receive either the full expected amount or a
portion of it.\\
A flow in the opposite direction (mobl rule, ­­­Figure 2) represents
carbon mobilization from the organs if the central sucrose pool
(\(\text{Cell}(c)\)) is reduced to a critical level. Thus each organ can
be either a net sink or source of carbon. For each organ, we also have a
flow leaving the system from the \(c\) pool for the cost of the
maintenance respiration and other processes of the organ (maint rule,
­­­Figure 2). Photosynthetic carbon fixation is represented by the
assimilation process (assim rule, ­­­Figure 2). The amount of assimilate
at every time unit is the product of the photosynthesis rate, which is a
function of environmental conditions at that time step, and the
projected area of the rosette. Here we use an observable,
\(a_{\text{ros}}\), for the effective rosette area, which is a function
of the global state of the rosette at the current time (derived from the
masses of all the current leaves) and takes into account the effect of
shading, as in Chew \emph{et al.} (2014\emph{b}). The carbon
partitioning function includes a baseline partitioning to starch, then
support of a target sucrose level, with excess sucrose supporting growth
and a final overflow to additional starch production, as in Chew
\emph{et al.(} 2014\emph{b}). At night, no photosynthesis occurs and
carbon from the starch pools flows to the sucrose pool (sdegr rule,
­­­Figure 2). Finally, we have the creation of new leaves, which impacts
the above processes indirectly by creating more demand for growth and
adding maintenance costs (leaf cr rule, ­­­Figure 2). Leaves are created
by the main apical meristem (\(\text{VAxis}\) agent) along with an
\(\text{LAxis}\) agent that can give rise to lateral branches after
flowering (see next section).

It is interesting to note that unlike FMv1 carbon partitioning between
processes and organs is done explicitly whereas in our Chromar
representation, partitioning is an emergent, stochastic effect of
competition for the finite amount of sucrose carbon in the main
reservoir. For example, partitioning of carbon among organs for growth
is done explicitly in FMv1 by dividing the demand of each organ by the
sum of the demands of all other organs
\(g\left( m \right) \cdot \ \frac{d(i,\ t)}{\sum_{}^{}{d(i,t\ )}}\). In
the Chromar representation we do not have this explicit division by the
global demand, which means that the amount of carbon that an organ gets
is higher at each growth event but growth events are rarer because not
all growth request are successful (competition). The competition
therefore recovers the explicit partitioning.

\emph{Modifications for natural conditions}\\
FMv1 was developed for lab conditions. As an initial approach to reflect
plant responses to the broader range of relevant conditions in nature,
we made the following changes:

\begin{itemize}
\item
  The rate of photosynthesis is set to 0 below 0 °C
\item
  The maintenance cost for an organ is also 0 below 0 °C
\item
  The rate of photosynthesis is affected by soil moisture through
  stomatal closure. The photosynthesis rate is affected by a stomata
  term \(f_{\text{stom}}(moist)\), which is a simple phenomenological
  function that relates soil moisture and stomatal closure (France
  \emph{et al.}, 1984).
\end{itemize}

These conservative changes give a lower bound on the effects of natural
weather conditions.

\paragraph{Comparison of FM-lite with
  FMv1}
\label{comparison-of-fm-lite-with-fmv1}

In addition to the weather responses, Wilczek flowering model and
emergent carbon partitioning among organs, our model representation uses
the stochastic rule-based Chromar as opposed to the deterministic Matlab
program of FMv1. In order to compare the model representations, we
simulated growth in the two models for a fixed number of hours in lab
conditions, where the modifications to weather responses have no effect.
The two models were simulated in lab conditions (22 °C, 12/12 light/dark
cycles) for \(800\) growth hours and showed comparable results (Figure
3). FMv1 was simulated in Matlab while FM-lite was simulated in the
Haskell implementation of Chromar and the results were averaged over
five runs. The rosette mass results are the closest since they represent
the development of multiple Leaf agents, masking the stochastic effects
on each Leaf. The difference between the final rosette mass of the FMv1
and FM-lite (averaged over 5 runs) simulations is within 10\% of the
final rosette mass in FMv1. The stochasticity is more apparent for the
root where the growth curves are further apart. The difference between
final root mass in FMv1 and FM-lite (averaged over 5 runs) is
\textasciitilde{}20\% of the final root mass in FMv1. Sucrose carbon
levels are also more variable in FM-lite, since the growth rule
(removing sucrose carbon from the central pool) provides organs with a
larger amount but less frequently than the small fixed amount at every
time step in FMv1 (see previous section).

\subsubsection{Reproductive stage model
  (C)}
\label{reproductive-stage-model-c}

\paragraph{Timing}
\label{timing-1}

The timing component is a thermal time model from Burghardt \emph{et
al.} (2015), representing the development of the inflorescence and seed
from \(dev = 0\) at flowering, to a threshold value, \(D_{s}\), where
the plant disperses its seeds. Here there is no genetic input and the
thermal units that accumulate at time \(t\) are simply the value of the
temperature at \(t\) above a base temperature \(T_{b}\):

\[tu(t) = \left\{ \begin{matrix}
T(t) - T_{b} & \text{if\ }T(t) > T_{b} \\
0 & \text{otherwise} \\
\end{matrix} \right.\ \]

Writing into Chromar we have an
\(\text{FPlant}(\text{dev}:\text{real})\) type for a flowered plant and
the following rule for its development that follows from the
\(\text{tu}\) definition above:

\[\text{FPlant}\left( \text{dev} = d \right) \rightarrow \text{FPlant}(\text{dev} = d + tu(temp))\]

Finally, the transition to seed happens when the accumulated development
reaches \(D_{s}\):

\[\text{FPlant}\left( \text{attr} = a,\text{\ dev} = d \right) \rightarrow \text{Seed}\left( \text{attr} = a,\ \text{dev} = 0,\ r = 0 \right)\ \ \lbrack d > D_{s}\rbrack\]

Note that the genotype attribute of the parent plant is transferred to
the seeds unchanged.

\paragraph{Growth}
\label{growth-1}

The growth component of the reproductive stage model is loosely related
to the Greenlab model (Christophe \emph{et al.}, 2008). The metabolic
processes affecting the carbon budget of the plant are the same as in
vegetative growth but with additional organ types to represent the
Arabidopsis inflorescences. Organs appear in units (metamers) with a
metamer identifier. Each growth unit on the main axis consists of an
internode (stem between leaves), a leaf, and a lateral meristem that can
give rise to a lateral axis. We consider only the primary axis and
secondary, lateral branches, thus metamers on the lateral axis lack a
further lateral meristem. All fruits on an axis are represented on its
last metamer, replacing the leaf; this metamer also lacks a meristem.
Two indices represent metamer position: the index of the metamer along
its axis and the index of the parent metamer along the primary axis
(left panel, Figure 4). We define the following new agent types to
represent this structure:

\begin{itemize}
\item
  \(\text{INode}(i,pi:\text{int},m:\text{real})\) to represent the
  internode (stem between successive leaves). Attribute \(i\) is the
  temporal index of appearance in its axis (primary or lateral) and
  attribute \(\text{pi}\) is the parent primary metamer. The cotyledons
  have indices 1 and 2 on the primary axis, for example.
\item
  \(\text{LLeaf}(i,pi:\text{int},m:\text{real})\) to represent a leaf on
  the lateral axes.
\item
  \(\text{Fruit}(i,pi:\text{int},m:\text{real})\) to represent a fruit
  on the axis.
\end{itemize}

The maximum number of inflorescence metamers on the main axis is taken
to be 20\% of the number of rosette leaves at flowering time (\(n_{f}\))
and given by \(v_{\max}(n_{f})\) (Pouteau and Albertini, 2009). The
maximum number of growth units on each lateral axis is given by
\(l_{\max}(i)\), a decreasing function of the index of the lateral axis
starting from a maximum of 6 at the axis after the cotyledons (index 3)
and going to a minimum of 1 at the topmost lateral branch (Mündermann
\emph{et al.}, 2005). The topmost lateral axis can only appear with a
delay after the apical fruit has appeared on the primary axis. Each
successive lateral branch going down can only start developing with a
delay after the fruit of the axis above it has appeared. The delay
associated with lateral axis growth, given in the rules by
\(t_{\text{del}}\), is a function of the metabolic state of the plant,
as described (Christophe \emph{et al.}, 2008).

The new organ types have associated sink strengths and demand functions.
The cauline leaves on the main axis contribute to the photosynthetically
active area and can shade the rosette leaves underneath them. The
lateral leaves contribute to photosynthesis without shading. Internodes
and fruits do not contribute to photosynthesis. Seeds are not directly
represented, so a birth function \(b(m)\) is required to calculate the
number of seeds for a given fruit mass \emph{m} at seed dispersal time,
as described below.

\subsection{Whole life cycle model,
  FM-life}
\label{whole-life-cycle-model-fm-life}

The Chromar framework allows us simply to concatenate the rules of
timing and growth components of the three models above, to represent the
whole life cycle. Then given an initial state with the genetic
attributes of the plant (\(\text{gntp}\) attribute of agents) and the
environmental conditions for a particular location, \(e(t)\), we can
simulate an entire life cycle from seed to seed. The timing components
of the model give us the timing within the year of the growth period
(vegetative + reproductive stages) and therefore the environmental
conditions that the plant is exposed to during growth. The growth
components predict growth at the individual organ level with these
environmental conditions and therefore give us the environmentally
determined seed number given by the \(b(m)\) function.

\subsection{Population level model and plotting
  conventions}
\label{population-level-model-and-plotting-conventions}

Since FM-life estimates the number of seeds at the end of the life
cycle, these can initiate multiple independent copies of the model in
the next generation. We then have a classical evolutionary birth
process, sometimes called a branching process since it unfolds in
tree-like way. The potential number of individuals in generation \(i\),
\(n_{i}\), is equal to the sum of the number of seeds produced by the
individuals in the previous generation (see Discussion). Dormant seed
never die in the model and may germinate after several years (Burghardt
et al., 2015).

Since we are using an individual-based model, \(n_{i}\) becomes
computationally prohibitive to simulate over decades of population
growth. In order to overcome this limitation, we simulated the timing
(phenology) and growth components sequentially and used conservative
birth functions \emph{b}(\emph{m}). Figure 5 introduces the plotting
conventions for these results. The timing components were first
simulated with \(b(m) = 1\), such that each plant makes one seed, as in
Burghardt et al. (2015). The phenological simulation results in an
unbranched sequence of developmental stage timings for each lineage
(Figure 5C). The simulation results for several decades typically
revealed a small number of life cycle growth strategies, from clusters
of individual life cycles. The clusters were generated using \(k\)-means
clustering, where \(k\) is chosen by visual inspection of the life cycle
plots (Figure 5A). Alternative clustering approaches might be an area
for future work. Figure 5A shows the distribution over a year of all
individual life cycles that conformed to two contrasting life cycle
strategies under environmental conditions for Valencia (see Results).
Cluster membership depends on the dates and durations of multiple
developmental stages. This is hard to visualize, because the timing of
any single developmental stage partially overlaps among different
strategies. Figure 5B therefore summarises the median dates of all three
developmental transitions in each strategy, here illustrated by 1. a
summer growth strategy and 2. a winter growth strategy. In the next
stage, the growth models were simulated once per cluster, with the
environmental conditions associated with the typical timing of that
cluster (median vegetative and reproductive stages). This returns the
typical biomass of organs over time, including the fruit mass at seed
dispersal (m1 for cluster 1, m2 for cluster 2; Figure 5D). Finally, each
life cycle is assigned the fruit mass \emph{m} associated with its
cluster, and thereby a growth-based, reproductive success
\emph{b}(\emph{m}) that evaluates to 0 in some cases. Thus, the second
stage recovers a version of the branching lineage tree, where some
lineages die out (Figure 5E).

Our output population measure is the total population of plants over all
lineages over all generations. For example, consider a lineage with
three generations starting with a plant with final fruit mass
\(m_{11}\). For the next generation we have \(b(m_{11})\) and then
\(b(m_{11}) \times b(m_{21})\). The population measure for that lineage
is \(1 + b(m_{11}) + b(m_{11}) \times b(m_{21})\). The population
measure for multiple lineages starting from multiple plants in the
initial population is the sum of the population measures of all the
lineages. This requires a birth function, which we use in a very simple
form, as follows:

\[b\left( m \right) = \left\{ \begin{matrix}
1 & \text{if\ }m > m_{0} \\
0 & \text{otherwise} \\
\end{matrix} \right.\ \]

A plant produces one seed or none, the latter in life cycles with fruit
mass at seed dispersal \emph{m} less than a threshold \(m_{0}\). Below,
we make some conservative choices for the value of the reproductive
threshold value, \(m_{0}\), to explore the effect on the output
population measure.

Finally, we distinguish three sources of variability in the population
model: (i) weather varies between years, (ii) genetic parameters can
vary among the initial population if their values are chosen
probabilistically, and (iii) simulation results vary due to
stochasticity in the model representation.

\subsection{Weather data}
\label{weather-data}

For the phenology model simulation we used the weather data that
accompanied the Burghardt et al. (2015) model, available from a Dryad
repository (Burghardt et al. 2014). In this dataset weather inputs over
60 years were generated stochastically for four locations in Europe:
Halle, Valencia, Norwich, and Oulu. The weather inputs include values
for temperature, moisture, and daylength.

For the growth simulations we used weather data from the ECMWF
ERA-Interim dataset over the years 2010-2011 (Dee \emph{et al.}, 2011).
A program was used to generate hourly inputs given daily averages from
the dataset for temperature and radiation. For the soil moisture input
used in the photosynthesis rate calculation we used a daily average of
soil moisture values from the dataset and assumed that is constant
throughout the day (swvl parameters in the ERA dataset).

\section{Results}
\label{results}

The population of FM-life models (see Methods) allows us to test how
growth processes that alter reproductive success affect the life history
strategies of Arabidopsis growing in different environmental conditions
(location) and with different genetic parameters in the initial
population. We can therefore explore the genotype x environment
interaction, using a population measure. To illustrate this potential,
we compare simulation results for two previously-studied locations,
Valencia (Spain) and Oulu (Finland), and two opposing combinations of
genetic parameters, high seed dormancy /high floral repression (HH) and
low seed dormancy /low floral repression (LL). Within an initial
population of 100 seeds, the seed dormancy levels, \(\psi_{i}\), were
assigned probabilistically, sampling from a normal distribution with
mean \(0.0\) and standard deviation, \(1\), for the Low dormancy level
(L) and mean \(2.5\) with the same standard deviation for the High
dormancy case (H). Floral repression was fixed at either \(0.598\) for
the Low level (L) and \(0.737\) for the High level (H), values that were
chosen to reflect the behaviour of natural populations of Arabidopsis in
Wilczek \emph{et al.} (2009). Both parameter choices follow Burghardt et
al. (2015). The simulation time period was 60 years and, as in Burghardt
\emph{et al (}2015), we discarded the first 15 years of the simulation
to focus on stable life history strategies. A key difference from the
earlier work is that even our conservative choice of birth function (see
Methods) allows some lineages to die out.

\subsection{Valencia}
\label{valencia}

Figure 6 shows the results of the two-stage simulation for a population
of the LL genotype in Valencia (Figure 6A). We identify four possible
life history strategies based on the timing (phenology) components of
the FM-life model:

\begin{enumerate}
\def\labelenumi{\arabic{enumi}.}
\item
  \emph{summer-only strategy} where the entire growth is in the summer.
  The growth period is quite short and the conditions unfavourably hot
  and dry. In the growth simulation, the rosette leaves senesce before
  the reproductive stage (blue curve). The drought effect on
  photosynthesis severely limits the carbon available for fruit mass
  (red curve).
\item
  \emph{spring strategy} where the entire growth period is in the
  spring. The growth period is only slightly longer than the
  \emph{summer-only} strategy but it falls in more favourable weather
  conditions. The rosette lifetime extends beyond flowering to support
  fruit growth, which combined with favourable weather gives high fruit
  mass.
\item
  \emph{winter-repr strategy} spans the winter/early spring period. A
  short vegetative period in the end of summer/early Autumn ends with
  flowering and a long reproductive stage over the winter/early spring.
  The rosette is senescing when favourable conditions return in early
  spring, seriously limiting fruit development.
\item
  \emph{winter-veg strategy} again spans the winter/early spring period.
  The life cycle duration is similar to strategy 3 but slightly later
  germination delays flowering until Spring. The rosette grows all
  winter, overlapping with a short reproductive stage and supporting
  high fruit mass.
\end{enumerate}

Plants with life cycle strategies 2 and 4 predicted orders of magnitude
more fruit mass than plants with life cycle strategy 3 and or the least
successful strategy 1 (Figure 6C). This result clearly ranked the
strategies available to plants of the LL genotype, although the absolute
values of the predicted biomass are less certain (see Discussion). The
100 plants amassed 4905 potential lifecycles over 45 years of
phenological simulation (Figure 6A). Without a minimum mass threshold
(\emph{m\textsubscript{0}}) for reproduction, 66\% of potential life
cycles followed the more successful \emph{spring} and \emph{winter-veg}
strategies (2 \& 4; Figure 6C). Figure 6D shows the sequential
transitions between strategies. For example, 60\% of potential plants
following the successful \emph{spring} strategy (2) disperse their seeds
early enough for the next generation to adopt the \emph{winter-veg}
strategy (4), achieving two generations per year. These transitions
underlie the bimodal distribution of life cycle times reported by
Burghardt et al. (2015) for this simulation.

Simulation of the HH genotype (Figure 6F, 6J) identified similar
strategies. Since the seed have longer dormancy, the population amassed
fewer potential life cycles (2954 as opposed to 4905 in the LL case;
Figure 6F). The growth and final fruit masses are different because of
slight variation in timing of the growth period but strategies 2 and 4
are again more successful than strategies 1 and 3 (Figure 6H). A higher
fraction of potential life cycles followed the successful strategies
(78\% as opposed to 66\% in the LL case; Figure 6G). Higher seed
dormancy reduced the germination in the summer and early autumn that led
to the less successful strategies 1 and 3, so any strategy was likely to
be followed by either strategy 2 or 4 in the next generation (Figure 6).

In order to calculate the population success we make two choices for the
reproduction mass threshold, \(m_{0}\), which eliminate one or both of
the least successful strategies. Choosing a value
\(m_{0} = 2 \times 10^{- 5}\) g (the mass of a single seed) eliminated
the \emph{summer-only} strategy from both genotypes, which gives a
population of 1210 plants over 45 years in the LL case (Figure 6C). The
HH genotype allows a larger percentage of viable life cycles but we
predict fewer plants (1020) since the number of potential life cycles
was lower (Figure 6H). Choosing a value \(m_{0} = 6 \times 10^{- 3}\) g
left only two viable strategies, 2 and 4, for both genotypes. Reciprocal
transitions between the strategies were still possible but
\emph{winter-veg} was strongly favoured (Figure 6E, 6J). The LL genotype
predicted 240 plants in total over 45 years, compared to 360 plants for
the HH genotype: G x E interaction favoured the HH genotype despite its
smaller number of potential life cycles. Thus, modelling the growth
processes not only distinguished among the potential life cycle
strategies within a genotype but also between the genotypes.



\subsection{Oulu}
\label{oulu}

The equivalent simulations were performed for conditions in Oulu,
Finland in the same LL and HH genetic backgrounds (Figure 7). The
results indicated 3 potential life cycle strategies (Figure 7A, 7B, 7E,
7F):

\begin{enumerate}
\def\labelenumi{\arabic{enumi}.}
\item
  \emph{summer-only strategy} where the entire life cycle occurs in the
  summer. The vegetative period is short, the rosette is very small and
  supports negligible fruit growth (Figure 7C).
\item
  \emph{winter-repr strategy} where a life cycle of almost a year has a
  very short vegetative stage, followed by a long reproductive stage
  over the winter. Again, the very small rosette supports little fruit
  growth in the following Spring.
\item
  \emph{winter-veg strategy} where the plant over-winters in the
  vegetative stage. Unlike in Valencia, the rosette grows little over
  the winter. Rapid rosette growth in the following spring supports a
  substantial inflorescence and fruit development, though the predicted
  fruit mass is smaller than in Valencia.
\end{enumerate}

The severe winter conditions limited the number of potential life cycles
to 2361 for the LL genotype or 363 for HH. A higher proportion of HH
life cycles followed the successful \emph{winter-veg} strategy (3; 32\%
against 24\% in LL; Figure 7B, 7G). Surprisingly, a majority of life
cycles for both genotypes followed the \emph{winter-repr} strategy (2).
Applying the reproductive threshold mass, \(m_{0}\), eliminated one or
both of strategies 1 and 2 (Figures 7C, 7G), suggesting a strong
selective pressure for greater floral repression to reduce the number of
\emph{winter-repr} life cycles. With \(m_{0}\) = \(2 \times 10^{- 3}\)
g, the LL genotype yielded 159 plants over 45 years compared to 53
plants for HH. All G x E combinations had actively-growing plants at the
end of the simulation. Interestingly, plants of the HH genotype had
higher average reproductive success per plant in Oulu yet the LL plants
were more successful by our population measure. The faster development
of LL plants allowed more, short lifecycles within the simulated
interval (consistent with the phenology model alone).

\section{Discussion}
\label{discussion}

We present a whole-life-cycle multi-model for growth and reproduction of
\emph{Arabidopsis thaliana}, FM-life, combining phenology models that
time the developmental stages and growth models to predict organ
biomass. The simple, FM-lite model of vegetative growth, and its
extension to the reproductive stage in FM-life, simulate broader,
mechanistically-founded components of fitness at the individual plant
level compared to the phenology models alone. Most insights from the
component models naturally remain (Rasse and Tocquin, 2006; Christophe
\emph{et al.}, 2008; Wilczek \emph{et al.}, 2009; Burghardt \emph{et
al.}, 2015). Multi-models are helpful in emphasising interactions. The
cauline leaves in our inflorescence model, for example, extend the
duration of photosynthetic competence. As cauline leaves can be produced
6 months later than early rosette leaves in the \emph{winter-veg}
strategy (Figure 7), they remain active photosynthetic sources (Earley
\emph{et al.}, 2009; Leonardos \emph{et al.}, 2014) when the rosette
leaves are senescing. The growth models provided the fruit mass that we
used as an indicator of reproductive success, such that metabolic and
developmental processes of growth informed a more mechanistic
understanding of ecological, population dynamics over multiple
generations.

The growth model allowed us to discriminate among alternative life cycle
strategies in each G x E combination, by selecting against strategies
that were compatible with the phenology models alone but had
qualitatively worse growth. In previous work, strategies with high seed
dormancy in southern Valencia and low dormancy in northern Oulu were
noted to align with the behaviour of the cognate wild populations
(Atwell \emph{et al.}, 2010; Chiang \emph{et al.}, 2011; Méndez-Vigo
\emph{et al.}, 2011; Burghardt \emph{et al.}, 2015). In each G x E
combination, individual plants in our simulations might adopt
alternative life cycle strategies. The less-successful strategies were
lethal in our model, eliminating \textgreater{}95\% of potential
lifecycles (simulated by the phenology model alone) for the LL genotype
in Valencia, for example (Figure 6A, 6C). Thus, our results supported
the observed genotypic distinction between Valencia and Oulu, because
the requirement for a minimum fruit mass eliminated more lineages of the
less-successful genotype in each case (Figures 6C, 6H and 7C, 7G).

Our approach might appear conservative, as the binary birth function
(one seed/no seed) ignored variation in seed mass among life cycle
strategies, which might otherwise reinforce the advantage of successful
strategies. The successful genotype LL in Oulu, however, had lower
reproductive success per plant than HH, suggesting a more subtle balance
of advantage. Genotypes with Low dormancy and High floral repression
(LH) are observed in far northern locations (Atwell \emph{et al.},
2010). We therefore simulated the LH genotype (Figure 8). LH plants
delayed flowering time enough to reduce the frequency of potential
\emph{summer-only} life cycles to 9\% compared to 15\% in LL (Figure 8B)
and increased the fruit mass of the successful \emph{winter-veg} life
cycle close to the HH genotype (Figure 8C). The LH model predicted
slightly higher reproductive success overall, returning 171 life cycles
(Figure 8C) compared to 159 for the LL variant (Figure 7C), consistent
with the observation of LH genotypes at this location.

A limitation of our work arises from the fact that the phenology
component models of FM-life have been validated against field data
(Wilczek \emph{et al.}, 2009; Burghardt \emph{et al.}, 2015) whereas the
growth component models have not (Rasse and Tocquin, 2006; Christophe
\emph{et al.}, 2008). Biomass simulations are inevitably sensitive to
the timing of the growth period, because a longer interval of
exponential growth in good conditions rapidly changes absolute biomass,
as illustrated in Figure 9D. Nonetheless, the FM-life model predicted
unreasonably high fruit mass in some cases. The binary birth function
ensured that this had no effect on our population measure. Among
possible gaps in understanding of the environmental effects on growth in
natural settings or in our representation, we repeat our previous
caution (Chew \emph{et al.}, 2014\emph{b}, 2017) that models of nutrient
balance for Arabidopsis will be helpful. Rosette biomass in the
Framework Model is understandably sensitive to photosynthetic parameters
(Chew \emph{et al.}, 2014\emph{b}) yet these have not been validated in
Arabidopsis across the wide range of photoperiods and temperatures
simulated here (Walker \emph{et al.}, 2013). FM-life predicts a
discretised fruit mass and hence reproductive success for a typical
representative of each life cycle strategy, approximating an underlying,
continuous distribution of fruit mass. The accuracy of this
approximation will depend on the variation within clusters. The benefit
lies in computational tractability, allowing us to simulate differential
reproductive success that is informed by understanding of growth
processes.

Our approach builds upon previous models that predict fitness and
population processes in Arabidopsis, which have focussed on
developmental components of fitness or on phenology (Prusinkiewicz
\emph{et al.}, 2007; Satake \emph{et al.}, 2013; Springthorpe and
Penfield, 2015). Linking these components sharpens ecological insight,
by understanding the performance of genetic variants in the environment
that underlies differences in fitness (see discussions in (Donohue
\emph{et al.}, 2015; Doebeli \emph{et al.}, 2017)) and can thus inform
evolutionary hypotheses. Adding genetic variation between generations
will in future model Arabidopsis evolution explicitly, perhaps after
competing genetic variants \emph{in silico} using adaptive dynamics
approaches (Brännström \emph{et al.}, 2013; Weiße \emph{et al.}, 2015).
Thus, the FM-life model offers a further tool to bridge among
disciplines in plant biology, ecology and evolution.

\singlespace

\printbibliography[heading=bibintoc]

%% ... that's all, folks!
\end{document}