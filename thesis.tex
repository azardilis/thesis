\documentclass[phd, logo, twoside]{infthesis}

\usepackage[utf8]{inputenc}
\usepackage[T1]{fontenc}
\usepackage[british]{babel}
\usepackage{microtype}
\usepackage[usenames,dvipsnames,svgnames,table]{xcolor}
\usepackage[english=british,autopunct=false]{csquotes}
\usepackage[natbib=true,style=authoryear-comp,maxnames=3,maxbibnames=6]{biblatex}
\usepackage{graphicx}
\usepackage{textcomp}
\usepackage{wrapfig}
\usepackage{xfrac}
\usepackage{xspace}
\usepackage{mathcommon}
\usepackage[sc]{mathpazo}
\usepackage{hyperref}
\usepackage{expl3}
\usepackage{enumitem}
\usepackage{booktabs}
\usepackage{tabularx}
\usepackage{mathpartir}
\usepackage{fancyvrb}
\usepackage{multirow}

\frenchspacing


% Bibliography
\addbibresource{thesis2.bib}
\bibliography{thesis2}

% Text
\newcommand{\ie}{i.e.\xspace}
\newcommand{\eg}{e.g.\xspace}

% Referencing
\newcommand{\chp}[1]{\S\ref{chp:#1}}
\newcommand{\sct}[1]{\S\ref{sec:#1}}
\newcommand{\ssec}[1]{\S\ref{subsec:#1}}
\newcommand{\eqn}[1]{Eq.~\ref{eq:#1}}
\newcommand{\eqns}[2]{Eq. \ref{eq:#1} and \ref{eq:#2}}
\newcommand{\lem}[1]{Lemma~\ref{lemma:#1}}
\newcommand{\lems}[2]{Lemmas \ref{lemma:#1} and \ref{lemma:#2}}
\newcommand{\thm}[1]{Th.~\ref{thm:#1}}
\newcommand{\fig}[1]{Fig.~\ref{fig:#1}}
\newcommand{\diagram}[1]{diagram~\ref{eq:#1}}
\newcommand{\app}[1]{Appendix~\ref{app:#1}}
\newcommand{\mcite}[1]{\textcolor{gray}{#1}} % missing cite
\newcommand{\defn}[1]{Def.~\ref{def:#1}}
\newcommand{\prop}[1]{Prop.~\ref{prop:#1}}

% Math
\renewcommand{\tuple}[1]{\left(#1\right)}
\DeclareMathOperator*{\expn}{exp}
\renewcommand*{\exp}[1]{e^{\,#1}} % \mathrm{e}^{#1}}
\renewcommand{\qedsymbol}{\ensuremath{\blacksquare}}
\newcommand{\partialto}{\rightharpoonup}
\newcommand{\id}{\vec{1}} % identity function
\newcommand{\mr}[1]{\mathrm{#1}}

\newcommand{\den}[1]{\llbracket #1 \rrbracket}
\newcommand{\m}[1]{\{\!| #1 |\!\}}
\newcommand{\M}[1]{\mathcal{#1}}
\newcommand{\MS}[0]{\mathrm{M}}
\newcommand{\SQ}[0]{\mathrm{S}}
\newcommand{\s}[1]{\underline{#1}}
\newcommand{\G}[0]{\Gamma}
\newcommand{\D}[0]{\Delta}
\newcommand{\mytt}{t\!t}
\newcommand{\myff}{f\!\!f}

\newcommand{\V}{\mathrm{V}}

\newcommand{\sel}{\mathrm{sel}}
\newcommand{\fold}{\mathrm{fold}}

\newcommand{\ms}{\mathrm{ms}}


\newtheorem{mydef}{Definition}
\def\dotminus{\mathbin{\ooalign{\hss\raise1ex\hbox{.}\hss\cr
  \mathsurround=0pt$-$}}}
\setlength{\tabcolsep}{8pt}
\renewcommand{\arraystretch}{1.0}

\newcommand{\match}{m}
\newcommand{\up}[1]{\uparrow\! #1}

\newcommand{\n}{\mathrm{n}}


% Other stuff
\newcommand{\maybe}[1]{\textcolor{gray}{#1}}
\newcommand{\todo}[1]{\textcolor{red}{TODO: #1}}

% rules
\newcommand{\ar}[2]{\mr{#1} \! = \! {#2}}

\setlength{\tabcolsep}{8pt}
\renewcommand{\arraystretch}{1.2}


% Header
\title{A modelling language for biology with applications}
\author{Argyris Zardilis}

\abstract{

  Understanding the links between biological processes at multiple scales, from
  molecular regulation to populations and evolution along with their
  interactions with the environment, is a major challenge in understanding
  life. Apart from understanding this is also becoming important in attempts to
  engineer traits, for example in crops, starting from genetics or from genomes
  and at different environmental conditions (genotype x environment ->
  trait). As systems become more complex relying on intuition alone is not
  enough and formal modelling becomes necessary for integrating data across
  different processes and allowing us to test hypotheses. The more complex the
  systems become, however, the harder the modelling process becomes and the
  harder the models become to read and write. In particular intuitive formalisms
  like Chemical Reaction Networks are not powerful enough to express ideas at
  higher levels, for example dynamic environments, dynamic state spaces, and
  abstraction relations between different parts of the model. Other formalisms
  are more powerful (for example general purpose programming languages) but they
  lack the readability of more domain specific approaches.

  The first contribution of this thesis is a modelling language with stochastic
  semantics, Chromar, that extends the visually intuitive formalisms of
  reactions, in which simple objects, called agents, are extended with
  attributes. Dynamics are given as stochastic rules that can operate on the
  level of agents (removing/adding) or at the level of attributes (updating
  their values). Chromar further allows the seamless integration of time and
  state functions with the normal set of expressions – crucial in multi-scale
  plant models for describing the changing environment and abstractions between
  different scales.  This leads to models that are both formal enough for
  simulations and easy to read and write.

  The second contribution of this thesis is a whole-life-cycle multi-model of
  the growth and reproduction of Arabidopsis Thaliana, FM-life, expressed in a
  declarative way in Chromar. It combines phenology models from ecology to time
  developmental processes and physical development, which allows to scale to the
  population and address ecological questions at different genotype x
  environment scenarios. This is a step in the path for mechanistic links
  between genotype x environment and higher-level crop traits.

  Finally, I show a way of using optimal control techniques to engineer traits
  of plants by controlling their growth environmental conditions. In particular
  we explore (i) a direct problem where the control is temperature – assuming
  homogeneous growth conditions and (ii) indirect problem where the control is
  the position of the plants -- assuming inhomogeneous growth conditions.
}

% Abstract
\begin{document}

%% First, the preliminary pages
\begin{preliminary}

  \maketitle

  \begin{frontenv}{Lay Summary}{\LARGE}
    Living organisms are often viewed as hierarchical. Entities at one level
    (\eg cells) function and organise giving rise to properties of entities at
    the next level (\eg tissues) and they in turn to the next until we get to
    the level of a living organism. If we want to understand life, which is the
    most fundamental question in biology, we need to understand the processes
    and the links between them across these levels. As the systems we are trying
    to understand become more complex, holding all the interactions in our heads
    as mental models becomes difficult (or impossible). One way to solve this is
    by using formal models written in some technical language (\eg mathematical
    equations). The advantage is that these can be manipulated in predefined
    ways, for example setting them in motion inside a computer, without the
    danger of missing anything. When trying to understand all the scales that
    make up an organism though, some of these languages become hard to read and
    write and can no longer serve as documentations of our understanding.

    In this thesis I first present a formal language, \emph{Chromar}, which has
    some features that make writing a class of multi-scale models easier. Some
    of these features include mixing discrete with continuous entities, for
    example you can describe how cells and how they are created or destroyed
    (discrete) but you can also describe how the size of the cell changes
    (continuous). It also allows the mixing of deterministic with stochastic
    dynamics.

    I then use this language to describe in a model the whole lifecycle (from
    seed to seed) of the plant Arabidopsis Thaliana, a commonly used organism in
    plant research, by combining or extending existing models. Such a model
    allows then multiple generations and the observation over time of
    populations of plants in environmental conditions of our choice. We can then
    compare populations of different types of plants in different scenarios and
    ask which type is more succesful in reproducing in some conditions. This
    might give us some hints on why some types of plants adapted to specific
    locations.

    Finally, since there are now solutions that allow more precise control over
    the surrounding environment (think more precise and controlled greenhouses),
    one might ask if it is possible to find and use optimal environmental
    conditions (\eg temperature) to achieve specific trais of plants? Here I use
    one of these comprehensive plant models along with other mathematical
    techniques to address this question.
\end{frontenv}\newpage

  \begin{acknowledgements}
Thanks to...
    \end{acknowledgements}

%% Next we need to have the declaration.
\standarddeclaration

%% Finally, a dedication (this is optional --
%% uncomment the following line if you want one).
% \dedication{To my mummy.}

%% Create the table of contents
\tableofcontents

%% If you want a list of figures or tables,
%% uncomment the appropriate line(s)
% \listoffigures
% \listoftables

\end{preliminary}

\chapter{Introduction}
\label{chp:intro}
%In light of this discussion I will also need to change the discussion in the
% next chapter
%somehow need to include the word constructive more, that's key
Life is produced and sustained through the organisation of entities at various
levels (\eg molecules, membranes) that interact to construct other autonomous
entities (such as cells). Organisation and structure are therefore fundamental
and defining properties of all living systems. Some go as far as to say that
(self-)organisation and the resulting self-maintenance are the only common and
defining characteristics of living organisms and that they are more important
than even natural selection in shaping the forms of life that we see
\citep[sometimes referred to as structuralism, for
example;][]{thompson1942growth, kauffman1992origins}. Even if we do not take
such an extreme view, it is still true that if we want to understand life we
have to understand the organisation that sustains it and the links between
processes at all scales defined by this organisation -- from genes to organisms
and ecosystems. Organisation here means both spatial organisation, for example
cells in a tissue, but also relations or interactions between processes that
give rise to other higher-level phenomena or between processes and the
environment.

% other less fundamental question could possibly be answered but at the core
% we need this
This view of organisation as a fundamental and defining principle of biology
makes, in my opinion, the subject distinct from other natural sciences, like
physics for example and precludes a natural mode of enquiry, which is
abstraction. The nature of the questions is different. How can we abstract the
details of an organism if those details are exactly what makes the organism
alive and in fact what we are trying to understand? Since abstraction or at
least complete abstraction is not available, in order to answer biological
questions we have to take a more holistic and multi-scale view of the natural
systems of interest.

%reductionist vs system (organisation)
Despite the importance of organisation in understanding life,a common focus in
biology has been on understanding individual mechanisms at the molecular
level. The reductionist view of life is that if the cell is the main unit of
life then if we understand all the processes inside the cell we will understand
life, as everything else follows.

There were, however, even if on the fringes or often completely outside
mainstream biology, few historical (and perhaps independent) threads of work
that recognised the importance of organisation and consequently of the systems
view. C.H. Waddington, for example, talked about the importance of the
interaction of parts in developmental systems \citep{allen_evolution_1977},
perhaps inspired by the metaphysics of Alfred North Whitehead \citep[same
book;][Autobiographical note]{allen_evolution_1977}. While the polarity between
constituents of systems and the importance of their organisation (systems view)
might have been a topic of discussion in the field at the time
\citep{waddington_nature_1961}, \citet{rashness's_topology_1954} explained the
need for a theoretical study organisation. He called this organisation the
`relational' aspects of biology that needs to be addressed in theoretical
studies to complement usual approaches that follow only the metric aspects of
physical systems. This is the usual abstraction in Physics where systems are
viewed only by their quantifiable properties (captured by numeric
variables). \citet{rosen_relational_1958} developed such a theory of
organisation, based on the (meta-)mathematics of category theory, for the
relational/organisational aspects of biology and was a proponent of a systems
view \citep{rosen1991life}. Again, thinking about questions on the origin of
life, \citet{varela_autopoiesis:_1974} developed the concept of `autopoieisis'
(self creation through organisation) as a necessary condition for autonomous
living entities. Similar questions regarding the origin of macromolecules in
cells lead to a study of self-organisation by
\citet{eigen_selforganization_1971}. Following this \citet{fontana_what_1994}
working with Peter Schuster (Eigen's student) highlighted the importance of
organisation as the defining characteristic of living systems and revisited
the problem of developing a theory of organisation as a basis for biological
understanding. For this theory they borrowed from computer science
\citep[$\lambda$-calculus in particular;][]{fontana_barrier_1996}. Finally, the
systems view entered the mainstream through the \emph{systems biology} movement
\citep{kitano2002systems} as a way to consolidate increasingly diverse
experimental molecular biology datasets.

% need for modelling
As the systems we are trying to understand become more complex, which is
particularly the case when we take the systems view, intuitive thinking and
mental models of the processes are not adequate tools to aid our
understanding. A useful tool in these cases is formal models where the systems
are represented using some formal language based on a mathematical
interpretation, either directly or indirectly. An interesting view is that of
formal models as machines that turn the assumptions that we put into them to
conclusions \citep{gunawardena_models_2014}. The unique advantage of formal
models compared to informal ones (mental models) in biology is that this machine
is guaranteed to function reliably therefore providing a good tool for testing our
hypotheses (assumptions) just like an experiment would. Mental models however do
not provide this guarantee and we therefore do not know if the fault is in our
assumption or in the animation of the model in our heads.

Historically, it is no surprise that formal models have been used in systems
models, for example for morphogenesis \citep{turing_chemical_1952} and neuronal
activity \citep{hodgkin_quantitative_1952} although in some cases good
abstraction theories have been used in evolutionary genetics \citep[by
abstracting away the organism;][]{dobzhansky1982genetics} and in the small scale
for chemical kinetics \citep{michaelis1913kinetics, gunawardena_lessons_2012},
\citep[see also;][]{gunawardena2013biology}. With the movement of systems
biology as the intellectual successor to the system and organisation theorists
of the 20th century we have seen an increase in mathematical models of cellular
processes, from metabolism to the circadian clock.

% multi-scale models
% modelling with explicity representation of processes at multiple levels of
% organisation to understand the links between scales and how life or particular
% phenomena in life emerge through this, which cannot be explained only as a
% sum of their parts.
% also cite 
Taking the systems and modelling view to the extreme and remembering the
fundamental goal of understanding living organisms through their organisation,
we can take a step further and expand our systems view from the cell to larger
scales, for example to the organism and beyond. These models that reach beyond
organisation on a single scale are usually called \emph{multi-scale models} and
they are more close to the historical spirit of understanding system
organisation to understand life, as we have seen above. For small
micro-organisms, comprehensive models link the metabolic and molecular level
with the cellular \citep{karr_whole-cell_2012} and population growth scales
\citep{weise_mechanistic_2015}. Work on more complex organisms has focused more
narrowly (Virtual liver, \citet{holzhutter2012virtual}; Virtual heart;
\citet{noble_modeling_2002}; Bone system, \citet{paoletti_multilevel_2012})
although there are ambitious collaborative project aiming at whole organisms
\citep[virtual rat, virtual human][]{beard_multiscale_2012, kohl_systems_2009}.

%multi-scale models in plants
%plants
Plant science research is concentrated on the laboratory model species
\emph{Arabibopsis thaliana}, which offers an opportunity for broad understanding
that includes mechanistic models \citep{chew_mathematical_2014,
  voss_modelling_2014}. This multi-scale view starting from a molecular
mechanism and metabolism is realised explicitly in the Framework Model that
represents vegetative growth starting from the a specific well-understood
molecular mechanism, the circadian clock \citep[FMv1]{chew_multiscale_2014}. On
the other end of the scales spectrum there is modelling work for plants in the
crop and ecology world where the organism is usually absent. While this work
usually is concerned with more crop or ecologically relevant species, recently
Arabidopsis is starting to appear in crop and ecology research in attempts to
bring the organism back into the equation and provide insight into relevant
mechanisms that affect ecological or crop phenotypes especially with respect to
the interaction with the environment.

The work of this thesis is situated exactly in this space, in the intersection
and modelling space between plant science and crop or ecological
research. We are particularly interested in linking multi-scale plant models
that consider growth through environmental interaction with ecological models
that consider ecosystems in the natural environment.  Apart from the scientific
challenges these multi-scale models present, in plants or elsewhere, there are
also some challenges regarding the technical language used for their
representation.

\section{Aims}
\label{sec:aims}
Therefore, the \emph{aims} of this thesis are:
\begin{itemize}
\item Develop a suitable language to address the problem of naturally and
  succinctly representing multi-scale plant models that go beyond the organism
  and act in a dynamic natural environment.
\item Use the above language to create such multi-scale plant models from the
  organism to the ecosystem.
\item Employ the models for answering questions in ecological models using the
  mechanistic understanding that comes from including a model of the organism in
  ecological studies.
\end{itemize}

In the rest of this chapter I will first give a simplified overview of modelling
in the plant domain starting from plant science and going to ecology and
crops. My objective here is to locate and motivate the challenges for modelling
languages, rather than a comprehensive survey of the state of the art in plant
models. I will then overview the language question in multi-scale models that
relates to the organisation view of living organisms we noted above. Finally, we
give the contributions of this thesis in reference to the outlined aims.

\section{From plants to crops and ecosystems}
Much contemporary work in plant biology research has been focused on
understanding molecular mechanisms at the foundation of important plant
processes like photosynthesis. Unlike mammalian systems, however, the effect of
environmental conditions is usually taken into account since plants are highly
dependent on those. The importance for crop traits also led to attempts at
linking molecular knowledge to organism traits, like growth. We can think of
these kind of questions of how a particular phenotypic trait arises from a
molecular mechanism as the \emph{how} questions (top part of cycle in
Figure~\ref{fig:circle}.

\begin{figure}[tb]
  \centering
  \includegraphics[width=0.8\textwidth]{figures/circle.eps}
  \caption{Plant biology research is usually concerned with the top part of the
    cycle and the 'how' of organisms. Evolution and ecology are concerned with
    the why although they usually use a limited view of the 'how' to make
    hypotheses about the 'why' \citep[adapted
    from][]{millar_intracellular_2016}.}
  \label{fig:circle}
\end{figure}

In the fields of ecology and evolution on the other hand, the question can also
address more on they \emph{why} organisms evolved the way they have (bottom part
of cycle in Figure~\ref{fig:circle}. This is usually done by observing genetic
variation and with some knowledge of the how this affect organism behaviour make
evolutionary hypotheses on the why.

Multi-scale plant models that go beyond the organism can be seen as tools for
understanding (evolutionary ecology) and tools for engineering (crop
science). Next, we start with plant models with an emphasis on the Framework
Model of \citet{chew_multiscale_2014} and go to an overview of crops and
evolutionary ecology models.

\subsection{Multi-scale plant models}
%main point here to give an overview, a feeling for plant models all the way to
%the multiscale FM, that represents the constructive organisation of the
%organism, then limitations on scaling beyond the organism to the evolutionary
%ecology level
Modelling in plant biology is usually concerned with individual mechanisms, for
example photosynthesis \citep{farquhar_biochemical_1980} or leaf stomata
conductance \citep{tardieu1993integration}. What perhaps makes it distinct from
modelling work in other species is the importance of the environment and the
attempt to link these molecular processes to organism traits directly important
for agriculture. For example photosynthesis models have been linked to sucrose
synthesis \citep{zhu_e-photosynthesis:_2013}, the circadian clock to flowering
time \citep{salazar_prediction_2009}, and leaf-stomata conductance to leaf
growth \citep{tardieu2015modelling}.

Unlike other organisms most of the development of plants happens throughout
their lifecycle and a lot of work was based on descriptive models of their
structural growth \citep{mundermann_quantitative_2005}. As the role of molecular
mechanisms into development became clearer there is more multi-scale work
linking molecular mechanisms to development, for example there are models of the
distribution of the hormone auxin along the plant structure, which is known to
act as a morphogen \citep{prusinkiewicz2009control,
  jonsson2006auxin}. Functional-structural models also try to link molecular
mechanisms, usually abstract views of metabolism, to structural growth at the
organ level \citep{christophe_model-based_2008}.

Plant development models usually ignore most of the underlying biochemistry of
the organism except for a very narrow description of directly relevant molecular
mechanisms. On the other hand models of molecular mechanisms, while sometimes
linked to the organism, they usually ignore the constructive organisation of the
plant. The question is can we construct the organism representing its
constructive organisation and underlying mechanisms? We next overview one
attempt at a digital organism, the Framework Model, which is inspired by the
functional-structural models.

% Then FM more multi-scale
% ->
\subsubsection*{Framework Model}
\label{subsec:fm}
The Framework Model represents vegetative growth of Arabidopsis in lab conditions
\citep{chew_multiscale_2014}, starting from four independent models that
represent photosynthesis and carbon storage \citep{rasse_leaf_2006}, plant
structure and carbon partitioning among organs
\citep{christophe_model-based_2008}, flowering phenology
\citep{chew_augmented_2012} and the circadian clock gene circuit and its output
to photoperiodic flowering \citep{salazar_prediction_2009}. Later updates
focussed on plant phenotypes controlled by the clock, such as tissue elongation
and starch metabolism \citep[FMv2;][]{chew_linking_2017}, or temperature and
organ-specific inputs to flowering
\citep{kinmonth-schultz_mechanistic_2018}. The Framework Models align with
community efforts to link understanding of crop plant processes at multiple
scales, for benefits in agriculture \citep{wu_connecting_2016, zhu_plants_2016}.

Among the limitations of the Framework Models, growth was limited to the
vegetative stage, ending upon flower induction. Without reproduction, the models
had no seed yield or link to evolutionary fitness. Without seed dormancy, they
lacked a major determinant of Arabidopsis life history in the field. Their
representation of the circadian clock was also unnecessarily detailed for many
studies outside chronobiology. These limitations mean that going beyond the
organism to the population level to answer ecological questions is not
practically possible without extra effort.

\subsection{Crop modelling}
%main point here is to give a feeling of how crop models are used, usually, for
%a  fundamentally different reason -- for engineering
%do we have any examples of this? I mean of actually doing this, genetic engineering
Crop models use many of the same ideas that we have seen in multi-scale plant
models but they have a different more engineering oriented goal in mind instead
of the fundamental biology goal of understanding. Usually the motivation is
understanding the effect of interventions to either plant genetics or crop
management on plant performance, which is usually growth related.

% traditional crop models
Crop models often regard the entire crop as as one big plant. The challenges
relate exactly to this assumption. For example, computing light interception of
the canopy has to take into account the geometry and assumed configuration and
positioning of plants in the field. The constructive organisation of plants is
absent and plant processes, like photosynthesis, are scaled to the entire crop
\citep{monteith_light_1965}. Genetics or molecular mechanisms are represented,
if at all, in a phenomenological way. For example, a mutant might have a
different set of model parameters and so on \citep{parent_can_2014}.

Since the goal in this models is prediction, as long as these models come
reasonably close to the behaviour of the crop under different environmental
conditions they are considered to be adequate representations. However, because
of their nature it becomes difficult to capture the responses of the systems
under multiple environmental inputs.

\subsubsection*{A flows-based view of a plant}
%Main point is to introduce the concept of this reservoir flow view, which is
%important for later
Other approaches start from a whole-plant model and then try to extrapolate to
the entire crop. The usual representation is of the plant as multiple reservoirs
of nutrients with flows between them \citep{france_mathematical_1984}. Growth,
which is the main goal usually, is calculated as a function of the levels in
these nutrient pools (see Figure~\ref{fig:plantFlow}).

\begin{figure}[tb]
\centering  
\includegraphics[scale=0.8]{figures/flowPlant.eps}
\caption{A generic plant model with the flows-based view of a plant
  \citep{france_mathematical_1984}. The plant is represented by several state
  variables representing the levels of various nutrients in different parts of
  the plant. The dynamics are then seen as flows between the pools. The pools
  could also represent entities outside the system, like the soil, for
  example. The variables represent, structural mass ($X$), nitrogen ($N$),
  carbon ($C$), and water ($Wa$). The output is usually the biomass of each part
  of the plant and especially that of fruits or seeds, which are important in
  agriculture.}
  \label{fig:plantFlow}
\end{figure}

This view has later found its way into more traditional plant models and it is
the metabolism level representation found in the Framework Model, for example,
although the flows are calculated more mechanistically and the pools are at the
organ level reflecting the explicit representation of plant organisation. Again.
in this whole plant flow-based models, the constructive organisation of the
plant is usually absent preventing fundamental questions. However, as long as
the goal is prediction for engineering, then often this kind of representation
is adequate.


\subsubsection*{Multi-scale models}
The multi-scale systems modelling approach has also been proposed in crop
modelling in attempts to engineer crop traits starting from genetics or from
genomes \citep{welch_merging_2005, yin_applying_nodate, yin_modelling_2010,
  parent_can_2014, wu_connecting_2016, chenu_integrating_2018}, where simpler
models have demonstrated both the potential of crop modelling in general and the
significant demands of detailed models for empirical data that varies in
availability \citep{hammer_models_2006,asseng_uncertainty_2013}.


\subsection{Ecology and evolution}
% Main point is to introduce phenology models that while they make evolutionary
% assumptions, they rarely include fitness or genetics (genetic variation) +
% usually developmental events in isolation -- organisms is absent
%
% on the other hand genetic studies on genetic variation dont have any modelling
The lifecycle of Arabidopis plants can be divided into three major
developmental stages where the transition between them is marked by
developmental events. Starting out as a seed the plant germinates transitioning
to the vegetative stage after which it bolts going to the reproductive stage
that ends with the dropping of the new seeds (Figure~\ref{fig:plantLife}.

In natural settings the timing of this events throughout the year
(\emph{phenology}) determines the environment that the plant is exposed to
during growth. Therefore phenology is a major determinant of plant fitness. If
growth coincides, for example, with unfavourable weather conditions the plant
might not have enough resources to put into making seeds or might not even
survive to reproduction.

\begin{figure}[tb]
\centering
\includegraphics[width=0.8\textwidth]{figures/phenology.eps}
  \caption{Plant developmental stages and timing in the year (phenology), which
    is a big determinant of fitness}
\label{fig:plantLife}
\end{figure}

Yet species have a natural geographic range that spans a wide range of climatic
conditions. For example, Arabidopsis in Europe grows naturally all the way from
Spain to Northern Finland. The hypothesis is then that the mechanisms that
control these developmental events might be under selective pressure to adapt to
different climates, , which can then be tested by experimental growth under
multiple environments. A lot of work in evolutionary ecology of plants is
focused on understanding the molecular mechanisms involved in plant adaptation
through the control of developmental events. The assumption is then that these
mechanisms reflect adaptation to different environments and therefore hint
towards the 'why' questions. \citet{mendez-vigo_altitudinal_2011} characterise
the variation in four flowering related genes across large geographical regions
to investigate their involvement in the variation of flowering related traits
across the region. Other studies focus on seed dormancy related genes
\citep{chiang_dog1_2011} or more broadly on general characterisation of genetic
variation for multiple traits across the natural habitats of Arabidopsis
\citep{atwell_genome-wide_2010}. The developmental events are correlated since
the length of one stage affects the next, which in turn affects the next
generation. Therefore the genetic mechanisms behind two related (and
consecutive) developmental events are sometimes studied together
\citep{debieu_co-variation_2013}.

\subsubsection*{Phenology models}
The hypothesis in phenology is that plants sense and integrate environmental
signals until they reach the optimal condition for triggering the transition to
the next developmental stage. There are experimental evidence of this
integration but evidence on the mechanism are still behind although there are is
some work pointing to the clock as a possible mechanism.

Phenology models are usually phenomenological and make no attempt to represent
the mechanism for sensing and integration of the environmental effects. Instead,
they use numerical quantities to represent the progress towards the transition,
which is an integral over time of the values of environmental indicators that
affect the particular stage under consideration \citep{chuine_plant_2013}. For
example, if we know that temperature affects a developmental stage, a phenology
model might use a temperature sum to track the progress towards the transition:
$$
d(t) = \sum_{i}^{t} T(i) - T_b
$$
The development at time $t$, $d(t)$, is the sum of the temperatures at every
time unit above a base temperature $T_b$. The transition time, $t_s$, is the
time such that the development sum reaches a threshold value, $d(t_s) >
D_s$. Incorporating genetics is usually represent by adding a sensitivity value
to the above, $ d(t) = \sum_{i}^{t} k \cdot (T(i) - T_b)$ or changing the base temperature,
$T_b$. The idea is that plants adapt by changing, $k$ or $T_b$, which reflects
some molecular mechanism, to change the length of the stage and time their
transition at the optimal time depending on the natural environment.

Modelling phenology usually goes in tandem with experimental observations about
timing but the models usually stop there and they do not deal with the
evolutionary hypotheses that usually follow. These models usually appear in
ecological studies and do not consider the genetics while studies of genetic
variation of developmental timing adaptation usually do not have a modelling
component.

Furthermore, dealing with each developmental stage in isolation means, while
being simple and inducing simple 'why' hypotheses, is sometimes not enough as one
developmental stage affects the timing and length of the next and the length of
the entire generation affects the next generation and so on.

\subsubsection*{System models}
In order to be useful in evolutionary ecology other work has incorporated
genetics and fitness into the simple phenology models of the kind we have seen
above. \citet{chuine_phenology_2001}, for example, propose a model that combines
fitness with traditional phenology models to predict the distribution of a
species. The model is simple, the phenology part predicts the timing and the
fitness, given as probabilities of survival under different weather stresses,
predicts overall survival rates.

While the simple phenology models consider species mean behaviour, other work
has made the links between the genetic studies of timing mechanisms to phenology
models to understand the range of expected behaviour both within but also
between populations. For example, \citet{wilczek_effects_2009} add genetic
variation of the FLC gene (involved in control of flowering time) to a flowering
phenology model to predict variation in flowering time and vegetative season
length.

\citet{burghardt_modeling_2015} takes a more systems approach by taking an
integrative look at the whole life cycle instead of events in
isolation. Variation is also considered explicitly by using an individual-based
population model. Fitness, however, is not considered either directly as the
reproductive success of the individual in the population or indirectly as
survival rates as we have seen before. The focus is more on the timing and the
consequences of the genetic environmental interaction.

In representations in both simple and more complex evolutionary ecology models
in this domain the organism is absent. While this gives tractable
representations of complex processes and allows understanding of experimental
results, a more fundamental understanding requires the organism to fill the gaps
between the scales from genes to the population. Adding the organism to the
population level recovers a more mechanistically founded reproductive success
(fitness) as a first step towards explicitly modelling evolutionary dynamics.

% People:
% Kathleen Donohue
% Amity Wilczek
% Justin Borevitz

% although there is more work on adaptation to other things directly like
% temperature
% See Caroline Dean

\section{Languages for multi-scale biology}
% uses of models + 
% hint towards problems of representation
We have already outlined the need for formal models in the description of
biological systems. The unique advantage is their formal nature, which means
that they can be used as tools for turning hypotheses into predictions using the
formal methods that are provided by the technical language we use to specify
them. Usually this involves simulating them on a computer. Their formal nature
also serves another purpose. Since they are externalisations of our knowledge
about a system, the representations can serve as documentation for our knowledge
when trying to communicating it to ourselves (for thinking) but also for others.

This view of notation or language as tool for thought and not solely as
technical object has been recognised by Kenneth Iverson who succinctly listed the
characteristics of a good notation: ``ease of expression of common constructs
in the domain, suggestivity, ability to subordinate detail, economy, and
amenability to formal proofs" \citep{iverson2007notation}.

Modelling in biology becomes necessary when systems are complex. However, the
more complex the systems are, the harder models become to read and write. This
means that, while they might serve their first purpose as technical objects,
they fail on their second purpose as tools for thought. One reason for this
might be that the standard technical languages from Physics, like Ordinary
Differential Equations, are not adequate for the representation of the
constructive nature of the organisation of living organisms. The standard
abstraction is to ignore the organisation and describe metric properties of the
system captured with their values captured in numeric variables and their
dynamics in equations. While this abstraction of systems as numeric variables is
sometimes enough, if we really want to fundamentally understand an organism or
aspects of it, which is the case in these complex systems, we also need a
description of their constructive organisation. This was already noted, as we
have seen, by the first `structuralists' and systems theorists like Rashevsky
who pointed out the need for the representation of the 'relational aspects of
biological systems' as well as their more metric properties
\citet{rashevsky_topology_1954}. Later work focused on finding suitable theories
of organisation analogous to the equations of motion in Physics. For that
reason Rashevsky's student, Robert Rosen, was called the Newton of biology
\citep{mikulecky2001robert}.

While we appreciate the organisation is important, especially at multi-scale
representations abstraction through quantifiable properties is also needed
especially with more complex systems where we do not have the capacity or the
desire to model everything mechanistically.

For complex models like the ones we have seen above from plant, crop, and
evolutionary ecology research they are either simple enough that they are
captured with ODEs or similar formalisms or they are complex that they cannot be
easily mapped to any existing technical language in which case more pragmatic
approaches are used where the model becomes a program in a general purpose
programming language. This is problematic because the knowledge is usually
obscured in the simulation details.

The more complex models are the ones we are interested in, representing
processes like whole-plant models beyond the organism reaching to ecology and
evolution. The next chapter~\ref{chp:relWork} is dedicated to a more in depth
overview of existing work on languages for this domain and goes into more
details in some of the issues we highlighted above using illustrative examples.


\section{Contributions}
The work of this thesis is concerned primarily with whole plant models that go
beyond the organism, languages for their description, and their applications. In
reference to the aims and discussion above, specifically my contributions are:
% This need to be refutable, like a contract for the thesis

\begin{itemize}
\item A notation, called \emph{Chromar}, for describing systems like the ones
  we hinted to above at the intersection of plant biology models in a natural
  environment. The language follows the structuralist tradition and is therefore
  object-based, which means that it can capture certain important aspects of the
  constructive organisation of living systems. At the same time quantifiable
  properties can also be represented allowing abstraction while further
  extensions allow an enriched expression language for these properties that
  combines regular mathematical expressions with (i) a flexible system of state
  observation based on database operations and (ii) a way of defining
  deterministically time-dependent values that can be used to describe parts of
  the systems or the natural environment that we do not wish to model
  mechanistically.
\item An implementation of the abstract version of Chromar as an embedded Domain
  Specific Language (EDSL) in the general-purpose programming language
  Haskell. This allows the set of expressions and types for quantifiable
  properties to come from the host language, which increases expressiveness while
  at the same time maintaining the naturalness of a domain-specific language.
\item An extension of the Framework Model (FM; \ssec{fm}) to the full-lifecycle
  of Arabidopsis Thaliana, including reproduction and see dormancy along with
  modifications that allow operation in realistic weather ranges. This is
  combined with full-lifecycle phenology models from ecology to provide a
  population level model that includes the organism (through the FM). This
  allows simulation experiments for determining ecological properties in
  different genotype x environment scenarios that are mechanistically founded
  and a step towards organism-centred models of evolution.
\item An application of the Framework Model with an engineering perspective,
  like in case of crop models. Instead of taking the traditional approach to
  engineer genomes, as we have seen, we take an alternative approach that aims
  at engineering the environment for achieving specific growth-related organism
  traits. We pose the problem as an optimal control problem and frame two
  instances of it, one where the control is direct and another one where the
  control is indirect through positioning.
\end{itemize}


\chapter{Background and Related Work}
\label{chp:relWork}
Formal models of physical systems serve two roles (i) documentation and
communication of our understanding and (ii) formal analysis (and/or simulation). The
most common language for describing the natural world is dynamical systems
theory. It has a long tradition in Physics starting from Newton and Poincare. It
relies on a conceptual abstraction where instead of describing the objects that
form the physical world, it describes quantifiable properties of such
objects. For example, when explaining the movement of planets one does not talk
about the moon but rather about the position of the moon. Then the evolution of
the system in time (or space) is described by (partial) differential (or
difference) equations that state the change in the numerical values of the
quantifiable properties that form the system as they interact with each
other. For example, one might write the equation of position of a particle or a
molecule diffusing in a medium. This abstraction of avoiding dealing with the
objects directly but rather with some numerical variables representing their
properties has been at the core of the success of dynamical systems theory. A
big ensemble of methods have been developed for the analysis of such systems,
which one gets for free if they choose to describe a physical system of interest
in these terms.

Dynamical systems languages have not been as successful in scenarios where the
structure of the described system is not static. While one might be able to find
a quantifiable property of the system to describe, in cases where the question
posed is about the changing structure itself or about how it gives rise to other
dynamics, such an abstraction is not adequate. Consider for example plant
development and its effect on carbon intake for the entire plant. While in some
cases writing an equation for the size of the plant is enough, a truly
mechanistic understanding requires treating the developing structure that gives
rise to plant size explicitly. How would one write equations for the size of the
leaves, for example, if new ones keep appearing?  In those cases one might be
better off with a language that allows the description of the objects
themselves, their interactions, and organisation. Unsurprisingly such languages
have been developed for scientific fields that deal with parts of the natural
world where structures are dynamic and self-organising. Languages of increasing
complexity exist for the description of biochemical systems starting from simple
molecular reactions. The field of plant development also has seen an independent
strand of work on languages with explicit description of objects. Both these
strands of work have seen a heavy interaction and inspiration from Computer
Science where there exist many formalisms in the theory of computation for the
description of object behaviour and interaction.
% especially in concurrent computation but also in sequential

%we need names for the two kinds of languages -- make the distinction so we can
%refer to them later
% Here we are interested in understanding a plant or particular aspects of a plant
% at the organism level starting from genomes and molecular mechanisms. In order
% to understand the links between all relevant processes we need models with
% explicit representation of processes at multiple levels. Apart from the
% scientific challenges that this presents there are also technical and social
% challenges, among which is the choice of representation (language) for
% describing our models. Since we are trying to recreate an organism in silico and
% especially one where development is plastic and happens throughout its life
% history, we need to be able to describe dynamics of discrete objects. At the
% same time we also have the need for abstraction through quantifiable properties
% since the complexity of the task is large. 

There are languages in both the dynamics and structure space that combine
aspects of both worlds. Chromar, the language we describe and that is a big part
of this thesis is situated in this space too. It comes from the biochemistry
tradition and has discrete objects at its core but at the same time also has
attributes to abstract away some internal structure of these objects. It further
has features for (i) linking the abstract (attributes) and the concrete
(observables) and (ii) explicitly using time, which is inspired, again, by the
dynamics world.

Since language design is as much art as it is science it can sometimes be hard
to properly situate a new language (Chromar) in the space of already existing
languages for the description of the natural world especially if one is only
familiar with one part of the world, perhaps the particular formalisms that are
standard for their disciplines. We therefore feel the need to include an
overview of existing languages (the ones we hinted to above) for Biology in
order to be able to place our work in the language design space. Since we have
particular requirements for structure representation, we will only deal with
languages coming from the discrete world or from the dynamics world that include
some support for defining discrete organisation as well.

While there are attempts for formal comparisons between languages
\citep{felleisen1991expressive}, here the design space is so large that such an
attempt is probably impossible. We instead will do our comparison informally
through the illustrative examples. We finally conclude with a summary of the
comparison. In the next chapter we introduce Chromar through the same examples
so we can properly place it and its novel characteristics in the same feature
space.

A more comprehensive overview of the above dichotomy between dynamics and
structure appears in \citet{fontana1996barrier}.

% Concretely, in this chapter we will do the following:
% \begin{itemize}

% \item Introduction of two modelling examples that have representation
%   requirements that are typical of the multi-scale comprehensive models that we
%   are interested in in this thesis.
% \item An overview and comparison of languages coming from (i) the discrete world
%   with an emphasis on languages from the biochemistry strand of work since
%   Chromar is a direct product of these and (ii) the dynamics world that have
%   some aspect of structural organisation. In order to place these languages and
%   later be able to compare them we will place them all in a feature space. The
%   features are related to their characteristics in regards to capturing complex
%   organisation, dynamics, but also meta-features like readability. The above
%   examples will serve to illustrate the features and limitations of each
%   formalism.
% \end{itemize}


\section{Examples}
\label{sec:examples}
In this section we will introduce two examples that we will use throughout our
language space overview to illustrate their limitations and the motivation
for their extensions with new features. We will not attempt a full
representation of the models in each formalism

\subsection{Root apical meristem and whole-plant effects}
\label{subsec:rootDev}
Cell production for plant development happens at sites of meristematic
activity. In the root this site is behind the root cap and it is called the Root
Apical Meristem (RAM). In root development this raises the question about the
mechanism through which this gives rise to the root architecture but also how
the meristematic cells maintain their position close to the root cap throughout
development. The hormone auxin and its distribution of concentration along the
root architecture has been pinpointed as a possible mechanism to explain both of
these phenomena.

Here we will consider auxin dynamics in a growing 1-D array of root cells
proposed by \citet{mironova_plausible_2010} as a possible mechanism to explain
the observed distribution of auxin concentration throughout the root. We will
further consider the effect of the resulting root architecture on above-ground
plant growth through its effect on water intake. We will only consider a very
abstract version of the shoot. This model is indicative of the kind of
comprehensive models that are the focus of this thesis. Here, for example, a
more detailed root model is placed in the context of the whole plant (more
abstract model) and the surrounding environment.

\begin{figure}
\centering
\includegraphics[width=0.7\linewidth]{figures/rootDev.eps}
\caption{Flows of auxin through the root.}
\label{fig:rootDev}
\end{figure}

More precisely we will consider the following dynamics
(Figure~\ref{fig:rootDev}):
\begin{itemize}
\item Auxin flow from the shoot. Since we're only considering an abstract shoot
  we can think of auxin flow from the shoot as auxin being produced in the first
  cell of the root.
\item Auxin degradation through cell expansion.
\item Auxin diffusion as passive transport between cells driven by concentration
  differences across the array.
\item Auxin active transport by the PIN transporter proteins.
\item Cell division.
\item Shoot growth, which depends on the water uptake through the root. The
  uptake depends on the length of the root.
\end{itemize}


\subsection{Plant development in a field}
\label{subsec:plantDev}
The example considers a very abstract view of plant development, but has
nevertheless enough details to demonstrate the main features of our
notation. Our model is inspired by the Framework Model (FM) of
\citet{chew_multiscale_2014}, a modular whole-plant model that connects
traditional plant biology representations of molecular processes with
representations of organ and whole-plant development processes. We will put this
model in a more ecologically relevant context (field).

The above-ground part of an Arabidopsis plant
architecture before flowering: a collection of leaves arranged in a circle. Each
leaf photosynthesises, creating the main currency, carbon; uses some carbon for
maintenance and some for growth; and transfers any remaining carbon to the other
leaves. The FM represents the Arabidopsis rosette (collection of leaves) with no
preference in the transfer, thus we have an all-to-all communication. Similarly
to the FM, in our model all the molecular processes in our model reside in a
central plant `cell' which allows us to keep the leaves as carbon sinks and
track their growth, while avoiding the per-leaf molecular processes and their
communication (see Figure~\ref{fig:fm}).

\begin{figure}
\centering
\includegraphics[width=0.7\linewidth]{figures/fms.eps}
\caption{ A Our simple plant development model. All the interactions, that in
  this case are transfer of carbon, happen between the central Cell that
  represents the molecular state of the entire plant and the leaves, which act
  as carbon sinks. The carbon that goes to the leaves is either used for growth,
  in which case it is transformed into new material (increase of mass), or to
  maintain the already existing Leaf by fuelling its life sustaining
  processes. New leaves are also created, forming new sinks and increasing
  competition for carbon between the leaves, but also increasing the production
  of carbon by providing new green areas for photosynthesis. B The simple plant
  model embedded in a 2-D field.}
\label{fig:fm}
\end{figure}

The processes that affect growth are as follows: we think of \textit{carbon
assimilation} per leaf as increasing the carbon concentration of the central
Cell depending on the photosynthesis level of a leaf (which will depend on its
size); we think of \textit{maintenance respiration} as the central Cell giving
some carbon to a leaf; and we think of \textit{growth respiration} as the
central Cell giving some carbon to a leaf and the leaf mass increasing. We will
also have \textit{new leaf creation}. There are interesting dynamics here such
as the interaction between growth and assimilation: the more we grow, the more
the leaves can photosynthesise, and the more carbon can go to the central Cell.

We assume that the plants are arranged in a two-dimensional field and that they
compete for light, which affects their growth. While we use Arabidopsis here,
such models are interesting for the interactions and competition between crops
and weeds in the field \citep{rajcan_understanding_2001}.


\section{Object-based languages}
\label{sec:objectLangs}
A significant line of work in languages from the discrete (object-based)
tradition has stemmed from the world of Biochemistry. A further distinction in
this camp is between \emph{rule-based} languages and \emph{process-calculi
  based} languages that differ in what they consider as their main unit of
description -- rule-based languages consider the event described as a rule
whereas process-based languages consider the individual object/process as the
main unit of description. Here we will focus more on languages from the
rule-based camp since Chromar follows that tradition. Our focus, however, is not
any specific language but rather the features of the languages and the modelling
requirements that motivated them. Both camps have undergone similar extensions
(starting from simple objects) of features driven by the same modelling
requirements so while we focus more on rule-based languages, our observations
should apply to the other camp as well, which we will only mention in passing in
the following text.

% cite Mjolness paper
In the following sections we categorise object-based languages based on their
features starting from simple objects to structure to dynamics. We will give an
overview and focus on languages that combine objects with dynamics since Chromar
is situated in this space too. Throughout this text we will use the above
examples to illustrate the features and limitations of the languages.

\subsection{Unstructured: Petri Nets}
\label{subsec:pns}
%description
The simplest language for unstructured collections of simple objects is that of
Petri Nets (PNs). At the description level PNs represent discrete objects with
types (species) where the dynamics are given as reactions that state how
different species interact to change the number of objects of the corresponding
species in the state of the system. The state of the system is an unstructured
collection (multiset) of copies (molecules) of each species. Qualitatively,
applying a reaction to a state (multiset) means removing from the state the
elements that appear in the left-hand side (lhs) of the reaction and adding the
elements that appear on ths right-hand side (rhs) of the reaction.

There is an equivalent graphical representation where the system is given as a
bipartite graph representing species as places (round nodes) and reactions as
transitions (rectangular nodes). Each place has an associated marking,
represented as tokens inside the node, that represents the multiplicity of that
species in the state. Applying a transition amounts to moving tokens between
places.

\begin{center}
    \includegraphics{figures/pns.eps}
\end{center}

A \emph{multiset} $m$ over a set $A$ is a function from $A$ to $\mathbb{N}$ counting
the multiplicity of each element $a \in A$ in the multiset. There is submultiset
relation on multisets where for two multisets $m$ and $m'$, $m \preceq m'$ if for each
element $a \in A$ $m(a) \leq m'(a)$. We write $M[A]$ for the set of all multisets
over a set $A$. Given a set of species $\Sigma$, a \emph{reaction} is a structure $\rho
= l \xrightarrow{k} r$ where $k \in R$ and both the left-hand and right-hand
sides are multisets over the species. 

\begin{definition}
A \emph{Petri Net} is a pair $(\Sigma, R)$ of sets of species $\Sigma$ and reactions $R$.
\end{definition}

The state of the Petri Net is a multiset over $\Sigma$. A reaction in $\rho \in R$ can be
applied to a state $s$ if $l(\rho) \preceq s$ giving a new state $s' = s -
l(\rho) + r(\rho)$, which we write as $\rho \bullet s$.

%semantics
Petri Nets can be given a stochastic interpretation as Continuous Time Markov
Chains (CTMCs). A CTMC is a triple $(S, Q, I)$ with a $S$ a set of states,
$Q: S \times S \rightarrow \mathbb{R}$ that gives the transition rate between
any two states, and $I$ the initial state. For each reaction apart from the base
rate given by $k$ we need to know how many times it can be applied in a give
state $s$ (\ie how many times its left-hand side appears in the state). This is
also called a \emph{match}. The multiplicity of a multiset $m$ in another
multiset $m'$ (number of matches) is given by:
$$
\mu(m, m') = \prod_{a \in m}  \binom{m'(a)}{m(a)}
$$
Given a PN $(\Sigma, R)$ and an initial state we can get a CTMC:
\begin{align*}
  S & = M[\Sigma] \\
  Q(s, s') &= \sum {k(\rho) \cdot \mu(l(\rho), s) | \rho \in R, r \bullet s = s'}
\end{align*}
% Have to say how that defines a master equation for the probabilities of the states
% doing anything with that is impossible but
% Gillespie to get trajectories?
%practically this means calculating the above.

%examples
PNs have been used widely for describing chemical reactions and other simple
systems. For systems like the ones in our examples they pose some limitations
though. Suppose for example that we wanted to write reactions for the diffusion
of auxin along the root (see first example: \ref{sec: rootDev}). We could write
the following reactions for the auxin molecules in each cell, where
$\mathrm{A}_1$ is an auxin molecule in the first cell, $\mathrm{A}_2$ is an
auxin molecule in the second cell, with the diffusion occurring at rate
$d$, and so on:
\begin{align*}
\mathrm{A}_1 &\xrightarrow{d} \mathrm{A}_2 \\
\mathrm{A}_2 &\xrightarrow{d} \mathrm{A}_1 \\
\mathrm{A}_2 &\xrightarrow{d} \mathrm{A}_3\\
 &  \hspace{6pt}  \vdots& \\
\varnothing & \xrightarrow{\alpha} \mathrm{A}_1 \\
 & \hspace{6pt}  \vdots &\\
\mathrm{A}_1 &\xrightarrow{\beta} \varnothing \\
  & \hspace{6pt}  \vdots 
\end{align*}
There are two problems with the above description. The first is that it is not
very compact. It grows with the number of cells since we have to write the
diffusion reaction for every pair of cells in both directions and
production/destruction reactions for every cell. The second is that it is hard
to see how to describe the creation of new cells because we need to create a new
auxin species for the new cell and new reactions, but the notation provides no
way to express such a possibility. We could try to make a cell species but then
we would have no way of linking the cell objects with the auxin objects and we
would have the same problems as above. We also have no way of representing
quantifiable properties of these objects, like the size of the cell, for
example, needed for an abstract description of growth. Furthermore properties of
bigger parts of the state, like the size of the entire root, needed for the
water-dependent shoot growth cannot be expressed and neither can parts of the
surrounding environment that we do not wish to model in detail like the water in
the soil.

\subsection{Structured}
\label{subsec:structLangs}
Extensions to simple reactions add support for organisation to the
unstructured multisets of objects in PNs. Organisation can be thought of as
explicitly adding relations over the collections of objects. There are many
relations that could be represented and the ones in this thread are,
unsurprisingly, inspired by cell biochemistry.

\subsubsection*{Hierarchy}
The first organisational principle is that of nesting inspired by the
compartment organisation inside a cell. Following this cell organisation,
languages usually distinguish two kinds of objects -- simple ones that cannot be
nested and more complex ones that can. P-systems is a language for membrane
computing and while it initially considered only simple objects enclosed in a
static (potentially multilevel) membrane structure \citep{puaun2000computing},
in later extensions membranes become first class and are equipped with dynamics
\citep{puaun2001p}. SMMR~\citep{oury_multi-level_2013} also makes this
distinction between simple objects (species) and complex objects that can be
nested (agents). The Calculus of Wrapped Compartments \citep{coppo_stochastic}
also considers nesting but it also adds notational features for expressing
computations that happen on the membrane. A later extension
\citep{coppo_hybrid_2010} adds names to compartments and they can then be
thought of as the agents of SMMR or the membranes of P-systems. The allowed
dynamics of these systems reflect the interpretation of the hierarchical
relation over collections objects as compartmental organisation inside
cells. Therefore sometimes the permitted operations are directly inspired by
this view while in other cases they seem more generic. In all cases, however,
the additional structure allows to express more stringent conditions on the
left-hand side of rules that select based on structure and type (as possible in
PNs).
\begin{center}
    \includegraphics{figures/nest.eps}
  \end{center}  
The nesting relation can also take other interpretations, for example as a
`part-of' relation. At a static level this might be reasonable but some of the
allowed dynamics might not be sensible any more as they only make sense for a
particular cell-inspired interpretation of the relation (see discussion in
\citet{artale_part-whole_1996}).

Semantics are given, like PNs, with similar interpretations as CTMCs. The
states, $S$, in this case are nested multisets and the transition rates between
them are given by appropriate counting of the matches of the nested multisets in
rule lhs's in the state. As states become more complex, however, practically
drawing trajectories from the stochastic process becomes harder as finding the
number of matches from the lhs of a rule to the state is more complex than
simple multiplication of the multiset multiplicities of the lhs elements.

Going to our examples, it looks possible to represent parts of the root
development + auxin dynamics. For example, we could have a $\mathrm{Cell_i}$
species per cell to represent root cells and $\mathrm{A}$ species to represent auxin
molecules. Using nested parentheses to represent the nesting relation we could
write the diffusion to the right reaction for the first position of the 1-d
array as:
$$
\mathrm{Cell_1}(A, x), \mathrm{Cell_2}(y) \rightarrow \mathrm{Cell_1}(x), \mathrm{Cell_2}(y, A)
$$
This description has the same problems as the simple PN description in that it is
not compact and the creation of new cells requires the creation of new
species. If we interpret the nesting relation as the 'next-to' relation we
can use a single $\mathrm{Cell}$ species and creation of new cells would not
require creating new species:
\begin{align*}
  \text{State (4 cells): } & \mathrm{Cell(Cell(Cell(Cell)))} \\
  \text{Division: } & \mathrm{Cell}(x) \rightarrow \mathrm{Cell}(\mathrm{Cell}(x))
\end{align*}
Then, however, we would not be able to use the given nesting relation for the
actual containment of the auxin molecules. The problem here is that there are
two relations, the 'next-to' relation over cells that we need since the
communication is only defined for neighbouring cells and a nesting relation
between cells and auxin molecules.

The same problems discussed above regarding abstraction of object attributes via
quantifiable properties (\eg cell size), abstractions over parts of the whole
state (\eg size of entire root), and representation of the surrounding
environment also appear here.

\subsubsection*{Links}
\label{subsec:links}
Another organisational principle sometimes considered explicitly is that of
complexation inspired by protein complexes. Kappa \citep{danos_formal_2004}
considers the state of the system as a graph, for example:
\begin{center}
    \includegraphics{figures/kappaG.eps}
  \end{center}
% semantics  
Reaction (or rules) have the usual qualitative interpretation: anything that
'matches` the lhs in the state can be replaced by the rhs. Kappa
also has a stochastic interpretation as a CTMC where the states are
site-graphs. Practically, again, the problem is finding and counting the matches
of lhs's of rules into the state, which is even harder in the case of
graphs. There are, however, certain properties of Kappa graphs that
allow for a more efficient simulation that does not require recomputing the
matches after every rule application \citep{danos_scalable_2007}.
  
% examples
Using this site-graph approach we can represent the 1-D arrangement of cells from
our example:
\begin{center}
    \includegraphics{figures/kappaCell.eps}
\end{center}
We, again, have the same problem we noted above regarding properties that cannot
easily be represented by discrete objects, like cell properties, observables
over states, and time. We note though that the simulator of Kappa, KaSim
\citep{kasimmanual2018}, has facilities for observation of states, which can be
used inside rules.

Other formalisms allow for more than one relation, for example bigraphs
\citep{milner1999communicating} and the more biologically relevant stochastic
bigraphs, which have a stochastic interpretation as CTMCs
\citep{krivine_stochastic_2008}. Using this we could represent both the
neighbour relation between cells and the nesting of auxin molecules inside
cells. The rest of the limitations we noted persist, however, since it is an
object-only language.

\subsection{Combining structure and dynamics}
\label{subsec:structDynLangs}
Going beyond the object world, other extensions to these rule-based languages
try to combine traditional features of object-based formalisms with dynamical
features to represent more abstract quantifiable properties of the objects.  It
is interesting to note that adding attributes allows one to implicitly represent
relations based on attribute selections on the left-hand side of rules (see
discussion in \ssec{cpns} and next chapter).

\subsubsection*{Coloured Petri Nets}
\label{subsec:cpns}
% description
As the name implies Coloured Petri Nets are an extension to Petri Nets
(Section~\ssec{pns}) that allows distinctions between objects of each species
(colouring of objects) by allowing them to have an associated data value
adhering to the type (colour set) of their
species~\citep{jensen_coloured_1987}. For example, for our root development
system instead of having a simple $\mathrm{Cell}$ species we can have a
$\mathrm{Cell}$ species with associated parameters (\eg position in the
array). Qualitatively CPNs work in the same way as simple PNs. Applying a
reaction moves tokens between places. Since we have more complex type of objects
the transitions have inscriptions to bind object colour values to variables
names so that they can be used in the lhs (out-arrows) of the transition. Note
that the same dynamical description is used for the dynamics of objects and
their attributes. In the case of object-based dynamics (adding, removing
objects) conceptually things are the same as before. For changing attributes we
simply remove an object and replace with its attribute changed.

%semantics
A stochastic version of this CPN formulation has also been used for biological
modelling before, for example for describing planar cell polarity in Drosophila
wings~\citep{gao_multiscale_2013} (and see~\citet{runge_application_2004,
gilbert_colouring_2013} for other examples). In these examples where the
stochastic version was used, its semantics are just given as a translation to
the corresponding simple Petri Net. CPNs are used in their graphical format and
most tools therefore use a graphical interface.

Going to our root development example we can use colours to represent the
`next-to' relation between cells, as well as the concentration of auxin and
other abstract quantifiable properties like cell size. To represent\emph{diffusion} we
employ the $\mr{Cell}$ colourset $\{\mr{id}: \mr{int}, \mr{l}:\mr{int},
\mr{a}:\mr{int} \} $. Apart from the number of auxin molecules, $a$, we also have
positional information. In order to determine the position of a cell in the
array we use its identifier colour $\mr{id}$ and the identifiers of its left
neighbour, $\mr{l}$ colour, and right neighbour $\mr{r}$ colour.
\begin{center}
    \includegraphics{figures/cpnDiff.eps}
\end{center}
This is an example of an implicit representation of a relation between objects
where the relation information is stored in the objects themselves. This is in
contrast to the explicit representation of relations that we have seen in the
languages with representation of structure (hierarchy and links, see
previous sections). While it is more flexible it places more burden on the user
to maintain this relation.

To represent \emph{growth} we add another colour,
$\mr{s}:\rm{real}$ to our $\mr{Cell}$ colourset and another colourset for the
entire root ($\mr{Root}$) so that we can store its size ($\rm{s}$)
\begin{center}
    \includegraphics{figures/cpnGrowth.eps}
\end{center}
Growth also degrades auxin concentration in the cell. Here we have used an
additional colourset to represent global information as the language does not
offer the possibility of explicitly defining the functional correspondence
between the size colours of the $\mr{Cell}$ tokens and the size of the entire
root. This means that the user has to manually keep track of this function and
update the global information.

For both of the transitions we consider we used some colours with $\mr{real}$
types. Since the semantics of CPNs are given with translations to simple PNs via
enumeration of the corresponding simple transitions, the unfolded simple Petri
Net has in many cases an infinite number of reactions. This means that in order
to do the unfolding at all, the types have to be finite sets, which further
means that real values are not allowed. This is also reflected in the
implementation of Coloured Petri Net tools where one can define a Stochastic
Coloured Petri Net but the definition is unfolded before it is
run~\citep{heiner_snoopyunifying_2012}. Petri Net tools further have a graphical
interface for defining the models, although there is a hybrid approach that
allows mixing the graphical definition with programming language constructs (ML
language)~\cite{jensen_coloured_1987}. While graphical notations are intuitive
for smaller models, we find that for larger models they become hard to read.

Finally, following the root development example there is no way to easily
represent the water availability in the surrounding soil.

\subsubsection*{Coloured Stochastic Multilevel Multiset Rewriting (CSMMR)}
CSMMR~\citep{oury_coloured_2011} is an extension to
SMMR~\citep{oury_multi-level_2013} that adds parameters to agents, motivated by
the limitations we have noted in the object-only languages that we have
considered (subsections~\ssec{structLangs} and~\ssec{pns}). Parameters can take
part in rules either passively by influencing rates rules, conditions and so on
but can also be actively changed. The addition of parameters gives it a very
similar flavour to CPNs. However, a CSMMR model, unlike a CPN model, is given
stochastic interpretation directly as a CTMC, without unfolding to a simpler
form.

Using a CSMMR agent type similar to the CPN $\mr{Cell}$ colourset for the root
development example (\ssec{rootDev}), we can write the following
for the diffusion (to the right) of auxin along the root:
$$
\mr{Cell}_{\mr{id}=i, \mr{a}=a}, \mr{Cell}_{r=k, a=a'} \xrightarrow{}
\mr{Cell}_{\mr{id}=i, \mr{a}=a+D(a'-a)}, \mr{Cell}_{r=k, a=a+D(a-a')} \, \,
[i=k]
$$
This is very similar to the CPN transition of the same dynamics but written in a
completely textual form.

Similar limitations to the CPNs (discussed above) regarding time dependent
values (\eg for representing water in the soil) and observational abstraction
over parts of the state (\eg size of the entire root) remain. The availability
of richer types for parameters would also work well with a more expressive
expression language to represent their dynamics perhaps utilising a
general-purpose programming language, like CPN tools do with ML.

\subsubsection*{Dynamical Grammars}
Dynamical Grammars of \citet{mjolsness2006stochastic} have a truly hybrid
approach where the quantifiable properties of objects can be defined by
differential equations as in dynamical systems theory.
$$
\{\tau_i[x_{i, 1}, \dots, x_{i, k[\tau_i]}] \} \rightarrow \{\tau_i[x_{i, 1}, \dots, x_i, k[\tau_i]]\} \, \,
\text{ solving } \{\frac{dx_{i,j}}{dt}=f_{i, j}(t)\}
$$
Everything else looks similar to other languages in this section, dynamics are
given as rules where each side of the rule is a multiset of discrete objects
with typed sequence of attributes. The interesting part comes from the
$\mr{solving}$ clause that can be used to express the dynamics of the attributes
as differential equations.

Going to our root development example, the diffusion can be expressed in a way
that is more familiar to the traditional diffusion models,
\begin{align*}
\{c_i=\mr{Cell}(i, a_i), c_{i+1}=\mr{Cell}(r, a_{i+1})\} \rightarrow \{c_i,
  c_{i+1}\} \\
  \text{ solving }\{\frac{da_i}{dt}=D(a_{i+1}-a_i),
  \frac{da_{i+1}}{dt}=D(a_i - a_{i+1} \}
\end{align*}
where we use the same $\mr{Cell}$ agent type as before.

Having the differential equations means we can also express time-dependent
values, for example to represent the change of moisture in the soil surrounding
the root in an abstract (non-mechanistic) way.
\begin{align*}
\{\mr{Shoot(s)}, \mr{Env}(w), \mr{Root(s)} \rightarrow \{\mr{Shoot(g(s))}, \mr{Env}(w),
  \mr{Root(s)} \} \\
  \text{ solving }\{\frac{dw}{dt}=f(w, t) \}
\end{align*}
However, if we are, as it is usually the case, taking the water values from a
table then it might be hard to find a mathematical description that fits the
differential equation framework. Mixing these two paradigms (object-based and
differential equations for the dynamics), while powerful, means that sometimes
rules do not correspond naturally to the conceptual view of biological events as
instantaneous flows of matter.

We note that there is no possibility for expressing observational abstractions
over parts of the state and a similar scheme to the one we used in the CPN
formulation, with an extra $\mr{Root}$ type object will have to be used.


\subsection{Plant development: L-systems}
\label{subsec:lSystems}
Another line of work started in the plant development tradition.
They followed roughly the same evolution of starting with simple objects and
then adding parameters for combining quantifiable properties with
objects. Traditional object-only L-systems are a string-rewriting formalism
where strings of characters (representing the objects) can be rewritten via
rules that specify how characters can be replaced by other character (or
sequences of characters). The lhs of the rules can be context-sensitive by
specifying the characters surrounding the character intended for replacement:
$$
a< b > c \rightarrow a < d > c
$$
For example, the above specifies that an object $b$ can be replaced by an object
$d$ in the string state of the system in places where it is surrounded by $a$ on
the left and $c$ on the right. Unlike the object languages we have seen before
that rely on multisets, L-systems rely on strings, which are ordered. Adding
parameters work in the same way as the transition from PNs to CPN, for example,
and it was motivated by the same modelling requirements.

%semantics
Unlike the PN family from the Biochemistry tradition, L-systems are not usually
interpreted as CTMCs. They instead rely on discrete time and multiple parallel
application of rules in cases where more than one is applicable.

% examples - limitations
The ordered parametric L-systems seem to work well for our root development
example since the cells along the root are in an ordered sequence. The
'next-to' relation then needs no further machinery (implicit representation) to
be represented. The diffusion rule will look like this, again using a
$\mr{Cell}$ object with similar parameters as before:
$$
\mr{Cell}(a_i)< \mr{Cell}(a_{i+1}) > \rightarrow \mr{Cell}(a_i+D(a_{i+1}-a_i)< \mr{Cell}(a_{i+1}+D(a_i-a_{i+1}) >
$$

L-systems might be not be as good in cases where the structure of the system is
not linear or cannot (easily) be linearised as a string. Take, for example, the
dynamics from our plant development example (Section~\ssec{plantDev}). The
leaves in that case and the 'cell` are not linearly related. It would therefore
be hard to write, for example, the rule for leaf growth that requires
information from both the leaves and cell objects.

Similar limitations regarding observational abstraction of parts of the state,
time-dependent value representation also apply here.

\section{Dynamics to structure}
\label{sec:dyns}
In the previous section we focused on languages that are object-based and
followed their evolution through adding features like adding extra structure on
the objects and then adding features that are more traditionally from the
dynamics world. Here, instead, we focus on languages that have dynamics at their
heart but have some features for representing (dynamic) structure as
well.

Modelica is an object-oriented language (or language standard) combining
traditional dynamics representation with structure
\citep{fritzson_modelicaunified_1998} with various implementations
\citep{otter_modeling_1996, li_hybrid_2007}. Dynamics are given as either
non-causal relations between variables or differential equations. While it is
object-based, the objects are intended for handling model complexity instead of
representation of domain entities. This is also reflected by the fact that the
object organisation is static. Later extensions
\citep{nytsch-geusen_mosilab:_2005} or inspired languages
\citep{zimmer2008introducing} add dynamic structure. The module dynamics are
given by boolean triggered transitions between model modes representing
different versions of the model with possibly different structure. This, again,
reflects the fact that modules are intended for structuring a model instead of
representation and they seem very far off from conceptual views of biological
systems especially if one is talking about the dynamics of populations of
objects. If we wanted, for example, to represent the root example with cells as
Modelica (+ dynamic structure) modules we would need one mode for each
population size. It is also very hard to see how to represent relations between
objects that affect the structure changes.

System dynamics is another popular approach to capturing continuous dynamics of
variables through the visual notation of stocks and flows. Stocks are state
variables and dynamics are given as flows between them. Recent commercial
projects integrating a modelling language based on systems dynamics with an
integrated development environment have started adding some discrete features as
well to combine dynamics with object-based representations \citep[Ventity,
AnyLogic;][]{yeager_entity-based_2014, borshchev2004system}. These have some
characteristics in comon with the Modelica family of languages but the object
dynamics are first-class and are represented through some language abstraction
that allows their definition. We next focus in more detail on one of these,
Simile, which has been used in Biology before.


\subsection{Simile}
\label{subsec:simile}
%description
Simile is another graphical language that has similarities to our approach
\citep{muetzelfeldt_simile_2003}. Simile is used mainly in the domains of
ecology and agricultural sciences but has also been used in systems biology
systems biology (for the whole-plant model \citep{chew_multiscale_2014} that was
the inspiration for our example in Section~\ssec{plantDev}) to exactly solve the
kind of limitations we noted in the object-only based languages. External inputs
are also supported, which is particularly important for adding weather data to
crop models. In Simile there are two levels of definition of a model. At the
first level we have continuous variables with rate equations and at the second
level we have discrete objects with discrete dynamics -- adding/removing. The
objects are grouped based on their types, and their behaviour is given at the
population level. The dynamics of the two types of entities, continuous
variables and objects, are not integrated as is the case in CPNs and CSMMR.

%semantics?
There is an obvious interpretation of the first-level continuous variables as a
systems of differential equations but there is no obvious interpretation as a
formal mathematical object for the entirety of the Simile features including the
object-level dynamics. At the implementation level a model generates a system of
Ordinary Differential Equations (ODEs) and then keeps track of the objects. Any
time there is a structural change to the model, the system of ODEs is
recalculated and integrated numerically until the next point of structural
change (\eg new object in population).

%limitations through examples
In order to illustrate how Simile features capture the modelling requirements from our
examples, let us start with the plant development example (Section~\ssec{plantDev}),
\begin{center}
  \includegraphics[scale=0.6]{figures/simileGrowth.eps}
\end{center}
We have a population of Leaf objects and a single Cell object representing what
we called $\mathrm{Cell}$ in our rules. Each Leaf in the population has a mass
that grows as a continuous variable. In order to define the use of carbon for
growth from the carbon variable in the Cell object we have to work at the
population level by summing the contribution of each Leaf. The population of
Leaf objects also grows (see creation box). This concise description is at the
level of objects instead of at the level of events, like in rule-based
languages, which means that the dynamics are distributed across the
model. Having the dynamics at the rule instead of the at the object level
sometimes seems more natural as they correspond more intuitevely to the
conceptual view of biological events. Even with mental models of processes, one
usually speaks about the dynamics of the system as interactions of components
instead of the behaviour of each component individually.

Object relations can also be encoded through conditions to restrict the dynamics
to a subset of the object types, similarly to conditional expressions available
in CPNs and CSMMR. For example, the diffusion dynamics that happen only between
neighbouring cells can be encoded.
\begin{center}
  \includegraphics[scale=0.6]{figures/simileRoot.eps}
\end{center}
Here the diffusion dynamics are restricted to neighbouring cells through the
association submodel, assoc, and a relevant condition (see `next-to' box).

Representing cell division is more challenging as the dynamics of objects can
only be represented at the population level. In the case of cell division the object
`creation' is not identical for all cells since the position matters for keeping
the next-to relation information (identifiers) updated.

%general meta-level limitations
Finally, like CPNs, the graphical notation of Simile becomes, in our experience,
problematic for models with more than a few variables. While CPNs have graphical
elements for species and transitions, Simile represents all variables and
interactions graphically further adding to the complexity of the
representation. This could, however, be mitigated by model hierarchy. Since the
representation of objects in Simile, which could correspond to entire models,
encourages model composition, further visual tools that allow hiding unnecessary
nested models could be used to handle model complexity.

\section{Pragmatic approaches}
\label{sec:pragmaticApproaches}
When modellers are confronted with complexity in models like the ones we have
seen in the Introduction or the example ones from this chapter, they sometimes
turn to more pragmatic approaches when they cannot map their understanding into
any suitable technical language.

Most of the pragmatic approaches are based on using an ad-hoc representation of
the model as a program in a general-purpose programming language. We next give
an overview of these methods with references to the examples as before.


\subsection{Models as programs}
By far the most popular approach when confronted with the limitations of either
object-based or dynamics-based languages we noted above, is to use a custom
simulation software. In this approach the domain knowledge is represented as a
program and it is mixed with the simulation code, especially if one is writing a
one-off program. The programs are not usually mapped or interpreted as
mathematical objects. Models like the Framework Model \citep{chew2014multiscale}
and the whole-cell of \citet{karr_whole-cell_2012}, which are examples of the
kinds of the comprehensive multi-scale models we are interested in, are written
in this way, for example (both given as Matlab programs).

Turning to our examples, in order to write the root development model one could
write something like the following in a general-purpose programming language:

\begin{BVerbatim}
type Mode = Idle | Growing
type Cell = {s, a: real, m: Mode}
type Shoot = {s : real}
type State = {root : [Cell], shoot : Shoot}

function simulate(s : State, n:Int) -> [State]
  ...
  for i = 1 to n
    s' = step s
    ...

function step(s: State) -> State
  s' = (diff . growth . cellDiv . ...) s
  return s'
\end{BVerbatim}

where we assume we are in a programming language that allows the definition of
compound data types that we use to represent the information about cells and the
shoot. We first decide on a representation for the state of the system where we
use a list to represent the array of cells in the root since lists are
ordered. Then in order to simulate the system we apply the \texttt{step}
function \texttt{n} times to some initial state. The \texttt{step} function
applies functions that represent all the biological processes that change the
systems of the system, diffusion, cell growth and division and so on.

One is faced not only with a decision on representation but also with a decision
of a simulation algorithm to follow, since the model is not mapped to a
well-defined technical language with existing implementation. Since
general-purpose programming languages are inherently sequential, for example,
one has to find a reasonable serialisation of the execution of the processes
that does not affect the results. \citet{karr_whole-cell_2012} deal with this by
using an appropriate time-step and in each time-step randomising the execution
of the processes.
%mention somewhere declaratively vs describing the process

This approach gives greater flexibility as one can use any existing programming
language constructs, either built-in or through external libraries, for the
construction of the model. At the same time, unless one spends a great deal of
effort, the resulting representation is hard to read in the first place and hard
to maintain or extend in the future. In contrast, the rule-based representations
we have seen above are inherently declarative and concurrent and extending a
model is just a matter of concatenating more rules to the existing ones.

\subsubsection{Multi-model simulation environments}
To deal with the problem of extension many models are written in a modular way
such that related components are packaged together. Each component is written in
the most convenient way. The problem comes in their combination.Sometimes models
are written in different programming languages, with different input/output
formats, different time-steps, units and other incompatabilities. What happens
in practice is that models are usually reimplemented in one language to create a
multi-model written in a single language. This is the approach taken, for
example, in the Framework Model where all constituent models are reimplemented
in Matlab along with other compatability changes, likes time resolution changes.

There has been some work on multi-model simulators that can deal with
heterogeneous models written in possibly different languages or have other
incompatibilities. The idea is that the user declares their intentions on how
the modules should interact and the rest, like interaction, conversion and
simulation, are handled by the framework. The modular integration of processes
in the \citet{karr_whole-cell_2012} model inspired the \emph{mois} tool
\citep{erbm_mois_2015} that uses various techniques, like variable time-steps
and backtracking \citep{bucher2013decomposition}, for the simulation of the
interacting modules. Another tool \citep{cis_2018} was developed as part of the
Plants in Silico initiative \citep{zhu_plants_2016}, a collaborative effort on
multi-scale plant and crop modelling.

\subsection{Simulation frameworks}
\label{sec:simFrameworks}
Going a step further, instead of the user having to write custom simulation
software, some frameworks provide libraries that expose common functionality for
the construction or combination of models.

\subsubsection*{Agent-based frameworks}
In Agent-based modelling (ABM) simulation frameworks, the description of the
process/model happens declaratively through the description of the behaviour of
classes of agents (see for example \cite{solovyev_spark:_2010} used in systems
biology).

While initially this looks similar to the object-based formalisms we have seen
above, in ABM systems the main unit of description is the entire behaviour of
each individual agent, whereas in rule-based formalisms it is the rule, which
can both describe a (possibly partial) behaviour of an individual agent and a
synchronised action of two or more agents. This ability to specify
synchronisation often leads to more natural descriptions and, more practically,
makes the resulting models easier to change and combine. The advantages of being
in a programming language remain.

\subsection{Crop Simulation frameworks}
Modelling has historically been used much more extensively in crop science than
in plant biology. Several large models incorporating significant environemental
components (from soil to ecosystem) have been developed. These have been
increasingly developed in a modular fashion providing the user with
functionality for common modelling scenarios, like calculation of water uptake
or light interception by the canopy \citep{keating_overview_2003}. In order to
increase re-usability and decrease user effort, several frameworks have been
developed on top of crop simulaiton software that offer a set of parameterised
components that can be used to model different scenarios or even different crops
\citep{brown_plant_2014}.

This modular approach of crop models has found its way in complex and
comprehensive biology models, like the \citet{karr_whole-cell_2012} whole-cell
model and the Framework Model \citep{chew2014multiscale}, which were built in a
modular fashion with exactly the same motivation -- re-usability and
extensibility. Writing these models requires familiarity the programming
language of the framework but, again, the advantages of a general-purpose
programming language remain.


\subsection{Model interchange formats (SBML)}
% SBML
The growth of the systems biology movement led to a large number of tools for
the representation and analysis of models of molecular processes. Most of the
models can be represented by all dealt with similar type of processes of genetic
networks and were represented similarly as systems of ODE's in the continuous
form or Petri Nets in a discrete form. The Systems Biology Markup Language
\citep[SMBL;][]{hucka_systems_2003} was developed as an interchange format
between the different tools for these kind of processes and had support for the
representation of the most commonly used constructs in these models, reactions,
compartments along with model information, like units and so on.

As models started becoming more complex it faced the same limitations we noted
in the Introduction and in the Petri Net section regading dynamic organisation
among others. This led to the development of multiple extensions to the original
format \citep[SBML L3;][]{sbmlL3} to handle the increased complexity. For
example, the `Dynamic Structures' package adds support for defining dynamic
structures and the `Hierarchical Model Composition' package adds support for
hierarchy representation. The L3 specification was put to the test for the
representation of the \citet{karr_whole-cell_2012} whole-cell model
\citep{waltemath_toward_2016}.

SBML models are only used as a description and interchange language and they do
not have any interpretation as mathematical objects. Using them directly for
modelling is probably impractical although there are some graphical tools that
allow the construction of SMBL models \citep{hoops_copasicomplex_2006}.






%\section{Conclusion}
%add some conclusion thing that summarises this chapter and gives a pass to the
%next one



\chapter{Chromar by example}
\label{chp:chromarEx}
In this chapter we will introduce Chromar (formally introduced in the next
chapter) by means of examples. The example serve two purposes: to introduce the
features of the languages but also as a means of comparing Chromar to the related
languages in the previous chapter by seeing how Chromar features handle the
modelling requirements that the examples present.

Chromar has the following main ideas, which we will explore in this chapter
through the examples:
\begin{itemize}
\item \emph{basic Chromar} is a rule-based notation with stochastic semantics
  yielding a Continuous Time Markov Chain (CTMC). It follows the object-based
  tradition (see previous section) and makes use of discrete objects called
  \emph{agents}. These have attributes that are defined at the type level, so
  that every agent of that type has these attributes. For a version of our auxin
  example a $\mr{Cell}$ agent type could be, similarly to other parametric
  languages,
  $\mathrm{Cell}(\mathrm{pos}:\mathrm{int}, \mathrm{a}: \mathrm{real})$ with
  attribute $\mathrm{pos}$ for positional information and attribute $\mathrm{a}$
  to keep its auxin concentration. Agents are instantiations of this type with
  concrete values for the attributes like
  $\mathrm{Cell}(\mathrm{pos}=1, \mathrm{a}=a_1)$ for the first cell,
  $\mathrm{Cell}(\mathrm{pos}=2, \mathrm{a}=a_2)$ for the second cell and so
  on. The states of the CTMC are multisets of agents. The rules describe how
  agents are added to or removed from states at a more abstract level than
  individual agents. This is done by using patterns on the rule left-hand sides
  specifying the group of agents for which the dynamics are defined. As we have
  seen, having parameters help us overcome some of the limitations of object-only
  languages.
\item \emph{extended Chromar} extends basic Chromar with two new features:
  \begin{itemize}
  \item[(i)] \textit{Fluents} --- the incorporation of deterministically
    changing time-dependent values. These are important for modelling dynamics
    in a changing environment, and
  \item[(ii)] \textit{Observables} --- values calculated from a global view of
    the system. These are important in cases where a coarse-grained view of the
    system is needed. This may be because we cannot acquire atomistic data, or
    because we do not wish to model everything at the same level of
    detail. Observables also give us a flexible way to observe the state of the
    system that can be used to report the results of model simulations, as we
    often need time series of some observable on the state of the system rather
    than time series of the state itself.
\end{itemize}
\item A concrete realisation of extended Chromar as an embedded Domain Specific
  Language (DSL) inside Haskell, a functional programming language
  \citep{gibbons_functional_2015}. The embedding means that we can use any valid
  Haskell expression where expressions are expected, for example in the rate
  expressions and in the right-hand sides of rules. Agent types and the outside
  environment (defining rate, condition function etc.) are defined directly as
  Haskell expressions but rules, fluents, and observables are defined using
  abstract Chromar syntax via quotation \citep{mainland_why_2007}.
\end{itemize}

The examples from the previous section are extensions to the examples that
appear in the articles describing Chromar \citep{honorato-zimmer_chromar_2017,
  honorato-zimmer_chromar_2018}. Therefore the following text is also in part
comes from the same sources. Note that the author order follows the traditional
alphabetical order of computer science literature. Ricardo Honorato-Zimmer
participated in the meetings/discussions of ideas with Prof Gordon Plotkin who
was my primary supervisor for this part of the work while Prof Andrew Millar was
secondary.


\section{Plant development in a field}
\label{sec:plantDev}

\subsection{Single plant}
Since agents are the main entities in our language, let us
consider what types of agents we should have to model the above
system, given the above discussion. We will need:
\begin{itemize}
\item a $\mathrm{Leaf}$ type with attributes for 
appearance: $\mathrm{Leaf}(\mathrm{age}:\mathrm{int},
\mathrm{mass}:\mathrm{real})$,
\item a $\mathrm{Cell}$ type that represents our main plant `cell' with an
attribute, $\mathrm{carbon}$, to keep the current carbon level:
$\mathrm{Cell}(\mathrm{carbon}:\mathrm{real})$ (there will only be one agent
\item a $\mathrm{Ros}$ type that represents the entire Rosette with an
attribute, $\mathrm{n}$, to keep the current number of leaves:
$\mathrm{Ros}(\mathrm{n}:\mathrm{int})$ (there will also only be one agent
\end{itemize}
We will use the current number of leaves relative to the index of
a particular leaf as a proxy for the leaf's age: the larger the difference
between the index of a leaf and the current number of leaves, the older that
leaf is.

For the \textit{carbon assimilation} from one particular leaf we need to
increase the carbon concentration of the central Cell. The bigger the leaf the
faster it contributes to the production of carbon:
\[\mathrm{Leaf}(\mathrm{mass}=m), \: \mathrm{Cell}(\mathrm{carbon}=c) \:
  \xrightarrow{f(m)} \: \mathrm{Leaf}(\mathrm{mass}=m), \:
  \mathrm{Cell}(\mathrm{carbon}=c+1)\] We can read this rule as saying that for
any two Leaf and Cell agents, the Leaf agent remains the same and the Cell agent
increases its carbon content by one. Note that we assign the values of the
attributes the left-hand side of the rule to the variables $m$ and $c$, so that
we can refer to them in the right-hand side of the rule and the rate
expression. Since the $\mathrm{age}$ is not used in the rest of the rule, it can
be omitted (see \textit{Missing attributes} syntactic extension,
\sct{syntaxExt}). If we were to write this in a traditional reaction notation we
would have to write a reaction for every possible Leaf agent which leads to the
compactness problem we have noted earlier (indeed, there would be infinitely
many such possibilities). The implicit `for-all' in the pattern on the left-hand
side allows the rule to be applied to new leaves when they are created. For
example if the state of the system has two leaves, the central cell, and the
rosette agent:
\begin{align*}
\m{ \mathrm{Leaf}(\ar{age}{1}, \ar{mass}{10}), \:
\mathrm{Leaf}(\ar{age}{2}, \ar{mass}{5}),
\mathrm{Cell}(\ar{carbon}{20}), \mr{Ros}(\ar{n}{2})}
\end{align*}
then the rule is applicable to the two substates
\begin{align*}
& \m{\mathrm{Leaf}(\mathrm{age}=1, \mathrm{mass}=10),
                 \mathrm{Cell}(\mathrm{carbon}=20)} \: \text{ and} \\
&\m{\mathrm{Leaf}(\mathrm{age}=2,
  \mathrm{mass}=5), \mathrm{Cell}(\mathrm{carbon}=20)}
\end{align*}

For \textit{maintenance respiration}, the central $\mathrm{Cell}$ agent gives
some carbon to a $\mathrm{Leaf}$ agent, with the amount of carbon needed for
maintenance depending on the size of the Leaf:
%
\begin{align*}
&\mathrm{Leaf}(\mathrm{mass}=m),\: \mathrm{Cell}(\mathrm{carbon}=c) \:
\xrightarrow{h(m)} \\ &\mathrm{Leaf}(\mathrm{mass}=m),\:
\mathrm{Cell}(\mathrm{carbon}=c-g(m)) \; [c \geq g(m)]
\end{align*}
Note the use of the condition $c \geq g(m)$ to make sure that the
carbon levels do not become negative. Also note that here we use both the rate
($h(m)$) and the increment size ($g(m)$) to control the amount by which carbon
is updated in a time unit. Since the rate controls the number of times the
update will happen, the product of the rate and update functions is the expected
amount that will be removed from the carbon pool in a time unit.

Next, \textit{leaf growth} depends on the mass of the leaf, its age (there is a
limit on how much a leaf can grow so older leaves stop growing at some point),
and the amount of carbon available:
%
\begin{align*}
  & \mathrm{Ros}(\mathrm{n}=n), \mathrm{Leaf}(\mathrm{age}=i,
\mathrm{mass}=m),\: \mathrm{Cell}(\mathrm{carbon}=c) \: \xrightarrow{h(n-i, m,
    c)}\: \\
  & \mathrm{Ros}(\mathrm{n} =n), \mathrm{Leaf}(\mathrm{age}=i,
\mathrm{mass}=m+1),\: \mathrm{Cell}(\mathrm{carbon}=c-1) \; [c \geq 1]
\end{align*} The mass of the leaf is also needed in the calculation to make sure
that the growth does not exceed observed physical constraints. The growth is
capped to a fraction of the current mass of the leaf.

Finally, for \textit{leaf creation} we have:
\begin{equation*} \mathrm{Ros}(\mathrm{n}=n) \: \xrightarrow{k} \:
\mathrm{Ros}(\mathrm{n}=n+1), \: \mathrm{Leaf}(\mathrm{age}=n+1,
\mathrm{mass}=m_0)
\end{equation*}

\subsubsection*{Fluents and Observables}
There are two problems with the above definition: 
\begin{itemize}
\item[(i)] There is no way to include the environment in the model, and so it is
  assumed constant. This makes the model very detached from reality.  For
  example, our plants photosynthesise all the time, whereas in reality they only
  do so during daylight, but we have no direct way to switch between day and
  night.
\item[(ii)] We have two representations of the same process at two different
  levels of abstraction that have to be kept consistent with each other by the
  user.  Specifically, the $\mathrm{n}$ attribute in the $\mathrm{Ros}$ agent
  keeps track of the total number of leaves in the plant. However, there is no
  way in the language to specify the global information the attribute is
  tracking, and check that its value is indeed what it should be.
\end{itemize}

I next introduce two new notational features
to solve these problems.

\textit{Fluents}

To solve the first problem I introduced time-dependent values called
\textit{Fluents}. These can be constructed through a combination of a small set
of primitives and general expressions, taken from Reactive Programming (FRP)
constructs in \citep{wan_functional_2000} (Fluents are usually called Behaviours
in FRP). For example we could write a fluent for light, which is a function from
time to the Booleans, and another one for temperature that depends on light:
\begin{align*}
& light = \mathbf{repeatEvery} \; 24.0 \; (\mathrm{True} \; \mathbf{when} \; (6 < \mathbf{time} < 18) \; \mathbf{else} \; \mathrm{False}) \\
& temp = 21.0 \; \mathbf{when} \; light \; \mathbf{else} \; 16.0
\end{align*}
where the $\mathbf{when} \dotso \mathbf{else}$ construct denotes conditional
behaviour and $\mathbf{repeatEvery}$ cycling behaviour. Cycling behaviour is
achieved with the modulo operation so the value of $\mathbf{repeatEvery} \; t_o
\; f$ at $t$ is the value of $f$ at $t \; mod \; t_0$.

The assimilation rule can then be written as follows:
\begin{align*}
& \mathrm{Leaf}(\mathrm{age} \!= \!i, \mathrm{mass} \!= \!m), \: \mathrm{Cell}(\mathrm{carbon} \!= \!c) \: \xrightarrow{f(m, temp)}  \\
&\mathrm{Leaf}(\mathrm{age} \!= \!i, \mathrm{mass} \!= \!m), \:
  \mathrm{Cell}(\mathrm{carbon} \!= \!c+1) \; [light ]
\end{align*}

Very often in biology we have empirical relationships of various quantities with
time. Fluents can be used to encode these as well. Such empirical relationships
do not provide any mechanistic insight but they can be useful when we either do
not have data or do not want to model more mechanistically using rules.

\textit{Observables}

To solve the second problem, I introduced functions on the state of the system
called \textit{observables}. Observables are constructed using a combination of
database-inspired $\mathbf{select}$ and $\mathbf{aggregate}$ operations: % we
think of our state as a kind of database where we have a collection of agents
rather than the more usual (and very similar) collection of records. The
$ \mathbf{select}$ operation selects agents of a particular type
from %specifies a
part of the state, producing a binding of each such agent's attribute values;
the $\mathbf{aggregate}$ operation then folds the resulting collection of
attribute value bindings into a single value, using a specified combining
function and initial value.

For example, to calculate the total mass  of the leafs in a state, we could write:
 \begin{align*}
lm = \mathbf{select} \, \mathrm{Leaf}(\mathrm{age} = i, \mathrm{mass} = m) \mathbf{;} \; \mathbf{aggregate} \;
 (total: \mathrm{int}.\, total + m) \; 0
\end{align*}
%
where we first select every agent that matches a $\mathrm{Leaf}$ pattern,
producing a binding of its $\mathrm{age}$ and $\mathrm{mass}$ attributes to $i$
and $m$, respectively, and then, starting from $0$, calculate the result using a
combining function that adds the value of $m$ in every such binding to the
running total.
 %
If we further wished to calculate the average mass of all the leafs in a state,
we could use the observable
\begin{align*}
nl = \mathbf{select} \, \mathrm{Leaf}(\mathrm{age} = i, \mathrm{mass} = m) \mathbf{;} \; \mathbf{aggregate} \;
 (count: \mathrm{int}.\, count + 1) \; 0
\end{align*}
%
to count the number of leafs in a state, and then divide $lm$ by $nl$.

Whenever an observable is used, one obtains its `fresh' value, even when the
underlying state has changed (\eg, by creating new leaves, in this case). For
example the \textit{leaf growth} rule now becomes:
%
\begin{align*}
&\mathrm{Leaf}(\mathrm{age} \!= \!i, \mathrm{mass} \!= \!m),\:
  \mathrm{Cell}(\mathrm{carbon} \!= \!c) \xrightarrow{h(nl-i, m, c)}\:   \\
  &
 \mathrm{Leaf}(\mathrm{age} \!= \!i, \mathrm{mass} \!= \!m+1),\:
    \mathrm{Cell}(\mathrm{carbon} \!= \!c-1) \; [c \geq  1]
\end{align*}
where we use our observable $nl$ directly in the rule rate. There is no need to
use the $\mathrm{Ros}$ agent any more as its only purpose was to keep track of
the number of leaves.

Note that observables, like fluents, can be used to model parts of the system in
cases where there is either not enough knowledge about a process, or else no
desire to model at a more mechanistic level. In this case, for example, the
leaves must have some mechanism for knowing their age. However this is not
central to our model, so we instead use global knowledge via observables to
abstract away from the details that would be involved in modelling that
mechanism.

The two features, fluents and observables, can be mixed arbitrarily along with
ordinary expressions. We might for example have a fluent inside an observable or
a fluent inside an observable and so on. For example we could write:
\begin{equation*}
f(m) \; \mathbf{when} \; nl > 10 \; \mathbf{else} \; f'(m)
\end{equation*}
where $nl$ is an observable for the number of leaves. This might be used to
introduce the crowding effect on rosette leaves; crowding affects the
assimilation rate as it reduces the photosynthetically active area.

\subsection{Field}
Going to the field where we have multiple plants we need some way of knowing
first which agents belong to the same plant and second the location of the
plant. We do not have an explicit way of representing these relations but we can
again use our attributes to encode them. I will use one identifier for the plant
and one for the patch of land. Every agent now has a further attribute to store
this information. For example, the $\mr{Leaf}$ agent type becomes:
$\mathrm{Leaf}(\mr{pos}:(\mr{int}, \mr{int}), \mathrm{age}:\mathrm{int},
\mathrm{mass}:\mathrm{real})$. The first element of the $\mr{pos}$ tuple is the
plant identifier and the second is the identifier of the patch it belongs to. I
further assume that we are in a suitably powerful programming language (\eg our
Haskell embedding) that I have (or can define) the functions $\mr{pid}$ to
select the first element of the tuple (plant id) and $\mr{cid}$ to select the
second element of the tuple (patch id).

The metabolic rules stay the same but add a further condition that the
agents on the lhs belong to the same plant. For example, maintenance respiration
now becomes:
\begin{align*}
  &\mathrm{Leaf}(\mathrm{pos}=p, \mathrm{mass}=m),\:
    \mathrm{Cell}(\mathrm{pos}=p', \mathrm{carbon}=c) \:
    \xrightarrow{h(m)} \\ &\mathrm{Leaf}(\mathrm{mass}=m),\:
                            \mathrm{Cell}(\mathrm{carbon}=c-g(m)) \; [c \geq g(m)
                            \, \land pid \, p = pid \, p']
\end{align*}

We assume that competition between plant affects the amount of light they
receive, which in turn affects the photosynthesis rate. The assimilation rule
will therefore change so that the light fluent is modulated by a competition
index, which will be an observable on the global state of leaves in an entire
patch. Here we assume that we are in our Haskell embedding and we have access to
the full power of the language. We define the following Haskell functions:
\begin{align*}
& \mr{comp} \: p = f (\mathbf{select} \; \mathrm{Leaf}(\mathrm{pos}=p' ,
  \mathrm{mass} = m) \; ; \; \mathbf{aggregate} \; (c: \mathrm{real}. \,
  \mr{sumComp}) \; 0 \\
& \mr{light} \: p = \mathbf{repeatEvery} \; 24.0 \; (\mathrm{light_{max}
                          } \cdot \mr{comp} \, p \; \mathbf{when} \; (6 < \mathbf{time} < 18) \; \mathbf{else} \; \mathrm{0})
\end{align*}
that take as input a $\mr{pos}$ tuple and return a relevant expression for that
particular plant in that patch. The combining function $\mr{sumComp}$ is defined
as: $\mr{sumComp} = \mathbf{if}\; cid \, p' = cid \, p \land pid \, p' \neq pid \, p
\;\mathbf{then} \; c + m; \; \mathbf{else} \; c$ that sums the masses of all
leaves in a patch except from the given plant. Then with the modulated light
definition the assimilation rules will change to:
\begin{align*}
& \mathrm{Leaf}(\mathrm{pos} \!= \!p, \mathrm{age} \!= \!i, \mathrm{mass} \!= \!m), \:
                 \mathrm{Cell}(\mathrm{carbon} \!= \!c) \: \xrightarrow{f(m,
                 temp, light \, p)}  \\
&\mathrm{Leaf}(\mathrm{age} \!= \!i, \mathrm{mass} \!= \!m), \:
  \mathrm{Cell}(\mathrm{carbon} \!= \!c+1) \; [light \, p > 0]
\end{align*}

\section{Root development}
\label{sec:rootDev}
For the root example, we again assume we are in our Haskell implementation and
we therefore have Haskell types and expressions in our disposal for the
definition of the model. We define the following agent types:
\begin{itemize}
\item a $\mr{Cell}$ type to represent the root cells in the array defined as:
  \begin{align*}
    \mr{Cell}(&\mr{id}: \mr{int}, \\
              &\mr{l, r}: \mr{link}, \\
              &\mr{s, a, d}: \mr{real}, \\
              &\mr{m}:\mr{mode})
    \end{align*}
    The type label $\mr{link}$ is interpreted as the set
    $\mathbb{N} \cup \{\epsilon\}$ and the relevant attributes ($\mr{l}$ for the
    left and $\mr{r}$ for the right neighbour) represent either a bound state
    (via an identifier) or a non-bound state ($\epsilon$). The type label
    $\mr{mode}$ is interpreted as the set $\{ 1, 2 \}$ and it represents the
    state of the cell, either growing (1) or idle (2). The other attributes
    represent the size of the cell ($\mr{s}$) and its auxin ($\mr{a}$) and
    division-factor ($\mr{d}$) concentrations (Figure~\ref{fig:rootDevChromar}).
  \item a $\mr{Shoot}$ agent type to represent the entire shoot of the plant
    defined as $\mr{Shoot}(\mr{s}: \mr{real})$ where the attribute $\mr{s}$
    represents the size of the shoot.
  \end{itemize}
  
\begin{figure}
    \centering
    \includegraphics[width=0.7\linewidth]{figures/rootChromar.eps}
    \caption{Scheme for capturing the `next-to' relation between cells and the
      division rule.}
    \label{fig:rootDevChromar}
  \end{figure}

In order to be able to give fresh identifiers to cells created by division we
also need to know the current number of cells in the array. We do this using
an observable:
\begin{align*}
nc = \mathbf{select} \; \mathrm{Cell}()\mathbf{;} \;\mathbf{aggregate} \; (count: \mathrm{int}.\, count + 1) \; 0
\end{align*}
As with our plant example in the previous section, we use a $\mathrm{count}$
expression that increases the running count by $1$ for each
$\mathrm{Cell}$. Note that like rules we can omit attribute bindings when the
values are not used in the subsequent definitions of the expression.

Turning to the dynamics of the system, for the auxin flow from the shoot we can
have the following rule:
\begin{align*}
\mr{Cell}(\mr{l} \!=\! \epsilon, a=a) \xrightarrow{} \mr{Cell}(\mr{a}=a + (a_{init} + \frac{0.17 \cdot \mathbf{time}}{t_{cc}}))
\end{align*}
where $t_{cc}$ is the time duration of the cell cycle. We represent auxin flow
from the shoot as auxin being produced in the first cell in the array. We
distinguish the first cell by requiring it to be unbound to the left. 

For the auxin diffusion we can write the following:
\begin{align*}
\mr{Cell}(\ar{r}{i}, \ar{a}{a}), \mr{Cell}(\ar{id}{i}, \ar{a}{a'})
  \xrightarrow{} \mr{Cell}(\ar{a}{a + D(a'-a)}), \mr{Cell}(\ar{a}{a' + D(a - a')})
\end{align*}

For the active transport of auxin in the cell by PIN proteins we have,
\begin{align*}
& \mr{Cell}(\ar{r}{i}, \ar{a}{a}), \mr{Cell}(\ar{id}{i}, \ar{a}{a'})
                 \xrightarrow{} \\
  &\mr{Cell}(\ar{a}{a - K_0 a PIN(a)}), \mr{Cell}(\ar{a}{a' + K_0
  a PIN(a)})
\end{align*}
Note that instead of modelling $\mr{PIN}$ concentration and its effect on auxin
directly, we do so implicitly through a function that modulates the
amount of auxin being transported. Unlike diffusion, which is bidirectional,
active transport takes place in the direction from shoot to root cap.

The production of the division factor is affected by the difference in auxin
concentration between neighbouring cells,
\begin{align*}
\mr{Cell}(\ar{id}{i}, \ar{a}{a'}), \mr{Cell}(\ar{l}{i}, \ar{a}{a}, \ar{d}{d})
  \xrightarrow{} \mr{Cell}(), \mr{Cell}(\ar{d}{d+\sigma(a'-a)})
\end{align*}

For the growth of the cell and degradation of the concentrations of the division
factor, $\mr{d}$, and auxin concentration, $\mr{a}$, we write:
\begin{align*}
\mr{Cell}(\ar{s}{s}, \ar{a}{a}, \ar{d}{d}) \xrightarrow{} \mr{Cell}(\ar{a}{a (1 - d +
  \frac{v(s)}{s})}, d=d (1 - k(a) + \frac{v(s)}{s}), \ar{s}{v(s)})
\end{align*}
The degradation of the division factor, again, depends on the auxin
concentration in the cell.

For the cell cycle we need to switch the mode of the cell, $\mr{m}$, from
growing to idle.
\begin{align*}
  \mr{Cell}(\ar{m}{1}, \ar{s}{s}) \xrightarrow{\rho(s)} \mr{Cell}(\ar{m}{2})
  \end{align*}
  with
  \begin{equation*}
\rho(s) = \frac{1}{1 + e^{- \frac{s-s_{\mr{min}}}{\mathbf{time}}}}
\end{equation*}

For cell division, we create a new cell on the right of the dividing cell and
split the concentrations of auxin and the dividing factor between the two cells:
\begin{align*}
  &\mr{Cell}(\ar{id}{i}, \ar{r}{k}, \ar{a}{a}, \ar{d}{d}, \ar{s}{s}, \ar{m}{2}),
  \mr{Cell}(\ar{id}{k}, \ar{l}{i}) \xrightarrow{\rho(d)} \\
  & \mr{Cell}(\ar{id}{i}, \ar{id}{nc}, \ar{a}{a/2}, \ar{d}{d/2}, \ar{s}{s/2},
    m=1), \\
  &  \mr{Cell}(\ar{id}{nc}, \ar{l}{i}, \ar{r}{k}, \ar{a}{a/2}, \ar{d}{d/2}, \ar{s}{s/2},
    m=1), \\
  &  \mr{Cell}(\ar{id}{k}, \ar{l}{nc})
\end{align*}
The cell divides only in the idle phase and the division sets the the mode back
to growing while the rate of division depends on the concentration of the
division factor. The dividing cell gets pushed to the left, keeping its id and
changing its right-neighbour identifier to the identifier of the new cell. The
new cell gets a fresh identifier from our cell counter observable $nc$ and a
right-neighbour identifier from the old neighbour of its mother cell
(Figure~\ref{fig:rootDevChromar}).

Finally, cells die at the root cap:
\begin{align*}
\mr{Cell}(\ar{r}{\epsilon}) \xrightarrow{\rho_{\mr{death}}} \emptyset
\end{align*}
We distinguish the end of the root by requiring the cell to be unbound on the
right.

Finally, the growth of the shoot follows an abstract functional form:
\begin{align*}
\mr{Shoot}(\ar{s}{s}) \xrightarrow{} \mr{Shoot}(\ar{s}{f(w, \mathbf{time})})
\end{align*}
The growth function is a phenomenological function of time and more
interestingly water uptake. Assuming that the water uptake depends on the root
size, we model it as follows:
\begin{align*}
& rs  = \mathbf{select} \; \mathrm{Cell}(\ar{s}{s}) \mathbf{;} \; \mathbf{aggregate} \;
                                                              (total:
                                                              \mathrm{real}.\,
                                                              total + s) \; 0 \\
& w = f(\mathbf{time}) \; \mathbf{when} \; rs > s_0 \; \mathbf{else} \;
f(\mathbf{time}) \cdot d                                       
\end{align*}
where $rs$ is an observable abstracting the size of the entire root from the
size of the constituent cells and $w$ is a fluent that uses that size to compute
the water uptake. The bigger the root the more water the plant takes. We use the
fluent here to model a process that while relevant for our process we do not
wish to model in detail and mechanistically via rules.

\section{Comparison to related work}
Having seen the properties of Chromar in action in this chapter and of other
languages in the same design space in the previous chapter, we can compare their
features. Here I will discuss properties relating to general usability features
that can be understood without the formal definitions. Comparison and discussion
related to more technical parts is done at the point where these are introduced
(see next Chapters for formal definition of Chromar and its embedding in
Haskell).

Table~\ref{tab:comp} shows a summary of the discussion points and a comparison
of the main features of Chromar and the languages from the previous chapter.

\begin{center}
  \resizebox {0.95\textwidth }{!}{%
  \begin{tabular}{p{0.13\linewidth}p{0.13\linewidth}p{0.13\linewidth}p{0.13\linewidth}p{0.13\linewidth}p{0.13\linewidth}p{0.13\linewidth}}
    \toprule
    & PN & SMMR & CPN & DG & Simile & Chromar\\
    \midrule
Objects \newline (dynamics) & Yes \newline (rule) & Yes \newline (rule) & Yes
                                                                          \newline
                                                                          (rule)
                      & Yes \newline(rule) & Yes \newline (population) & Yes \newline (rule)\\
    \addlinespace[0.2cm]
Relations \newline (explicit) & No & Yes \newline (hierarchy) & No \newline
                                                                (implicit via
                                                                attrs) & No \newline (implicit via attrs) & No \newline (implicit via attrs) & No \newline (implicit via attrs)\\
    \addlinespace[0.2cm]
    Properties \newline (dynamics) & No & No & Yes \newline
(transitions) & Yes \newline (transitions or ODEs) & Yes \newline (ODEs) & Yes \newline
(transitions) \\
    \addlinespace[0.2cm]
    State \newline abstraction & No & No & No & No & Yes & Yes\\
    \addlinespace[0.2cm]
    Time & No & No & No & Yes \newline (via ODEs) & Yes & Yes\\
    Interpretation & CTMC& CTMC& CTMC& CTMC + \newline others & - & CTMC\\
    \bottomrule
    \label{tab:comp}
  \end{tabular}}
\end{center}

\subsection{Declarative modelling}
A distinction exists in programming languages between imperative and declarative
styles. In the imperative style a computer program consists of a series of
instruction of how the described process should be computer. On the other hand
in more declarative languages the programmer defines what the program should do
but not how that should be done. Mainstream programming languages follow the
imperative style, which has prevailed, perhaps since computer hardware is
inherently imperative \citep{backus2007can}.

A similar distinction appears in modelling where some models use a declarative
style while others are more imperative mixing the what with the how. The usual
mathematical languages used in the description of physical processes, like
differential equations, for example, are declarative statements about how
functions of time (and/or space) on the described quantities relate to the rates
of change of these functions. On the other hand, as we have seen, many models
are represented as programs in imperative languages, which means that the
represented biological knowledge is coupled with the how the process should be
simulated along with other statements relating to particular choices for the
representation of the state of the system (Models as programs
\sct{pragmaticApproaches}). While libraries or frameworks provide ways to
separate to an extent the biology from the implementation of the simulation (see
\sct{simFrameworks}), this is not guaranteed.

If we accept the view that models should serve as both tools for simulation as
well as tools for thought for our understanding then procedural representations
are not adequate. Chromar follows the declarative style. Rules correspond to
biological events and the user declares how the state changes during the
event. The simulation need not be defined as the interpretation as a CTMC gives
that without any extra effort. Here I gave a particular implementation of the
language as an embedded language inside Haskell using the Haskell expressions as
an expression language. While Haskell follows the declarative style, certain
properties of the biology might still be obscured depending on how much of the
model lives in the rules and how much in the expressions implementing changes to
the agent attributes. Different choices can easily be made though, for example
we could restrict the expressions to small set of mathematical expressions
\citep[perhaps a subset of MathML][]{ausbrooks2003mathematical} if the increased
power of a general-purpose programming language is not required.

Another common advantage of the declarative style over a more procedural style
is model composability. This exists in Chromar too and it is crucial for the
more comprehensive multi-models that we are interested in in this work. In
Chromar composing two models is just a matter of concatenating their rule
sets. An example of this is the multi-model lifecycle model of Arabidopsis I
present in Chapter~\ref{chp:fms}. This is not as straightforward in procedural
model implementations where the whole simulation and representation would have
to be altered.


\subsection{Basic Chromar}
The idea of extending simple objects with fields to represent some of their
attributes has been used before, as we have seen, for example in Coloured Petri
Nets \citep{jensen_coloured_1987}, in a rule-based setting, in CSMMR
\citep{oury_coloured_2011} and Dynamical Grammars \citep{mjolsness2006stochastic},
and in parametric L-systems. Our notation is inspired by these.

The correspondence of CPNs to basic Chromar is straightforward, colour sets are
our agent types (records with named attributes), tokens are our agents, and
transitions are our rules. CPN transitions also have predicates that are the
same as our conditions. One can think of it as a minimal rule-based (and
textual) version of stochastic coloured petri nets, where the richer types are
first class and not merely a means of translation to a non-coloured version; one
can also think of it as a simpler version of CSMMR with only colours left. While
graphical notations are intuitive for smaller models, we have found that for
larger models they become hard to read whereas text-based approaches like our
language produce much more readable representations.

Like the development of other languages we have seen in the previous chapter,
basic Chromar provides an extension to the representation of reactions, where
the simple named species are upgraded to agents employing rich types, namely
records with named typed attributes. Writing rules using these richer types
yields a more compact representation than one would get by writing reactions on
the simpler types in the traditional reaction setting. Moreover it sometimes
helps with writing systems that are difficult to write otherwise as it permits
the dynamic creation of agents and therefore their attributes. These are the
limitations we noted for the simple reactions in PNs.

Attributes can be used to encode relations like the `next-to' relation in the
root development example.  However, whenever we use them to encode binding we
would probably be better off using a language that represents these directly
such as Kappa~\citep{danos_rule-based_2008} or
BioNetGen~\citep{blinov_bionetgen:_2004}. In general Chromar rule left-hand
sides are simple and can only select using types, which means the only relations
we can encode directly are products of types (we can think of types as sets
containing all the agents of that type). This is often unproblematic, for
example in our Plant system (\sct{plantDev}) \textit{all} $\mathrm{Leaf}$
agents interact with \textit{all} $\mathrm{Cell}$ (only one in this case) agents
so writing the left-hand at the type level works because the rules are then
applicable to exactly the pairs of agents we want,
$\{ (\mathrm{Leaf}_1, \mathrm{Cell}), (\mathrm{Leaf}_2, \mathrm{Cell}), \dots
\}$.

In other cases though, the relation we want is some subset of the product of the
types. For example in our array of cells example the diffusion rule is not
applicable to all pairs of $\mathrm{Cell}$s so writing
$\mathrm{Cell}(\dots), \mathrm{Cell}(\dots)$ on the left-hand side gives us more
pairs than we want.  In that case we had to encode the relation through
identifiers and the next-to relation pairs were stored inside our agents. The
selection was then restricted using attribute conditional expressions. While it
works in this simple example, it might become problematic in large models when
multiple relations are present, for example.

However, note that in the dividing cell and diffusion model of the previous
section we use colours in other ways that cannot be easily represented as
binding, or any relation offered by other object-only languages. In particular,
the division of the contents of a cell would be hard to express in any of the
object-only languages.


\subsection{Extended Chromar}
Our use of database inspired operations for the observables is also new and we
have found it very useful in model building. The declarative nature of our
multiset query primitives makes the definition of observables very
intuitive. Similar database-inspired query operations on top of collections are
used in LINQ~\citep{budiu_compiler_2013} although the collections are usually taken
to be lists not multisets. Buneman's comprehension
syntax~\citep{buneman_comprehension_1994}, a collection query language similar to
practical database query languages, considers other types of collections,
including multisets. These are, however, not used in the modelling world.

Observables could be developed further and made into first-class entities in
the language, for example by making them attributes of types. For instance in
our plant growth example we would keep our initial $\mathrm{Ros}$ type and use
our observable primitives to define its $n$ attribute instead of defining the
observables outside our agents. First-class observables would work particularly
well with an extension for a native representation of levels (the `nested-in'
relation we noted earlier), in which case the idea of the agent attributes at a
higher level being observables of agents at a lower-level would be both
intuitive and powerful.

Fluents allow the convenient expression of changing values, which is
important as models become more comprehensive and start reaching outside their
domain, for example traditional plant biology models reaching out to the outside
world and to crop science.

Fluents can be expressed, albeit in the ODE formulation, in Dynamical
grammars. Simulators of some of the languages in the overview in the previous
chapter also allow observables for result reporting. However, the lifting of
fluents and observables to first-class entities and their combination with
expression coming (potentially) from a general purpose programming language is
new and makes for a very powerful extension to the language.


\subsection{Haskell}
Our embedding in a programming language increases expressive power, and fits
with the availability of rich types.  There have been other languages that,
while not giving the full power of a programming language, still allow for
complex expressions to appear, e.g., inside rate expressions. For example in
React(C) \citep{john_biochemical_2011} rate expressions can be build from a subset
of a functional programming language with a reflection option to get access to
the full state of the system. This allows conditions to be encoded inside rates
by setting the rates to zero if they are not met. Simulators for other widely
used languages like KaSim (the simulator for Kappa~\citep{danos2008}) and
BioNetGen~\citep{blinov_bionetgen:_2004} also allow more sophisticated rate
expressions than traditional mass-action kinetic rates.

Our embedding in Haskell lifts some of the constraints of modelling languages
and, we think, gets the best of both worlds: it naturally and succinctly
captures some elements in our domain of interest but, at the same time, when
greater expressive power is needed we can turn to the programming language. This
increase in expressive power might come at the expense of the ability to do
general analysis of models since we cannot say much about what is happening in
the Haskell \texttt{exprs} inside the rules. There seems to be a trend, though,
in the direction of mixing domain-specificity and general purpose programming
languages, for example \citet{pedersen_high-level_2015} allow embedded F\#
scripts inside LBS-$\kappa$, and in PySB \citep{lopez_programming_2013} Kappa
models are defined inside Python. Embedding a domain-specific language inside a
programming language, as we did, is in some cases better than doing the opposite
-- embedding a programming language inside a domain-specific language -- because
we have fewer constraints on our definitions and more generally full access to
the language for structuring our model definitions. We also note, as we have
seen, that there is a hybrid approach that allows mixing the graphical
definition of CPNs with programming language constructs (ML
language)~\citep{jensen_coloured_1987}.

\section{Conclusion}
In conclusion, extended Chromar and its concrete realisation in Haskell give a
highly expressive language that is particularly good for the description of
models with a simple structure, such as the Plant example in
\sct{plantDev} and the root development example in
\sct{rootDev}. The main ideas of Chromar --- two levels of dynamics
(attribute and agent levels), a flexible system of observation, and a
combination of domain-specific and general programming languages --- should
carry to other frameworks or languages, for example ones with connection and
nesting and perhaps other complex structure.



\chapter{Chromar: formal definition}
\label{chp:chromarForm}
In the previous chapter I gave some examples illustrating how Chromar works. In
this chapter I formally introduce the syntax and semantics of Chromar. Formal
definition of programming languages is a common practice in the computer science
world. It has the many of the same advantages over informal, natural language
descriptions as we have noted earlier regarding formal against informal
descriptions of biological models. The formal description acts as a tool for the
language designer to understand and design the language by making design choices
explicit and understanding the relation between different features of the
language. This usually also then makes the implementation process easier while
the language is not tied to that implementation.

Most of the work in this chapter appears in the articles describing Chromar
\citep{honorato-zimmer_chromar_2017, honorato-zimmer_chromar_2018}.


\section{Abstract syntax of basic Chromar}
\label{sec:syntax}
I next define the abstract syntax of basic Chromar.

Regarding notation, we write $\s{a}$ for a possibly empty sequence of objects,
whose typical elements are given by mathematical expressions $a$. We write
$\s{a}.i$ for the $i$th element of the sequence and $|\s{a}|$ for its length,
and write the empty sequence as $\varepsilon$ and use commas for the composition
of sequences.  We assume we have a countable set of names $N$ ranged over by
$n_a, n_f, id$ where $n_a$ will be used for agent names, $n_f$ for attribute
names, and $id$ for variables.

For generality we assume that basic Chromar is parameterised by a set $E$ of
\emph{expressions} (an expression language) ranged over by $e, e_r, e_c$, and
including variables, together with a set of \emph{base types}
$B \supseteq \{\mathrm{bool}, \mathrm{real} \}$ ranged over by $b$; base types are used to
specify agent attributes (we use $e_c$ to indicate condition expressions and
$e_r$ to indicate rate expressions).
%
We further have \emph{agent types} $\s{n_f:b}$ that are sequences of distinct
typed attribute names.

We may say that basic Chromar is \emph{abstract} in that no choice is made of an
expression language and its types; the same holds of extended Chromar
(\sct{extSyntax}).  In contrast, in a \emph{concrete} realisation of the
language a particular choice is made.  For example, the Haskell embedding of
extended Chromar (\sct{extChromar}) provides such a concrete realisation, where
the expressions and types are the Haskell expressions and types.

A \emph{base type environment} $\G = \s{id:b}$ is a sequence of variable type
bindings, and an \emph{agent type environment} $\D = \s{id:at}$ is a sequence of
agent type bindings; we require that the variables in environments are distinct.
We write $\G[\G']$ for the new base type environment produced by extending or
overriding $\G$ with new type bindings $\G'$, and define $\D[\D']$ similarly.

Turning to typing, as regards the expression language we assume we have typing
axioms of the form $\G \vdash e : b$ that assign unique base types to expressions in
some base type environments, with the following being the only axiom for
variables:
 \[\G \vdash id:b  \quad (id:b \, \in \, \G)\]

 The syntax and typing rules for the rest of basic Chromar are given in
 Figure~\ref{fig:syntax}. I next go through the various syntactic constructs
 and explain their syntax and typing rules:

\begin{figure}[!htbp]
\centering
\resizebox{0.9\linewidth}{!}{%
\begin{tabularx}{\textwidth}{llr|c} 
\toprule
\addlinespace[0.25cm]
&\textsf{\emph{Syntax}} & & \textsf{\emph{Typing rules}} \\ 
& & & \\
$a_d ::=$ & $\mathbf{agent} \; n_a(\s{n_f:b})$ &\textsf{agent decl}
%
& $\inferrule*[Right=(T-Intro)]{}{%\G 
\vdash \mathbf{agent} \; n_a(\s{n_f :b}) : (n_a : \s{n_f : b})}$\\
& & &  (all $n_f$s distinct) \\
& & & \\
$a_l ::=$ & $n_a(\s{n_f=id})$ & \textsf{left agent} &
%
$\inferrule*[Right=(T-L-agent)]{}{\D \vdash n_a(\s{n_f=id}) : (\s{id:b})} $\\
& & \textsf{expr}& ($n_a: (\s{n_f:b}) \in  \D$, all $id$s distinct) \\
& & & \\
$a_r::=$ & $n_a(\s{n_f=e})$ &\textsf{right agent} & 
$\inferrule*[Right=(T-R-agent)]{\G \vdash \s{e :b}}
{\G \,|\, \D \vdash n_a(\s{n_f=e})}$ \\
& & \textsf{expr}& ($ n_a : (\s{n_f:b}) \in \D$)\\
& & &\\
$r ::=$& $\s{a_l} \xrightarrow{e_r} \s{a_r} \; [e_c]$ & \textsf{rule expr} &  
%
$\inferrule*[Right=(T-Rule)]{\D \vdash \s{a_l} : \G' \\
 \G[\G']|\D \vdash \s{a_r} \\\\ \G[\G'] \vdash e_r : \mathrm{real}  \\ \G[\G'] \vdash e_c : \mathrm{bool} }{\G \,|\,\D \vdash \s{a_l} \xrightarrow{e_r} \s{a_r} \; [e_c]}$\\
& & & \\
$m ::=$ & $ \s{a_d};  \; \mathbf{init} \, (\s{a_r});\; \s{r}$ & \textsf{model def} & $\inferrule*[Right=(T-Model)]{ \vdash \s{a_d}: \D \\\\ \G \,|\, \D \vdash \s{a_r} \\ \G \,|\, \D \vdash \s{r}}{\G \vdash \s{a_d}; \; \mathbf{init}\, (\s{a_r}); \;   \s{r}}$\\
& & & \\
\addlinespace[0.25cm]
\bottomrule
\end{tabularx}}
\caption{Abstract syntax of Chromar with corresponding typing rules for each syntactic construct}
\label{fig:syntax}
\end{figure}

\subsubsection*{Agent declarations}
Agent declarations $a_d$ define the sequence of typed attributes that all agents
with some name $n_a$ should have (\textsf{agent decl},
Figure~\ref{fig:syntax}). Using the model of the previous section as an example
we had a $\mathrm{Leaf}$ type with attributes $\mathrm{mass}$ of type
$\mathrm{real}$ and $\mathrm{age}$ of type $\mathrm{int}$, that could be
declared with:
$$\mathbf{agent} \; \mathrm{Leaf}(\mathrm{mass}:\mathrm{real}, \mathrm{age}:\mathrm{int})
$$
We use the sequence of typed attributes $\s{n_f:b}$ as the \emph{agent
type}. The typing rule \textsc{T-Intro} produces a new such type binding
associating the agent name to the agent type.

The typing  for models uses the judgment $\vdash \s{a_d}: \D$ for sequences of agent declarations.  This conjoins the agent environments given by the individual agent declarations, provided that no agent name is declared twice. It can be given (inductively on the length of the sequence) by the following axiom and rule:
%
\[\vdash \varepsilon:\varepsilon   \qquad \frac{\vdash a_d: (n_a: \s{n_f:b})  \quad \vdash \s{a'_d}: \D}{\vdash a_d \s{a'_d}: (n_a: \s{n_f:b})  \D} \quad (n_a \mbox{ not bound by } \D)\]

\subsubsection*{Agent expressions}
Agent expressions have a name and a finite sequence of \emph{attribute expressions} of the form $n = e$. (\textsf{agent expr}s, Figure~\ref{fig:syntax}). Agent expressions have a dual role: they are used in the left-hand side of rules as a way of binding values to variables and on the right-hand side of rules as a way of expressing the change in the left-hand side agents. We therefore have two distinct syntactic classes, one for each use:
\begin{itemize}
\item[-] left  agent expressions, $a_l$, where all attribute expressions are of the form $n_f = id$, 
 binding the value of an $n_f$ to a variable $id$, and
\item[-] right agent expressions, $a_r$, where all attribute expressions are of the form $n_f = e$ where  the expression $e$ gives the new value of the attribute $n_f$.
%change in the attribute values.
\end{itemize}



The typing rule \textsc{T-L-agent} introduces new type bindings in the environment for the variables in the left agent expressions according to the agent type. Specifically if we have an agent type in the environment $n_a : (n_1: b_1, \dots, n_k: b_k)$ and a left agent expression $n_a(n_1 = id_1, \dots, n_k = id_k)$, each attribute expression $n_i = id_i$ of the %interface of 
the left agent expression produces a type binding, $id_i:b_i$ according to the agent type; this results  in bindings for the entire expression $(id_1: b_1, \dots, id_k: b_k)$. In Figure~\ref{fig:syntax} we use a shorthand notation where we show the typing for a typical element of each sequence in the rule and then underline for the natural extension to the entire sequence. 

The typing rule \textsc{T-R-agent} makes sure that all expressions assigned to attribute names have the correct type according to the agent type. If we have an agent type binding 
$n_a: (n_1:b_1, \dots, n_k:b_k)$, each attribute expression $n_i = e_i$ of a right agent expression with name $n_a$ should have type $b_i$ in a base type environment $\G$. Again we use the shorthand sequence notation, indicating the rule for typical elements of each sequence and then underlining for the extension to the entire sequence. 

A further restriction implied by the typing rules is that agent expressions should 
%define their interface fully, which means that they should 
define all the attributes indicated by their type.


\subsubsection*{Rule expressions}
Rules, $r$, as we have seen from the examples, have a left-hand side (lhs) consisting of a sequence of left agent expressions, $a_l$, a right-hand side (rhs) consisting of a sequence of right agent expressions, $a_r$, a rate expression $e_r$, and a condition expression $e_c$ (\textsf{rule expr}, Figure~\ref{fig:syntax}). 

The typing rule \textsc{T-Rule} first makes sure via the judgment $\D \vdash \s{a_l} : \G'$ that each left agent expression is well-formed and collects all the basic environments these left agent expressions produce into the environment $\G'$. The judgement can be given (inductively on the length of the sequence) by the following axiom and rule:
%
\[\D \vdash \varepsilon:\varepsilon   \qquad \frac{\D  \vdash a_l : \G  \quad \D  \vdash \s{a'_l}: \G'}{\D  \vdash a_l \s{a'_l}: \G\G'} \quad (\mbox{$\G$ and $\G'$ do not overlap} )\]
%
where two environments are said to \emph{overlap} if there is a variable they both bind.

%agent expressions are well typed (i.e., that agents bind variables to all their attributes, and to no others). Each left agent expression produces a basic type 
%environment bindings for its variables (and no others), and in %GDPwe write
 %$\D \vdash \s{a_l} : \G'$, it is intended that the basic type environment $\G'$ is the concatenation of the environments produced by each element of the 
 %sequence and that no two of these environments bind the same variable (analogously to the case of agent declaration sequences). 
 
 Then, in the environment extended with environment produced by the lhs,  it is checked that the rest of the rule is well formed; in particular, in $ \G[\G']|\D \vdash \s{a_r} $ it is intended that $ \G[\G']|\D \vdash a_r $ holds  for each of the $a_r$.   The extended type environment is needed since the variables of the rule lhs may appear in the rhs, rate, and condition expressions.

\subsubsection*{Model definitions}
A full basic Chromar model $m$ is just a sequence of definitions. We have a sequence of agent declarations, $\s{\mathrm{a_d}}$ followed by the initial state definition, which is just a sequence of right agent expressions, and finally a sequence of rule expressions (\textsf{model def}, Figure~\ref{fig:syntax}). The typing rule \textsc{T-Model} constructs new agent type bindings for all the agent declarations in the model. 
%where $\D$ is obtained by extending the empty binding by the bindings %the concatenation of agent type bindings 
%produced by each agent declaration in the sequence (in more detail, if $\s{a_d} = (a^1_d, \ldots , a^k_d)$, for some $k \geq 0$, and $\vdash a^i_d:  \D^i$, %then $\s{a_d} \vdash  \D$ where $ \D = \varepsilon[ \D^1] \ldots [ \D^k]$).
%
Then in a type environment extended with the defined agent types, the rule checks the initial state definition and the sequence of rules.

\subsection{Syntactic extensions}

My concrete realisation of Chromar provides two syntactic extensions to the above language which we explain informally below. Chromar with these extensions can be translated to Chromar without them; we omit a formal treatment of the translation.


\subsubsection*{Missing attributes} \label{sec:syntaxExt} We can relax the constraint of having to specify the all the attributes %full interfaces 
of agents in rules by establishing %defining 
a correspondence between left and right-hand sides of rules as follows:
%
\begin{itemize}
\item First, bindings to fresh variable are added to left agents for any missing attributes (these attributes are known from its agent type).
%
%The interface of any agent expressions on the left-hand side of rules with incomplete interfaces is %filled with bindings to fresh names ($id$s) for the attribute names missing in the expression %compared to the attribute names in its agent type. 
%
For example, given an agent type $\mathrm{A}(\mathrm{a}: \mathrm{int}, \mathrm{b} : \mathrm{int})$, the incomplete left agent expression $\mathrm{A}(\mathrm{a}=x)$ becomes $\mathrm{A}(\mathrm{a}=x, \mathrm{b}= y)$, where $y$ is a fresh variable %identifier 
(i.e., one not occurring anywhere else in the lhs). 

\item Then, for every $\s{a_l}.i$ in the left-hand side we establish a
  positional correspondence with an $\s{a_r}.j$ in the right-hand side, if it
  exists, and complete the corresponding right agent expressions with the same
  fresh name bindings, if needed.  Any right agent expressions that do not have
  a corresponding left agent expression need to be fully defined. Continuing
  with our example, the rule
  $\mathrm{A}(\mathrm{a}=x) \rightarrow \mathrm{A}(\mathrm{a}=x+1)$ is completed
  to
  $\mathrm{A}(\mathrm{a}=x, \mathrm{b}= y) \rightarrow \mathrm{A}(\mathrm{a}=x +
  1, \mathrm{b}= y)$.
\end{itemize}
\subsubsection*{Attribute conditional expressions} 
Condition expressions restrict the applicability of rules by establishing a
relation between agents. These relations are often functional dependencies. In
such cases, rules can be made easier to read and write by using expressions ---
not just variables --- inside left agent expressions.

For example, consider an agent type $\mathrm{A}(\mathrm{a}:\mathrm{int})$. If we
want a rule $\mathrm{A}(\dots), \mathrm{A}(\dots) \rightarrow \dots$ to be applicable to only a subset
of the $(\mathrm{A}(\dots), \mathrm{A}(\dots))$ pairs in the state, we might use the
condition expression $y = f(x)$ to encode a functional dependency of $y$ on $x$,
as in, say, a rule of the form
%
\[\mathrm{A}(\mathrm{a}=x), \mathrm{A}(\mathrm{a}=y) \rightarrow \dots \; [y = f(x)]\]
%
However, instead of binding to $y$ in the second left agent expression and then
putting the condition $y = f(x)$ at the end, we can write the expression $f(x)$
directly in the left agent expression, obtaining the rule
%
\[\mathrm{A}(\mathrm{a}=x), \mathrm{A}(\mathrm{a}=f(x)) \rightarrow \dots\]


This is just syntactic sugar as any rule left-hand side with such expressions
can be rewritten as a rule with a binding-only left-hand side.  This is done as
follows: any non-variable expression in a left agent expression of the rule is
rewritten as a binding to a fresh variable, and a corresponding equality
condition on this variable is added to the conditional expression of the rule.

\section{Semantics of basic Chromar}\label{sec:semantics}
First, regarding notation, we write
$\{ a_1 \mapsto b_1, a_2 \mapsto b_2, \dots \}$ for finite partial maps, from
$A$ to $B$, with $a_1, a_2, \dots$ distinct elements of $A$ and
$b_1, b_2, \dots$ in $B$. For two such functions $\sigma$ and $\sigma'$ we write
$\sigma[\sigma']$ for the partial function produced by using $\sigma'$ to extend or overwrite
the values produced by $\sigma$.
%
We write $\m{a_1, \dots, a_n}$ for the (finite) multiset of elements of a set
$A$ whose elements, counting repetitions, are $a_i, \ldots, a_n$, and take such
multisets to be functions from $A$ to $\mathbb{N}$ counting multiplicities; note
that such a function $m$ is zero for all but finitely many arguments, and that
$a$ is an element of $m$, written $a \in m$, exactly when $m(a) \neq 0$. We write
$\ms(\s{a})$ for the multiset corresponding to the sequence $\s{a}$. We write
$m \preceq m'$ for the submultiset relation; this takes multiplicities into account
and is defined pointwise. Finally, we write $\MS[A]$ for the set of all finite
multisets over a set $A$ and we write $\SQ[A]$ for the set of all finite
sequences over $A$.


For each basic type $b \in B$ we assume we have an associated set of values
$\V_b$, including $\V_{\mathrm{bool}} = \{\mytt, \myff\} $ for $\mathrm{bool}$
and $\V_{\mathrm{real}} = \mathbb{R}$ for $\mathrm{real}$. We have
$\V=\bigcup_{b \in B} \V_b$ the union of these sets of values ranged over by
$v$. A \emph{value environment} $\sigma = id_1: v_1, \dots, id_n: v_n$ with
$v_1, \dots, v_n \in \V$ is a sequence of variable value bindings, where the
variables are required to be distinct.  We will sometimes treat value
environments as functions with finite domain; for example to get the value of
$id_i$ in $\sigma$ we may write $\sigma(id_i)$. We say that
$\sigma= id_1:v_1, \dots, id_n:v_n$ is a $\G$-environment for a base type environment
$\G= id_1:b_1, \dots, id_n:b_n$ if each $id_i:v_i$ in $\sigma$ has an entry for that
$id_i$ in $\G$ of the correct type, i.e., $\sigma(id_i) \in \V_{b_i}$.

Regarding the semantics of expressions, we assume that for each $\G$-environment
$\sigma$, base type $b$, and expression $e$ such that $\G\vdash e:b$, the expression has an
evaluation $\den{e}_{\G}(\sigma) \in \V_b$ (below, we often omit the environment
subscripts on evaluation functions). In particular, for typings $\G\vdash id:b$ of
variables $id$, we assume that $\den{id}_{\G}(\sigma) = \sigma(id)$.

We next give the semantics of the other syntactic constructs of basic Chromar:
agent expressions, rules, and models.

\subsubsection*{Agents}
\textit{Agents} are modelled as structures with a name and a finite partial map
from attribute names to base values, $ (n_a, \{ n_f \mapsto v \})$. We write
$\mathrm{Av}$ for the set of such agents, ranged over by $av$. We define the set
of \emph{states}, ranged over by $s$, to be $\MS[\mathrm{Av}]$, the set of all
finite multisets of agents.
%
When writing agents in examples, we use left agent expressions, but with
constants in place of variables. %identifiers.
So for example, we write $ \mathrm{Leaf}(\mathrm{age}=1, \mathrm{mass}=10)$ for
the agent $(\mathrm{Leaf},\{\mathrm{age} \mapsto 1, \mathrm{mass} \mapsto 10\}) $.

We evaluate sequences of left and right agent expressions as states.  For left
agent expressions, %as states, %multisets of agents,
suppose we have a typing $\D \vdash \s{n_a(\s{n_f=id})} : \G$ of a sequence of left
agent expressions. Then $\G$ is determined by $\D$, and gives basic types for
the variables in the various left agent expressions (and no more). So, for any
$\G$-environment $\sigma$ we may define the evaluation
$\den{\s{n_a(\s{n_f = id})}}_{\D}(\sigma) \in \MS[\mathrm{Av}]$ of the sequence
as follows:
\begin{align*}
\den{\s{n_a(\s{n_f = id})}}_{\D}(\sigma) = \ms(\s{n_a(\s{n_f =  \sigma(id)})} 
\end{align*}

For right agent expressions, suppose that we have a typing
$\G \,|\, \D \vdash \s{n_a(\s{n_f = e})}$ of a sequence of right agent expressions. Then
for any $\G$-environment $\sigma$ we define the evaluation
$\den{\s{n_a(\s{n_f = e})}}_{\G,\D}(\sigma) \in \MS[\mathrm{Av}]$ of the sequence as
follows:
\begin{align*}
\den{\s{n_a(\s{n_f = e})}}_{\G,\D}(\sigma) = \ms(\s{n_a(\s{n_f = \den{e}_{\G}(\sigma)})}
\end{align*}

\subsubsection*{Rules}
\label{sec:ruleSem}
Rules induce (stochastic) reactions, and I discuss those first. A
\emph{reaction} is a structure $\rho \, = \, l \xrightarrow{k} r$ with $l, r$
states and $k$ a positive real number. We will assume that these names also act
as accessors for the parts of the reaction structure, so for a reaction $\rho$,
$l(\rho)$ returns its left-hand side and so on. We can apply the reaction $\rho$ to a
state $s$ if $l(\rho) \preceq s$, when we obtain a new state
$s - l(\rho) + r(\rho) $, which we write as $\rho \bullet s$.

A \emph{stochastic matrix} (aka \emph{rate function}) over a set of states $S$
is a function $Q: S \times S \rightarrow \mathbb{R}_+$ with $Q$ zero on all but finitely many
entries of any row.  Note that the sum of two stochastic matrices is also a
stochastic matrix, as is the zero matrix.  We will use stochastic matrices over
$\MS[\mathrm{Av}]$ to encode the rate at which reactions occur in different
states.

The rate at which a reaction $\rho$ occurs in a state $s$ depends on the number of
ways in which $l(\rho)$ occurs in $s$.  For any set $A$, we define the
\emph{multiplicity} of a submultiset $m \in \MS[A]$ of another multiset
$m' \in \MS[A]$ as the number of distinct times $m$ appears as a submultiset of
$m'$, defined as:
%
$$
\mu(m, m') = \prod_{a \in m}  \binom{m'(a)}{m(a)}
$$
Consider for example multisets $m = \m{a, a, b}$ and $m'= \m{a_1, a_2, a_3, b}$
where we label the $a$'s in $m'$ to distinguish them.  The multiplicity of $m$
in $m'$ is three as $m$ appears as a submultiset of $m'$ in three distinct ways:
$$
\mu(m, m') = |\{ \m{a_1, a_2, b}, \m{a_1, a_3, b}, \m{a_2, a_3, b} \}|$$ With these
preliminaries out of the way, we can define the semantics of rules as stochastic
matrices on $\MS[\mathrm{Av}]$, given suitable environments.  Suppose that we
are given a rule typing
$\G\,|\, \D \vdash \s{a_l} \xrightarrow{e_r} \s{a_r} \; [e_c]$. Then we further
have $\D \vdash \s{a_l} : \G'$ for a unique $\G'$.  We also have
$\overline{\G} \vdash e_r : \mathrm{real}$, $ \overline{\G}|\D \vdash \s{a_r}$, and
$ \overline{\G} \vdash e_c : \mathrm{bool}$, where $\overline{\G} = \G[\G']$.

In order to define the stochastic matrix over $\MS[\mathrm{Av}]$ associated to
the rule $ \s{a_l} \xrightarrow{e_r} \s{a_r} \; [e_c]$, we first need to explain
when it matches a state and to define the reaction it denotes.
%
For the first, we say that a $\G'$-environment $\match$ is a \emph{match} for the
rule in a state $s= \m{av_1, \dots, av_n}$ if $\den{a_l}_{\D}( \match)$ is a
submultiset of $s$. Note that there are only finitely many possible such rule
matches.
%
For the second, suppose we are given a $\G$-environment $\sigma$ and a
$\G'$-environment $\sigma'$, then we set\hspace{-2pt}
%
\footnote{We use Kleene equality in this definition: for any two, possibly undefined, mathematical expressions $e$ and $e'$, $e \simeq e'$ holds if (i) $e$ is defined if, and only if, $e'$ is defined, and (ii) they are equal if they are both defined.}
%
\begin{equation*}
\mathcal{R}\den{\s{a_l} \xrightarrow{e_r} \s{a_r} \, [e_c]}(\sigma, \sigma') \simeq \left\{ 
\begin{array}{ll}
\den{a_l}_{\D}(\sigma') \xrightarrow{\den{e_r}_{\overline{\G}}(\overline{\sigma})} \den{a_r}_{\overline{\G}, \D}(\overline{\sigma})  &  
                                                                                                                  (\overline{\sigma} = \sigma[\sigma'],  \den{e_r}_{\overline{\G}}(\overline{\sigma}) > 0, \\
  & \den{e_c}_{\overline{\G}}(\overline{\sigma}) = \mytt)\\
\text{undefined } & (\text{otherwise})
\end{array}
\right .
\end{equation*}
%
Note that this reaction only exists if (the denotation of) the rule rate is
positive and (the denotation of) the rule condition holds. In case the reaction
exists and $\sigma'$ is a match for the rule in a state $s$, we say that the match is
a \emph{proper} match for the rule in state $s$, given $\sigma$.
%
We can now define the stochastic matrix associated to the rule, given a $\G$-environment $\sigma$:
%
\begin{equation*}
\begin{split}
    \den{
{\s{a_l} \xrightarrow{e_r} \s{a_r} \, [e_c]}}_{\G,\D}(\sigma) (s,s')  =   \sum\{\den{e_r}_{\overline{ \G}}(\overline{\sigma})\cdot \mu(\den{a_l}_{ \D}(\match),s)\mid 
& \match \mbox{ is a proper match for } a_l \mbox{ in } s,\, \\
& \overline{\sigma} = \sigma[\match], \\
& s' = \mathcal{R}\den{{\s{a_l} \xrightarrow{e_r} \s{a_r} \, [e_c]}}_{\G,\D}(\sigma,\match)\bullet s \}
\end{split}
\end{equation*}
For example, consider the left-hand expression
$\mathrm{A}(\mathrm{a}=x), \mathrm{A}(\mathrm{a}=y)$ and state
$\m{ \mathrm{A}(\mathrm{a}=1),\mathrm{A}(\mathrm{a}=2)}$. We have two distinct
matches $\match_1= \{x \mapsto 1, y \mapsto 2\}$ and
$\match_2 = \{x \mapsto 2, y \mapsto 1\}$.
%
Next, consider the rule
%
\[r \;=\; \mathrm{A}(\mathrm{a}=x),\mathrm{A}(\mathrm{a}=y) \xrightarrow{f(x, y)} \mathrm{A}(\mathrm{a}=x+y), \mathrm{A}(\mathrm{a}=y-1) \: [g(x,y)]\]
%
assumed well-typed given the agent type environment $A: (a: \mathrm{int})$ and
empty base type environment.
%
Then, assuming, for example, that $f(x,y)$ denotes 3 in $\match_1$ and the
condition $g(x,y)$ holds there, $\match_1$ yields a reaction from our rule $r$,
namely:
%
\[\mathcal{R}\den{r}(\{\},\match_1)      \;=\;       \mathrm{A}(\mathrm{a}=1), \mathrm{A}(\mathrm{a}=2) \xrightarrow{3} \mathrm{A}(\mathrm{a}=3), \mathrm{A}(\mathrm{a}=1)\]
%
Assuming, further, that the condition does not hold in $\match_2$, for the
stochastic matrix of $r$ we will have:
%
\[\den{r}(\{\})(\m{ \mathrm{A}(\mathrm{a}=1),(\mathrm{a}=2)},\m{ \mathrm{A}(\mathrm{a}=3),\mathrm{A}(\mathrm{a}=1)}) = 3\]
%
If, on the other hand, the condition does hold in $\match_2$ and $f(x,y)$ denotes
3 there, then the stochastic matrix of $r$ will have value 6.  Thus we may count
the same reaction several times when applying a rule.

We should remark that there is an alternative way of counting
multiplicities. Rather than regarding two occurrences of the same agent as
identical and using the binomial coefficient to count up to symmetry, one
instead regards them as distinct entities and uses the falling factorial. The
former approach is used in the standard biochemical literature and in stochastic
multiset rewriting~\citep{anderson_continuous_2011, barbuti_intermediate_2009}; the
latter approach is used in the rule-based kappa system~\cite[Section
4.2.3]{danos_rule-based_2008}.
%
In the case of the usual chemical reactions, the two approaches are the same up
to a multiplicative change in the reaction rate. However this is not the case
for rule-based systems. Consider, for example, a rule with left-hand side
$\mathrm{A}(\mathrm{a}=x),\mathrm{A}(\mathrm{a}=y)$, and the state
$\m{ \mathrm{A}(\mathrm{a} =1), \mathrm{A}(\mathrm{a} =1), \mathrm{A}(\mathrm{a}
  = 2)}$.  The two ways of counting multiplicities of the match
$\{x \mapsto 1,y \mapsto 1\}$, give 1 and 2, whereas for the match
$\{x \mapsto 1,y \mapsto 2\}$ they both give 2.
 %
Nonetheless, a more complex translation between the two approaches is possible,
assuming available a sufficiently rich stock of conditional expressions for rate
expressions to distinguish the various possible cases.

\subsubsection*{Stochastic semantics of models}
\label{sec:stoch}
A \emph{continuous time Markov chain (CTMC)} is a tuple $(S, Q, I)$ with $S$ a
set of states, $Q: S \times S \rightarrow \mathbb{R}$ a stochastic matrix on
$S$, and $I \in S$ the initial state.  Basic Chromar models are evaluated as CTMCs
with set of states $\MS[\mathrm{Av}]$, given a suitable environment.
%
So suppose we have a model typing 
$\G \vdash \s{a_d}; \; \mathbf{init}\, (\s{a_r}); \; \s{r}$. Then, for a unique $\D$, we have $\ \vdash \s{a_d}: \D$, $\G \,|\, \D \vdash \s{a_r}$, and $\G \,|\, \D \vdash \s{r}$. 
%
Then, for any $ \G$-environment $\sigma$,  the CTMC associated to the model is:
%
\[\den{ \s{a_d}; \; \mathbf{init}\, (\s{a_r}); \; \s{r}}_{\G,\D}(\sigma) = (\MS[\mathrm{Av}], \sum_{r \in \s{r}} \den{r}_{\G,\D}(\sigma), \den{\s{a_r}}_{\G,\D}(\sigma))\]
%
We remark that, in the implementation of Chromar, the $ \G$-environment needed
for the semantics of a model, is supplied by the Haskell context in which the
model is defined. We also remark that, with this definition, the same reaction
may be counted more than once as it can occur in two different ways, either when
applying a given rule (a possibility noted above), or when applying two
different rules.

\subsection{A simulation algorithm}
If we try to expand basic Chromar rules to the equivalent reactions for
simulation, we may obtain infinitely many reactions (unless we constrain the
types of the attributes in our agent types to be finite). However, for a given
state only finitely many of these reactions will apply, so we can still use the
normal Stochastic Simulation Algorithm (SSA) to get sample paths from the CTMC,
without constraining the available attribute types.

Specifically our algorithm is the usual SSA (direct
method~\cite{gillespie1977exact}), but with an extra step (1) that dynamically
creates the reactions based on the current state of the system. We assume a
model and an environment $\sigma$, as above. The algorithm keeps a state $s$ and a
time $t$, starting with state $s_0 = \den{\s{a_r}}_{\G,\D}(\sigma)$ and an initial time
$0$, then iterates the following sequence of steps as many times as desired.
\begin{enumerate}
\item For current state $s$, generate the multiset of all possible reactions for
  every rule:
$$R  = \m{ \mathcal{R}\den{\s{r}.i}(\sigma, m) \; | \; i = 1,\ldots, |\s{r}|, \, m \mbox{ is a proper match of rule $\s{r}.i$ in $s$, given $\sigma$}}$$ 
\item Calculate the total rate
  $k_T = \sum_{\rho \in R} R(\rho)\cdot (\mu(l(\rho), s)\cdot k(\rho))$. Halt if this is zero.
\item Pick the waiting time $t_w$ for the next reaction event from the
  exponential distribution with cumulative distribution function
  $F(t) = 1- e^{-k_T t}$.
\item Pick exactly one of the reactions, choosing reaction $\rho \in R$ with
  probability $\frac{R(\rho)\cdot \mu(l(\rho), s) \cdot k(\rho)}{k_T}$.
\item If reaction $\rho$ is picked then update the state to
  $s' = \rho \bullet s$, and the time to $t + t_w$.
\end{enumerate}

Note that we take account of the multiplicity of reactions here: the same
reaction can occur from two different rules, or from one rule using different
matches.

\section{Extended Chromar} 
\label{sec:extChromar} 
I now extend basic Chromar to include the fluent and observable features that
were introduced informally in Chapter~\ref{chp:chromarEx}. Observables and
fluents extend the given set of (\emph{ordinary}) expressions, $e \in E$, to
give what we call \emph{enriched expressions} $e_r$. These consist of ordinary
expressions, fluents and observables, and any combinations of them. Enriched
expressions give access to the current state and time, and so we take the value
of an enriched expression with type $b$ to be a function from states and times
to $\V_b$, rather than an element of $\V_b$ (in the language of intensional
logic we work with their \emph{intensions} rather than their
\emph{extensions}~\cite{Stan,Fit}).

As enriched expressions allow an arbitrary mixing of ordinary expressions,
fluents, and observables, we might have an observable inside a fluent or a
fluent inside an observable and so on. For example we could write:
\begin{equation*}
f(m) \; \mathbf{when} \; nl > 10 \; \mathbf{else} \; f'(m)
\end{equation*}
where $nl$ is an observable for the number of leaves. This might be used to
introduce the crowding effect on rosette leaves; crowding affects the
assimilation rate as it %affects
reduces the photosynthetically active area.

We begin with the syntax of extended Chromar and then proceed to its semantics.

\subsection{Abstract syntax of extended Chromar}
\label{sec:extSyntax}

The grammar and typing rules are given in Figure~\ref{fig:extSyntax}, and I
discuss them next. The discussion begins with enriched expressions and then
moves on to agents, rules, and models. These are much as in basic Chromar, but
adapted to allow enriched expressions; in particular there is a new syntactic
class of declarations that allows (the intensional values of) enriched
expressions to be used in models.

\subsubsection*{Enriched expressions}
\begin{itemize}

\item[-]%\subsubsection*{Ordinary expressions}
 Any ordinary expression $e \in E$
 is an enriched expression (\textsf{ord-expr}, Figure~\ref{fig:extSyntax}). 
%
\item[-] Any variable in an enriched expression can refer to (the value of)
  another enriched expression using the \textsf{where} expression.

  For example we can write
  $t > 5 \; \mathbf{where} \; t : \mathrm{real} \; \mathbf{is} \; \mathrm{temp}$
  to use the $\mathrm{temp}$ fluent inside, in this case, an ordinary
  expression. %, in this case coming from the original expression language.
  These expressions %$e$
  are typed inside a type environment extended using the type of the enriched
  expression in the where clause (\textsc{T-Where},
  Figure~\ref{fig:extSyntax}). In our examples we generally proceed informally,
  omitting the where clause and instead using the where clause expression
  directly; % inside the language expressions,
  for example we might write $\mathrm{temp} > 5$, although that is not
  officially an enriched expression.

\item[-] %\subsubsection*{Fluents}
  The \textsf{time}, \textsf{condition}, and \textsf{repeat} expressions are
  fluents. They are used to provide ways of expressing common motifs of
  time-dependent behaviour, and appeared in examples given
  above. % which we have seen before in the examples.
  The $\mathbf{time}$ constructor gives access to time. The $\mathbf{when}$
  $\mathbf{else}$ fluent is used to specify conditional behaviour over time,
  where, if a condition is met, we choose one fluent and, if not, we choose a
  different one. The typing rule \textsc{T-Cond} ensures that the conditional is
  of type $\mathrm{bool}$ and that the two alternatives have the same type; the
  type of the whole expression is then the common type of the two
  alternatives. The $\mathbf{repeatEvery}$ fluent is used to define a repeating
  behaviour over time, where the first enriched expression is the period of the
  cycling and the second is the behaviour to be cycled. The \textsc{T-Repeat}
  typing rule ensures that the first such expression has type $\mathrm{real}$
  since it is meant to denote time; the type of the whole expression is then the
  type of the cycling behaviour.

\item[-]
  The last way to construct enriched expressions is through the
  database-inspired observables expression (\textsf{obs},
  Figure~\ref{fig:extSyntax}). The $\mathbf{select}$ part is followed by a left
  agent expression. This is used to `filter out' all the agents that have the
  agent name of the left agent expression from the current state, while
  producing bindings of the left agent variables to the values of the
  corresponding agent fields. For example, to select only $\mathrm{Leaf}$ agents
  from the state, we write
  $\mathbf{select} \; \mathrm{Leaf}(\mathrm{age} = i, \mathrm{mass} = m)$,
  thereby producing a multiset of bindings of $i$ and $m$ to the values of the
  $\mathrm{age}$ and $\mathrm{mass}$ fields of the selected agents, with one
  such binding for each occurrence of a leaf agent in the state multiset.  Since
  multiple agents can have the same attribute values, we need a mutiset of
  bindings rather than just a set of them.

  The $\mathbf{aggregate}$ part contains an enriched expression $e_r$, used to
  obtain a combining function to `fold' the state, and a second such expression
  $e'_r$, used to obtain the initial value of the folding. The combining function
  takes the field values from each of the bindings produced by the select part
  and an initial value, and returns an update to the folding value. To enable
  this, $e_r$ (but not $e'_r$) is in the scope of the variables %identifiers
  in the left agent expression.

  We remark that it is in typing observables expressions that the need for the
  agent type environment $\D$ arises in the typing rules for enriched
  expressions.

\end{itemize}


\subsubsection*{Agent expressions, rules, and models}
Enriched expressions can appear in the same places where expressions can be
expected and can also be bound to variables in models. In more detail:

\begin{itemize}

\item[-] Regarding agent introduction and left agent expressions, we keep the
  rules \textsc{T-Intro} and \textsc{T-L-Agent} of Figure~\ref{fig:syntax}, but
  do not repeat them in Figure~\ref{fig:extSyntax}. Regarding right agent
  expressions and rules, we adapt the rules \textsc{T-R-Agent} and
  \textsc{T-Rule} of Figure~\ref{fig:syntax} to Figure~\ref{fig:extSyntax} to
  allow enriched expressions to appear in places where expressions are expected:
  in right agent expressions, rates, and conditions.

\item[-] Variables can be bound to the values of enriched expressions so that
  they can be referred to later (\textsf{declare}, Figure~\ref{fig:extSyntax}).
  A declaration $e_d$ produces an environment in which the type of the enriched
  expression is bound to the variable. These environments are used to produce an
  environment from a sequence of declarations; the rule for this
  (\textsc{T-Dec}, Figure~\ref{fig:extSyntax}) takes account of the facts that
  declared variables can be used in subsequent expressions and that the same
  variable may be declared more than once.
%
\item[-] As before, models $m$ have agent declarations, initial states, and
  rules, but may now use enriched expressions, except in the right agent
  expression for the initial state, as it would make no sense to have enriched
  expressions there (for this reason the judgment
  $\G | \D \vdash_{\mathrm{b}} \s{a_r}$ is used in the \textsc{T-Model} rule, with the
  subscript intended to indicate that the judgement is one of basic, not
  extended, Chromar).

  Models additionally have a sequence of declarations, $\s{e_d}$, and the base
  type environment produced by this sequence is added to the initial base type
  environment when checking the rules (\textsf{model def},
  Figure~\ref{fig:extSyntax}).  The environment produced by the declaration
  sequence is given inductively by the following axiom and rule:
%
\[\G|\D\vdash \varepsilon:\varepsilon \quad \frac{
\G | \D \vdash e_d: \G' \quad\G[\G'] | \D \vdash \s{e_d}:\G''}{\G | \D \vdash e_d, \s{e_d}: \G'[\G'']}\]
%
Note that, unlike the cases of agent declarations or rule lhs's, the
environments here are `overlapped' ($\G[\G']$), rather than `conjoined'
($\G,\G'$), as in the rule for sequences of left agent expressions. This is to
enable identifiers bound to enriched expressions declared earlier in a
declaration sequence to be used later in it.
\end{itemize}

\begin{figure}[!h]
\centering
\resizebox{0.85\linewidth}{!}{%
\begin{tabularx}{1.08\textwidth}{llr|c} 
\toprule
\addlinespace[0.25cm]
& \emph{\textsf{Syntax}} & & \emph{\textsf{Typing rules}} \\
& & & \\
%Ordinary expressions
$e_r ::=$  & $e\quad (e \in E)$ & \textsf{ord-expr} & $\inferrule*[Right=(T-OrdExpr)]{\G  \vdash e: b }{\G | \D\vdash e : b}$
%& $id$ & \textsf{ref} & $\inferrule*[Right=(T-Ref)]{ \G | \D(id) = b}{ \G | \D  \D \vdash id : b}$ \\
\\
& & & \\
%Where
& $e'_r \; \mathbf{where} \; id:b \; \mathbf{is} \; e_r$ & \textsf{where} & $\inferrule*[Right=(T-Where)]{  \G | \D \vdash e_r : b \\   \G[id:b] |  \D \vdash e'_r : b'}{ \G | \D \vdash e'_r \; \mathbf{where} \; id:b \; \mathbf{is} \; e_r : b'}$ \\
& & & \\
%Time
& $\mathbf{time}$ & \textsf{time} & $\inferrule*[Right=(T-Time)]{}{ \G |  \D \vdash \mathbf{time} : \mathrm{real}}$ \\
& & & \\
%Conditional
& $e_r \; \mathbf{when} \; e'_r\; \mathbf{else} \; e''_r$ & \textsf{condition} & $\inferrule*[Right=(T-Cond)]{ \G | \D \vdash e_r : b \\
 \G | \D \vdash e'_r: \mathrm{bool} \\\\ \G | \D \vdash e''_r : b}{\G | \D \vdash e_r \; \mathbf{when} \; e'_r \; \mathbf{else} \; e''_r : b}$\\
& & & \\
%Repeat
& $\mathbf{repeatEvery} \; e_r \; e_r'$ & \textsf{repeat} & $\inferrule*[Right=(T-Repeat)]{\G | \D\vdash e_r : \mathrm{real} \\ 
\G | \D \vdash e_r': b}{\G| \D \vdash \mathbf{repeatEvery} \; e_r \; e_r' : b}$ \\ 
%%Aggregate (old)
%& $\mathbf{select} \; n_a \, ; \mathbf{aggregate} \; e_r \; e_r'$ & \textsf{obs} & $\inferrule*[Right=(T-Obs)]{\G \vdash e_r : (\s{n_f:b}) \rightarrow b' \rightarrow 
%b' \\ \G \vdash e_r' : b'}{\G | \D \vdash \mathbf{select} \; n_a \; ; \mathbf{aggregate} \; e_r \; e_r' : b'}$\\
%\addlinespace[-0.1cm]
%& & & $(n_a : (\s{n_f : b})) \in \D$ \\
%& & & \\
& & & \\
%%Aggregate (new)
%& $\mathbf{select} \; n_a \, ; \mathbf{aggregate} \; (\s{id:b}, id':b').\, e_r, \; e_r'$ & \textsf{obs} & 
%  $\inferrule*[Right=(T-Obs)]{\G,\s{id:b}, id':b' \vdash e_r : b' \\ \G \vdash e_r' : b'}{\G | \D \vdash \mathbf{select} \; n_a \; ; 
%\mathbf{aggregate} \; (\s{id:b}, id':b').\, e_r, \; e_r' : b'}$\\
%\addlinespace[-0.1cm]
%& & & ($(n_a : (\s{n_f : b})) \in \D$) \\
& & & \\
%Aggregate (newnew)
& $\mathbf{select} \; a_l; $ & \textsf{obs} & 
  $\inferrule*[Right=(T-Obs)]{\D \vdash a_l : \G' \\\\ \G[\G'[id:b]] \vdash e_r : b \\ \G \vdash e_r' : b}
  {\hspace{-2.5cm} \G | \D \vdash   \mathbf{select} \; a_l; }$\\
%\addlinespace[0.1cm]
& $ \mathbf{aggregate} \; (id:b.\, e_r) \; e_r'$& & $\hspace{1.55cm} \mathbf{aggregate} \; (id:b.\, e_r) \; e_r' : b$ \\ %($id'$ not bound in $\G'$) \\
& & & ($id'$ not bound in $\G'$)\\
& & & \\
% Declaration
$e_d :: =$ & $id = e_r$ & \textsf{declare} & $\inferrule*[Right=(T-Dec)]{\G | \D \vdash e_r : b }{\G | \D \vdash id = e_r: (id:b)}$ \\
&&&\\
% Declarations
%$\s{e_d} ::=$ & $\varepsilon \; \mid \; e_d, \s{e_d}$ & \textsf{declare seq}& $\G|\D\vdash \varepsilon:\varepsilon \quad \inferrule*[Right=(T-Decs)] {
%\G | \D \vdash e_d: \G' \\ \G[\G'] | \D \vdash \s{e_d}:\G''}{\G | \D \vdash e_d, \s{e_d}: \G'[\G'']}$ \\
%& & & \\
% Right Agent
$a_r::=$ & $n_a(\s{n_f=e_r})$ &\textsf{right agent} & 
$\inferrule*[Right=(T-R-agent)]{\G|\D \vdash \s{e_r :b}}
{\G \,|\, \D \vdash n_a(\s{n_f=e_r})}$ \\
%\addlinespace[-0.3em]
& & \textsf{expr}& ($ n_a : (\s{n_f:b}) \in \D$)\\
& & & \\
% Rule
$r ::=$& $\s{a_l} \xrightarrow{e_r} \s{a_r} \; [e'_r]$ & \textsf{rule expr} &  
%
$\inferrule*[Right=(T-Rule)]{\D \vdash \s{a_l} : \G' \\
 \G[\G']|\D \vdash \s{a_r} \\\\ \G[\G']\,|\,\D \vdash e_r : \mathrm{real}  \\ \G[\G']\,|\,\D \vdash e'_r : \mathrm{bool} }{\G \,|\,\D \vdash \s{a_l} \xrightarrow{e_r} \s{a_r} \; [e'_r]}$\\
& & & \\
% Model 
$m :: = $ & $ \s{a_d}; \; \mathbf{init} (\s{a_r}); \; \s{e_d}; \; \s{r}$ & \textsf{model def} & 
$\inferrule*[Right=(T-Model)]{ \vdash \s{a_d} : \D \\  \G| \D  \vdash_{\mathrm{b}} \s{a_r} \\\\ \G | \D \vdash \s{e_d}:\G' \\ \G[\G'] | \D \vdash \s{r}}{\G 
 \vdash  \s{a_d}; \; \mathbf{init} (\s{a_r}); \; \s{e_d}; \; \s{r}   }$  
\\\\
\bottomrule
\end{tabularx}}
\caption{Abstract syntax of enriched expressions and corresponding typing rules.}
\label{fig:extSyntax}
\end{figure}

\subsection{Semantics of extended Chromar}
\label{sec:extSem}
We next give the semantics of Chromar extended with fluents and observables,
beginning with that of enriched expressions and then moving on to the constructs
that use enriched expressions. We need to use functional environments: we say
that $\sigma= id_1:f_1, \dots, id_n:f_n$ is a \emph{functional $\G$-environment} for a base
type environment $\G= id_1:b_1, \dots, id_n:b_n$ if, for $i = 1,\ldots,n$, $f_i$ is a
function from $\MS[\mathrm{Av}] \times \mathbb{R}$ to $\V_{b_i}$. We work with these
environments as we do with others, in particular treating them as finite
functions. Given a functional $\G$-environment
$\sigma= id_1:f_1, \dots, id_n:f_n$ and a state $s$ and a time $t$ we can obtain
a $\Gamma$-environment by setting
$\sigma(s,t) = id_1:f_1(s,t), \dots, id_n:f_n(s,t)$. In the other direction we
can `promote' a $\G$-environment $\sigma= id_1:v_1, \dots, id_n:v_n$ to a
functional $\G$-environment $\up{\sigma} = id_1: f_1, \dots, id_n : f_n$, by setting
$f_i(s,t) = v_i$, for $i = 1,\ldots,n$.

We need some further definitions in order to give the semantics of observables.  
%First, for any sequence $\s{b} = b_1\ldots b_n$ of base types we set  $\V_{\s{b}} = \V_{b_1} \times \ldots \times \V_{b_n}$.
For any basic environment $\G$ we write $\V_{\G}$ for the set of $\G$-environments.
%For any agent type $\s{n_f:b}$, we define a function $sel_{\s{n_f:b}}:{\MS[\mathrm{Av}] \rightarrow \MS[V_{\s{b}}]}$ by:
%%
%\[\sel_{\s{n_f:b}}(s)  = \m{ \s{v} \in \V_{\s{b}}\mid (n_a, \varphi) \in s , \varphi(\s{n_f}) = \s{v} }\]
%%
Then, given any left agent  expression $\D\vdash a_l: \G$, where $a_l = n_a(\s{n_f = id})$, 
%$n_a(\s{n_f = id})$ and sequence $\s{b}$ of basic types of the same length as $\s{n_f = id}$, 
we define a `selection function' 
%
\[\sel_{\D,a_l}:{\MS[\mathrm{Av}] \rightarrow \MS[\V_{\G}]}\]
%
that takes a state and returns a multiset of bindings ($\G$-environments). This
multiset has a binding for each matching agent in the state with multiplicity
that of the matching agent.  The selection function is defined by:
%
%\[\sel_{\D, a_l}(s)  = \m{\, \sigma \in \V_{\G}\mid \den{a_l}_{\D}(\sigma) \in s }\]
%
%\[\sel_{\D, a_l}(s)  = \m{\, \sigma \in \V_{\G}\mid (n_a, \varphi) \in s ,  \den{a_l}_{\D}(\sigma) = (n_a, \varphi)}\]
%%
%\[\sel_{\D, a_l}(s)  = \m{\, \sigma \in \V_{\s{id:b}}\mid (n_a, \varphi) \in s , \s{\varphi(n_f)} = \s{\sigma(id)}\, }\]
%%
%making use of the semantics of basic Chromar and an informal finite multiset comprehension syntax intended to respect %multiplicities (for which, cf.~\cite{buneman1994comprehension}).
% In terms of multisets as functions, we have: 
%
\[\sel_{\D, a_l}(s) =  \sigma \in \V_{\G} \mapsto s(n_a(\s{n_f =  \sigma(id)}))\]
%
%\[\sel_{\D, a_l}(s)(\sigma) =  s(\den{a_l}_{\D}(\sigma))\]
%making use of the semantics of basic Chromar and 
recalling that we treat finite multisets as functions that have value 0 except at finitely many arguments. In this respect, note that, as required,  $\sel_{\D, a_l}(s)$ is 0 except for finitely many arguments as $s$ is, and the function 
$\sigma \mapsto n_a(\s{n_f =  \sigma(id)})$ is 1-1.

We recall the list folding function \citep[and
see][]{hutton_fold_1998}. Identifying sequences and lists, for any sets $X$, $Y$,
combining function $f: X \times Y \rightarrow Y$ and $y \in Y$, we define:
%
$\fold(u,f,y) \in Y$ by induction on the length of $u$ as follows:
%
\[\fold(u,f,y) =  \left \{\begin{array}{ll}
                                   y & (u = \varepsilon) \\
                                   f(x,\fold(u',f,y)) & (u = xu')
                                 \end{array} \right .\]
%
Under a natural condition on the combining function, we can use list folding to define finite multiset folding. Say that $f: X \times Y \rightarrow Y$ has a \emph{commutative (left) action} if,  for any $x,x'\in X$ and $y \in Y$, we have:
%
\[f(x,f(x',y)) =  f(x',f(x,y))\]
%
For such a function $f$, we can unambiguously define $\fold(m,f,y)$, for a
finite multiset $m \in \MS[X]$, to be $\fold(u,f,y)$, choosing any
$u \in \SQ[Y]$ such that $\ms(u) = m$.

Finally we assume available a `linearisation function'
%
\[\langle\!\langle - \rangle\!\rangle:\MS[\V_{\s{id:b}}] \longrightarrow \SQ[\V_{\s{id:b}}] \]
%
making such a choice for multisets of $\s{id:b}$-environments. It puts the
elements of a finite multiset of $\s{id:b}$-environments in some standard order,
repeating them according to their multiplicity, so that
$\ms(\langle\!\langle m \rangle\!\rangle) = m$ holds for any $m \in \MS[\V_{\s{id:b}}]$.


\subsubsection*{Semantics of enriched expressions}
As discussed above we take the values of enriched expressions to be functions of
states and time.  So, suppose that we have a typing $\G|\D \vdash e_r$ of an enriched
expression $e_r$. Then for any functional $\G$-environment $\sigma$, we define its
evaluation
$\den{e_r}_{\G,\D}(\sigma): \MS[\mathrm{Av}] \times \mathbb{R} \rightarrow \V_b$ as follows, where we
divide the definition into cases according to the form of the enriched
expression:

\begin{align*}
%ordinary expressions
& \den{e}_{\G,\D}(\sigma ) (s, t) = \den{e}_{\G}(\sigma(s,t)) \\[0.2cm]\\
&\den{e'_r \; \mathbf{where} \; id:b \; \mathbf{is} \; e_r}_{\G,\D}(\sigma )  = \den{e'_r}_{\G,\D}(\sigma[x \mapsto \den{e_r}_{\G|\D}(\sigma )]) \\[0.2cm]\\
%time
& \den{\mathbf{time}}_{\G,\D}(\sigma ) (s, t) = t \\[0.2cm]
%when
& \den{e_r \; \mathbf{when} \; e'_r \; \mathbf{else} \; e''_r}_{\G,\D}(\sigma ) (s, t) \;  = \; 
\left \{ \begin{array}{ll}
\den{e_r}_{\G,\D}(\sigma ) (s, t)  & (\mbox{if } \den{e'_r}_{\G|\D}(\sigma ) (s, t) = \mytt) \\[0.2cm]
\den{e''_r}_{\G,\D}(\sigma ) (s, t)  & (\mbox{otherwise})\end{array} \right . \\[0.2cm]\\
%repeat
& \den{\mathbf{repeatEvery} \; e_r \; e_r'}_{\G,\D}(\sigma ) (s, t) = 
          \left \{\begin{array}{ll}    
                       \den{e_r'}_{\G,\D}(\sigma) (s, t') & (\den{e_r}_{\G,\D}(\sigma ) (s, t) > 0)\\
                       0 & (\mbox{otherwise})
                               \end{array}
                      \right . \\[0.2cm]                  
&\hspace{10pt} \mbox{where } t' = t \; \; \mathrm{mod} \; \; \den{e_r}_{\G,\D}(\sigma ) (s, t) \\ \\
&\den{\mathbf{select} \; a_l;\; \mathbf{aggregate} \; ( id:b.\, e_r) \; e_r'}_{\G,\D}(\sigma)(s,t) = \fold(\langle\!\langle
 \sel_{\D,a_l}(s)\rangle\!\rangle, f, \den{e'_r}_{\G,\D}(\sigma)(s,t)) \\
&\hspace{10pt} \mbox{where }\D \vdash a_l:\G' \mbox{ and:} \\
&\hspace{50pt} f(\sigma',v) = \den{e_r}_{\G[\G'[ id:b]],\D}(\sigma[\sigma'[id:v]])(s,t) \quad (\sigma': \V_{\G'}, v : \V_{b})\\
\end{align*}
Here:
\begin{itemize}

\item[-] Ordinary expressions, $e \in E$, are evaluated using the provided evaluation function. 

\item[-] Enriched expressions $e_r' \; \mathbf{where} \; id:b \; \mathbf{is} \; e_r$
are evaluated by evaluating the first expression $e'_r$ in a functional environment in which $id$ is bound to the evaluation of the second expression $e_r$.

\item[-] The time expression is evaluated by returning the given time (so ignoring the environment and the state).

\item[-] The conditional expression is evaluated by evaluating the first expression if the second expression (the condition) evaluates to true, and otherwise  by evaluating the third expression. 

\item[-] The expression $\mathbf{repeatEvery} \; e_r \; e_r'$ is evaluated at time
  $t$ by evaluating $e_r'$ at $t$ modulo the evaluation of $e_r$ at $t$, if the
  evaluation of $e_r$ is $>0$, and returning $0$, if the evaluation of $e_r$ at
  $t$ is $\leq 0$. While the semantics always gives a result, it is only sensible
  if $e_r$'s value is not $0$; a more complex semantics for Chromar would produce
  an error in that case.

\item[-] 

  Finally, given a state, the observable expression
  $\mathbf{select} \; a_c; \; \mathbf{aggregate} \; (id:b.\,e_r) \; e_r'$ is
  evaluated in two stages.  First, the left agent expression
  $a_l = n_a(\s{n_f = id})$ is used to filter out all the $n_a$ agents from the
  state and transform them into environments holding the field values of the
  agent; this produces a multiset of such environments. Second, the value of the
  observable expression is produced by a fold on this multiset. The fold uses a
  combining function obtained from the semantics of $e_r$, and an initial value
  given by evaluating $e'_r$. Note that it employs the linearisation function
  $\langle\!\langle - \rangle\!\rangle$ assumed above. While the semantics gives a result for every such
  combining function, the result of such a fold using a fixed, but arbitrary,
  selection function, is only sensible if the combining function has a
  commutative left action. We leave it to the programmer to make sure their
  combining function has such an action.
\end{itemize}


\subsubsection*{Declarations}
Declarations $id = e_r$ bind the values of enriched expressions $e_r$ to variables
$id$.  A series of such expressions evaluates to an environment binding the
variables to the values of the corresponding enriched
expressions.

So, for a single declaration $\G | \D \vdash id = e_r: (id:b)$, for every functional
$\G$-environment $\sigma$, we define its evaluation
$\den{id = e_r: (id:b)}_{\G ,\D}(\sigma)$ as a functional $id:b$-environment as follows:
%
\begin{align*}
\den{id = e_r}_{\G,\D}(\sigma) = (id : \den{e_r}_{\G,\D}(\sigma))
\end{align*}
%
and for a sequence of declarations $\G | \D \vdash \s{e_d}:\G'$, for every functional
$\G$-environment $\sigma$, we define its evaluation
$\den{\s{e_d}}_{\G , \D}(\sigma)$ as a functional $\G'$-environment as follows:
%
\[\den{\s{e_d}}_{\G , \D}(\sigma) = 
\left \{\begin{array}{ll}
 \varepsilon & (\s{e_d} = \varepsilon)   \\
\den{e'_d}_{\G,\D}(\sigma) [ \den{\s{e''_d}}_{\G[\G'],\D} (\sigma[\den{e'_d}_{\G,\D}(\sigma)]) ] & (\s{e_d} = e'_d \s{e''_d},\, \G|\D \vdash e'_d: \G') 
\end{array}\right .\]
%
Note that this definition recurses on $\s{e_d}$, but, by induction on the length
of $\s{e_d}$, is nonetheless proper.


\subsubsection*{Agents, Rules, and Models}  
Enriched expressions can appear in places where simple expressions appeared:
right agent expressions, conditions, and rates. This does not change our
semantics except that the semantics of all the constructs that involve the
enriched expressions have to be parameterised by state and time since their
evaluation will depend on those.

The semantics of sequences of left agent expressions is the same as that of
basic Chromar. For right agent expressions we make the evident change to
accommodate enriched expressions. To this end, to evaluate sequences of right
agent expressions as states, %to multisets of agents,
suppose that we have a right agent expression typing
$\G \,|\, \D \vdash \s{n_a(\s{n_f = e_r})}$. Then for any functional
$\G$-environment $\sigma$, state $s$, and time $t$, we define its evaluation
$\den{\s{n_a(\s{n_f = e_r})}}_{\G,\D}(\sigma,s,t) \in \MS[\mathrm{Av}]$ as follows:
\begin{align*}
\den{\s{n_a(\s{n_f = e_r})}}_{\G,\D}(\sigma,s,t) = \ms(\s{n_a(\s{n_f = \den{e_r}_{\G,\D}(\sigma)(s,t)})}
\end{align*}

For the semantics of rules, suppose that we are given a rule typing
$\G\,|\, \D \vdash \s{a_l} \xrightarrow{e_r} \s{a_r} \; [e'_r]$. Then we further
have $\D \vdash \s{a_l} : \G'$ for a unique $\G'$.  We also have
$\overline{\G}|\D \vdash e_r : \mathrm{real}$,
$ \overline{\G}|\D \vdash \s{a_r}$, and
$ \overline{\G}|\D \vdash e'_r : \mathrm{bool}$, where $\overline{\G} = \G[\G']$.

For the reaction denoted by the rule, suppose we are given a functional
$\G$-environment $\sigma$, a $\G'$-environment $\sigma'$, and a state $s$ and a time
$t$. Then we set
\begin{equation*}
\mathcal{R}\den{\s{a_l} \xrightarrow{e_r} \s{a_r} \, [e'_r]}(\sigma, \sigma', s, t) \simeq \left\{ 
\begin{array}{ll}
\den{\s{a_l}}_{\D}(\sigma') \xrightarrow{\den{e_r}_{\overline{\G},\D}(\overline{\sigma})(s, t)} \den{\s{a_r}}_{\overline{\G}, \D}(\overline{\sigma}, s, t)  &  \! \big(\overline{\sigma} = \sigma[\up{\sigma'}],  \\ 
  & \den{e_r}_{\overline{\G},\D}(\overline{\sigma})( s, t) > 0, \\ 
  & \den{e'_r}_{\overline{\G},\D}(\overline{\sigma})( s, t) = \mytt \big) 
 \\
\text{undefined } & (\text{otherwise})
\end{array}
\right .
\end{equation*}

Regarding matches, we say that a $\G'$-environment $\match$ is a \emph{proper
  match} of the rule in a state $s$, given a functional $\G$-environment
$\sigma$ and a time $t$, if $\den{a_l}_{\D}( \match)$ is a submultiset of $s$, and
$\mathcal{R}\den{\s{a_l} \xrightarrow{e_r} \s{a_r} \, [e'_r]}(\sigma, m, s, t) $
exists (i.e., $ \den{e_r}_{\overline{\G},\D}(\overline{\sigma})( s, t) > 0$ and
$\den{e'_r}_{\overline{\G},\D}(\overline{\sigma})( s, t) = \mytt $, where
$\overline{\sigma} = \sigma[\up{m}]$).


The stochastic matrix denoted by the rule, given a functional $\G$-environment $\sigma$ and a time $t$, is:
\begin{equation*}
\begin{split}
\den{
{\s{a_l} \xrightarrow{e_r} \s{a_r} \, [e'_r]}}_{\G,\D}(\sigma, t) (s,s') = & \sum\{\den{e_r}_{\overline{ \G},\D}(\overline{\sigma})(s, t) \cdot \mu(\den{a_l}_{ \D}(\match),s) \mid \\
 & \hspace{2.0cm} \match \mbox{ is a proper match for } a_l \mbox{ in } s,\\ 
 & \hspace{2.0cm} \overline{\sigma} = \sigma[\up{ \match}],   \\
 & \hspace{2.0cm} s' = \mathcal{R}\den{{\s{a_l} \xrightarrow{e_r} \s{a_r} \, [e'_r]}}_{\G,\D}(\sigma,\match, s, t) \bullet s \}
\end{split}
\end{equation*}

Turning to models, suppose we have a model typing
$\G \vdash \s{a_d}; \; \mathbf{init}\, (\s{a_r}); \, \s{e_d};\, \; \s{r}$.
Then, for a unique $\D$ and $\G'$, we have $\ \vdash \s{a_d}: \D$ and
$\G | \D \vdash \s{e_d}:\G' $, and then also
$\G | \D \vdash_{\mathrm{b}} \s{a_r}$ and $\overline{\G} | \D \vdash \s{r}$, where
$\overline{\G} = \G[\G']$.
%
Then, for any $\G$-environment $\sigma$ and time $t$ the CTMC associated to the model is:
%

\[\den{ \s{a_d}; \; \mathbf{init}\, (\s{a_r}); \; \s{e_d}; \; \s{r}}_{\G,\D}(\sigma,t) \; =  \; 
                 (\MS[\mathrm{Av}], \sum_{r \in \s{r}} \den{r}_{\overline{\G},\D}(\up{\sigma}[\den{\s{e_d}}_{\G,\D}(\up{\sigma})],t), \den{\s{a_r}}_{\G,\D}(\sigma))\]
%
               where we use the semantics of basic Chromar for the evaluation of
               the basic Chromar initial state right agent expression $a_r$.
               
\subsection{A simulation algorithm}
\label{sec:simT}
Using fluents means that the Continuous Time Markov Chain that a Chromar model
generates becomes inhomogeneous in time. Therefore the standard Stochastic
Simulation algorithm (SSA) does not apply since it assumes constant propensities
between reactions. In practice, however, this discrepancy is often not
prohibitive and gives very similar results while avoiding the extra
computational cost added by non-homogeneity (we discuss this further below).

We therefore use an approximate simulation method where the fluents, and
therefore all time-varying expressions in rules, are only evaluated at the times
the usual SSA would choose for the next reaction. This is equivalent to keeping
their values constant between reactions, at the expense of accuracy. The only
change in our algorithm from that for basic Chromar is then in the reaction
generation step (1) where the reaction generation happens at the current state
and time in an environment extended with the evaluation of defined enriched
expressions.

So assume a model and an environment $\sigma$, as above, and set $\sigma'$ to be the
functional $\G$-environment $\up{\sigma}[\den{e_d}_{\G,\D}(\up{\sigma})]$. As before, the
algorithm keeps a state $s$ and a time $t$, starting with state
$s_0 = \den{\s{a_r}}_{\G,\D}(\sigma)$ and an initial time $0$. It then iterates the same
sequence of steps as before, as many times as desired, except that step (1) is
changed to

\begin{itemize}
\item[1.]  For current state  $s$ and current time $t$, 
 generate the multiset of all possible reactions for every rule:
 \begin{equation*}
 \begin{split}
R  = \m{ \mathcal{R}\den{\s{r}.i}(\sigma', m, s, t) \; | \; & i = 1,\ldots,|\s{r}|, \\ 
                                 & m \mbox{ is a proper match of rule $\s{r}.i$ in $s$}, \\ 
                                 & \mbox{ given $\sigma'$ and $t$}}
\end{split}
\end{equation*}
\end{itemize}

Since fluents are only sampled according to the discrete time-jumps followed by
the simulation clock that means that in practice we only get a discrete
approximation of the continuous functions denoted by the fluent definitions (see
for example Figure~\ref{fig:fluentSamp}). The accuracy of the approximation will
depend on the sampling interval and how fast the fluent changes. If, for
example, the fluent changes on a faster timescale than that of the model then
the approximation will be poor. In practice, however, the fluents are usually on
the same or a slower timescale than that of the model since they are usually
used to model the physical context of the model, which means we get acceptable
approximations. In the case of boolean fluents (like our light fluent from
Figure~\ref{fig:fluentSamp}) similar decisions on how to handle discontinuous
changes have to be made in, for example, simulators that support SBML events. A
common approach there is to have explicit handling of events in the simulation
loop. The stochastic simulation in the iBioSim tool, for example, checks the
time of the next event and executes it if it happens before the time of the next
reaction \citep[see][Algorithm 7]{watanabe_hierarchical_2014}. It should be
possible to extend these methods to Fluents of any type and add to our
simulation algorithm if increased accuracy is needed. We note too that there are
promising new methods for simulating CTMCs with time-varying propensities that
alleviate some of the computational cost \citep{voliotis_stochastic_2016}.

\begin{figure}[tb]
\centering
\includegraphics[width=0.8\textwidth]{figures/fluentSample.png}
\caption{ Ideal (solid line) versus practical approximation during simulation
  (dotted line) of the fluent for the light conditions in a day:
  $ \mathrm{True} \; \mathbf{when} \; (6 < \mathbf{time} < 18) \; \mathbf{else}
  \; \mathrm{False}$, where we take $\mathrm{True}$ to be $1.0$ and
  $\mathrm{False}$ to be $0.0$. The fluent definition denotes a continuous
  function of time but in practice the function will only be sampled at the time
  points the SSA visits (red points). This means that practically we only get a
  discrete approximation of the fluent.}
\label{fig:fluentSamp}
\end{figure}

\subsection{Simulation efficiency}
\label{subsec:simEff}
In terms of simulation, our implementation is simple and follows the steps we
presented in \sct{semantics}. One area of improvement could be in the reaction
generation step where we currently generate all the active reactions at every
step. In practice we do not need to completely regenerate the reactions at every
step since only a subset of them changes between steps. The performance gains
will depend on the efficiency of calculating the change in the reactions set
after a change in the state. This idea was already used, to great effect, in the
implementation of Kappa \cite{danos_scalable_2007}.


\section{Conclusion}
Here I defined the formal syntax and semantics of Chromar. This process of
definition uncovered and made explicit several design choices that were implicit
in the informal version of the language in my head. It furthermore made the
extensions and implementation of enriched expression possible by clarifying its
relation to the basic parts of the language. The abstract nature of Chromar
(parametric to expression and types) was also made easier by this formal
definition. In the next chapter I make particular choices regarding these
(Haskell expressions and types respectively) but the availability of the formal
abstract definition of this chapter means that other choices can be made with
less effort.




\chapter{Implemented Chromar}
\label{chp:chromarImpl}
\documentclass[phd]{infthesis}
\usepackage[utf8]{inputenc}
\usepackage[T1]{fontenc}
\usepackage[british]{babel}
\usepackage{microtype}
\usepackage[usenames,dvipsnames,svgnames,table]{xcolor}
\usepackage[english=british,autopunct=false]{csquotes}
\usepackage[natbib=true,style=authoryear-comp,maxbibnames=6]{biblatex}
\usepackage{graphicx}
\usepackage{textcomp}
\usepackage{wrapfig}
\usepackage{xfrac}
\usepackage{xspace}
\usepackage{mathcommon}
\usepackage[sc]{mathpazo}
\usepackage{hyperref}
\usepackage{expl3}
\usepackage{enumitem}
\usepackage{booktabs}
\usepackage{tabularx}
\usepackage{mathpartir}
\usepackage{fancyvrb}
\usepackage{inconsolata}

\frenchspacing

% Bibliography
\addbibresource{chromarImpl.bib}
\bibliography{chromarImpl}

% Text
\newcommand{\ie}{i.e.\xspace}
\newcommand{\eg}{e.g.\xspace}

% Referencing
\newcommand{\chp}[1]{\S\ref{chp:#1}}
\newcommand{\sct}[1]{\S\ref{sec:#1}}
\newcommand{\ssec}[1]{\S\ref{subsec:#1}}
\newcommand{\eqn}[1]{Eq.~\ref{eq:#1}}
\newcommand{\eqns}[2]{Eq. \ref{eq:#1} and \ref{eq:#2}}
\newcommand{\lem}[1]{Lemma~\ref{lemma:#1}}
\newcommand{\lems}[2]{Lemmas \ref{lemma:#1} and \ref{lemma:#2}}
\newcommand{\thm}[1]{Th.~\ref{thm:#1}}
\newcommand{\fig}[1]{Fig.~\ref{fig:#1}}
\newcommand{\diagram}[1]{diagram~\ref{eq:#1}}
\newcommand{\app}[1]{Appendix~\ref{app:#1}}
\newcommand{\mcite}[1]{\textcolor{gray}{#1}} % missing cite
\newcommand{\defn}[1]{Def.~\ref{def:#1}}
\newcommand{\prop}[1]{Prop.~\ref{prop:#1}}

% Math
\renewcommand{\tuple}[1]{\left(#1\right)}
\DeclareMathOperator*{\expn}{exp}
\renewcommand*{\exp}[1]{e^{\,#1}} % \mathrm{e}^{#1}}
\renewcommand{\qedsymbol}{\ensuremath{\blacksquare}}
\newcommand{\partialto}{\rightharpoonup}
\newcommand{\id}{\vec{1}} % identity function
\newcommand{\mr}[1]{\mathrm{#1}}

\newcommand{\den}[1]{\llbracket #1 \rrbracket}
\newcommand{\m}[1]{\{\!| #1 |\!\}}
\newcommand{\M}[1]{\mathcal{#1}}
\newcommand{\MS}[0]{\mathrm{M}}
\newcommand{\SQ}[0]{\mathrm{S}}
\newcommand{\s}[1]{\underline{#1}}
\newcommand{\G}[0]{\Gamma}
\newcommand{\D}[0]{\Delta}
\newcommand{\mytt}{t\!t}
\newcommand{\myff}{f\!\!f}

\newcommand{\V}{\mathrm{V}}

\newcommand{\sel}{\mathrm{sel}}
\newcommand{\fold}{\mathrm{fold}}

\newcommand{\ms}{\mathrm{ms}}


\newtheorem{mydef}{Definition}
\def\dotminus{\mathbin{\ooalign{\hss\raise1ex\hbox{.}\hss\cr
  \mathsurround=0pt$-$}}}
\setlength{\tabcolsep}{8pt}
\renewcommand{\arraystretch}{1.0}

\newcommand{\match}{m}
\newcommand{\up}[1]{\uparrow\! #1}

\newcommand{\n}{\mathrm{n}}

% Other stuff
\newcommand{\maybe}[1]{\textcolor{gray}{#1}}
\newcommand{\todo}[1]{\textcolor{red}{TODO: #1}}

% rules
\newcommand{\ar}[2]{\mr{#1} \! = \! {#2}}

\setlength{\tabcolsep}{8pt}
\renewcommand{\arraystretch}{1.2}

\begin{document}


\chapter{Implemented Chromar}
\section{Haskell embedding}
\label{sec:impl}
As we have seen, extended Chromar is abstract in that no choice is made of the
expression language or its types. In this section we describe the implementation
via embedding in Haskell of one concrete realisation of Chromar where we fix the
expression language $E$ and set of types $T$ to Haskell expressions and types
respectively.

Embedding a domain-specific language inside a host general purpose programming
languages has the advantage that it allows the reuse of the functionality of the
host language for the implementation therefore relieving some of the effort
involved in creating the infrastructure for a new language. From the
programmer's point of view it also allows the possibility of mixing features of
the host language with features of the hosted language thus getting a more
expressive result. There are two basic ways of embedding a language: (i) shallow
embedding where the domain-specific constructs are directly expressed in host
language terms (usually through a library) and (ii) deep embedding where
embedded language terms are built in some representation in the host language
and are then given meaning via interpretation \citep{hudak_modular_1998}. There
are advantages and disadvantages to both approaches. Here we choose a middle
ground where we use shallow embedding for expressing agent types, initial state,
and outside value environment (for the definitions of constants, functions and
so on) and deep embedding (via quasiquotations) using domain-specific syntax for
rules and enriched expressions. Such a mixed-level embedding in Haskell has been
explored before for another modelling language, Hydra, in the context of
non-causal, hybrid modelling \citep{giorgidze_mixed-level_2010}.

We next go through the different entities in extended Chromar and see how they
are represented in Haskell. We will use \texttt{typeface} font for Haskell code
to distinguish it from the abstract Chromar syntax.


\subsection{Agents}
Agent type declarations of the form $\mathbf{agent} \; n_a = (\s{n_f : b})$
become Haskell types with $n_a$ as the name of the type constructor and a
sequence of named, typed fields (records). Sequences of agent declarations,
$\s{a_d}$, can be collected in a union type. For example our $\mathrm{Leaf}$ and
$\mathrm{Cell}$ agent declarations from Section~\ref{sec:overview} are written
thus:
\begin{center}
\begin{BVerbatim}
data Agent
    = Leaf { age :: Int
           , mass :: Double }
    | Cell { carbon :: Double }
\end{BVerbatim}
\end{center}
where the \texttt{data} keyword introduces a new datatype.  Agents are values of
the \texttt{Agent} type, for example \texttt{Leaf\{age=1, mass=1.5\}} and
\texttt{Cell\{carbon=1.2\}}.  To construct multisets of Agents, we introduce the
parametric type \texttt{Multiset a = [(a, Int)]}.  Then the concrete type
\texttt{Multiset Agent} maps agents to their counts.
Given a function \texttt{mset} that constructs a multiset from a list (analogous to our function $\ms$ from Section~\ref{sec:semantics}),
our initial state definition becomes:
\begin{center}
\begin{BVerbatim}
init = mset [ Leaf{age=1,mass=0.2}, 
              Leaf{age=2, mass=0.1}, 
              Cell{carbon=0.5} ]
\end{BVerbatim}
\end{center}


\subsection{Rules (via Quasiquotation)}
\label{sec:ruleQQ}
We use a deeper embedding of rules where we allow the expression of rules
directly in domain-specific syntax using Quasi-quotes. Quasi-quotes is a Haskell
language extension that allows the use of special syntax (i.e., non standard
Haskell syntax) inside \texttt{[\$quoter| ... |]} brackets along with the name
of a quoter function (\texttt{\$quoter}) that is a function that takes the
string inside the brackets and produces Haskell abstract syntax. The produced
abstract syntax is injected by the Haskell compiler in place of the quotes.

The implementation here provides the \texttt{rule} quoter function that takes
the string inside the quotes and produces a reaction generation function (or
rather the abstract syntax of such a function) that given a state and time
produces a multiset of reactions. We therefore take the rule to be a reaction
generating function of type: \texttt{Time -> Multiset a -> [Reaction a]} where a
reaction is naturally defined as a record type parameterised on an agent type:
\begin{center}
\begin{BVerbatim}
data Reaction a = Reaction
    { lhs :: Multiset a
    , rhs :: Multiset a
    , rate :: Double
    }
\end{BVerbatim}
\end{center}
Our quoted rules have the following syntax that is almost the same as the one we presented earlier:
$$\mathrm{r_q} ::= \; \s{n_a\{\s{n_f \mathbf{=} id}\}} \; \rightarrow \; \s{n_a'\{\s{n'_f \mathbf{=} e}\}} \; @\,e_r \; {[} e_c {]} $$                         
For example our growth rule from the plant model example becomes:
\begin{center}
\begin{BVerbatim}
growth = [rule| Leaf{mass=m}, Cell{carbon=c} -->
                Leaf{mass=m}, Cell{carbon=c-1} @g m [c >= 1] |]
\end{BVerbatim}
\end{center}
This looks very close to our abstract syntax, but with some minor syntactic
differences such as the placement of the rate expression at the end of the rule
preceded by the \texttt{@} symbol. Crucially, being inside a programming
language means that we can use any valid Haskell expression in the places where
expressions are expected, i.e., in the values of fields in the right-hand side
of rules, rates, and conditions. A very wide range of expressions are therefore
supported, without further effort. We next look at the rule quoter function in
more detail.

\subsubsection*{Rule quoter function}
The \texttt{rule} quoter function takes a quoted rule expression, $r_q$, and
produces a Haskell function of type \texttt{Time -> Multiset a -> [Reaction a]}
where \texttt{[Reaction a]} is a list of reactions and \texttt{Time} is a
synonym for \texttt{Double}. The rule function needs to find all matches
(according to the rule lhs) and then generate a concrete reaction for each.
This leads to the two parts of the rule functions we need to construct: the
query part --- for finding the matches --- and the reaction generation part. For
the query part we use list comprehensions and pattern matching for binding the
variables in the rule lhs. This structure follows the definition of the
reactions denoted by a rule we have seen before (see $\mathcal{R}$,
Section~\ref{sec:ruleSem}).

The following example illustrates the translation from quoted rule expression to
rule functions. Consider the agent declaration
$\mathbf{agent} \; \mathrm{A}(\mathrm{a}:\mathrm{int})$ and rule
$\mathrm{A}(\mathrm{a}=x), \mathrm{A}(\mathrm{a}=y) \xrightarrow{f(x)}
\mathrm{A}(\mathrm{a}=x) \; [x > y]$. For the query part we make one
comprehension generator statement for each agent on the left-hand side. The
statement for the first agent becomes: \texttt{(A\{a=x\}, \_) <- s} for some state
$s$ of type \texttt{Multiset Agent}. Pattern matching ensures that we only look
at $A$'s in the state and that we also bind the \texttt{x} variable. Note that
we ignore the count of each agent-count pair since we later calculate the
multiplicity of the entire lhs inside the state in the reaction generation part
of the comprehension.  For each such successful match of the first statement we
continue to the second statement with \texttt{x} bound. The statement for the
second element of the rule lhs is similarly \texttt{(A\{a=y\}, \_) <- s}.  With
\texttt{x} and \texttt{y} bound we have our match $m$ (as defined in
Section~\ref{sec:ruleSem}) and we produce a reaction if the condition is true in
the environment extended with $m$. The full comprehension statement is thus:
\begin{center}
\begin{BVerbatim}
r :: Multiset Agent -> Time -> [Reaction Agent]
r s t =
     [ Reaction
      { lhs = mset [A{a = x}, A{a = y}]
      , rhs = mset [A{a = x}]
      , rate = rr
      }
     | (A {a = x}, _) <- s
     , (A {a = y}, _) <- s
     , x > y
     , let rr = (f x) * m (mset [A{a = x}, A{a = y}]) s
     , rr > 0 ]
\end{BVerbatim}
\end{center}
Each rule generates a multiset of reactions, as noted earlier, since the same
reaction can occur from two different matches. The multiset is given as a list
where the order is not important. Function \texttt{m} is the $\mu$ function that
calculates the multiplicity of the match (defined in
Section~\ref{sec:stoch}). The multiplicity also makes sure that the two patterns
do not match the same agent in the state. The comprehension allows that, which
means we create some spurious reactions, but the rate of those will be 0. The
rate function, \texttt{f}, is provided by the outside Haskell environment. This
is an example where we leverage the strength of the host language.


\subsection{Fluent and Observable features (enriched expressions)}
Enriched expressions are also embedded using Quasi-quotes that allows the direct
use of their syntax. The implementation provides a quoter function, \texttt{er},
for the translation of the surface syntax to its interpretation. The
interpretation of an enriched expression is a function from states and time to
values in $V$, exactly like our formal definition in
Section~\ref{sec:extSem}. In Haskell we use the following type:
\begin{center}
  \texttt{data Er a b = Er \{ at :: Multiset a -> Time -> b\}}
\end{center} 
parameterised by the agent type (\texttt{a}) and the return type (\texttt{b}).

The surface syntax for enriched expressions is the same as the abstract one with
the exception of the $\mathrm{where}$ construct. The most common use-case of the
$\mathrm{where}$ construct is to mix expressions from the provided language $E$
with fluents and observables. For example being able to write $nl + 1$ for some
observable $nl$. In the current implementation we only cover this particular
sub-case where variables referring to enriched expressions are embedded inside
Haskell expressions. Instead of being `tagged' with the $\mathrm{where}$ clause,
we surround them with \texttt{\$..\$} symbols. This resolves the problem of
mixing simple and functional values. The functional values (inside
\texttt{\$..\$}) will be interpreted to their simple value at current state and
time. Other local definitions of enriched expressions can be done using the
restricted scope declaration mechanisms of the host language (\texttt{let} and
\texttt{where} in Haskell). All other forms of enriched expressions remain the
same as in the grammar provided in Section~\ref{sec:extSyntax}.

We omit a formal translation from surface syntax to the Haskell \texttt{Er}'s
but go through an example to illustrate the \texttt{er} quoter
function. Consider for instance our enriched expression example from
Section~\ref{sec:extChromar}, giving alternative rate functions depending on the
current number of leaves in the plant:
$$
f \; \mathbf{when} \; nl  > 10 \; \mathbf{else} \; f'
$$
where $nl$ is an observable for the number of leaves. This will be written as:
\begin{center}
\begin{BVerbatim}
rateF = [er| f when $nl$ > 10 else f'|]
  where
    nl = [er| select Leaf{age=i,mass=m}; aggregate (count. count + 1) 0 |]
\end{BVerbatim}
\end{center}
Note that the \texttt{\$nl\$ > 10} to embed an enriched expression inside a
normal Haskell expression instead of the
$\mathrm{n} \; \mathbf{where} \; n \; \mathbf{is} \; \mathrm{nl}$ construct from
our abstract syntax. The number of leaves observable ($nl$) is given using
Haskell's \texttt{where} clause. The \texttt{f} and \texttt{f'} functions come
from the outside Haskell environment. The above \texttt{rateF} quote will be
translated to:
\begin{center}
\begin{BVerbatim}
[er| f when $nl$ > 10 else f'|] = Er {
  at = \s t -> if at [er|$nl$ > 10|] s t
                 then at [er|f|] s t
                 else at [er|f'|] s t }
               
[er|$nl$ > 10|] = Er { at = \s t -> at nl s t > 10 }
[er|f|] = Er { at = \s t -> f }
[er|f'|] = Er { at = \s t -> f' }
\end{BVerbatim}
\end{center}
The first equality is an implementation of the semantics of conditional
expressions in Section~\ref{sec:extSem}.  The rest follow the semantics of
ordinary expressions.  Note how we unfold the enriched expression,
$\mathrm{nl}$, to its evaluation (\texttt{at nl s t}) to achieve the same effect
as evaluating the expression in an extended environment (as defined in the
semantics of the $\mathrm{where}$ construct, Section~\ref{sec:extSem}).  The
interpretation of observables will depend on the ordering imposed by our list
representation of multisets.  While this is not consistent with the semantics,
we do not see any problems arising from that as the difference only occurs for
non-commutative fold functions, which, in any case, the programmer should not
employ (see \textit{Semantics of enriched expressions},
Section~\ref{sec:extSem}, for a discussion of the validity of the folding
function when using linearised multisets).  The translation for other types of
enriched expressions similarly follows the semantics given in
Section~\ref{sec:extSem}.


\singlespace


\printbibliography[heading=bibintoc]

%% ... that's all, folks!
\end{document}

\chapter{Framework Models}
\label{chp:fms}
As we have seen in the Introduction plant models are rarely comprehensive enough
to reconstruct the organism and evolutionary ecology models usually abstract the
organism away considering only its development as a numerical variable
(phenology). The first version of the Framework Model
\citep[FMv1;][]{chew_multiscale_2014} is an attempt at reproducing a `digital
Arabidopsis' inspired by functional-structural models that consider molecular
mechanisms as well as development. It only represents vegetative growth, which
limits it applicability in ecological studies (\ssec{fm}). Other models
have considered reproductive success through growth, including for
Arabidopsis. One simplified approach relates growth and fitness only to the
duration of the developmental period and not to its timing in the year, ignoring
environmental influences \citep{prusinkiewicz_evolution_2007}. On the other hand
recent ecology models have taken a more systems view considering the phenology
of the entire lifecycle of Arabidopsis plants in a natural setting and even
scaled this view to the population level using individual-based models
\citep{burghardt_modeling_2015}. However, the organism is only represented in
the usual ecological way as a numerical variable to represent its conceptual
development over time ignoring physical aspects of growth and development.

Here I combine the two approaches to present a population level model where the
organism is present. In particular in this chapter I do the following:
\begin{itemize}
\item Present \emph{FM-lite}, a simplified version of FMv1
  \citep{chew_multiscale_2014} represented in Chromar for studies that do not
    focus on circadian timing. It represents three of the four constituent
    models of FMv1 with modifications for real weather condition and without the
    clock and Flowering Time (FT) genetic circuits (\ssec{fm-lite}).
  \item Present \emph{FM-life}, an extension of the FM-lite to the whole
    Arabidopsis life cycle represented declaratively in Chromar. FM-life
    includes FM-lite to represent vegetative growth, and a new model of
    inflorescence growth including reproduction. As well as growth, timing of
    the developmental stages is given by phenology models, for example for seed
    dormancy. (\sct{compModels}).
\item Scale FM-life to the population level by introducing a clustering
  approximation in order to simulate FM-life tractably at the population scale
  over decades (\sct{pop}).
\item Show simulation results (ecological measure of interest) of the
  population-level model with contrasting environmental and genetic inputs to
  highlight that ecological questions can increasingly be informed by mechanistic
  understanding of growth processes \citep{millar_intracellular_2016,
    doebeli_towards_2017} (\sct{results}).
\end{itemize}

The work in the chapter is under review for the Journal of Experimental Botany
and appears as a bioRxiv pre-print \citep{zardilis2018multi}.

\section{Phenology models in Chromar}
\label{phenMod}
In many phenology models, the simulated plant accumulates a conceptual
development indicator in every time unit as a function of the
contributing environmental factors, until a threshold is reached for
transition to the next developmental stage. For example, in a seed type
\(\mr{Seed}(\mr{dev}:\mr{real})\), the dev attribute measures
development towards germination. A phenology rule for germination
affected by temperature and moisture, starting from dev value $d$,
could be:
\[\mr{Seed}(\ar{dev}{d}) \xrightarrow{} \mr{Seed}(\ar{dev}{d + f(temp,moist)}) \]
On average once every time unit the $\mr{dev}$ attribute of a
particular seed will be increased from the present value, \(d\), by a
function of the contributing factors $temp$ and
$moist$. Further parameters might represent how sensitive the
seed is to the environmental factors. At the threshold
$D_t$, the seed germinates to a plant and resets the
development measure to 0:
%
\[\mr{Seed}(\ar{dev}{d}) \xrightarrow{} \mr{Plant}(\ar{dev}{0}) \, \lbrack d > D_{t}\rbrack \]
%
where the expression inside the square brackets is used to indicate
conditional activity of the rule. The rule is active only when the
expression evaluates to true.

\section{FM-life: the component models}
\label{sec:compModels}
The models presented here represent the full life cycle in three stages: seed
dormancy (A, left panel, Figure~\ref{fig:overview}), vegetative growth up to
flowering (B, left panel, Figure~\ref{fig:overview}), and the reproductive stage
up to seed dispersal (C, left panel, Figure~\ref{fig:overview}). Each model (A,
B, C) includes a phenology component that represents only timing
(Section~\ref{phenMod}). The vegetative and reproductive
stage models also represent biomass growth at the organ level, based on the
carbon budget of the plant. I varied genetic parameters that affect only the
timing components of A (seed dormancy, \(\psi_{i}\)) and B (floral repression
during vegetative growth, \(f_{i}\)), for comparison to
\citet{burghardt_modeling_2015}.  Each parameter value for an individual plant
can be fixed or selected probabilistically from a distribution as described
\citep{burghardt_modeling_2015}. The three models were integrated in a whole
life-cycle model of one plant (FM-life), and then extended to a population of
such plants.

\begin{figure}[tb]
  \centering
\includegraphics[width=\textwidth]{figures/modelOverviewFig/modelOverview.eps}
\caption{Overview of the FM-life and population models used in this study. Left
  Overview of the models used for the growth and timing components for the three
  developmental stages: seed dormancy (A), vegetative period (B), and
  reproductive period (C) and Right Sketch of the population level model. Inputs
  to the model are the distribution of values of the two genetic parameters
  (\(f_{i}\), \(\psi_{i}\)) and weather data from some location for a number of
  years. The output is some population measure of interest, an example might be
  the total number of plants after \(k\) years.}
\label{fig:overview}
\end{figure}

\subsection{Seed dormancy model (A)}
\label{seed-dormancy-model-a}

The seed dormancy model is the Chromar version of the model of
\citet{burghardt_modeling_2015}, which is based in turn on
\citet{alvarado_hydrothermal_2002}. It represents the development of a
newly-dispersed seed from \(dev = 0\) to a threshold value, \(D_{g}\), where the
seed germinates. Above baseline levels of temperature $T_b$ and of moisture (see
below), increasing moisture and temperature speed the progress towards
germination. The additional developmental units added (hydrothermal units,
\(\mr{htu}\)) at every time unit are described by:
%
\[\mr{htu}( t ) = \ \left\{ \begin{matrix}
( \Psi( t ) - \Psi_{b}( t ) ) \cdot ( T( t ) - T_{b} ) \\
\ ( \Psi( t ) - \Psi_{b}( t ) )\  \cdot ( T_{o} - T_{b} ) \\
0 \\
\end{matrix} \right.\ \ \begin{matrix}
\mr{\ \ if\ \ }T_{b} < T( t ) \leq T_{o}\ \mr{and}\ \Psi_{b}( t ) < \ \Psi(t) \\
\mr{if\ \ T}( t ) > T_{o}\ \mr{and}\ \Psi_{b}( t ) < \ \Psi( t ) \\
\mr{otherwise} \\
\end{matrix}\]
%
where \(\Psi(t)\) (in MPa) and \(T(t)\) (in \textdegree C) give the moisture
(water potential) and temperature levels at time \(t\) respectively. The
definition distinguishes between operating in suboptimal and supraoptimal
temperatures (below or above $T_{o}$ respectively). The baseline
moisture is used to represent the dormancy level of the seed. If \(\Psi_{b}\) is
high, the seed accumulates htu slowly for a given set of environment conditions,
whereas if \(\Psi_{b}\) is low, development is faster in the same
conditions. From an initial dormancy level, \(\psi_{i}\), seeds lose dormancy
(\(\Psi_{b}\) becomes smaller) over time at a rate $r$ that is also a function
of the environmental conditions, moisture and temperature, and represents the
observed process of after-ripening.  \(\psi_{i}\) is also used to represent the
genetic effect on dormancy, where high \(\psi_{i}\) represents stronger
dormancy.

In Chromar, the \(\mr{Seed}\) type captures information about the seed
development process:
\(\mr{Seed}(\mr{gntp}:(\mr{real},\mr{real}),\mr{dev}:\mr{real},r:\mr{real})\).  The
\(\mr{gntp}\) attribute stores the genotype of the organism, \(\psi_{i}\) (seed
dormancy level) and \(f_{i}\) (floral repression level), which is passed on to
the agents representing the later stages of development and transmitted
unchanged to the next generation.  \(\mr{dev}\) stores the cumulative
development indicator (sum of \(\mr{htu}\) up to the current timepoint), and
\(r\) stores the after-ripening up to the current timepoint. The development
rule is the following:
\begin{equation*}
\begin{split}
& \mr{Seed}(\ar{gntp}{a}, \ar{dev}{d}, \ar{r}{r}) \xrightarrow{1.0} \\ 
&\mr{Seed}(\ar{dev}{d + htu(temp, moist, f(r,temp,moist), a.\psi_i)}, 
\ar{r}{f(r, temp, moist)})
\end{split}
\end{equation*}
where \(\mr{temp}\) and \(\mr{moist}\) are fluents describing
temperature and moisture. We use the 'dot' (\(.\)) operator for
accessing the two genetic parameters of the \(\mr{gntp}\) attribute.
The following rule represents germination, starting the vegetative
stage:
\begin{equation*}
\begin{split}
& \mr{Seed}(\ar{gntp}{a},\ar{dev}{d},\ar{r}{r} ) \xrightarrow{1.0} \\
& \mr{Plant}(\ar{gntp}{a}), \mr{Root}(\ldots),\mr{Leaf}( \ldots ), \mr{Leaf}(\ldots) \, \, \lbrack d > D_{g}\rbrack
\end{split}
\end{equation*}
The abstract \(\mr{Plant}\) agent represents the plant at the vegetative stage,
along with agents for the root and the two cotyledon leaves. The initial
configuration of the organs at germination is as introduced by
\citet{chew_multiscale_2014}. Note that the genotype attribute is passed from
seed to emerged plant unchanged.

In order to illustrate the functioning of the model I show some illustrative
simulations of the $\mr{htu}$ accumulation over a year for real weather data in
two locations in Europe, Valencia in Spain and Oulu in Finland, and for two
values of the initial base moisture level, $\psi_i$. In the case of Valencia we
can see that there are two periods during the year that are favourable for
development towards germination, Spring and Autumn (Figure~\ref{fig:seedInd}
C). Development progresses faster in the low dormancy case ($\psi_i=0$) than in
the high dormancy case ($\psi_i=2.5$) as expected. The constraints of
development to two periods of the year and the binary threshold to germination
mean that sometimes if a seed misses the window for germination it has to wait
until the next favourable season (Figure~\ref{fig:seedInd} A).
\begin{figure}[tb]
  \centering
  \includegraphics[width=\textwidth]{figures/phenInd/seed.eps}
  \caption{Indicative runs of the seed dormancy model for two location and two
    genetic backgrounds (two values of the initial seed dormancy level) run with
    a daily timestep. A Two runs of the seed dormancy model for yearly weather
    in Valencia (C) for two different value of $\psi_i$. The red line is the
    germination threshold. Stochasticity comes from the model representation and
    can sometimes be amplified by the binary threshold and weather constraints
    to lead to different germination behaviour of otherwise identical seeds (H
    case) B Same as A for weather data from Oulu (D) C Yearly weekly averages of
    moisture and temperature levels in Valencia D Weather data as in (C) for
    Oulu.  }
  \label{fig:seedInd}
\end{figure}

In the case of Oulu only summer provides favourable conditions for germination
(Figure~\ref{fig:seedInd} D). In the low dormancy case there is progress to
germination during the summer while in the high dormancy case there is no
development (Figure~\ref{fig:seedInd} B).

\subsection{Vegetative growth model (FM-lite) (B)}
\label{subsec:fm-lite}
For the vegetative stage I introduce a simplified version of FMv1
\citep{chew_multiscale_2014} for use in studies that do not focus on circadian
timing. FM-lite has three constituent models represented in Chromar with
modifications to environmental responses (see below), and without the fourth,
circadian clock and FT gene expression model of FMv1.

\subsubsection*{Timing}
\label{timing}

The timing component is the simpler flowering phenology model of
\citet{wilczek_effects_2009} rather than the augmented version in FMv1
(combination of \citet{chew_augmented_2012} with
\citet{salazar_prediction_2009}). Vegetative development extends from \(dev = 0\)
to a threshold value, \(D_{f}\), where the plant flowers. The main contributing
environmental factors are photoperiod, ambient temperature and vernalisation,
giving the modified photothermal units, \(\mr{mptu}\), at a time \(t\) as:
\[\mr{mptu}(t) = \mr{photoperiod}(t) \cdot \mr{thermal}(t) \cdot \mr{vernalisation}(t)\]
The \(\mr{vernalisation}\) term accounts for both the observed requirement for
a specific duration of exposure to cold and is also used to represent the
genetic effect on the progress towards flowering, modelled as
\(\mr{vernalisation}(t) = f(wc,f_{i})\), where $wc$ is the exposure to
cold accumulated up to \(t\) and \(f_{i}\) is the genetic parameter for the
initial floral repression, as in \citet{wilczek_effects_2009}.

In Chromar, the plant type:
\(\mr{Plant}(\mr{gntp}:(\mr{real},\mr{real}),\mr{dev}:\mr{real},\mr{wc}:\mr{real})\)
includes the genotype attributes as noted above, the development so far
(\(dev)\), and finally the accumulated winter chilling (\(wc)\). The
development rule is then:
\begin{equation*}
\begin{split}
&\mr{Plant}(\ar{gntp}{a},\ar{dev}{d}, \ar{wc}{w}) \xrightarrow{1.0} \\ 
&\mr{Plant}(\ar{dev}{d + mptu( temp, dl, a.f_{i}, w )},\ar{wc}{f(w)})
\end{split}
\end{equation*}
where $temp$ and $dl$ are fluents for temperature and day
length respectively, and $w$ is the present value of $\mr{wc}$. The transition to a
flowering plant, \(\mr{FPlant}\), follows:
$$
\mr{Plant}(\ar{gntp}{a}, \ar{dev}{d}, \ar{wc}{w}) \xrightarrow{1.0} \mr{\ FPlant}(\ar{gntp}{a}) \, \, \lbrack d > D_{f} \rbrack
$$

I again plot some illustrative runs of this model in two different locations in
Europe, Valencia and Oulu, and for two values of the initial floral repression,
$f_i$ (Figure~\ref{fig:vegInd}). In Valencia there are favourable conditions
throughout the year for progress towards flowering even in winter and the floral
repression level does not affect the development much (Figure~\ref{fig:vegInd}
A, C). In Oulu progress happens rapidly in spring when temperatures rise above 0
\textdegree C coinciding with longer photoperiods (Figure~\ref{fig:vegInd} B,
D).

\begin{figure}[tb]
  \centering
  \includegraphics[width=\textwidth]{figures/phenInd/plant.eps}
  \caption{Indicative runs of the flowering model for two locations (Valencia,
    Oulu) and two genetic backgrounds ($f_i=0.598$, $f_i=0.737$) run with a
    daily time-step. A Two runs of the flowering phenology model with yearly
    weather for Valencia for two values of the initial floral repression
    level. The red line indicates the flowering threshold B Similar to (A) for
    Oulu C Weekly averages of photoperiod and temperature for a year in Valencia
    D Weekly averages of photoperiod and temperature for a year in Oulu.}
  \label{fig:vegInd}
\end{figure}


\subsubsection*{Growth}
As in FMv1 \citep{chew_multiscale_2014}, the growth component includes a carbon
budget for the plant from \citet{rasse_leaf_2006}, which in turn includes
photosynthesis rate equations based on the
\citet{farquhar_biochemical_1980}. Growth at the organ level (rosette leaves and
root) is represented based on the Greenlab model
\citep{christophe_model-based_2008}. We will consider a sucrose carbon pool
(\(c\)), a starch carbon pool (\(s\)), and one pool for the biomass of the root
and each of the rosette leaves (left panel, Figure~\ref{fig:fmv1}). In Chromar
we have the following agents to store the state (amount of carbon, or total
biomass) of these pools:
\begin{figure}[p]
\centering
\resizebox{\textwidth}{!}{
\begin{tabularx}{1.1\textwidth}{c|ll} 
\toprule
\addlinespace[0.25cm]
\textsf{\emph{Flows}} & & \textsf{\emph{Rules}} \\ 
\multirow{10}[50]{*}{\includegraphics[width=0.45\textwidth]{figures/agentFlowsFig/agentFlows.pdf}} & \textsf{growth}&  $\mathrm{Leaf}(\ar{m}{m},\ar{i}{i}), \mathrm{Cell}(\ar{c}{c}) \xrightarrow{d(i, t)}$ \\
& & $\mathrm{Leaf}(\ar{m}{m+g(m)}), \mathrm{Cell}(\ar{c}{c-g(m)})$ \\ 
& & $[c-g(m) > c_b]$ \\
\addlinespace[0.2em]
& & similarly for other organs \\ 
\addlinespace[-0.5em]
& & \\
& \textsf{maint} & $\mathrm{Leaf}(\ar{m}{m}), \mathrm{Cell}(\ar{c}{c}) \xrightarrow{}$ \\
& & $\mathrm{Leaf}(\ar{m}{m}), \mathrm{Cell}(\ar{c}{c-f_m(m)})$ \\
& & $[c-f_m(m) > 0]$ \\
\addlinespace[0.2em]
& & similarly for other organs \\ 
\addlinespace[-0.5em]
& & \\
& \textsf{mobl} & $\mathrm{Leaf}(\ar{m}{m}), \mathrm{Cell}(\ar{c}{c}) \xrightarrow{}$ \\
& & $\mathrm{Leaf}(\ar{m}{m-f_t(m)}), \mathrm{Cell}(\ar{c}{c+f_t(m)})$ \\
& & $[c \leq c_b]$ \\
\addlinespace[0.2em]
& & similarly for other organs \\ 
\addlinespace[-0.5em]
& & \\
& \textsf{assim} & $\mathrm{Cell}(\ar{c}{c}, \ar{s}{s}) \xrightarrow{\phi(e(t))}$ \\
& & $\mathrm{Cell}(\ar{c}{c+k_c a_{\mathrm{ros}}}, \ar{s}{s+(1-k_c) a_{\mathrm{ros}}}) $ \\
& & $[c \leq c_b]$ \\
\addlinespace[-0.5em]
& & \\
& \textsf{sdegr} & $\mathrm{Cell}(\ar{c}{c}, \ar{s}{s}) \xrightarrow{}$ \\ 
& & $\mathrm{Cell}(\ar{c}{c+f_d(e(t)}), \ar{s}{s-f_d(e(t)})$ \\
& & $[\mathrm{night} \land s-f_d(e(t)) > 0]$ \\ 
\addlinespace[-0.5em]
& & \\
& \textsf{leaf cr} & $\mathrm{Plant}(), \mathrm{VAxis}(\ar{n}{n}) \xrightarrow{f_o(e(t))} $ \\ 
& & $\mathrm{Plant}(), \mathrm{VAxis}(\ar{n}{n+1}), \mathrm{LAxis}(\ar{i}{n+1}),$ \\
& & $\mathrm{Leaf}(\dots)$ \\
\addlinespace[0.25cm]
\bottomrule
\end{tabularx}} %}
\caption{An overview of the dynamics on FM-lite (growth component of vegetative
  stage). The dynamics take the form of flows between different reservoirs of
  carbon, here shown in a graphical way in the left panel with the corresponding
  Chromar rules on the right.  }
\label{fig:fmv1}
\end{figure}
%
\begin{itemize}
\item
  \(\mr{Cell}(c,s:\mr{real})\) An agent that stores the amount of
  carbon in the sucrose (\(c\) attribute) and starch pools (\(s\)
  attribute). The amounts are carbon totals at the whole plant level.
\item
  \(\mr{Leaf}(\mr{m}:\mr{real},\mr{i}:\mr{int})\) An agent that represents a
  rosette leaf. It has attributes for its mass (\(m\)) and its index of
  appearance (\(i\)).
\item
  \(\mr{Root}(\mr{m}:\mr{real})\) An agent that represents the root with
  an attribute for its mass (\(m\)).
\end{itemize}

For each organ we have a growth flow from the sucrose carbon pool to the mass of
the organ (\textsf{growth} rule, Figure~\ref{fig:fmv1}. The growth amount
depends on the demand function of the organ (\(d(i,t)\) rule rate function) and
its 'sink strength' (\(g(m)\)), which varies among organs. The value of the
demand function varies over time between 1 (maximum demand) and 0 (no demand) at
the end of the expansion period of the organ. The amount of carbon requested by
an organ at every time unit is \(g( m ) \cdot d(i,t)\). Depending on the
metabolic status of the whole plant (level of \(c\) pool) and the requests from
other organs, an organ will receive either the full expected amount or a portion
of it.

A flow in the opposite direction (\textsf{mobl} rule, Figure~\ref{fig:fmv1})
represents carbon mobilisation from the organs if the central sucrose pool
(\(\mr{Cell}(c)\)) is reduced to a critical level. Thus each organ can be either
a net sink or source of carbon. For each organ, we also have a flow leaving the
system from the \(c\) pool for the cost of the maintenance respiration and other
processes of the organ (\textsf{maint} rule,
Figure~\ref{fig:fmv1}). Photosynthetic carbon fixation is represented by the
assimilation process (\textsf{assim} rule, Figure~\ref{fig:fmv1}). The amount of
assimilate at every time unit is the product of the photosynthesis rate, which
is a function of environmental conditions at that time step, and the projected
area of the rosette. Here I use an observable, \(a_{\mr{ros}}\), for the
effective rosette area, which is a function of the global state of the rosette
at the current time (derived from the masses of all the current leaves) and
takes into account the effect of shading, as in
\citet{christophe_model-based_2008}. The carbon partitioning function includes a
baseline partitioning to starch, then support of a target sucrose level, with
excess sucrose supporting growth and a final overflow to additional starch
production, as in \citet{rasse_leaf_2006}. At night, no photosynthesis occurs
and carbon from the starch pools flows to the sucrose pool (\textsf{sdegr} rule,
Figure~\ref{fig:fmv1}). Finally, we have the creation of new leaves, which
impacts the above processes indirectly by creating more demand for growth and
adding maintenance costs (\textsf{leaf cr} rule, Figure~\ref{fig:fmv1}). Leaves
are created by the main apical meristem (\(\mr{VAxis}\) agent) along with an
\(\mr{LAxis}\) agent that can give rise to lateral branches after flowering (see
next section).

It is interesting to note that in FMv1 carbon partitioning between processes and
organs is done deterministically whereas in our Chromar representation,
partitioning is an emergent, stochastic effect of competition for the finite
amount of sucrose carbon in the main reservoir. For example, partitioning of
carbon among organs for growth in FMv1 is done explicitly by dividing the demand
of each organ by the sum of the demands of all other organs
\(g( m ) \cdot \ \frac{d(i,\ t)}{\sum_{}^{}{d(i,t\ )}}\). In the Chromar
representation we do not have this explicit division by the global demand, which
means that the amount of carbon that an organ gets is higher at each growth
event but growth events are rarer because not all growth request are successful
(competition). In order to avoid 'starvation' of organs due to the increased
carbon amount I decreased the growth request size to $g(m)/10$ and increased the
rate of the growth rule to $10 \cdot d(i, t)$. If $c - g(m)/10 > c_b$ then the
request is successful and the organ gets the whole amount for growth otherwise
the request is unsuccessful and the organ gets nothing. The competition recovers
the explicit partitioning of FMv1 albeit with greater variability even with the
smaller request size (see Figure~\ref{fig:comp}).

\emph{Modifications for natural conditions}\\
FMv1 was developed for lab conditions. As an initial approach to reflect
plant responses to the broader range of relevant conditions in nature,
we made the following changes:

\begin{itemize}
\item
  The rate of photosynthesis is set to 0 below 0 °C
\item
  The maintenance cost for an organ is also 0 below 0 °C
\item
  The rate of photosynthesis is affected by soil moisture through
  stomatal closure. The photosynthesis rate is affected by a stomata
  term \(f_{\mr{stom}}(moist)\), which is a simple phenomenological
  function that relates soil moisture and stomatal closure \citep{france_mathematical_1984}.
\end{itemize}

These conservative changes give a lower bound on the effects of natural
weather conditions.

\subsubsection*{Comparison of FM-lite with FMv1}
\label{comparison-of-fm-lite-with-fmv1}

In addition to the weather responses, Wilczek flowering model and emergent
carbon partitioning among organs, our model representation uses the stochastic
rule-based Chromar as opposed to the deterministic Matlab program of FMv1. In
order to compare the model representations, I simulated growth in the two
models for a fixed number of hours in lab conditions, where the modifications to
weather responses have no effect.  The two models were simulated in lab
conditions (22 °C, 12/12 light/dark cycles) for \(800\) growth hours and showed
comparable results (Figure~\ref{fig:comp}). FMv1 was simulated in Matlab while
FM-lite was simulated in the Haskell implementation of Chromar and the results
were averaged over five runs. The rosette mass results are the closest since
they represent the development of multiple Leaf agents, masking the stochastic
effects on each Leaf. The difference between the final rosette mass of the FMv1
and FM-lite (averaged over 5 runs) simulations is within 10\% of the final
rosette mass in FMv1. The stochasticity is more apparent for the root where the
growth curves are further apart. The difference between final root mass in FMv1
and FM-lite (averaged over 5 runs) is \textasciitilde{}20\% of the final root
mass in FMv1. Sucrose carbon levels are also more variable in FM-lite, since the
growth rule (removing sucrose carbon from the central pool) provides organs with
a larger amount but less frequently than the small fixed amount at every time
step in FMv1 (see previous section).

\begin{figure}[tb]
  \centering
  \includegraphics[width=0.9\textwidth]{figures/fmCompFig/fmCompLabs.jpg}
  \caption{A comparison of the original FM implementation (FMv1) with the
    adaptation used in this work (FM-lite) for 800 hours of growth. FM-lite
    simulations were performed in the Haskell implementation of Chromar and
    results were averaged over five runs. FMv1 simulations were carried out in
    the Matlab environment. A Comparison of simulated rosette mass trajectories
    between FMv1 and FM-lite B Comparison of simulated root mass trajectories
    between FMv1 and FM-lite C Comparison of simulated sucrose carbon between
    FMv1 and FM-lite and D Comparison of simulated starch carbon between FMv1
    and FM-lite.}
  \label{fig:comp}
\end{figure}

\subsection{Reproductive stage model (C)}
\label{reproductive-stage-model-c}

\subsubsection*{Timing}
\label{timing-1}

The timing component is a thermal time model from
\citet{burghardt_modeling_2015}, representing the development of the
inflorescence and seed from \(dev = 0\) at flowering, to a threshold value,
\(D_{s}\), where the plant disperses its seeds. Here there is no genetic
input. Moreover, there is no input from the vegetative or growth model from the
same stage so for example a plant with a small rosette that makes very little
fruit will mature its seed at the same time as a very branchy plant with a lot
of fruit. This is a simplification and in fact throughout the FM-life model we
assume independence of timing with growth although the duration of the period
will indirectly affect growth (see indicative simulations below,
Figure~\ref{fig:reprGInd}). The thermal units that accumulate at time \(t\) are
simply the value of the temperature at \(t\) above a base temperature \(T_{b}\):

\[tu(t) = \left\{ \begin{matrix}
T(t) - T_{b} & \mr{if\ }T(t) > T_{b} \\
0 & \mr{otherwise} \\
\end{matrix} \right.\ \]

Writing into Chromar we have an
\(\mr{FPlant}(\mr{dev}:\mr{real})\) type for a flowered plant and
the following rule for its development that follows from the
\(tu\) definition above:
$$
\mr{FPlant}( \ar{dev}{d} ) \xrightarrow{} \mr{FPlant}(\ar{dev}{d + tu(temp)})
$$

Finally, the transition to seed happens when the accumulated development
reaches \(D_{s}\):
\[\mr{FPlant}( \mr{attr} = a,\mr{\ dev} = d ) \rightarrow \mr{Seed}( \mr{attr} = a,\ \mr{dev} = 0,\ r = 0 )\ \ \lbrack d > D_{s}\rbrack\]

Note that the genotype attribute of the parent plant is transferred to
the seeds unchanged.

As in the other phenology models I show some illustrative simulations of this
model in the two standard locations, Valencia and Oulu
(Figure~\ref{fig:reprInd}). In Valencia there is development throughout the year
whereas in Oulu the conditions are only favourable for only part of the year
from spring to autumn where the temperatures are above freezing.

\begin{figure}[tb]
  \centering
  \includegraphics[width=0.8\textwidth]{figures/phenInd/fplant.eps}
  \caption{Indicative runs of the timing to seed dispersal model run with daily
    timestep. A Since temperature is favourable in Valencia throughout the year
    development takes place continuously even in the winter. B In Oulu
    development towards seed dispersal can happen only in late spring, summer,
    and early autumn when temperatures are favourable. Development then starts
    accumulating rapidly in late spring. C Weekly temperature averages for year
    in Valencia D Weekly temperature averages for a year in Oulu.}
  \label{fig:reprInd}
\end{figure}

\subsubsection*{Growth}
\label{growth-1}

I developed the growth component of the reproductive stage model, which is
loosely related to the Greenlab model \citep{christophe_model-based_2008}. The
metabolic processes affecting the carbon budget of the plant are the same as in
vegetative growth but with additional organ types to represent the Arabidopsis
inflorescences. Organs appear in units (metamers) with a metamer
identifier. Each growth unit on the main axis consists of an internode (stem
between leaves), a leaf, and a lateral meristem that can give rise to a lateral
axis. I consider only the primary and secondary axes (lateral branches), thus
metamers on the lateral axis lack a further lateral meristem. All fruits on an
axis are represented on its last metamer, replacing the leaf; this metamer also
lacks a meristem.  Two indices represent metamer position: the index of the
metamer along its axis and the index of the parent metamer along the primary
axis (left panel, Figure~\ref{fig:reprModel}). We define the following new agent
types to represent this structure:

\begin{itemize}
\item
  \(\mr{INode}(i,pi:\mr{int},m:\mr{real})\) to represent the
  internode (stem between successive leaves). Attribute \(i\) is the
  temporal index of appearance in its axis (primary or lateral) and
  attribute \(\mr{pi}\) is the parent primary metamer. The cotyledons
  have indices 1 and 2 on the primary axis, for example.
\item
  \(\mr{LLeaf}(i,pi:\mr{int},m:\mr{real})\) to represent a leaf on
  the lateral axes.
\item
  \(\mr{Fruit}(i,pi:\mr{int},m:\mr{real})\) to represent a fruit
  on the axis.
\end{itemize}

\begin{figure}[p]
\centering
\resizebox{\textwidth}{!}{
\begin{tabularx}{1.05\textwidth}{c|ll} 
\toprule
\addlinespace[0.25cm]
\textsf{\emph{Structure}} & & \textsf{\emph{Rules}} \\ 
\multirow{20}[0]{*}{\includegraphics[scale=0.7]{figures/struct.pdf}} & \textsf{vgrowth}&  $\mathrm{FPlant}, \mathrm{VAxis}(\ar{n}{n}) \xrightarrow{f(e(t))}$ \\
& & $\mathrm{FPlant}, \mathrm{VAxis}(\ar{n}{n+1})$ \\
& & $\mathrm{INode}(\ar{i}{n+1}, \ar{pid}{V \, n+1}),$ \\ 
& & $\mathrm{Leaf}(\ar{i}{n+1}, \dots)$ \\
& & $[n < v_{\mathrm{max}}(n_f)]$ \\
\addlinespace[-0.15cm]
& & \\
& \textsf{vfruit} & $\mathrm{FPlant}, \mathrm{VAxis}(\ar{n}{n}) \xrightarrow{f(e(t))}$\\
& & $\mathrm{FPlant}, \mathrm{VAxis}(\ar{n}{n+1}), \mathrm{Fruit}(\ar{pi}{V \; n+1})$ \\
& & $[n = v_{\mathrm{max}}(n_f)]$ \\
\addlinespace[-0.15cm]
& & \\
& \textsf{lgrowth} & $\mathrm{FPlant}, \mathrm{LAxis}(\ar{i}{i},\ar{n}{n}), \mathrm{Fruit}(\ar{pi}{V \, i+1}) \xrightarrow{f(e(t))}$ \\
& & $\mathrm{FPlant}, \mathrm{LAxis}(\ar{n}{n+1}), \mathrm{Fruit},$\\
& & $ \mathrm{INode}(\ar{i}{n+1}, \ar{pi}{L \, i}), \mathrm{LLeaf}(\ar{i}{n+1}, \ar{pi}{L \, i})$ \\
& & $[n < l_{\mathrm{max}}(n_f, i) \wedge t_{\mathrm{del}}]$ \\
\addlinespace[-0.15cm]
& & \\
& \textsf{lfruit} & $\mathrm{FPlant}, \mathrm{LAxis}(\ar{i}{i}, \ar{n}{n}) \xrightarrow{f(e(t))}$ \\
& & $\mathrm{FPlant}, \mathrm{LAxis}(\ar{n}{n+1}), \mathrm{Fruit}(\ar{pi}{L \, i})$ \\
& & $[n = l_{\mathrm{max}}(n_f, i)]$ \\
& & \\
& & + metabolic processes (see Figure~\ref{fig:fmv1}) \\
\addlinespace[0.25cm]
\bottomrule
\end{tabularx}}
\caption{Overview of the structural part of the growth component of the
  reproductive stage model. The numbering scheme used to keep track of the
  positions of the organs in the inflorescence architecture is shown on the
  left. On the right the Chromar rules used to grow a structure like the one on
  the left panel (see main text for details).}
\label{fig:reprModel}
\end{figure}

The maximum number of inflorescence metamers on the main axis is taken to be
20\% of the number of rosette leaves at flowering time (\(n_{f}\)) and given by
\(v_{\max}(n_{f})\) \citep{pouteau_significance_2009}. The maximum number of
growth units on each lateral axis is given by \(l_{\max}(i)\), a decreasing
function of the index of the lateral axis starting from a maximum of 6 at the
axis after the cotyledons (index 3) and going to a minimum of 1 at the topmost
lateral branch \citep{mundermann_quantitative_2005}. The topmost lateral axis
can only appear with a delay after the apical fruit has appeared on the primary
axis. Each successive lateral branch going down can only start developing with a
delay after the fruit of the axis above it has appeared. The delay associated
with lateral axis growth, given in the rules by \(t_{\mr{del}}\), is a function
of the metabolic state of the plant, as described
\citep{christophe_model-based_2008}.

The new organ types have associated sink strengths and demand functions.  The
cauline leaves on the main axis contribute to the photosynthetically active area
and can shade the rosette leaves underneath them. The lateral leaves contribute
to photosynthesis without shading. Internodes and fruits do not contribute to
photosynthesis. Seeds are not directly represented, so a birth function \(b(m)\)
is required to calculate the number of seeds for a given fruit mass \emph{m} at
seed dispersal time, as described below.

I plot two indicative runs of the vegetative + reproductive stage models at two
different temperatures to show the interplay between growth and seed dispersal
timing (Figure~\ref{fig:reprGInd}). At 22 \textdegree C we get two fruit agents and
one lateral branch at seed dispersal (1379 hours after sowing) with a total of
14 out of 25 non-senesced leaves on the main axis (Figure~\ref{fig:reprGInd}A,
B). While the photosynthesis rate is higher at 25\textdegree C, the growth
period is shorter, which means less growth time in the exponential growth period
resulting in a smaller plant and less fruit mass (Figure~\ref{fig:reprGInd} C,
D).

\begin{figure}[p]
  \centering
  \includegraphics[width=\textwidth]{figures/reprInd/reprSim.eps}
  \caption{Indicative runs of the vegetative + reproductive stage models
    (germination to seed dispersal). A 5-run averages of growth period indicative
    biomass traces (whole plant, main axis, lateral axes) for growth at 22
    \textdegree C (and lab conditions, light
    intensity=120$\mu \mathrm{mol} \cdot m^{-2} \cdot s^{-1}$, CO2=420ppm). B
    Sketch of the structure of the inflorescence at seed dispersal (1379 h) C Growth
    period simulations with same conditions as in (A) but at 25 \textdegree C. D
    Same as in (B) for growth at 25 \textdegree C.}
  \label{fig:reprGInd}
\end{figure}

\section{From the lifecycle to the population}
\label{sec:pop}
The Chromar framework allows us simply to concatenate the rules of timing and
growth components of the three models above, to represent the whole life
cycle. Then given an initial state with the genetic attributes of the plant
(\(\mr{gntp}\) attribute of agents) and the environmental conditions for a
particular location, \(e(t)\), we can simulate an entire life cycle from seed to
seed. The timing components of the model give us the timing within the year of
the growth period (vegetative + reproductive stages) and therefore the
environmental conditions that the plant is exposed to during growth. The growth
components predict growth at the individual organ level with these environmental
conditions and therefore give us the environmentally determined seed number
given by the \(b(m)\) function, which can depend on growth (mass $m$ of fruits).


\subsection{Population level model and plotting conventions}
\label{population-level-model-and-plotting-conventions}

\begin{figure}[p]
\centering
\includegraphics[width=\textwidth]{figures/popLevelFig/popLevel1.pdf}
\caption{The two-stage simulation of the population level model. A The
  distributions of developmental events (germination, flowering) from the
  phenology-only simulation of the population model (C) with the identification
  of two clusters representing two distinct strategies (blue, red) B The 25-th,
  50-th, and 75-th percentiles of the distributions of developmental events of
  the two clusters from A (cluster labels in y-axis). Germination time
  distributions are in brown, flowering time distributions are in green and seed
  dispersal distributions in red. C Illustration of the phenology only simulation
  with growth-independent birth function b(m)=1 (each plant makes a single
  seed). The choice of, $\psi_i$ (seed dormancy level) and $f_i$ (initial floral
  repression) is done probabilistically through a normal distribution D Results
  of simulations of the growth models for the median dates of developmental
  events (B) for the clusters identified in (A). $m_1$, $m_2$ are the final
  fruit masses for a plant with median developmental event timings for clusters
  1 and 2 respectively (as identified and shown in A, B) E Illustration of the
  assignment of fruit masses to the lifecycles of the phenology-only simulation
  (C) according to their clusters. $m_1$, $m_2$ refer to final fruit masses for
  a median plant in clusters 1 and 2 (from growth simulations in D and same 2
  clusters in A, B). This recovers the full branching population process where
  the horizontal bars indicate a lineage dying out ($b(m)=0$ at that generation
  -- no reproduction) and the potential rest of the lineages is shaded. The
  breadth of the ellipses indicates the reproductive success when there is
  reproduction ($b(m) > 0$).}
\label{fig:popModel}
\end{figure}

Since FM-life estimates the number of seeds at the end of the life cycle, these
can initiate multiple independent copies of the model in the next generation. We
then have a classical evolutionary birth process, sometimes called a branching
process since it unfolds in tree-like way. The potential number of individuals
in generation \(i\), \(n_{i}\), is equal to the sum of the number of seeds
produced by the individuals in the previous generation (see Discussion). Dormant
seed never die in the model and may germinate after several years
\citep{burghardt_modeling_2015}.

Since we are using an individual-based model, \(n_{i}\) becomes computationally
prohibitive to simulate over decades of population growth. In order to overcome
this limitation, I simulated the timing (phenology) and growth components
sequentially and used conservative birth functions \emph{b}(\emph{m}) where $m$
is the final fruit mass at seed dispersal. Figure~\ref{fig:popModel} introduces
the plotting conventions for these results. The timing components were first
simulated with \(b(m) = 1\), such that each plant makes one seed regardless of
fruit mass (since it is not available), as in \citet{burghardt_modeling_2015}
but with a daily time-step. The phenological simulation results in an unbranched
sequence of developmental stage timings for each lineage
(Figure~\ref{fig:popModel} C). The simulation results for several decades
typically revealed a small number of life cycle growth strategies, from clusters
of individual life cycles. The clusters were generated using \(k\)-means
clustering, where \(k\) is chosen by visual inspection of the life cycle plots
(Figure~\ref{fig:popModel}A). Alternative clustering approaches might be an area
for future work. Figure~\ref{fig:popModel}A shows the distribution over a year
of all individual life cycles that conformed to two contrasting life cycle
strategies under environmental conditions for Valencia (see Results).  Cluster
membership depends on the dates and durations of multiple developmental
stages. This is hard to visualise, because the timing of any single
developmental stage partially overlaps among different
strategies. Figure~\ref{fig:popModel}B therefore summarises the median dates of
all three developmental transitions in each strategy, here illustrated by 1. a
summer growth strategy and 2. a winter growth strategy. In the next stage, the
growth models were simulated once per cluster, with the environmental conditions
associated with the typical timing of that cluster (median vegetative and
reproductive stages). This returns the typical biomass of organs over time,
including the fruit mass at seed dispersal (m1 for cluster 1, m2 for cluster 2;
Figure~\ref{fig:popModel}D). Finally, each life cycle is assigned the fruit mass
\emph{m} associated with its cluster, and thereby a growth-based, birth
function \emph{b}(\emph{m}) that evaluates to 0 in some cases (no
reproduction). Thus, the second stage recovers a version of the branching
lineage tree, where some lineages die out (Figure~\ref{fig:popModel}E).

The output population measure I will use for the results in the next section and
to compare scenarios is the total population of plants over all lineages over
all generations. I define the population measure for one lineage as follows:
\begin{definition}[single-lineage population measure]
  The \emph{single-lineage population measure}, $p_i$, for a lineage $i$ is the
  total number of plants over all generations.
\begin{equation*}
p_i = 1 + \sum_{k=2}^{n} \, (\prod_{j=1}^{k-1} b(m_{ij}))
\end{equation*}
where $n$ is the total number of generations, $m_{ij}$ is the final fruit mass
of the plant (given by its cluster membership) in generation $j$ of lineage $i$,
and $b(m)$ is a birth function returning the number of offspring given a fruit
mass.
\end{definition}
For example, consider lineage $1$ with three generations starting
with a plant with final fruit mass \(m_{11}\). For the next generation we have
\(b(m_{11})\) plants and then \(b(m_{11}) \times b(m_{21})\). The population measure
for that lineage is \(1 + b(m_{11}) + b(m_{11}) \times b(m_{21})\).

The population measure for multiple lineages starting from multiple plants in
the initial population is the sum of the population measures of all the
lineages.
\begin{definition}[multiple-lineage population measure]
  The \emph{multiple-lineage population measure}, $P$, for $N$ lineages is the
  sum of the single-lineage population measure for all of them.
  $$
    P = \sum_{i=1}^{N} p_i
  $$
\end{definition}

For the results in the next section and throughout the rest of this chapter I
report, unless otherwise stated, the multiple-lineage population measure.  The
above definitions require a birth function, which I use in a very simple form,
as follows:
\[b( m ) = \left\{ \begin{matrix}
1 & \mr{if\ }m > m_{0} \\
0 & \mr{otherwise} \\
\end{matrix} \right.\ \] A plant produces one seed or none, the latter in life
cycles with fruit mass at seed dispersal $m$ less than a threshold
\(m_{0}\). Below, I make some conservative choices for the value of the
reproductive threshold value, \(m_{0}\), to explore the effect on the output
population measure.

To illustrate the resulting process consider the populations plots over time in
Figure~\ref{fig:pops} for two locations, Valencia and Oulu, and two combinations
of genetic parameters for dormancy level and initial floral repression. I use
the above binary birth function. This does not show the cumulative population
measure from above but rather how many plants are present at any instance for a
period from year 15 to year 60 of a 60 year simulation. Since the output of the
birth function is less than 1 the population either stays constant or decreases
over time as lineages die out when a plant in a generation does not have fruit
mass above the threshold value. For any other function that at least
distinguishes the reproductive success of individuals (\eg b(m) = \{0, 1, 2\}) I
could have used the population at the end of the simulation as the population
measure. With the current birth function, however, it is impossible to
distinguish between strategies so I use the cumulative population measure (see
above) that carries, even if only conservatively, the reproductive success over
time.

\begin{figure}[tb]
  \centering
  \includegraphics[width=\textwidth]{figures/popExps/pops.eps}
  \caption{Populations plots over time for two locations, Valencia and Oulu, and
    two genetic backgrounds (two values of seed dormancy and initial floral
    represssion levels). A Number of plants present at any instance from year 15
    to year 60 of a 60-year simulation in Valencia. B Similar to A but in Oulu.}
  \label{fig:pops}
\end{figure}

Finally, I distinguish three sources of variability in the population model: (i)
weather varies between years, (ii) genetic parameters can vary among the initial
population if their values are chosen probabilistically, and (iii) simulation
results vary due to stochasticity in the model representation.

\subsection{Weather data}
\label{weather-data}
For the phenology model simulations I used the weather data that accompanied the
\citet{burghardt_modeling_2015}, available from a Dryad repository
\citep{burghardt_data_2014}. In this dataset weather inputs over 60 years were
generated stochastically for four locations in Europe: Halle, Valencia, Norwich,
and Oulu. The weather inputs include values for temperature (in \textdegree C),
moisture (water potential in MPa), and daylength (in hours).

For the growth simulations I used weather data from the ECMWF ERA-Interim
dataset over the years 2010-2011 \citep{dee_era-interim_2011}.  A program
provided by Mathew Williams and Luke Smallman (School of GeoSciences, University
of Edinburgh) that uses methods from \citet{williams_primary_2001} was used to
generate hourly inputs given daily averages from the dataset for temperature and
radiation. For the soil moisture input used in the photosynthesis rate
calculation I used a daily average of soil moisture values from the dataset and
assumed that is constant throughout the day (swvl parameters in the ERA
dataset). The soil moisture parameter here is a number in arbitrary units from 0
to 1 that represents the 'wetness' of the soil while the soil moisture used
above measure water potential and is given in MPa.

\section{Results}
\label{sec:results}
The population of FM-life models (see previous section) allows us to test how
growth processes that alter reproductive success affect the life history
strategies of Arabidopsis growing in different environmental conditions
(location) and with different genetic parameters in the initial population. We
can therefore explore the genotype x environment interaction, using a population
measure. To illustrate this potential, I compare simulation results for two
previously-studied locations, Valencia (Spain) and Oulu (Finland), and two
opposing combinations of genetic parameters, high seed dormancy /high floral
repression (HH) and low seed dormancy /low floral repression (LL). Within an
initial population of 100 seeds, the seed dormancy levels, \(\psi_{i}\), were
assigned probabilistically, sampling from a normal distribution with mean
\(0.0\) and standard deviation, \(1\), for the Low dormancy level (L) and mean
\(2.5\) with the same standard deviation for the High dormancy case (H)
(Figure~\ref{fig:psisCorVal} A, C). Floral repression was fixed at either
\(0.598\) for the Low level (L) and \(0.737\) for the High level (H), values
that were chosen to reflect the behaviour of natural populations of Arabidopsis
in \citet{wilczek_effects_2009}. Both parameter choices follow
\citet{burghardt_modeling_2015}. The simulation time period was 60 years and, as
in \citet{burghardt_modeling_2015}, we discarded the first 15 years of the
simulation to focus on stable life history strategies. A key difference from the
earlier work is that even our conservative choice of birth function (see
Methods) allows some lineages to die out.

\subsection{Valencia}
\label{valencia}
Figure~\ref{fig:valComp} shows the results of the two-stage simulation for a population
of the LL genotype in Valencia (Figure~\ref{fig:valComp}A). We identify four possible
life history strategies based on the timing (phenology) components of
the FM-life model:
\begin{enumerate}
\def\labelenumi{\arabic{enumi}.}
\item
  \emph{summer-only strategy} where the entire growth is in the summer.
  The growth period is quite short and the conditions unfavourably hot
  and dry. In the growth simulation, the rosette leaves senesce before
  the reproductive stage (blue curve). The drought effect on
  photosynthesis severely limits the carbon available for fruit mass
  (red curve).
\item
  \emph{spring strategy} where the entire growth period is in the
  spring. The growth period is only slightly longer than the
  \emph{summer-only} strategy but it falls in more favourable weather
  conditions. The rosette lifetime extends beyond flowering to support
  fruit growth, which combined with favourable weather gives high fruit
  mass.
\item
  \emph{winter-repr strategy} spans the winter/early spring period. A
  short vegetative period in the end of summer/early Autumn ends with
  flowering and a long reproductive stage over the winter/early spring.
  The rosette is senescing when favourable conditions return in early
  spring, seriously limiting fruit development.
\item
  \emph{winter-veg strategy} again spans the winter/early spring period.
  The life cycle duration is similar to strategy 3 but slightly later
  germination delays flowering until Spring. The rosette grows all
  winter, overlapping with a short reproductive stage and supporting
  high fruit mass.
\end{enumerate}

\begin{figure}[p]
\includegraphics[width=1.05\textwidth]{figures/valCompFig/valComp.eps}
\caption{Population experiments in Valencia in two different genetic backgrounds
  (LL:low dormancy, Low floral repression and HH: High dormancy, High floral
  repression). A Results of the phenology-only simulation results for the LL
  genotype. The 25-th, 50-th, and 75-th percentile of the distribution of
  developmental events are shown for each identified cluster B Illustration of
  growth stages over a year for each cluster from A according to the median time
  of the distribution of developmental events for each cluster. C Growth
  simulations over the growth period shown in panel B for each cluster.  D, E, F
  Equivalently for the HH genotype in the same location.}
\label{fig:valComp}
\end{figure}


\begin{figure}[tb]
\includegraphics[width=\textwidth]{figures/valCompFig/valSankeys.eps}
\caption{Transition probabilities between life history strategies from
  Figure~\ref{fig:valComp} in Valencia. A Probabilities of successive strategies
  for the LL genotype B Probabilities of successive strategies for the LL
  genotype after eliminating life cycles with strategies 1 and 3 using a
  reproductive threshold. C, D Similarly for the HH genotype.}
\label{fig:valSank}
\end{figure}

Plants with life cycle strategies 2 and 4 predicted orders of magnitude more
fruit mass than plants with life cycle strategy 3 or the least successful
strategy 1 (Figure~\ref{fig:valComp}C). This result clearly ranked the
strategies available to plants of the LL genotype, although the absolute values
of the predicted biomass are less certain (see Discussion). The 100 lineages
amassed 4905 potential lifecycles over 45 years of phenological simulation
(Figure~\ref{fig:valComp}A). Without a minimum mass threshold ($m_0$) for
reproduction, 66\% of potential life cycles followed the more successful
\emph{spring} and \emph{winter-veg} strategies (2 \& 4;
Figure~\ref{fig:valComp}C). Figure~\ref{fig:valSank}A shows the sequential
transitions between strategies. For example, 60\% of potential plants following
the successful \emph{spring} strategy (2) disperse their seeds early enough for
the next generation to adopt the \emph{winter-veg} strategy (4), achieving two
generations per year. These transitions underlie the bimodal distribution of
life cycle times reported by \citet{burghardt_modeling_2015} for this
simulation.

Simulation of the HH genotype (Figure~\ref{fig:valComp}D, E) identified similar
strategies. Since the seed have longer dormancy, the population amassed
fewer potential life cycles (2954 as opposed to 4905 in the LL case;
Figure~\ref{fig:valComp}A). The growth and final fruit masses are different because of
slight variation in timing of the growth period but strategies 2 and 4
are again more successful than strategies 1 and 3 (Figure~\ref{fig:valComp}F). A higher
fraction of potential life cycles followed the successful strategies
(78\% as opposed to 66\% in the LL case; Figure~\ref{fig:valComp}D). Higher seed
dormancy reduced the germination in the summer and early autumn that led
to the less successful strategies 1 and 3, so any strategy was likely to
be followed by either strategy 2 or 4 in the next generation
(Figure~\ref{fig:valSank} C).

In order to calculate the population success we make two choices for the
reproduction mass threshold, \(m_{0}\), which eliminate one or both of the least
successful strategies. Choosing a value \(m_{0} = 2 \times 10^{- 5}\) g (the
mass of a single seed) eliminated the \emph{summer-only} strategy from both
genotypes, which gives a population of 1210 plants over 45 years in the LL case
(Figure~\ref{fig:valComp}C). The HH genotype allows a larger percentage of
viable life cycles but we predict fewer plants (1020) since the number of
potential life cycles was lower (Figure~\ref{fig:valComp}D). Choosing a value
\(m_{0} = 6 \times 10^{- 3}\) g left only two viable strategies, 2 and 4, for
both genotypes. Reciprocal transitions between the strategies were still
possible but \emph{winter-veg} was strongly favoured (Figure~\ref{fig:valSank}
B, D). The LL genotype predicted 240 plants in total over 45 years, compared to
360 plants for the HH genotype: G x E interaction favoured the HH genotype
despite its smaller number of potential life cycles. Thus, modelling the growth
processes not only distinguished among the potential life cycle strategies
within a genotype but also distinguished between the genotypes.

The effect of seed dormancy on lineage length can be seen in
Figure~\ref{fig:psisCorVal}. For the low dormancy distribution (mean $\psi_i=0$)
taking into account only phenology, the distribution of dormancy values is
negatively correlated with lineage length, as expected (more dormant lineages
have fewer generations, Figure~\ref{fig:psisCorVal} B -- timing only). Applying
a growth-based birth function, which can sometimes be $0$, gives a different
picture as higher dormancy is positively correlated with lineage length, perhaps
reinforcing the genotype distinction we observed above
(Figure~\ref{fig:psisCorVal} B -- timing+growth). In the high dormancy case
(mean $\psi_i =2.5$), again, there is a negative correlation between dormancy
and lineage length. Applying the birth function changes the correlation where
lineages at the two ends of the dormancy distribution do better than the ones
close to the mean (Figure~\ref{fig:psisCorVal} D). This requires further
investigation as it is not clearly interpretable like the result in the low
dormancy case.

\begin{figure}[tb]
  \centering
  \includegraphics[width=\textwidth]{figures/paramDistrs/psisCorVal.eps}
  \caption{Dormancy level correlation with lineage length in the Valencia
    experiments A Distribution of dormancy levels for the Low (L) dormancy
    ($\psi$ mean=0, sd=1) case. B The correlation of dormancy level values with
    lineage length first when considering only timing (phenology model alone) or
    when considering timing and growth (applying growth-based birth function to
    population). C Distribution of dormancy levels for the High (H) dormancy
    case ($\psi$ mean=2.5, sd=1) D The correlation of dormancy level values with
    lineage length first when considering only timing (phenology model alone) or
    when considering timing and growth (applying growth-based birth function to
    population).}
  \label{fig:psisCorVal}
\end{figure}




\subsection{Oulu}
\label{oulu}
The equivalent simulations were performed for conditions in Oulu,
Finland in the same LL and HH genetic backgrounds (Figure~\ref{fig:oulComp}). The
results indicated 3 potential life cycle strategies (Figure~\ref{fig:oulComp}A, B, E,
F):

\begin{enumerate}
\def\labelenumi{\arabic{enumi}.}
\item
  \emph{summer-only strategy} where the entire life cycle occurs in the
  summer. The vegetative period is short, the rosette is very small and
  supports negligible fruit growth (Figure~\ref{fig:oulComp}C).
\item
  \emph{winter-repr strategy} where a life cycle of almost a year has a
  very short vegetative stage, followed by a long reproductive stage
  over the winter. Again, the very small rosette supports little fruit
  growth in the following Spring.
\item
  \emph{winter-veg strategy} where the plant over-winters in the
  vegetative stage. Unlike in Valencia, the rosette grows little over
  the winter. Rapid rosette growth in the following spring supports a
  substantial inflorescence and fruit development, though the predicted
  fruit mass is smaller than in Valencia.
\end{enumerate}

\begin{figure}[p]
\centering
\includegraphics[width=\textwidth]{figures/oulCompFig/oulComp.eps}
\caption{Population experiments in Oulu in two different genetic backgrounds (LL
  ? low dormancy, Low floral repression and HH-High dormancy, High floral
  repression). A Results of the phenology-only simulation results for the LL
  genotype. The 25-th, 50-th, and 75-th percentile of the distribution of
  developmental events are shown for each identified cluster B Illustration of
  growth stages over a year for each cluster according to the median time of the
  distribution of developmental events for each cluster. C Growth simulations
  over the growth period shown in panel B for each cluster.  D Probabilities of
  successive strategies. E, F, G, H, J Equivalently for the HH genotype in the
  same location.}
\label{fig:oulComp}
\end{figure}

The severe winter conditions limited the number of potential life cycles to 2361
for the LL genotype or 363 for HH. A higher proportion of HH life cycles
followed the successful \emph{winter-veg} strategy (3; 32\% against 24\% in LL;
Figure~\ref{fig:oulComp}B, F). Surprisingly, a majority of life cycles for both
genotypes followed the \emph{winter-repr} strategy (2).  Applying the
reproductive threshold mass, \(m_{0}\), eliminated one or both of strategies 1
and 2 (Figures~\ref{fig:oulComp} C, G), suggesting a strong selective pressure
for greater floral repression to reduce the number of \emph{winter-repr} life
cycles. With \(m_{0}\) = \(2 \times 10^{- 3}\) g, the LL genotype yielded 159
plants over 45 years compared to 53 plants for HH. All G x E combinations had
actively-growing plants at the end of the simulation. Interestingly, plants of
the HH genotype had higher average reproductive success per plant in Oulu yet
the LL plants were more successful by our population measure. The faster
development of LL plants allowed more, short lifecycles within the simulated
interval (consistent with the phenology model alone).

The distinction between genotypes can perhaps also be appreciated by looking at
the correlation between dormancy values in the distributions and lineage length
(Figure~\ref{fig:psisCorOul}). In both low and high dormancy case, with $0$ and
$2.5$ mean $\psi_i$ respectively, the correlation does not change when
considering timing and growth as opposed to timing alone
(Figure~\ref{fig:psisCorOul} B, D). This can be interpreted by the higher
reproductive success of LL plants we have identified above.

\begin{figure}[tb]
  \centering
  \includegraphics[width=\textwidth]{figures/paramDistrs/psisCorOul.eps}
  \caption{Dormancy level correlation with lineage length in the Oulu
    experiments A Distribution of dormancy levels for the Low (L) dormancy
    ($\psi$ mean=0, sd=1) case. B The correlation of dormancy level values with
    lineage length first when considering only timing (phenology model alone) or
    when considering timing and growth (applying growth-based birth function to
    population). C Distribution of dormancy levels for the High (H) dormancy
    case ($\psi$ mean=2.5, sd=1) D The correlation of dormancy level values with
    lineage length first when considering only timing (phenology model alone) or
    when considering timing and growth (applying growth-based birth function to
    population).}
  \label{fig:psisCorOul}
\end{figure}

\section{Discussion}
\label{sec:discussion}

I present a whole-life-cycle multi-model for growth and reproduction of
\emph{Arabidopsis thaliana}, FM-life, combining phenology models that time the
developmental stages and growth models to predict organ biomass. The simple,
FM-lite model of vegetative growth, and its extension to the reproductive stage
in FM-life, simulate broader, mechanistically-founded components of fitness at
the individual plant level compared to the phenology models alone. Most insights
from the component models naturally remain
\citep{rasse_leaf_2006,christophe_model-based_2008,
  wilczek_effects_2009,burghardt_modeling_2015}. Multi-models are helpful in
emphasising interactions. The cauline leaves in the inflorescence model, for
example, extend the duration of photosynthetic competence. As cauline leaves can
be produced 6 months later than early rosette leaves in the \emph{winter-veg}
strategy (Figure~\ref{fig:oulComp}), they remain active photosynthetic sources
\citep{earley_inflorescences_2009, leonardos_photosynthetic_2014} when the
rosette leaves are senescing. The growth models provided the fruit mass that I
used as an indicator of reproductive success, such that metabolic and
developmental processes of growth informed a more mechanistic understanding of
ecological, population dynamics over multiple generations.

The growth model allowed us to discriminate among alternative life cycle
strategies in each G x E combination, by selecting against strategies that were
compatible with the phenology models alone but had qualitatively worse
growth. In previous work, strategies with high seed dormancy in southern
Valencia and low dormancy in northern Oulu were noted to align with the
behaviour of the cognate wild populations \citep{atwell_genome-wide_2010,
  chiang_dog1_2011, mendez-vigo_altitudinal_2011, burghardt_data_2014}. In each
G x E combination, individual plants in our simulations might adopt alternative
life cycle strategies. The less-successful strategies were lethal in my model,
eliminating \textgreater{}95\% of potential lifecycles (simulated by the
phenology model alone) for the LL genotype in Valencia, for example
(Figure~\ref{fig:valComp}A, C). Thus, my results supported the observed
genotypic distinction between Valencia and Oulu, because the requirement for a
minimum fruit mass eliminated more lineages of the less-successful genotype in
each case (Figures~\ref{fig:valComp}C,F and \ref{fig:oulComp}C,G).

My approach might appear conservative, as the binary birth function (one seed/no
seed) ignored variation in seed mass among life cycle strategies, which might
otherwise reinforce the advantage of successful strategies. The successful
genotype LL in Oulu, however, had lower reproductive success per plant than HH,
suggesting a more subtle balance of advantage. Genotypes with Low dormancy and
High floral repression (LH) are observed in far northern locations
\citep{atwell_genome-wide_2010}. I therefore simulated the LH genotype
(Figure~\ref{fig:oulLH}) . LH plants delayed flowering time enough to reduce the
frequency of potential \emph{summer-only} life cycles to 9\% compared to 15\% in
LL (Figure~\ref{fig:oulLH}B) and increased the fruit mass of the successful
\emph{winter-veg} life cycle close to the HH genotype
(Figure~\ref{fig:oulLH}C). The LH model predicted slightly higher reproductive
success overall, returning 171 life cycles (Figure~\ref{fig:oulLH}C) compared to
159 for the LL variant (Figure~\ref{fig:oulComp}C), consistent with the
observation of LH genotypes at this location.

\begin{figure}[tb]
\centering
\includegraphics[width=\textwidth]{figures/conclFigure/oulLH.eps}
\caption{Simulation results for a combined variant in Oulu (LH - low dormancy,
  high floral repression). A Results of the phenology-only simulation
  results. The 25-th, 50-th, and 75-th percentile of the distribution of
  developmental events are shown for each identified cluster B Illustration of
  growth stages over a year for each cluster according to the median time of the
  distribution of developmental events for each cluster. C Growth simulations
  over the growth period shown in panel B for each cluster.  D Probabilities of
  successive strategies.}
\label{fig:oulLH}
\end{figure}

\begin{figure}[tb]
\centering
\includegraphics[width=\textwidth]{figures/growthInvFigure/growthInv.eps}
\caption{Robustness of timing and growth simulation results.  A Sources of
  variability for the timing results. B Timing results for variable weather (60
  years) and variable genetic parameters chosen from a distribution and timing
  results for simulations with constant weather (same 1-year weather over 60
  years) and constant parameters. The timings of the strategies are very similar
  but strategy 1 life cycles have moved to strategy 3. This could be because of
  the particular year of weather data I used for the simulations. C Sources of
  variability for growth results D Growth simulation results starting from three
  different germination dates (corresponding to 25-th, 50-th, and 75-th
  percentiles of the distribution of germination times) of strategy 4 in
  Valencia LL (Figure~\ref{fig:valComp})}
\label{fig:growthInv}
\end{figure}

Variability in timing results, as we have noted, appears as a result of variable
weather conditions each year, difference in genetic parameter values, and the
stochasticity inherent in the model representation (Figure~\ref{fig:growthInv}
A, B). Both parameters and variable weather inevitably have some effect on the
timing results and in turn on the growth results and overall population measures
we considered (Figure~\ref{fig:growthInv} B, C). As for the stochasticity in the
model representation, we can get a hint on the effect from the indicative
simulations of the seed dormancy model. There the stochasticity might, for
example, drastically change the germination behaviour of otherwise identical
seeds because of the hard transition thresholds and weather constraints during
the year (Figure~\ref{fig:seedInd} A). It would be interesting to quantify this
effect perhaps by changing the phenology simulations from a daily to an hourly
timestep. Perhaps this also suggests that this type of phenomenological models
that are meant to reproduce field experimental data are best suited to
deterministic representations. I could have used the fluents of Chromar, for
example, to represent the timing parts of the models (enriched expressions of
Chromar \sct{extSyntax}).

A limitation of my work arises from the fact that the phenology component models
of FM-life have been validated against field data
\citep{wilczek_effects_2009,burghardt_modeling_2015} whereas the growth
component models have not
\citep{rasse_leaf_2006,christophe_model-based_2008}. Biomass simulations are
inevitably sensitive to the timing of the growth period, because a longer
interval of exponential growth in good conditions rapidly changes absolute
biomass, as illustrated in Figure~\ref{fig:growthInv}D. I therefore modified the
growth models conservatively to account for severe winter conditions and to
limit photosynthesis in dry conditions (\ssec{fm-lite}). The moisture effect on
seed maturation (through photosynthesis) in particular is important and extends
the effects of water availability, an important field environmental condition,
to the whole lifecycle as opposed to only the seed dormancy stage in
\citep{burghardt_modeling_2015}, for example. Nonetheless, the FM-life model
predicted unreasonably high fruit mass in some cases. The binary birth function
ensured that this had no effect on our population measure. Among possible gaps
in understanding of the environmental effects on growth in natural settings or
in my representation, I repeat previous caution \citep{chew_multiscale_2014,
  chew_linking_2017} that models of nutrient balance for Arabidopsis will be
helpful. Rosette biomass in the Framework Model is understandably sensitive to
photosynthetic parameters \citep{chew_multiscale_2014} yet these have not been
validated in Arabidopsis across the wide range of photoperiods and temperatures
simulated here \citep{walker_temperature_2013}. FM-life predicts a discretised
fruit mass and hence reproductive success for a typical representative of each
life cycle strategy, approximating an underlying, continuous distribution of
fruit mass. The accuracy of this approximation will depend on the variation
within clusters. The benefit lies in computational tractability, allowing us to
simulate differential reproductive success that is informed by understanding of
growth processes.

My approach here builds upon previous models that predict fitness and population
processes in Arabidopsis, which have focussed on developmental components of
fitness or on phenology
\citep{prusinkiewicz_evolution_2007,satake_forecasting_2013,springthorpe_flowering_2015}. Linking
these components sharpens ecological insight, by understanding the performance
of genetic variants in the environment that underlies differences in fitness
\citep[see discussions in][]{burghardt_modeling_2015, doebeli_towards_2017} and
can thus inform evolutionary hypotheses. Adding genetic variation between
generations will in future model Arabidopsis evolution explicitly, perhaps after
competing genetic variants \emph{in silico} using adaptive dynamics approaches
\citep{brannstrom_hitchhikers_2013, weise_mechanistic_2015}.  Thus, the FM-life
model offers a further tool to bridge among disciplines in plant biology,
ecology and evolution.

\chapter{Optimal control}
\label{chp:oc}
Biological dynamics, from single cell behaviours to emergent properties of
populations, are the result of the complex interplay between genomes and the
environment. Efforts to tightly regulate them have so far largely been focused
on (re)engineering genomes. In this chapter I instead address the problem of
controlling the plant biomass manipulating the environment around it. While the
automatic climate control problem has been studied before for greenhouses, the
focus of such efforts was on the minimisation of energy consumption and/or
achieving assigned climate regimes. Seldom those works attempted to optimise
crop traits \citep{Chalabi1996, udinktenCate1983, Challa_1990, Aaslyng2003}.

More insulated and sophisticated growth environments with more precise control
are becoming available as part of the urban and controlled-environment
agriculture movement \citep{mok_strawberry_2014, despommier_farming_2013}. There
are many commercial examples (\href{https://aerofarms.com/}{Aerofarms},
\href{https://motorleaf.com/}{Motorleaf}) and others like Intelligent Growth
Solutions (\href{https://www.intelligentgrowthsolutions.com/}{IGS}) that are
based on novel technologies, for example, for reducing energy consumption and
increasing control over the growth space (lighting patent
\cite{aykroyd_novel_2016}, automated growth tower patent
\cite{aykroyd_automated_2018}). These more sophisticated growth environment
allow us to focus more on crop traits and even more precisely at specific
quality standards (size, uniformity) that crops need to meet \citep[see for
example EU marketing standards on fruit and vegetables;][]{eu-543-2011}.

In this chapter, I present a formulation of the climate control problem for
\textit{Arabidopsis thaliana} plants as an optimal control problem
\citep{kirk_optimal_2012} where the control variables are climate conditions
(temperature) and the performance criterion is a crop trait. It has already been
proposed that more mechanistic models are needed when linking between multiple
scales instead of the usually empirical models used in crop modelling
\citep{yin_role_2004, yin_modelling_2010}. Here I use the Framework Model
\citep[FMv1;][]{chew2014multiscale} as a mathematical description of vegetative
plant growth of \textit{Arabidopsis thaliana} that provides mechanistic links
from molecular regulation all the way to whole-plant traits, like biomass. In
particular, in this chapter I do the following:
\begin{itemize}
\item Formulation and solution of a \textit{direct problem} of offline climate
  control where the control variable is the temperature of a growth chamber in
  order to achieve a particular plant biomass at a particular time (predicted
  from the FM). The solution to the problem uses standard gradient-based
  optimisation techniques by discretising the growth period and assuming
  constant temperature within each time interval.
\item Formulation and solution of an \textit{indirect problem} of offline
  climate control. Here I assume that we cannot control temperature precisely in
  the growth space, which leads to temperature inhomogeneities. I further assume
  that the position of plants can be controlled (e.g. switch their positions) in
  a linear array that has a temperature gradient. The control variable is the
  position of the plants and the performance criterion is homogeneity in final
  plant biomass (as predicted from the FM) for the population of plants in the
  array. For the solution of the problem I use techniques from combinatorial
  optimisation since the control variables are discrete.
\end{itemize}

\section{The model and main idea}
For this work I use the Framework Model (FM) as described in
\citet{chew2014multiscale}. The FM takes environmental inputs (CO2 level,
temperature, light intensity, and temperature) and outputs several
growth-related quantities over its simulation time. In the following work,
however, I will treat the FM model a Single Input Single Output system
accepting temperature as input and outputting biomass by regarding all the other
inputs as constant and ignoring the rest of the output signals. Despite the
``black box'' approach we take to model plant growth, it is instructive to look
at two particular metabolic processes that are highly temperature dependent.

\begin{figure}[tb]
\centering
\includegraphics[width=\linewidth]{figures/fmFig/agentFlows.eps}
\caption{Metabolic processes as flows in the Framework Model (FM). A The FM
  keeps track of carbon movement in different pools, one for each organ (leaves
  and root) and two whole-plant reserves of sucrose carbon (c) and starch carbon
  (s). Maintenance respiration and assimilation (intake of carbon) are
  explicitly temperature dependent (indicated by red dots) B Photosynthesis and
  maintenance rate trends for a range of temperature values.}
\label{fig:fm}
\end{figure}

All the metabolic processes keep track of the main building block of new mass,
carbon, and are represented as flows between various pools representing the
organs (root and leaves only at the vegetative stage) and whole plant reserves
of carbon as sucrose (C) and starch (S) (Figure~\ref{fig:fm}A). The two
metabolic processes that are temperature dependent are (i) assimilation, the
intake of carbon from the environment represented as an in-flow into the central
reservoir of sucrose carbon (C) and (ii) maintenance respiration, the use of
carbon for maintaining the life-sustaining processes inside the plant
represented as an out-flow from the main sucrose carbon
reservoir. Figure~\ref{fig:fm}B shows the rates of these flows for a range of
temperature while keeping the other environmental inputs constant (light
intensity,120 $\mu \mathrm{mol} \cdot m^{-2} \cdot s^{-1}$, CO2=420ppm).

While temperature directly affects only two of the processes it also indirectly
affects the other since all of them compete for the sucrose carbon in the main
reservoir. Therefore and not surprisingly temperature is a major determinant of
growth rate and final biomass. The \emph{main idea} of this work is to control
temperature and therefore indirectly control growth to achieve a particular
growth-related objective of interest using the FM as our ground truth to
evaluate the growth of an \textit{Arabidopsis thaliana} plant at different
temperature inputs.

In the FM the growing plants starts as seeds and wait for a period of time
(emergence period), which is a function of temperature, before they emerge and
start vegetative growth. In the following we only control temperature in the
vegetative growth phase and not in the emergence period. Therefore we assume
that seeds are kept in a constant temperature of 22\textdegree C during the
emergence period.


\section{Direct problem}
Here I seek to design a control scheme that modulates temperature to reach a
specific biomass after a particular period of time. This is relevant to growers,
for example, that have specific requirements both in terms of time and crop
attributes.

Given the characteristic of the problem at hand I formulate is as an optimal
control problem, i.e. find a time-profile of temperature function, $T^*(t)$,
over a period $[t_0, t_f]$ that minimises the difference between the final
biomass of the plant (as predicted by the FM) and a target biomass $m_0$:
$$
J = m_{T^*}(t_f) - m_0
$$
where $m_{T*}(t)$ is the temperature dependent biomass trajectory predicted from
the FM. To reflect physical constraings, we impose upper and lower bounds on the
temperature $T_l < T(t) < T_u$. Additionally, here I consider the case of a
``single plant control''. Notably, this approach can be extended, with
appropriate modifications, to the control of crops.

\begin{figure}[tb]
\centering
\includegraphics[scale=0.25]{figures/directPFig/ocProb}
\caption{Direct problem of climate control. For the \emph{direct problem} we try
  to find a temperature function or sequence of temperature values in the
  optimisation formulation in order for the final biomass value as predicted by
  the FM (using this temperature function as input) to reach a specific
  predefined value, $m_0$.}
\label{fig:directP}
\end{figure}

As conventional in optimal control, we can convert the optimal control into a
standard non-linear optimisation problem by discretising the time domain
$[t_0, t_f]$ into $k$ intervals and assuming a constant temperature value inside
each interval \citep{kraft_converting_1985}.
\begin{definition}[biomass-only problem]
The \emph{biomass-only problem} is the problem of finding a sequence of
temperature values, $T_{1, k}=T_1, \dots, T_k$, that minimises the square of the
% why did we choose the square of the difference if we are considering a single point?
difference between the final biomass value of the plant at $t_f$,
$m_{T_{1, k}}(t_f)$, as predicted by the FM, for the sequence $T_{1, k}$.
\begin{align*}
& \argmin_{T_1, \dots T_k} \; (m_{T_{1, k}}(t_f) - m_0)^2 \; \; \text{subject to} \\
& T_l < T_{i} < T_u; \; \; \; i=1 \dots k
\end{align*}
\end{definition}
Note that I use $T_{i, k}$ and $T_1, \dots, T_k$ interchangeably for a sequence of
$k$ temperature values.

Since repeated, large switchings of the temperature introduce wear and tear of
the actuators and use energy, we extend our control problem to take into account
the control effort as well.

\begin{definition}[biomass+control effort problem]
  The \emph{biomass+control effort} problem is an extension to the
  \emph{biomass-only problem} where the sought sequence of temperatures
  $T_{1, k}=T_1, \dots, T_k$ minimises both the distance to the target biomass,
  $m_0$, and the control effort defined as the average jump between successive
  temperatures in the sequence, $\underline{T}$.
$$ \argmin_{T_1, \dots T_k} \; (m_{T_{1, k}}(t_f) - m_0)^2 \; + \; \frac{1}{k} \sum_{i=2}^{k} T_i - T_{(i-1)}$$
\end{definition}
% what about multi-objective optimisation instead of single-objective with
% linear combination of two?
The bound constraints are the simple lower/upper bounds as before. 

\subsection{Results}
In this section I solve the two formulations of the optimisation problems
(direct problem) using the \texttt{fmincon} function for constrained non-linear
optimisation from the Matlab optimisation toolbox for some choices of $m_0$,
$t_f$, and $k$. While I vary $m_0$ to explore different instances of the
problem, I generally leave $t_f$ and $k$ constant ($t_f=512h$, $k=4$). There
are methods that adapt $k$ that we could have used that are implemented in more
sophisticated tools \citep[mesh refinement, AMIGO
tool;][]{balsa-canto_amigo2_2016}). The final time is dictated by the bolting
time of the plant under the range of temperatures that we consider since the FM
only considers vegetative growth (before bolting). Bolting time is more than
$512h$ for all the temperature values in the range that I consider
($[10, 30]$).

I use a performance metric, $\mathrm{log}(\sqrt{L(T_{1, k}, t_f, m_0)} / m_0)$,
to assess particular optimisations for different problem instances (specific
$k$, $t_f$, and $m_0$) that states the final value of the biomass objective
function $L(T_{1, k}, t_f, m_0))=(m_{T_{1, k}}(t_f) - m_0)^2$ as a percentage of
the target biomass.

\begin{definition}[$\epsilon$-reachable]
  A target biomass value,$m_0$, defining a particular instance of the
  biomass-only or biomass+control effort problems is $\epsilon$-\emph{reachable}
  if there exist a $k$ and a particular sequence of temperature
  $T_{1, k}=T_1, \dots, T_k$ such that the ratio of the performance metric to the
  target biomass, $m_0$, is less than a tolerance value, $\epsilon$.
$$
\frac{\sqrt{L(T_{1, k}, m_0, t_f)}}{m_0} < \epsilon
$$
\end{definition}
In the following I assume that a specific sequence of temperatures that
satisfies our tolerance, $\epsilon$, exists if the particular optimisation
algorithm I use can converge to that sequence.

Figure~\ref{fig:directPRes} shows results of the optimisation procedure for both
the biomass-only and biomass+control effort problems defined in the previous
section for a range of target biomasses. For the biomass-only problem there is a
set of reachable target biomasses $[0.03, 0.25]$ (Figure~\ref{fig:directPRes}A)
for a tolerance $\epsilon=0.1$ , $k=4$, and $t_f=512$h. The upper and lower
bound constraints for the temperature values are set to 10 \textdegree C and 30
\textdegree C respectively. The optimisation converges to different solutions on
different runs (Figure~\ref{fig:directPRes}B and C, D, E for details of
particular solutions for $m_0=0.05$, $m_0=0.1$, and $m_0=0.15$ respectively). It
is also interesting to explore if the optimisation increases the reachable set
compared to a naive exhaustive simulation of the FM with constant temperature
inputs in the constrained space [10 \textdegree C, 30 \textdegree C]. Exhaustive
simulation of the FM in the space [10 \textdegree C, 30 \textdegree C] with a
$0.5$ \textdegree C step gives final (at $t_f=512$h) biomasses between $0.0093$g
at 10 \textdegree C and $0.1806$g at 17.5 \textdegree C while more than one
temperature input can give the same (or very close) final biomass (orange line,
Figure~\ref{fig:directPRes}B).


\begin{figure}[p]
  \centering \resizebox{1.06\linewidth}{!}{
    \includegraphics[width=\linewidth]{figures/directResFig/res}}
  \caption{ Results for both biomass-only and biomass+control effort
    formulations of the direct problems over a range of target biomasses with
    $k=4$ and $t_f=512$. A The performance metric for the biomass-only problem
    formulation over a range of target biomasses from 0.01 to 0.4. The other
    environmental inputs are constant: CO2, 420ppm; light intensity,
    120$\mu\mathrm{mol} \cdot m^{-2} \cdot s^{-1}$; 12:12-h light/dark
    cycle. For the first 139 hours of simulations the plants are still seeds and
    are assumed to be kept in a constant temperature of 22\textdegree C. The
    reachable set is under the dotted line. B Comparison between the solutions
    obtained with the optimisation and a naive exhaustive simulation with
    constant temperatures over the interval [10 \textdegree C, 30 \textdegree
    C]. For the optimisation solutions per $m_0$ I give the two most commonly
    occurring temperatures in the solution space (10 runs of the optimisation
    per $m_0$). C, D, E Four example solutions and corresponding biomass time
    series (as obtained from FM simulation) for target biomasses $m_0=0.05$ (C),
    $m_0=0.1$ (D), and $m_0=0.15$ (E). F Performance metric for the
    biomass+control effort problem formulation over a range of target
    biomasses. G Same as B but for the biomass+control effort problem
    formulation H, I, J Similar to C, D, E but for the biomass+control effort
    formulation.}
\label{fig:directPRes}
\end{figure}

For the biomass+control effort problem I find a smaller reachable set
$[0.03, 0.21]$ compared to the biomass-only problem
(Figure~\ref{fig:directPRes}F) with the same parameters. The optimisation
procedure though in this case mostly converges to the same solution at different
runs and the solutions mostly keep the temperature constant
(Figure~\ref{fig:directPRes}F and H, I, J for details of particular solutions
for $m_0=0.05$, $m_0=0.1$, and $m_0=0.15$ respectively). Comparing the solutions
to the naive exhaustive search as before we can see the optimisation converges
to constant temperature solutions that are very close to the ones given by the
simulation with a similar reachable set. In cases where two constant temperature
inputs give the same final biomass the optimisation cannot distinguish between
the two and can converge to either (for example, Figure~\ref{fig:directPRes}H
and temperature distributions in G).

Finally, we can compare the results of the optimal strategy with a random
strategy (as a control) where the sequence of temperatures, $T_1, \dots, T_k$ is
picked at random (Figure~\ref{fig:compsAllDir}). For 100 simulation runs for
both optimal and random strategies, the optimal strategy performs significantly
better for target biomasses $m_0=0.05$ and $m_0=0.1$.

\begin{figure}[tb]
\centering
\includegraphics[width=\linewidth]{figures/directResFig/compsAll}
\caption{
  Comparison between optimal and random strategies for both formulations of the
  direct problem. The comparison is for two target biomasses in the controllable
  range $m_0=0.05$ and $m_0=0.1$ over 100 runs of the optimisation 100 runs of
  the random strategy where temperatures are chosen at random for the 4
  intervals.
}
\label{fig:compsAllDir}
\end{figure}


\section{Growth space inhomogeneities experiment}
\label{sec:exp}
I have so far assumed that we can perfectly control the environmental
conditions inside the growth chamber such that all the plants are exposed to the
same temperature. In actual growth spaces, this is rarely possible. To quantify
the effect of temperature inhomogeneity on plant growth we conducted an
experiment designed to test the difference in growth (biomass) of two sets of
\textit{Arabidopsis thaliana} plants grown in two different locations on a shelf
(Figure~\ref{fig:expRes}E). The first group of plants is grown in the middle of
the shelf and the second on the side of the shelf.

% In the direct problem
% formulation and the entire premise of this work we have the assumption that we
% can perfectly control the environmental conditions inside the growth space such
% that all the growing plants can grow in perfectly homogeneous conditions. Based
% on our experience of growing plants in academic growth spaces, which are
% specifically designed to reduced experimental technical variability, this
% assumptions is rarely true. To test this we conducted an experiment in a growth
% room designed to test the difference in growth (biomass) of two sets of
% \textit{Arabidopsis thaliana} plants grown in two different locations on a shelf
% (Figure~\ref{fig:expRes}E) in a typical growth room. The first group of plants
% is grown in the middle of the shelf and the second on the side of the shelf.

\begin{figure}[tb]
\centering
\includegraphics[width=\linewidth]{figures/exp/expRes.eps}
\caption{ Results of experiment designed to investigate growth inhomogeneities
  resulting from environmental inhomogeneities in a controlled growth room. A
  Mean biomass for the two groups (middle, side) for the two samples (23 and 27
  days) and fitted exponential curves. B Full distribution for the two samples
  (23 and 27 days) C Average temperature time series over the three sensors
  placed along the positions of the middle and side groups. D Average light time
  series averaged over the three sensors places along the positions of the
  middle and side groups. Note that light is given in lux instead of the
  $\mu\mathrm{mol}$ that the models use since the readouts from the sensors are
  given in lux. E Experimental setup showing the positions of the plants and
  sensors used to get the environmental data in C, D.  }
\label{fig:expRes}
\end{figure}

The environmental conditions in terms of light and temperature are very
different between the two sets of plants (Figure~\ref{fig:expRes}C, D). We took
two samples at two different stages of vegetative development and before
bolting, one at 23 days after germination (plants M1/1[1-6], M2/2[1-6] for the
middle group and plants S1/1[1-6], S2/2[1-6] for the side group --
Figure~\ref{fig:expRes}) and one 27 days after germination (plants M1/2[1-6],
M2/1[1-6] for the middle group and plants S1/2[1-6], S2/1[1-6] for the side
group -- Figure~\ref{fig:expRes}). The differences in the environmental
conditions affected the growth and resulted in differences in biomass of the two
groups (Figure~\ref{fig:expRes}A, B). While the differences in biomass is not
statistically significant in the 23-day sample, in the 27-day sample the
difference in biomass between the two groups is significant. The environmental
differences are amplified during the exponential growth of the plants.


\section{Indirect problem}
In the direct problem formulation I assumed perfect control inside the growth
space of a crop such that the temperature is homogeneous and all the plants are
exposed to the same conditions. This is rarely true even in academic growth
environments (see previous section, for example). Here I address the problem of
controlling plant growth acting on the position of plants in a linear array: a
linear temperature gradient is imposed on such an array. As in this case we can
only change the temperature the plant is growing indirectly, \ie by changing its
position in the linear array, we call this problem ``indirect''. This
formulation more closely resembles industrially relevant problems.

% In the direct problem formulation we assumed perfect control inside the growth
% space of a crop such that the temperature is homogeneous and all the plants are
% exposed to the same conditions. Despite progress in the technology of growth
% chambers this assumption is rarely true even in academic growth environments
% (see previous section, for example). This is problematic because it means growth
% inhomogeneities along the population. We therefore turn into a related problem
% where we try to reduce as much as possible temperature and therefore growth
% inhomogeneities in a population of plants by controlling their positions in a
% growth space during development. This, again, could be relevant for growers
% where there are requirements for homogeneity or even in academic settings where
% technical variability is expected to be minimal.

\begin{figure}
\centering
\includegraphics[width=\linewidth]{figures/indirectPFig/indirectOC.eps}
\caption{Indirect problem of climate control. In the \emph{indirect problem} we
  try to find a position function (map from time and plant id to position in a
  linear array) such that the final biomasses of the plants as predicted by the
  FM (using the position dependent temperature of the plants as input) to be as
  close to each other as possible (minimum Gini-index).}
\label{fig:indirectP}
\end{figure}

I assume that have $n$ plants and the growth space is a linear array of $n$
positions such that each plant occupies one position. I further assume that I
am given a global temperature perhaps coming from an optimisation procedure
(for example, see the direct problem before) but due to non-precise control this
gives rise to a temperature gradient along the array where the plants are
positioned. The optimal control problem becomes then to find an optimal position
function $P^*(i, t)$ that gives the position of plant $i$ at time $t$ over a
period $[t_0, t_f]$ such that some non-uniformity index over the final biomasses
of the $n$ plants is minimised:
$$
J = G(m_1(t_f), \dots, m_n(t_f))
$$
Here I write $m_i(t_f)$ for the biomass of $i$-th plant at the final time $t_f$
and we use the Gini index (denoted by $G$) as our non-uniformity metric.

As in the case of the direct problem we can turn the above into an optimisation
problem by discretising the time domain $[t_0, t_f]$ into $k$ intervals and
assuming that the positions of the plants are constant inside each interval. The
objective is to obtain by $t_f$ a set of plants with similar (satisfactory)
biomass. The problem then becomes to find the a sequence of $n \cdot k$ values,
$$
P = \begin{bmatrix} 
    p_{11} & p_{12} & \dots \\
    \vdots & \ddots & \\
    p_{n1} &        & p_{nk} 
    \end{bmatrix}
$$
,where $p_{i, j}$, is the position of plant $j$ at interval $i$:
\begin{align*}
& \min_{P} \; G(m_1(t_f), \dots, m_n(t_f)) \; \text{subject to} \\
& 1 < p_{ij} < n \\
& p_{ij} \; \text{integers}
\end{align*}

The positioning of the plant indirectly determines the temperature function,
$T(t)$, over $[t_0, t_f]$ for the plants. Unlike the direct problem the decision
variables of the optimisation problem are discrete. We therefore cannot use
standard gradient-based optimisation techniques. In the results in the following
sections we use a technique from combinatorial optimisation (see next section)
that uses some heuristics to search the space of all solutions while following
our objective function from above.

A constraint that I have overlooked in the formulation above is the need to
have one plant, and only one, in each position at every single time point. We
can formalise this as: for any interval $j$ all the values in the sequence
$p_{1j}, \dots, p_{nj}$ should be unique. In order to not have to deal with this
extra constraint we make a change to the formulation of the problem such that
the solutions of the combinatorial optimisation procedure are guaranteed to
satisfy this ``non-overlap constraint'' without having the constraint explicitly
in the optimisation procedure. In particular suppose that we have a permutation
matrix, $\Pi(n)$, that lists all the possible permutation of $n$ numbers:

$$
\Pi = \begin{bmatrix} 
    1 & 2 & 3 & \dots n \\
    2 & 1 & 3 & \dots n \\
    \vdots &  & \vdots \\
    n &  n-1 & \dots  &  1 \\
    \end{bmatrix}
$$
I redefine the optimisation with the decision variables being indices into the
permutation matrix so for example if at interval $j$ one we get $\pi_j$ then we
have the positions of each plant, $i$, at that interval $P_{ij} = \Pi_{(\pi_j, i)}$

\begin{definition}[permutation-index problem]
The \emph{permutation-index problem} is a variant of the indirect problem of
climate control where we seek a sequence of values $\pi_1, \dots \pi_k$ that minimise the
inhomogeneity in the final biomasses (after $t_f$ hours, as predicted by the FM)
of $n$ plants.
\begin{align*}
& \min_{\pi_1, \dots \pi_k} \; G(m_1(t_f), \dots, m_n(t_f)) \; \text{subject to} \\
& 1 < \pi_i < n! \\
& \pi_i \; \text{integers}
\end{align*}
\end{definition}

The number of all possible permutations for $n$ numbers is given by $n!$. The
order of permutations in $\Pi$ is important here as we need to make sure that
solutions that are close together in the permutation index space are also close
together in the positions space. Here we assume that every permutation listed in
$\Pi$ is one flip away from the permutation above it. Therefore consecutive
points in permutation space are adjacent in position space as well. For the
instances of the problems we dealt with this solution was adequate. However, for
larger instances the search space might become prohibitively large ($n!$) in
which case we can go back to the previous formulation of the problem and
introduce an overlap penalty in the objective function.

\subsection{Results} 
In this section we explore solutions to the permutation-index problem defined in
the previous section using combinatorial optimisation techniques. In particular
we use a Matlab implementation \citep[MEIGO too;][]{banga2014} of the Variable
Neighbourhood Search algorithm \citep[VNS;][]{mladenovic1997variable}.

\begin{figure}[tb]
\centering
\includegraphics[width=\linewidth]{figures/indirectResFig/indRes}
\caption{
  Comparison between optimal and random strategies for the indirect problem. A
Four example solutions returned by the optimisation procedure for four
intervals, five plants, and a growing space of five positions arranged
linearly. Each colour represents one plant and a line the movement of a plant
across the growing space over the four intervals. The temperature gradient
starts from 20\textdegree C in one end of the array and goes up to 24\textdegree
C on the other end. B Comparison of the objective function values (normalised
Gini-index of final biomasses of plants) over 100 runs of the optimal and random
strategies where the plants take random positions at each intervals. The
Gini-index for the static strategy is shown with a dotted line.
}
\label{fig:compsAllInDir}
\end{figure}

For the following experiment described in this section we assumed a growth space
of $5$ position with $5$ growing plants, one for each position, $k=4$ intervals
and $t_f=512$ hours. The temperature gradient goes from 20 \textdegree C at
position 1 on one end of the growth array to 24 \textdegree C at position $5$ at
the other end of the array. The positions in between are assumed have a
temperature linearly interpolated between the values at the two ends of the
array. We use a normalised performance metric $\tilde{G}=G/G_s$ where $G_s$ is
the value of the Gini-index for a position matrix where the positions of the
plants are unchanged over the growth period.

Unsurprisingly the optimisation procedure returns results where the plant
positions are shuffled over the time intervals (example solutions:
Figure~\ref{fig:compsAllInDir}A). The optimal strategy is significantly better
though than a naive strategy where the plants are randomly assigned positions at
each interval (Figure~\ref{fig:compsAllInDir}B). Both strategies are better over
100 runs than the static strategies where the plant positions are unchanged in
the growing period (static $G$ value indicated by dotted line,
Figure~\ref{fig:compsAllInDir}B).


\section{Discussion}
I present a formulation of the climate control problem for achieving particular
growth-related plant (or population of plants) attributes as an optimal control
problem and solution after transforming it to an optimisation problem. Optimal
control of climate in greenhouses has been studied for a long time but since
greenhouses are not insulated to weather conditions, greater effort (and
therefore energy) is needed for climate control. Optimal control studies have
therefore mainly been focused on reducing energy consumption \citep{fisher1997,
ramirezArias2012, delSagrado2016} with very few also taking into account plant
processes like photosynthesis for optimising growth as well as energy
consumption\citep{harun2015, Aaslyng2003}. Here I assume that our growth space
is more insulated so that more precise control is available at less effort. This
allows us to focus on crop traits and even at more precise quality control
standards like size and uniformity. I do, however, consider a control effort
quantity in the biomass+control effort problem formulation of the direct problem
that should be related to energy consumption. In the indirect problem we do not
have that even though switching the plant positions will require significant
energy even at sophisticated growth environments.  %related work


% %, only one with some kind of crop %trait in mind (photosynthesis rate; and only
% one for indoor %farming with LEDs %Even the one that considers photosynthesis
% only does so to exploit regularities in the photosynthesis to minimise energy
% consumption. For example, observation that temperature increase does not
% increase photosynthesis rate at low light so no need to use energy to increase
% temperature.  %No focus on energy but more on crop traits although we do
% consider control effort (energy related) in the first problem.

%discussion of results
In order to address the climate control problem I start with a \emph{direct
  problem} where I try to find an optimal input temperature signal to the FM so
that the output biomass signal has a particular value at some final time,
$t_f$. The formulation of the problem with an added penalty for control effort
along with exhaustive simulations with constant temperature inputs suggest that
a single constant temperature input is almost enough to achieve the same results
as the optimal strategy (Figure~\ref{fig:directPRes}A, B, F, G). There is very
little increase in the reachable set using the optimal strategy as opposed to
naive enumeration of final biomasses using single-temperature simulations of the
FM (Figure~\ref{fig:directPRes}B, G).  %tomato result optimal
temperature, find paper As we have seen from experimental results though even if
I set a global optimal temperature (for example as given by a solution of an
optimisation problem representing the direct problem) we are not guaranteed that
all the plants will have the same conditions. Distances as little as 1m (same
shelf), even in spaces designed specifically to reduce climate inhomogeneities,
can lead to significant environmental and therefore growth differences among a
population of plants. I therefore turned to a formulation of an \emph{indirect
  problem} of climate control where we accept the inhomogeneities and try to
minimise them (and therefore also growth inhomogeneities) by switching the
positions of plants along a space with a temperature gradient.

%limitations -- model
For the solutions of both problem statements, direct and indirect, I use the FM
as our ground truth to predict plant behaviour at different temperature input
profiles. The FM starts with biochemical models that have been validated only in
a narrow range of experimentally relevant temperatures so even in our seemingly
conservative temperature range ([10\textdegree C, 30 \textdegree C]) there might
be gaps in our understanding \citep{walker_temperature_2013}. This limits the
predictive power of the model across the range that we consider. Moreover there
might be other temperature effects on plant physiology that I do not
consider. The FM is deterministic but often there is variability in growth
\citep{abley2016developmental}, which might suggest an adaptive online strategy
in a possible implementation with a continuous monitoring of the growing
plants. Instead of a priori assigning the temperature signal like I do here, in
an adaptive online setting the input signal could be recomputed during growth
taking into account the development of the plant up to that point.

% limitations -- assumptions on growth space
%temperature switches, 

%conclusion + future
Our approach builds on ideas from traditional climate control in greenhouses and
the availability of more sophisticated growth spaces to suggest using optimal
control of climate for control of crop traits and quality standards, like size
and uniformity. Of course while the application of such methods is most
interesting for crop species that are practically relevant, here we chose the
model species \textit{Arabidopsis thaliana}. It will be interesting first to
test our offline in-silico optimisation experiments in an actual prototype
implementation for Arabidopsis plants given the limitations of the model-only
approach outlined above. Application of such techniques for more practically
relevant crops will also be very attractive and there is a number of models to
back such an application for commercial crops (potato models,
\cite{fleisher2017potato}; tomato model, \cite{heuvelink1999evaluation}; wheat
models, \cite{martre2015multimodel}). These are plant growth models that are
used for understanding but similar (although simpler) crop models have been used
more practically for optimal growth in greenhouses, for example with cucumber
crops \citep{Challa_1990}.  Finally, another possible application of, for
example, the indirect problem formulation might be in academic settings where
technical variability is expected to be minimal and there is an increase in the
use of automated phenotyping and growth platforms (PHENOPSIS
\cite{granier_phenopsis_2006}; platforms in the European Plant Phenotyping
Network, \url{https://eppn2020.plant-phenotyping.eu/}).



\chapter{Conclusions}
\label{chp:concl}
After a brief summary of the work presented in this thesis, I next go through
and discuss aspects of my work in order to understand its limitations,
implications that follow, and possible future work that it suggests.

\section{Summary}
% summary
No complex system can be understood by extrapolation of the properties of its
elementary components. In order to understand a complex system, like an
organism, one needs to consider explanations at various levels of detail. This
is especially true in biological systems where entities (processes and so on)
and their organisation give rise to properties of processes at the next level of
detail and they in turn to the next until you get to the level of a living
entity, the organism. In order to understand the organism (and consequently
life) we therefore need to understand the links between the explanations at the
various levels of detail.

The importance of organisation between components has been stressed before and
it is in fact the fundamental concept of systems theory. The stress on
organisation and its constructive nature also leads to questions about
representation. Abstraction to quantifiable properties alone is not adequate to
represent the rich organisation of biological systems. I presented
\emph{Chromar} a language that combines organisational aspects through the
explicit representation of discrete entities, agents, and non-organisational
aspects through abstract properties of these entities, attributes. The concept
of abstraction is there in extensions to the expression language used for the
dynamics of the attributes. Observables recognise the need for explanations at
different levels by allowing an explicit functional link between abstract
quantifiable properties and more explicit representations. Fluents, again, allow
abstraction via the description of system properties using time-dependent
deterministic value for describing parts of the system we do not wish to model
mechanistically.
% While the focus in
% Biology has been traditionally on single components the systems view has found
% its way into mainstream biological thinking with the systems biology
% movement.  The most widely used technical languages for the
% representation of physical systems come from the dynamics world. Since the
% questions there are usually different, abstracting a system to its properties
% through variables and ignoring organisation is usually adequate. The
% constructive organisation of biological systems though poses different
% requirements although the value of abstraction still remains. 

In contemporary plant research the focus has been on the model species
Arabidopsis. Functional-structural models have focused on the constructive
organisation of plant organs (development) as well as functional aspects like
metabolism. The Framework Model \citep{chew_multiscale_2014} is inspired by
these but adds more breadth by explicitly including a genetic circuit and
phenology models to time the conceptual development towards flowering. The FM
covers one part of plant development but does not extend to reproduction, which
means that it has limited applicability in ecological studies. Ecological
studies on the other hand consider the full lifecycle but of an abstract version
of plants where its construction and physical development are absent. I
presented \emph{FM-life} an extension to the FM to the full lifecycle including
reproduction and its scale-up to the population level via a clustering approach
to tractably simulate population of plants over multiple decades in different
genotype x environment scenarios.

Finally, I presented another more engineering oriented use of multi-scale plant
model as devices for optimising plant traits. Instead of the more traditional
genome engineering approach we focused on engineering the environment during
their development to achieve specific growth-related traits.

\section{Chromar and representations of multi-scale biology}
The `structuralists' that we have overviewed in Chapter~\ref{chp:intro} and were
the proponents of organisation and the systems view regarded biology (and life)
as inherently modular with this being a defining (if not the most defining)
characteristic of living systems. Whether biology is inherently modular is
debated \citep{hartwell1999molecular}. %need better ref here!!
Nevertheless, treating as such is a
justifiable choice to advance understanding. Modules includes `visibly'
differentiated substructures like the cell or organs, for example, given in
spatial terms but also other entities that form units in a non-spatial way that
could be identified by the strength of the interactions between their components
or their functional goal. For example, one might identify functional units in
reaction networks \citep{ederer_approach_2003} and hierarchical modules in
metabolic networks \citep{ravasz_hierarchical_2002}. These modules are organised
either by interaction -- more so the case in non-spatial modules -- or by
physical forces -- more the case in spatially distinguished modules like cells
\citep[`Biological and Physical systems' section]{simon1962architecture}.

As we have pointed out, in order to understand complex biological systems, like
entire organisms, we need explanations at multiple levels of detail. At each
level of detail we can identify different modules, which can either be treated
as undivided wholes or as organisations of their elementary components/modules
\citep{varela_autopoiesis:_1974}. The links between the functioning of modules
at different levels of detail are crucial for fundamental questions in
biology. We will call these aspects of biology that deal with modules, their
organisation (interaction), and hierarchy as the \emph{organisational} aspects
of systems. Importantly this organisation changes dynamically.

Explicitly representing the functioning of modules in organisational only terms
in detail, however, is probably impractical in many cases. Even when considering
the elementary components of a module and their organisation, certain aspects of
the functioning of a module are still more practically captured through some
abstraction to pick out properties of interest that perhaps are important for
the links to the next level of detail. The abstraction that one uses to
understand the functioning of a module will depend on the level of understanding
and practical considerations.
%For example, we might not want to describe the
%mechanism of a particular process ('how') but only 'what' it does at a more
%descriptive level.
We will call these the \emph{non-organisational} or
\emph{abstract} aspects of a system (or its model).

Given that in multi-scale explanations of systems require representation of both
non-organisational and organisational aspects, the question of the technical
language used for this representation is interesting. For the organisational
parts since we are considering discrete entities, we need collections of objects
\citep{fontana_barrier_1996}. Other aspects of the organisation possibly also
then need the description of relations on these collections of objects
\citep{rashevsky_topology_1954, rosen_relational_1958}. The constructive nature
of this organisation means that we need a way to describe the
creation/destruction of objects and the creation or deletion of pairings of
objects in the defined relations. In Chromar we have simple representations of
organisational aspects where objects, agents, can be defined along with their
creation/destruction. Explicit definition of relations is not possible, but we
have seen that different kinds of relations on objects exist in object-based
languages (\sct{objectLangs}). Implicit representation is possible through agent
attributes. While possible, this puts an extra burden on the user to first
decide on a representation of the relation through agent attributes and then to
maintain this relation consistent when the state changes when rules are
applied. This burden increases if one needs to keep track of more than one
dynamic relation. This might suggest ways our notation could be
extended. Ideally the language would track such relations and have a special
notation for the most common types of relations -- for example in biology the
`connects-to' and `nested-in' relations seem natural, as we have seen
(\sct{objectLangs}). We could then write rule left-hand sides that say `this
rule is applicable to any two leaves that are connected' or `this rule is
applicable to any two cells inside the same leaf'. A combination of Chromar with
Kappa features resulting in a version of Kappa with richer types, for example,
would be very powerful.

For the non-organisational aspect the question of representation is more
difficult since the point of abstraction is to use the most convenient or
practical explanation for a particular module. Chromar allows agents (modules)
to have attributes, which are lists of typed values. While attributed objects
(colours and so on) have been used in languages before, in Chromar these take an
`elevated' status where they could stand for entire models. Their dynamics come with
discrete transitions, which is the same way the dynamics of objects
(deletion/creation) are defined. This is very general and could satisfy a range
of applications, and we have seen that the Dynamical Grammars of
\citet{mjolsness2006stochastic} also allow the description in terms of the more
familiar ODEs. 

What if we wanted to use the most convenient abstraction for the abstract parts
of each module though? Pragmatic approaches like multi-model simulators
\citep{erbm_mois_2015, cis_2018} allow that through concurrent running of
executables representing the simulation of each module. The organisation of the
modules, for example communicating the level of variables (interaction), is
usually represented externally in a configuration file. While this achieves the
goal of simulation, the resulting multi-model does not achieve the second goal
of models that we have identified regarding understanding since the organisation
is opaque. The connections are merely identification of variables and it is not
clear how these affect each other. Furthermore, these connections are expressed
in a static way, which encourages larger modules that are guaranteed to be
static.

\begin{figure}[tb]
  \centering
  \includegraphics[width=0.9\textwidth]{figures/hypothetical.eps}
  \caption{A hypothetical system inspired by Chromar that combines
    non-organisational and organisational descriptions. The state of the system
    is given by discrete objects with types (coloured circles) organised in a relationfor the
    organisational parts and with abstract dynamics attaches to these objects
    (rectangles next to circles) for the non-organisational parts. Dotted lines
    show interactions between parts of the state. The dynamics are given as
    rules that can change the organisation, synchronise variables between the
    non-organisational parts of modules or simply represent the internal
    dynamics of modules.}
    \label{fig:hyp}
\end{figure}

What the above suggest for possible future work is a system that handles both
the organisational aspects in an object-based language and the
non-organisational aspects possibly in many languages. The combination of these
in Chromar through enriched expressions suggests a way that both organisational
and abstract aspects could co-exist naturally in the same system. Consider for
example the system shown in Figure~\ref{fig:hyp}. Like Chromar, the
organisational aspects are represented by discrete objects that have types
(colours in the diagram). Non-organisational aspects exist as part of these
modules to describe aspects of their functioning that we want to capture in a
convenient abstraction. This is like the attributes of agents in Chromar, which
can represent entire models (rectangles attached to objects,
Figure~\ref{fig:hyp}). Different levels of explanation can be combined with
observables, for example a variable in one level might be a function of the
state at another level (dotted line between objects and variables in abstract
models, Figure~\ref{fig:hyp}). Other dynamics can be defined independently of
any module, like Chromar does with fluents ($f$, Figure~\ref{fig:hyp}). Both of
these can coexist with the dynamics of state variables inside each module, in a
similar way that attribute dynamics are given using enriched expressions, which
combine regular expressions, observables, and fluents.

Both the organisational and non-organisational dynamics can be given with rules,
like we have seen with Chromar. The object rules can add/delete or
relate/unrelate objects. Rules can also be used for the interaction of the
internal dynamics of each module. For example, if a state variable from one
module is needed to compute the state variable for another, this interaction can
be captured by synchronising them through a rule. Internal dynamics are captured
through rules referring only to single modules.

Unlike Chromar, however, the abstractions in each module or the independent ones
should be written in any language and explicit relations should also be
representable. In order to achieve this we will either need formal mappings
between the mathematical interpetations of the abstractions used in each module
as suggested by \citet{mjolsness_prospects_2018} or a more pragmatic system
using simulation synchronisation, for example through message passing, which is
the approach taken by \citet{cis_2018}. This does not answer the question of
explicit representation of organisation (through relations) but this could come
from any of the object-based languages we have overviewed that do have these
features (\sct{objectLangs}).

The system of organisation of processes reminds us of the graph of process
interactions of \citet{rashevsky_topology_1954}, who was one of the first to
point to the importance of considering the `relational' aspects of Biology (what
we called organisational aspects here). It is also comparable to the multiscale
topological structures of \citet{godin_multiscale_1998}, which is the main
representation structure used in another tool for simulation of plant systems
based on data-flows on graphs \citep[OpenAlea;][]{pradal_openalea:_2008}.

%\todo{refer to work on previous chapters that motivated observables etc.}

\subsection{Practical considerations}
While there is a lot of work on theoretical aspects of modelling languages,
practical aspects related to the practical use of these languages can be
overlooked especially in more theoretical (modelling) languages from computer
science. This might be because these are usually tied to short-lived research
projects or because there is too much focus on theoretical aspects of these
languages. Some exceptions exist, for example Kappa (\ssec{links}) has seen a
continuous development of a suite of practical tools for its use for a number of
years \citep[Kappa platform;][]{boutillier2018kappa}.

Chromar could similarly benefit from practical tools to overcome practical
hurdles to its use. For example, in the current implementation one would have to
have familiarity with Haskell to write Chromar rules and have the Haskell
ecosystem (compiler and so on) installed to compile models or access the
simulator. Other languages have used more mainstream and familiar languages,
like Python \citep[Kappa or L-systems in Python;][]{lopez_programming_2013,
  boudon2012py} but that still leaves the cost of having a working installation,
which is sometimes prohibitive.

Having these practical tools will be especially important for systems like the
one we outlined in the previous section that allows the combination of multiple
modules. In comprehensive models it is becoming rare that a single person writes
the entire model. A practical modelling tool that combines model definition with
a library of existing modules would be very powerful. This is the approach taken
by simulation frameworks (see \sct{simFrameworks}) in general-purpose
programming languages like CHASTE \citep[suite of software for heart
simulation][]{mirams2013chaste} or APSIM \citep[crop
modelling][]{keating_overview_2003}. Drawing from the ideas in the previous
section and the above framework examples, one could imagine a (possibly visual)
interface where existing or new modules can be connected like jigsaw puzzle
pieces similar to the way programs are defined in visual programming languages
like Scratch \citep{maloney_scratch_2010}. This perhaps could work on the web,
like a version of the Kappa platform, to lower the adoption curve.

\subsection{Database ideas}
For the observables we use an analogy of the state of the system (multiset of
agent values) to a database. The first part of the observables is defined with
a `select' statement that picks out parts of the state similarly to the
select statement in database query languages. Similar techniques appear already
in trace-query languages where we have observables over state traces (sequences
of states) instead of single states \citep{laurent_trace_2018}. The left-hand
sides of rules can similarly be thought of as select statements as they pick out
subsets of the state where the transformation represented by the rule can be
applied.

This connection can be exploited to allow the adoption of techniques from
database research for rule-based systems. One area could be to use techniques
from database research to increase the performance of language
implementations. We highlighted some possible ways our implementation could be
improved by avoiding recomputing the matches of each rule at every step of the
simulation (\ssec{simEff}). This is similar to the \emph{view maintenance
  problem}, a well studied problem in database theory. Views are query results
over a database that need to be kept consistent under changes to the underlying
database. There are efficient solutions to the problem for databases with
duplicates (database is taken as a multiset) that could be exploited in the
calculation of our matches to avoid the current naive recomputation at each step
\citep{griffin_incremental_1995}.

Apart from practical results a large amount of theory work also becomes
available, for example for query languages on multisets, which are ubiquitous in
the object-based modelling languages \citep{libkin_query_1997,
  buneman_comprehension_1994}.


%This is the key!!!!!

% made the tie between non-organisational and organisational aspects
% stronger. through fluents and observables
% abstract vs non-abstract spaces
% Make the variables more first-class
% So that they represent whole models perhaps and their connection over multiple
% level of explanation the abstract to the more concrete..enriched expressions
% suggest a way that this could be done
% could be combined

% show how examples to see the challenges to advance their resolution
% refer back to how things in the previous chapter lead to the realisation
% discussed here
% If I had more time + future

% (mention database ideas that could mean borrowing ideas from databases for, say,
% efficient implementation etc.)
% + Laurent paper
% observables over a single agent
% which suggests --->
% --->

% There are ways (see Ricardo's work to abstract a system that includes some form
% of organisation and one level of detail to another.
% Observables are not like that


%\subsection{Multi-scale models that combine}
% What does the Chromar experience suggest?
% Organisational aspects through Chromar or an object-based lang to capture either
% things that we consider objects or abstract objects like processes (see
% Barabasi)
% The non-organisational aspects can be captured in any relevant abstraction
% depending on the level of understanding Can be even treated as undivided wholes
% if needed

% Practically will need either mappings between formalisms or more pragmatically
% something like mois cis\_interface

% mois cis etc. combination of models is not a model!!


\section{Organism-centred evolutionary ecology}
\label{sec:ecology}
It is not necessary to model everything at the mechanism level ('how' level). If
the intension is to understand at an abstract level what a particular process
does as a computational unit and especially if there is lack of data (or
knowledge) then a more phenomenological description at some abstract level is
adequate and probably even preferable \citep[see levels of description
of][Chapter 1]{marr_vision:_1982}. For example the theory of syntax of
\citet{chomsky_aspects_2014} is a purely computational description of language
and does not describe how this is physically realised in neurons in the brain.

Similarly, many models in plant ecology and evolution abstract away the organism
and have an abstract view of the population as a whole. Understanding the
mechanism though is also important. This has always been true but recent
advances in understanding, the availability of detailed datasets across scales
along with developments in computational tools (\eg high-performance computing)
make this practically achievable \citep{zhu_plants_2016,
  marshall-colon_crops_2017}.

The connections between the abstract computational view of a process and its
physical realisation is vital in answering the 'how' questions of life (see
Introduction, Figure~\ref{fig:circle}). As more knowledge of the mechanism
becomes available for plants, it becomes attractive to replace the abstract with
the mechanistic to get a more mechanistically-founded view of evolutionary
processes, as we have seen in Chapter~\ref{chp:fms}.

On the opposite side of this however, connecting a low-level mechanism up the
scales to the population level might help understand and answer 'why' questions
about the evolution of the mechanism too. If evolution is a population process
\citep{doebeli_towards_2017} then modelling a mechanism in detail in the context
of the organism and the population might explain its evolution
\citep[\enquote{Nothing in biology makes sense except in light of population
  genetics};][]{lynch_origins_2007}.

As we go from individual detailed mechanisms to the population level it is
always important to experimentally test our understanding and hypotheses that we
formalise in models. While population experiments in natural conditions are
common for crop species, they usually only consider one generation. The short
length of the Arabidopsis lifecycle along with detailed molecular and genetic
understanding might make it possible to perform experiments to see the links
between genomes and the population over multiple generations in natural
conditions. \citet{rubin_circadian_2018}, for example, do field experiments
(natural weather conditions) with populations of circadian clock mutant
Arabidopsis plants to observe the effect on developmental timing and shoot
architecture. This study is again for a single generation but interestingly
includes traits like total fruit mass that could be used as proxies for
reproductive success therefore allowing validation of multi-scale models and the
evolutionary hypotheses they suggest.

% Experimental validation? From genes to populations?

% Effect of one mechanism on context? This theme is also important for the
% previous section, abstract then mechanisms can be added later. It might be
% interesting to see if we understand the effects of a single mechanism on
% higher-levels of organisation.
% Not always at the mechanism level -- see Marr's levels

% Since we're plant biologists and the plant is where we have mechanims information
% we would like to see the effect on ecology.

% Next, including population genetics and genomic variation should give
% understanding of the evolution of the clock gene circuit

% With phenology we kind of understand at the first level, going
% to the second level of mechanism (how this is realised) we need the organism.




% Apart from helping with the how questions by adding mechanism to higher level
% processes, they can also help with why question when considering a particular
% mechanism in detail.

% \section{Multi-scale organism models for engineering}
% Finally,
% for these I'm not sure we need the organism. The organism is there for the
% fundamental questions
% We can engineer the environemnt etc.

% Why do we use a multi-scale detail model?



\singlespace

%% Specify the bibliography file. Default is thesis.bib.
% \bibliography{thesis}

\printbibliography[heading=bibintoc]

%% ... that's all, folks!
\end{document}

%%% Local Variables:
%%% mode: latex
%%% TeX-master: t
%%% End:
