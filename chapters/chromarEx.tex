In this chapter we will introduce Chromar (formally introduced in the next
chapter) by means of examples. The example serve two purposes: to introduce the
features of the languages but also as a means of comparing Chromar to the related
languages in the previous chapter by seeing how Chromar features handle the
modelling requirements that the examples present.

Chromar has the following main ideas, which we will explore in this chapter
through the examples:
\begin{itemize}
\item \emph{basic Chromar} is a rule-based notation with stochastic semantics
  yielding a Continuous Time Markov Chain (CTMC). It follows the object-based
  tradition (see previous section) and makes use of discrete objects called
  \emph{agents}. These have attributes that are defined at the type level, so
  that every agent of that type has these attributes. For a version of our auxin
  example a $\mr{Cell}$ agent type could be, similarly to other parametric
  languages,
  $\mathrm{Cell}(\mathrm{pos}:\mathrm{int}, \mathrm{a}: \mathrm{real})$ with
  attribute $\mathrm{pos}$ for positional information and attribute $\mathrm{a}$
  to keep its auxin concentration. Agents are instantiations of this type with
  concrete values for the attributes like
  $\mathrm{Cell}(\mathrm{pos}=1, \mathrm{a}=a_1)$ for the first cell,
  $\mathrm{Cell}(\mathrm{pos}=2, \mathrm{a}=a_2)$ for the second cell and so
  on. The states of the CTMC are multisets of agents. The rules describe how
  agents are added to or removed from states at a more abstract level than
  individual agents. This is done by using patterns on the rule left-hand sides
  specifying the group of agents for which the dynamics are defined. As we have
  seen, having parameters help us overcome some of the limitations of object-only
  languages.
\item \emph{extended Chromar} extends basic Chromar with two new features:
  \begin{itemize}
  \item[(i)] \textit{Fluents} --- the incorporation of deterministically
    changing time-dependent values. These are important for modelling dynamics
    in a changing environment, and
  \item[(ii)] \textit{Observables} --- values calculated from a global view of
    the system. These are important in cases where a coarse-grained view of the
    system is needed. This may be because we cannot acquire atomistic data, or
    because we do not wish to model everything at the same level of
    detail. Observables also give us a flexible way to observe the state of the
    system that can be used to report the results of model simulations, as we
    often need time series of some observable on the state of the system rather
    than time series of the state itself.
\end{itemize}
\item A concrete realisation of extended Chromar as an embedded Domain Specific
  Language (DSL) inside Haskell, a functional programming language
  \citep{gibbons_functional_2015}. The embedding means that we can use any valid
  Haskell expression where expressions are expected, for example in the rate
  expressions and in the right-hand sides of rules. Agent types and the outside
  environment (defining rate, condition function etc.) are defined directly as
  Haskell expressions but rules, fluents, and observables are defined using
  abstract Chromar syntax via quotation \citep{mainland_why_2007}.
\end{itemize}

The examples from the previous section are extensions to the examples that
appear in the articles describing Chromar \citep{honorato-zimmer_chromar_2017,
  honorato-zimmer_chromar_2018}. Therefore the following text is also in part
comes from the same sources. Note that the author order follows the traditional
alphabetical order of computer science literature. Ricardo Honorato-Zimmer
participated in the meetings/discussions of ideas with Prof Gordon Plotkin who
was my primary supervisor for this part of the work while Prof Andrew Millar was
secondary.


\section{Plant development in a field}
\label{sec:plantDev}

\subsection{Single plant}
Since agents are the main entities in our language, let us
consider what types of agents we should have to model the above
system, given the above discussion. We will need:
\begin{itemize}
\item a $\mathrm{Leaf}$ type with attributes for 
appearance: $\mathrm{Leaf}(\mathrm{age}:\mathrm{int},
\mathrm{mass}:\mathrm{real})$,
\item a $\mathrm{Cell}$ type that represents our main plant `cell' with an
attribute, $\mathrm{carbon}$, to keep the current carbon level:
$\mathrm{Cell}(\mathrm{carbon}:\mathrm{real})$ (there will only be one agent
\item a $\mathrm{Ros}$ type that represents the entire Rosette with an
attribute, $\mathrm{n}$, to keep the current number of leaves:
$\mathrm{Ros}(\mathrm{n}:\mathrm{int})$ (there will also only be one agent
\end{itemize}
We will use the current number of leaves relative to the index of
a particular leaf as a proxy for the leaf's age: the larger the difference
between the index of a leaf and the current number of leaves, the older that
leaf is.

For the \textit{carbon assimilation} from one particular leaf we need to
increase the carbon concentration of the central Cell. The bigger the leaf the
faster it contributes to the production of carbon:
\[\mathrm{Leaf}(\mathrm{mass}=m), \: \mathrm{Cell}(\mathrm{carbon}=c) \:
  \xrightarrow{f(m)} \: \mathrm{Leaf}(\mathrm{mass}=m), \:
  \mathrm{Cell}(\mathrm{carbon}=c+1)\] We can read this rule as saying that for
any two Leaf and Cell agents, the Leaf agent remains the same and the Cell agent
increases its carbon content by one. Note that we assign the values of the
attributes the left-hand side of the rule to the variables $m$ and $c$, so that
we can refer to them in the right-hand side of the rule and the rate
expression. Since the $\mathrm{age}$ is not used in the rest of the rule, it can
be omitted (see \textit{Missing attributes} syntactic extension,
\sct{syntaxExt}). If we were to write this in a traditional reaction notation we
would have to write a reaction for every possible Leaf agent which leads to the
compactness problem we have noted earlier (indeed, there would be infinitely
many such possibilities). The implicit `for-all' in the pattern on the left-hand
side allows the rule to be applied to new leaves when they are created. For
example if the state of the system has two leaves, the central cell, and the
rosette agent:
\begin{align*}
\m{ \mathrm{Leaf}(\ar{age}{1}, \ar{mass}{10}), \:
\mathrm{Leaf}(\ar{age}{2}, \ar{mass}{5}),
\mathrm{Cell}(\ar{carbon}{20}), \mr{Ros}(\ar{n}{2})}
\end{align*}
then the rule is applicable to the two substates
\begin{align*}
& \m{\mathrm{Leaf}(\mathrm{age}=1, \mathrm{mass}=10),
                 \mathrm{Cell}(\mathrm{carbon}=20)} \: \text{ and} \\
&\m{\mathrm{Leaf}(\mathrm{age}=2,
  \mathrm{mass}=5), \mathrm{Cell}(\mathrm{carbon}=20)}
\end{align*}

For \textit{maintenance respiration}, the central $\mathrm{Cell}$ agent gives
some carbon to a $\mathrm{Leaf}$ agent, with the amount of carbon needed for
maintenance depending on the size of the Leaf:
%
\begin{align*}
&\mathrm{Leaf}(\mathrm{mass}=m),\: \mathrm{Cell}(\mathrm{carbon}=c) \:
\xrightarrow{h(m)} \\ &\mathrm{Leaf}(\mathrm{mass}=m),\:
\mathrm{Cell}(\mathrm{carbon}=c-g(m)) \; [c \geq g(m)]
\end{align*}
Note the use of the condition $c \geq g(m)$ to make sure that the
carbon levels do not become negative. Also note that here we use both the rate
($h(m)$) and the increment size ($g(m)$) to control the amount by which carbon
is updated in a time unit. Since the rate controls the number of times the
update will happen, the product of the rate and update functions is the expected
amount that will be removed from the carbon pool in a time unit.

Next, \textit{leaf growth} depends on the mass of the leaf, its age (there is a
limit on how much a leaf can grow so older leaves stop growing at some point),
and the amount of carbon available:
%
\begin{align*}
  & \mathrm{Ros}(\mathrm{n}=n), \mathrm{Leaf}(\mathrm{age}=i,
\mathrm{mass}=m),\: \mathrm{Cell}(\mathrm{carbon}=c) \: \xrightarrow{h(n-i, m,
    c)}\: \\
  & \mathrm{Ros}(\mathrm{n} =n), \mathrm{Leaf}(\mathrm{age}=i,
\mathrm{mass}=m+1),\: \mathrm{Cell}(\mathrm{carbon}=c-1) \; [c \geq 1]
\end{align*} The mass of the leaf is also needed in the calculation to make sure
that the growth does not exceed observed physical constraints. The growth is
capped to a fraction of the current mass of the leaf.

Finally, for \textit{leaf creation} we have:
\begin{equation*} \mathrm{Ros}(\mathrm{n}=n) \: \xrightarrow{k} \:
\mathrm{Ros}(\mathrm{n}=n+1), \: \mathrm{Leaf}(\mathrm{age}=n+1,
\mathrm{mass}=m_0)
\end{equation*}

\subsubsection*{Fluents and Observables}
There are two problems with the above definition: 
\begin{itemize}
\item[(i)] There is no way to include the environment in the model, and so it is
  assumed constant. This makes the model very detached from reality.  For
  example, our plants photosynthesise all the time, whereas in reality they only
  do so during daylight, but we have no direct way to switch between day and
  night.
\item[(ii)] We have two representations of the same process at two different
  levels of abstraction that have to be kept consistent with each other by the
  user.  Specifically, the $\mathrm{n}$ attribute in the $\mathrm{Ros}$ agent
  keeps track of the total number of leaves in the plant. However, there is no
  way in the language to specify the global information the attribute is
  tracking, and check that its value is indeed what it should be.
\end{itemize}

I next introduce two new notational features
to solve these problems.

\textit{Fluents}

To solve the first problem I introduced time-dependent values called
\textit{Fluents}. These can be constructed through a combination of a small set
of primitives and general expressions, taken from Reactive Programming (FRP)
constructs in \citep{wan_functional_2000} (Fluents are usually called Behaviours
in FRP). For example we could write a fluent for light, which is a function from
time to the Booleans, and another one for temperature that depends on light:
\begin{align*}
& light = \mathbf{repeatEvery} \; 24.0 \; (\mathrm{True} \; \mathbf{when} \; (6 < \mathbf{time} < 18) \; \mathbf{else} \; \mathrm{False}) \\
& temp = 21.0 \; \mathbf{when} \; light \; \mathbf{else} \; 16.0
\end{align*}
where the $\mathbf{when} \dotso \mathbf{else}$ construct denotes conditional
behaviour and $\mathbf{repeatEvery}$ cycling behaviour. Cycling behaviour is
achieved with the modulo operation so the value of $\mathbf{repeatEvery} \; t_o
\; f$ at $t$ is the value of $f$ at $t \; mod \; t_0$.

The assimilation rule can then be written as follows:
\begin{align*}
& \mathrm{Leaf}(\mathrm{age} \!= \!i, \mathrm{mass} \!= \!m), \: \mathrm{Cell}(\mathrm{carbon} \!= \!c) \: \xrightarrow{f(m, temp)}  \\
&\mathrm{Leaf}(\mathrm{age} \!= \!i, \mathrm{mass} \!= \!m), \:
  \mathrm{Cell}(\mathrm{carbon} \!= \!c+1) \; [light ]
\end{align*}

Very often in biology we have empirical relationships of various quantities with
time. Fluents can be used to encode these as well. Such empirical relationships
do not provide any mechanistic insight but they can be useful when we either do
not have data or do not want to model more mechanistically using rules.

\textit{Observables}

To solve the second problem, I introduced functions on the state of the system
called \textit{observables}. Observables are constructed using a combination of
database-inspired $\mathbf{select}$ and $\mathbf{aggregate}$ operations: % we
think of our state as a kind of database where we have a collection of agents
rather than the more usual (and very similar) collection of records. The
$ \mathbf{select}$ operation selects agents of a particular type
from %specifies a
part of the state, producing a binding of each such agent's attribute values;
the $\mathbf{aggregate}$ operation then folds the resulting collection of
attribute value bindings into a single value, using a specified combining
function and initial value.

For example, to calculate the total mass  of the leafs in a state, we could write:
 \begin{align*}
lm = \mathbf{select} \, \mathrm{Leaf}(\mathrm{age} = i, \mathrm{mass} = m) \mathbf{;} \; \mathbf{aggregate} \;
 (total: \mathrm{int}.\, total + m) \; 0
\end{align*}
%
where we first select every agent that matches a $\mathrm{Leaf}$ pattern,
producing a binding of its $\mathrm{age}$ and $\mathrm{mass}$ attributes to $i$
and $m$, respectively, and then, starting from $0$, calculate the result using a
combining function that adds the value of $m$ in every such binding to the
running total.
 %
If we further wished to calculate the average mass of all the leafs in a state,
we could use the observable
\begin{align*}
nl = \mathbf{select} \, \mathrm{Leaf}(\mathrm{age} = i, \mathrm{mass} = m) \mathbf{;} \; \mathbf{aggregate} \;
 (count: \mathrm{int}.\, count + 1) \; 0
\end{align*}
%
to count the number of leafs in a state, and then divide $lm$ by $nl$.

Whenever an observable is used, one obtains its `fresh' value, even when the
underlying state has changed (\eg, by creating new leaves, in this case). For
example the \textit{leaf growth} rule now becomes:
%
\begin{align*}
&\mathrm{Leaf}(\mathrm{age} \!= \!i, \mathrm{mass} \!= \!m),\:
  \mathrm{Cell}(\mathrm{carbon} \!= \!c) \xrightarrow{h(nl-i, m, c)}\:   \\
  &
 \mathrm{Leaf}(\mathrm{age} \!= \!i, \mathrm{mass} \!= \!m+1),\:
    \mathrm{Cell}(\mathrm{carbon} \!= \!c-1) \; [c \geq  1]
\end{align*}
where we use our observable $nl$ directly in the rule rate. There is no need to
use the $\mathrm{Ros}$ agent any more as its only purpose was to keep track of
the number of leaves.

Note that observables, like fluents, can be used to model parts of the system in
cases where there is either not enough knowledge about a process, or else no
desire to model at a more mechanistic level. In this case, for example, the
leaves must have some mechanism for knowing their age. However this is not
central to our model, so we instead use global knowledge via observables to
abstract away from the details that would be involved in modelling that
mechanism.

The two features, fluents and observables, can be mixed arbitrarily along with
ordinary expressions. We might for example have a fluent inside an observable or
a fluent inside an observable and so on. For example we could write:
\begin{equation*}
f(m) \; \mathbf{when} \; nl > 10 \; \mathbf{else} \; f'(m)
\end{equation*}
where $nl$ is an observable for the number of leaves. This might be used to
introduce the crowding effect on rosette leaves; crowding affects the
assimilation rate as it reduces the photosynthetically active area.

\subsection{Field}
Going to the field where we have multiple plants we need some way of knowing
first which agents belong to the same plant and second the location of the
plant. We do not have an explicit way of representing these relations but we can
again use our attributes to encode them. I will use one identifier for the plant
and one for the patch of land. Every agent now has a further attribute to store
this information. For example, the $\mr{Leaf}$ agent type becomes:
$\mathrm{Leaf}(\mr{pos}:(\mr{int}, \mr{int}), \mathrm{age}:\mathrm{int},
\mathrm{mass}:\mathrm{real})$. The first element of the $\mr{pos}$ tuple is the
plant identifier and the second is the identifier of the patch it belongs to. I
further assume that we are in a suitably powerful programming language (\eg our
Haskell embedding) that I have (or can define) the functions $\mr{pid}$ to
select the first element of the tuple (plant id) and $\mr{cid}$ to select the
second element of the tuple (patch id).

The metabolic rules stay the same but add a further condition that the
agents on the lhs belong to the same plant. For example, maintenance respiration
now becomes:
\begin{align*}
  &\mathrm{Leaf}(\mathrm{pos}=p, \mathrm{mass}=m),\:
    \mathrm{Cell}(\mathrm{pos}=p', \mathrm{carbon}=c) \:
    \xrightarrow{h(m)} \\ &\mathrm{Leaf}(\mathrm{mass}=m),\:
                            \mathrm{Cell}(\mathrm{carbon}=c-g(m)) \; [c \geq g(m)
                            \, \land pid \, p = pid \, p']
\end{align*}

We assume that competition between plant affects the amount of light they
receive, which in turn affects the photosynthesis rate. The assimilation rule
will therefore change so that the light fluent is modulated by a competition
index, which will be an observable on the global state of leaves in an entire
patch. Here we assume that we are in our Haskell embedding and we have access to
the full power of the language. We define the following Haskell functions:
\begin{align*}
& \mr{comp} \: p = f (\mathbf{select} \; \mathrm{Leaf}(\mathrm{pos}=p' ,
  \mathrm{mass} = m) \; ; \; \mathbf{aggregate} \; (c: \mathrm{real}. \,
  \mr{sumComp}) \; 0 \\
& \mr{light} \: p = \mathbf{repeatEvery} \; 24.0 \; (\mathrm{light_{max}
                          } \cdot \mr{comp} \, p \; \mathbf{when} \; (6 < \mathbf{time} < 18) \; \mathbf{else} \; \mathrm{0})
\end{align*}
that take as input a $\mr{pos}$ tuple and return a relevant expression for that
particular plant in that patch. The combining function $\mr{sumComp}$ is defined
as: $\mr{sumComp} = \mathbf{if}\; cid \, p' = cid \, p \land pid \, p' \neq pid \, p
\;\mathbf{then} \; c + m; \; \mathbf{else} \; c$ that sums the masses of all
leaves in a patch except from the given plant. Then with the modulated light
definition the assimilation rules will change to:
\begin{align*}
& \mathrm{Leaf}(\mathrm{pos} \!= \!p, \mathrm{age} \!= \!i, \mathrm{mass} \!= \!m), \:
                 \mathrm{Cell}(\mathrm{carbon} \!= \!c) \: \xrightarrow{f(m,
                 temp, light \, p)}  \\
&\mathrm{Leaf}(\mathrm{age} \!= \!i, \mathrm{mass} \!= \!m), \:
  \mathrm{Cell}(\mathrm{carbon} \!= \!c+1) \; [light \, p > 0]
\end{align*}

\section{Root development}
\label{sec:rootDev}
For the root example, we again assume we are in our Haskell implementation and
we therefore have Haskell types and expressions in our disposal for the
definition of the model. We define the following agent types:
\begin{itemize}
\item a $\mr{Cell}$ type to represent the root cells in the array defined as:
  \begin{align*}
    \mr{Cell}(&\mr{id}: \mr{int}, \\
              &\mr{l, r}: \mr{link}, \\
              &\mr{s, a, d}: \mr{real}, \\
              &\mr{m}:\mr{mode})
    \end{align*}
    The type label $\mr{link}$ is interpreted as the set
    $\mathbb{N} \cup \{\epsilon\}$ and the relevant attributes ($\mr{l}$ for the
    left and $\mr{r}$ for the right neighbour) represent either a bound state
    (via an identifier) or a non-bound state ($\epsilon$). The type label
    $\mr{mode}$ is intepreted as the set $\{ 1, 2 \}$ and it represents the
    state of the cell, either growing (1) or idle (2). The other attributes
    represent the size of the cell ($\mr{s}$) and its auxin ($\mr{a}$) and
    division-factor ($\mr{d}$) concentrations (Figure~\ref{fig:rootDevChromar}).
  \item a $\mr{Shoot}$ agent type to represent the entire shoot of the plant
    defined as $\mr{Shoot}(\mr{s}: \mr{real})$ where the attribute $\mr{s}$
    represents the size of the shoot.
  \end{itemize}
  
\begin{figure}
    \centering
    \includegraphics[width=0.7\linewidth]{figures/rootChromar.eps}
    \caption{Scheme for capturing the `next-to' relation between cells and the
      division rule.}
    \label{fig:rootDevChromar}
  \end{figure}

In order to be able to give fresh identifiers to cells created by division we
also need to know the current number of cells in the array. We do this using
an observable:
\begin{align*}
nc = \mathbf{select} \; \mathrm{Cell}()\mathbf{;} \;\mathbf{aggregate} \; (count: \mathrm{int}.\, count + 1) \; 0
\end{align*}
As with our plant example in the previous section, we use a $\mathrm{count}$
expression that increases the running count by $1$ for each
$\mathrm{Cell}$. Note that like rules we can omit attribute bindings when the
values are not used in the subsequent definitions of the expression.

Turning to the dynamics of the system, for the auxin flow from the shoot we can
have the following rule:
\begin{align*}
\mr{Cell}(\mr{l} \!=\! \epsilon, a=a) \xrightarrow{} \mr{Cell}(\mr{a}=a + (a_{init} + \frac{0.17 \cdot \mathbf{time}}{t_{cc}}))
\end{align*}
where $t_{cc}$ is the time duration of the cell cycle. We represent auxin flow
from the shoot as auxin being produced in the first cell in the array. We
distinguish the first cell by requiring it to be unbound to the left. 

For the auxin diffusion we can write the following:
\begin{align*}
\mr{Cell}(\ar{r}{i}, \ar{a}{a}), \mr{Cell}(\ar{id}{i}, \ar{a}{a'})
  \xrightarrow{} \mr{Cell}(\ar{a}{a + D(a'-a)}), \mr{Cell}(\ar{a}{a' + D(a - a')})
\end{align*}

For the active transport of auxin in the cell by PIN proteins we have,
\begin{align*}
& \mr{Cell}(\ar{r}{i}, \ar{a}{a}), \mr{Cell}(\ar{id}{i}, \ar{a}{a'})
                 \xrightarrow{} \\
  &\mr{Cell}(\ar{a}{a - K_0 a PIN(a)}), \mr{Cell}(\ar{a}{a' + K_0
  a PIN(a)})
\end{align*}
Note that instead of modelling $\mr{PIN}$ concentration and its effect on auxin
directly, we do so implicitly through a function that modulates the
amount of auxin being transported. Unlike diffusion, which is bidirectional,
active transport takes place in the direction from shoot to root cap.

The production of the division factor is affected by the difference in auxin
concentration between neighbouring cells,
\begin{align*}
\mr{Cell}(\ar{id}{i}, \ar{a}{a'}), \mr{Cell}(\ar{l}{i}, \ar{a}{a}, \ar{d}{d})
  \xrightarrow{} \mr{Cell}(), \mr{Cell}(\ar{d}{d+\sigma(a'-a)})
\end{align*}

For the growth of the cell and degradation of the concentrations of the division
factor, $\mr{d}$, and auxin concentration, $\mr{a}$, we write:
\begin{align*}
\mr{Cell}(\ar{s}{s}, \ar{a}{a}, \ar{d}{d}) \xrightarrow{} \mr{Cell}(\ar{a}{a (1 - d +
  \frac{v(s)}{s})}, d=d (1 - k(a) + \frac{v(s)}{s}), \ar{s}{v(s)})
\end{align*}
The degradation of the division factor, again, depends on the auxin
concnetration in the cell.

For the cell cycle we need to switch the mode of the cell, $\mr{m}$, from
growing to idle.
\begin{align*}
  \mr{Cell}(\ar{m}{1}, \ar{s}{s}) \xrightarrow{\rho(s)} \mr{Cell}(\ar{m}{2})
  \end{align*}
  with
  \begin{equation*}
\rho(s) = \frac{1}{1 + e^{- \frac{s-s_{\mr{min}}}{\mathbf{time}}}}
\end{equation*}

For cell division, we create a new cell on the right of the dividing cell and
split the concentrations of auxin and the dividing factor between the two cells:
\begin{align*}
  &\mr{Cell}(\ar{id}{i}, \ar{r}{k}, \ar{a}{a}, \ar{d}{d}, \ar{s}{s}, \ar{m}{2}),
  \mr{Cell}(\ar{id}{k}, \ar{l}{i}) \xrightarrow{\rho(d)} \\
  & \mr{Cell}(\ar{id}{i}, \ar{id}{nc}, \ar{a}{a/2}, \ar{d}{d/2}, \ar{s}{s/2},
    m=1), \\
  &  \mr{Cell}(\ar{id}{nc}, \ar{l}{i}, \ar{r}{k}, \ar{a}{a/2}, \ar{d}{d/2}, \ar{s}{s/2},
    m=1), \\
  &  \mr{Cell}(\ar{id}{k}, \ar{l}{nc})
\end{align*}
The cell divides only in the idle phase and the division sets the the mode back
to growing while the rate of division depends on the concentration of the
division factor. The dividing cell gets pushed to the left, keeping its id and
changing its right-neighbour identifier to the identifier of the new cell. The
new cell gets a fresh identifier from our cell counter observable $nc$ and a
right-neighbour identifier from the old neighbour of its mother cell
(Figure~\ref{fig:rootDevChromar}).

Finally, cells die at the root cap:
\begin{align*}
\mr{Cell}(\ar{r}{\epsilon}) \xrightarrow{\rho_{\mr{death}}} \emptyset
\end{align*}
We distinguish the end of the root by requiring the cell to be unbound on the
right.

Finally, the growth of the shoot follows an abstract functional form:
\begin{align*}
\mr{Shoot}(\ar{s}{s}) \xrightarrow{} \mr{Shoot}(\ar{s}{f(w, \mathbf{time})})
\end{align*}
The growth function is a phenomenological function of time and more
interestingly water uptake. Assuming that the water uptake depends on the root
size, we model it as follows:
\begin{align*}
& rs  = \mathbf{select} \; \mathrm{Cell}(\ar{s}{s}) \mathbf{;} \; \mathbf{aggregate} \;
                                                              (total:
                                                              \mathrm{real}.\,
                                                              total + s) \; 0 \\
& w = f(\mathbf{time}) \; \mathbf{when} \; rs > s_0 \; \mathbf{else} \;
f(\mathbf{time}) \cdot d                                       
\end{align*}
where $rs$ is an observable abstracting the size of the entire root from the
size of the constituent cells and $w$ is a fluent that uses that size to compute
the water uptake. The bigger the root the more water the plant takes. We use the
fluent here to model a process that while relevant for our process we do not
wish to model in detail and mechanistically via rules.

\section{Comparison to related work}
Having seen the properties of Chromar in action in this chapter and of other
languages in the same design space in the previous chapter, we can compare their
features. Here I will discuss properties relating to general usability features
that can be understood without the formal definitions. Comparison and discussion
related to more technical parts is done at the point where these are introduced
(see next Chapters for formal definition of Chromar and its embedding in
Haskell).

Table~\ref{tab:comp} shows a summary of the discussion points and a comparison
of the main features of Chromar and the languages from the previous chapter.

\begin{center}
  \resizebox {0.95\textwidth }{!}{%
  \begin{tabular}{p{0.13\linewidth}p{0.13\linewidth}p{0.13\linewidth}p{0.13\linewidth}p{0.13\linewidth}p{0.13\linewidth}p{0.13\linewidth}}
    \toprule
    & PN & SMMR & CPN & DG & Simile & Chromar\\
    \midrule
Objects \newline (dynamics) & Yes \newline (rule) & Yes \newline (rule) & Yes
                                                                          \newline
                                                                          (rule)
                      & Yes \newline(rule) & Yes \newline (population) & Yes \newline (rule)\\
    \addlinespace[0.2cm]
Relations \newline (explicit) & No & Yes \newline (hierarchy) & No \newline
                                                                (implicit via
                                                                attrs) & No \newline (implicit via attrs) & No \newline (implicit via attrs) & No \newline (implicit via attrs)\\
    \addlinespace[0.2cm]
    Properties \newline (dynamics) & No & No & Yes \newline
(transitions) & Yes \newline (transitions or ODEs) & Yes \newline (ODEs) & Yes \newline
(transitions) \\
    \addlinespace[0.2cm]
    State \newline abstraction & No & No & No & No & Yes & Yes\\
    \addlinespace[0.2cm]
    Time & No & No & No & Yes \newline (via ODEs) & Yes & Yes\\
    Intepretation & CTMC& CTMC& CTMC& CTMC + \newline others & - & CTMC\\
    \bottomrule
    \label{tab:comp}
  \end{tabular}}
\end{center}

\subsection{Declarative modelling}
A distinction exists in programming languages between imperative and declarative
styles. In the imperative style a computer program consists of a series of
instruction of how the described process should be computer. On the other hand
in more declarative languages the programmer defines what the program should do
but not how that should be done. Mainstream programming languages follow the
imperative style, which has prevailed, perhaps since computer hardware is
inherently imperative \citep{backus2007can}.

A similar distinction appears in modelling where some models use a declarative
style while others are more imperative mixing the what with the how. The usual
mathematical languages used in the description of physical processes, like
differential equations, for example, are declarative statements about how
functions of time (and/or space) on the described quantities relate to the rates
of change of these functions. On the other hand, as we have seen, many models
are represented as programs in imperative languages, which means that the
represented biological knowledge is coupled with the how the process should be
simulated along with other statements relating to particular choices for the
representation of the state of the system (Models as programs
\sct{pragmaticApproaches}). While libraries or frameworks provide ways to
separate to an extent the biology from the implementation of the simulation (see
\sct{simFrameworks}), this is not guaranteed.

If we accept the view that models should serve as both tools for simulation as
well as tools for thought for our understanding then procedural representations
are not adequate. Chromar follows the declarative style. Rules correspond to
biological events and the user declares how the state changes during the
event. The simulation need not be defined as the intepretation as a CTMC gives
that without any extra effort. Here I gave a particular implementation of the
language as an embedded language inside Haskell using the Haskell expressions as
an expression language. While Haskell follows the declarative style, certain
properties of the biology might still be obscured depending on how much of the
model lives in the rules and how much in the expressions implementing changes to
the agent attributes. Different choices can easily be made though, for example
we could restrict the expressions to small set of mathematical expressions
\citep[perhaps a subset of MathML][]{ausbrooks2003mathematical} if the increased
power of a general-purpose programming language is not required.

Another common advantage of the declarative style over a more procedural style
is model composability. This exists in Chromar too and it is crucial for the
more comprehensive multi-models that we are interested in in this work. In
Chromar composing two models is just a matter of concatenating their rule
sets. An example of this is the multi-model lifecycle model of Arabidopsis I
present in Chapter~\ref{chp:fms}. This is not as straighforward in procedural
model implementations where the whole simulation and representation would have
to be altered.


\subsection{Basic Chromar}
The idea of extending simple objects with fields to represent some of their
attributes has been used before, as we have seen, for example in Coloured Petri
Nets \citep{jensen_coloured_1987}, in a rule-based setting, in CSMMR
\citep{oury_coloured_2011} and Dynamical Grammars \citep{mjolsness2006stochastic},
and in parametric L-systems. Our notation is inspired by these.

The correspondence of CPNs to basic Chromar is straightforward, colour sets are
our agent types (records with named attributes), tokens are our agents, and
transitions are our rules. CPN transitions also have predicates that are the
same as our conditions. One can think of it as a minimal rule-based (and
textual) version of stochastic coloured petri nets, where the richer types are
first class and not merely a means of translation to a non-coloured version; one
can also think of it as a simpler version of CSMMR with only colours left. While
graphical notations are intuitive for smaller models, we have found that for
larger models they become hard to read whereas text-based approaches like our
language produce much more readable representations.

Like the development of other languages we have seen in the previous chapter,
basic Chromar provides an extension to the representation of reactions, where
the simple named species are upgraded to agents employing rich types, namely
records with named typed attributes. Writing rules using these richer types
yields a more compact representation than one would get by writing reactions on
the simpler types in the traditional reaction setting. Moreover it sometimes
helps with writing systems that are difficult to write otherwise as it permits
the dynamic creation of agents and therefore their attributes. These are the
limitations we noted for the simple reactions in PNs.

Attributes can be used to encode relations like the `next-to' relation in the
root development example.  However, whenever we use them to encode binding we
would probably be better off using a language that represents these directly
such as Kappa~\cite{danos2008} or BioNetGen~\cite{blinov2004bionetgen}. In
general Chromar rule left-hand sides are simple and can only select using types,
which means the only relations we can encode directly are products of types (we
can think of types as sets containing all the agents of that type). This is
often unproblematic, for example in our Plant system (from \sct{overview})
\textit{all} $\mathrm{Leaf}$ agents interact with \textit{all} $\mathrm{Cell}$
(only one in this case) agents so writing the left-hand at the type level works
because the rules are then applicable to exactly the pairs of agents we want,
$\{ (\mathrm{Leaf}_1, \mathrm{Cell}), (\mathrm{Leaf}_2, \mathrm{Cell}), \dots
\}$.

In other cases though, the relation we want is some subset of the product of the
types. For example in our array of cells example the diffusion rule is not
applicable to all pairs of $\mathrm{Cell}$s so writing
$\mathrm{Cell}(\dots), \mathrm{Cell}(\dots)$ on the left-hand side gives us more
pairs than we want.  In that case we had to encode the relation through
identifiers and the next-to relation pairs were stored inside our agents. The
selection was then restricted using attribute conditional expressions. While it
works in this simple example, it might become problematic in large models when
multiple relations are present, for example.

However, note that in the dividing cell and diffusion model of the previous
section we use colours in other ways that cannot be easily represented as
binding, or any relation offered by other object-only languages. In particular,
the division of the contents of a cell would be hard to express in any of the
object-only languages.


\subsection{Extended Chromar}
Our use of database inspired operations for the observables is also new and we
have found it very useful in model building. The declarative nature of our
multiset query primitives makes the definition of observables very
intuitive. Similar database-inspired query operations on top of collections are
used in LINQ~\citep{budiu_compiler_2013} although the collections are usually taken
to be lists not multisets. Buneman's comprehension
syntax~\citep{buneman_comprehension_1994}, a collection query language similar to
practical database query languages, considers other types of collections,
including multisets. These are, however, not used in the modelling world.

Observables could be developed further and made into first-class entities in
the language, for example by making them attributes of types. For instance in
our plant growth example we would keep our initial $\mathrm{Ros}$ type and use
our observable primitives to define its $n$ attribute instead of defining the
observables outside our agents. First-class observables would work particularly
well with an extension for a native representation of levels (the `nested-in'
relation we noted earlier), in which case the idea of the agent attributes at a
higher level being observables of agents at a lower-level would be both
intuitive and powerful.

Fluents allow the convenient expression of changing values, which is
important as models become more comprehensive and start reaching outside their
domain, for example traditional plant biology models reaching out to the outside
world and to crop science.

Fluents can be expressed, albeit in the ODE formulation, in Dynamical
grammars. Simulators of some of the languages in the overview in the previous
chapter also allow observables for result reporting. However, the lifting of
fluents and observables to first-class entities and their combination with
expression coming (potentially) from a general purpose programming language is
new and makes for a very powerful extension to the language.


\subsection{Haskell}
Our embedding in a programming language increases expressive power, and fits
with the availability of rich types.  There have been other languages that,
while not giving the full power of a programming language, still allow for
complex expressions to appear, e.g., inside rate expressions. For example in
React(C) \citep{john_biochemical_2011} rate expressions can be build from a subset
of a functional programming language with a reflection option to get access to
the full state of the system. This allows conditions to be encoded inside rates
by setting the rates to zero if they are not met. Simulators for other widely
used languages like KaSim (the simulator for Kappa~\citep{danos2008}) and
BioNetGen~\citep{blinov_bionetgen:_2004} also allow more sophisticated rate
expressions than traditional mass-action kinetic rates.

Our embedding in Haskell lifts some of the constraints of modelling languages
and, we think, gets the best of both worlds: it naturally and succinctly
captures some elements in our domain of interest but, at the same time, when
greater expressive power is needed we can turn to the programming language. This
increase in expressive power might come at the expense of the ability to do
general analysis of models since we cannot say much about what is happening in
the Haskell \texttt{exprs} inside the rules. There seems to be a trend, though,
in the direction of mixing domain-specificity and general purpose programming
languages, for example \citet{pedersen_high-level_2015} allow embedded F\#
scripts inside LBS-$\kappa$, and in PySB \citep{lopez_programming_2013} Kappa
models are defined inside Python. Embedding a domain-specific language inside a
programming language, as we did, is in some cases better than doing the opposite
-- embedding a programming language inside a domain-specific language -- because
we have fewer constraints on our definitions and more generally full access to
the language for structuring our model definitions. We also note, as we have
seen, that there is a hybrid approach that allows mixing the graphical
definition of CPNs with programming language constructs (ML
language)~\citep{jensen_coloured_1987}.

\section{Conclusion}
In conclusion, extended Chromar and its concrete realisation in Haskell give a
highly expressive language that is particularly good for the description of
models with a simple structure, such as the Plant example in
Section~\ref{sec:plantDev} and the root development example in
Section~\ref{sec:rootDev}. The main ideas of Chromar --- two levels of dynamics
(attribute and agent levels), a flexible system of observation, and a
combination of domain-specific and general programming languages --- should
carry to other frameworks or languages, for example ones with connection and
nesting and perhaps other complex structure.

