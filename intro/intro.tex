\documentclass[phd]{infthesis}
\usepackage[utf8]{inputenc}
\usepackage[T1]{fontenc}
\usepackage[british]{babel}
\usepackage{microtype}
\usepackage[usenames,dvipsnames,svgnames,table]{xcolor}
\usepackage[english=british,autopunct=false]{csquotes}
\usepackage[natbib=true,style=authoryear-comp,maxbibnames=6]{biblatex}
\usepackage{graphicx}
\usepackage{textcomp}
\usepackage{wrapfig}
\usepackage{xfrac}
\usepackage{xspace}
\usepackage{mathcommon}
\usepackage[sc]{mathpazo}
\usepackage{hyperref}
\usepackage{expl3}
\usepackage{enumitem}
\usepackage{booktabs}
\usepackage{tabularx}
\usepackage{mathpartir}
\usepackage{fancyvrb}
\usepackage{inconsolata}
\usepackage{multirow}

\frenchspacing

% Bibliography
\addbibresource{intro.bib}
\bibliography{intro}

% Text
\newcommand{\ie}{i.e.\xspace}
\newcommand{\eg}{e.g.\xspace}

% Referencing
\newcommand{\chp}[1]{\S\ref{chp:#1}}
\newcommand{\sct}[1]{\S\ref{sec:#1}}
\newcommand{\ssec}[1]{\S\ref{subsec:#1}}
\newcommand{\eqn}[1]{Eq.~\ref{eq:#1}}
\newcommand{\eqns}[2]{Eq. \ref{eq:#1} and \ref{eq:#2}}
\newcommand{\lem}[1]{Lemma~\ref{lemma:#1}}
\newcommand{\lems}[2]{Lemmas \ref{lemma:#1} and \ref{lemma:#2}}
\newcommand{\thm}[1]{Th.~\ref{thm:#1}}
\newcommand{\fig}[1]{Fig.~\ref{fig:#1}}
\newcommand{\diagram}[1]{diagram~\ref{eq:#1}}
\newcommand{\app}[1]{Appendix~\ref{app:#1}}
\newcommand{\mcite}[1]{\mrcolor{gray}{#1}} % missing cite
\newcommand{\defn}[1]{Def.~\ref{def:#1}}
\newcommand{\prop}[1]{Prop.~\ref{prop:#1}}

% Math
\renewcommand{\tuple}[1]{(#1)}
\DeclareMathOperator*{\expn}{exp}
\renewcommand*{\exp}[1]{e^{\,#1}} % \mathrm{e}^{#1}}
\renewcommand{\qedsymbol}{\ensuremath{\blacksquare}}
\newcommand{\partialto}{\rightharpoonup}
\newcommand{\id}{\vec{1}} % identity function
\newcommand{\mr}[1]{\mathrm{#1}}

\newcommand{\den}[1]{\llbracket #1 \rrbracket}
\newcommand{\m}[1]{\{\!| #1 |\!\}}
\newcommand{\M}[1]{\mathcal{#1}}
\newcommand{\MS}[0]{\mathrm{M}}
\newcommand{\SQ}[0]{\mathrm{S}}
\newcommand{\s}[1]{\underline{#1}}
\newcommand{\G}[0]{\Gamma}
\newcommand{\D}[0]{\Delta}
\newcommand{\mytt}{t\!t}
\newcommand{\myff}{f\!\!f}

\newcommand{\V}{\mathrm{V}}

\newcommand{\sel}{\mathrm{sel}}
\newcommand{\fold}{\mathrm{fold}}

\newcommand{\ms}{\mathrm{ms}}


\newtheorem{mydef}{Definition}
\def\dotminus{\mathbin{\ooalign{\hss\raise1ex\hbox{.}\hss\cr
  \mathsurround=0pt$-$}}}
\setlength{\tabcolsep}{8pt}
\renewcommand{\arraystretch}{1.0}

\newcommand{\match}{m}
\newcommand{\up}[1]{\uparrow\! #1}

\newcommand{\n}{\mathrm{n}}

% Other stuff
\newcommand{\maybe}[1]{\mrcolor{gray}{#1}}
\newcommand{\todo}[1]{\mrcolor{red}{TODO: #1}}

% rules
\newcommand{\ar}[2]{\mr{#1} \! = \! {#2}}

\setlength{\tabcolsep}{8pt}
\renewcommand{\arraystretch}{1.2}

\begin{document}
\chapter{Introduction}
% In light of this discussion I will also need to change the dicsussion in the
% next chapter

Life is produced and sustained through the organisation of entities at various
levels that interact to construct other autonomous entities. Organisation and
structure are therefore fundamental and defining properties of all living
systems. Some go as far as to say that (self-)organisation and the resulting
self-maintenance are the only common and defining characteristics of living
organisms and that they are primary to even natural selection in shaping the
forms of life that we see \citep[sometimes referred to as structuralism, for
example;][]{thompson1942growth, kauffman1992origins}. Even if we do not take such an
extreme view, it is still true that if we want to understand life we have to
understand the organisation that sustains it and the links between processes at
all scales defined by this organisation -- from genes to organisms and
ecosystems. Organisation here means both spatial organisation, for example cells
in a tissue, but also relations or interactions between processes that give rise
to other higher-level phenomena or between processes and the environment

% other less fundamental question could possibly be answered but at the core
% we need this
This view of organisation as a fundamental and defining principle of Biology
makes the subject distinct from other natural sciences, like Physics for example
and precludes a natural mode of enquiry, which is abstraction. The nature of the
questions is different. How can we abstract the details of an organism if those
details are exactly what makes the organism alive and in fact what we are trying
to understand? Since abstraction or at least complete abstraction is not
available, in order to answer biological questions we have to take a more
holistic and multi-scale view of the natural systems of interest.

%reductionist vs system (organisation)
Despite the importance of organisation in understanding life, historically the
focus in Biology has been on understanding individual mechanisms at the
molecular level. The reductionist view of life is that if cell is the main unit
of life then if we understand all the processes inside the cell we will
understand life as everything else follows.

There were, however, even if on the fringes or often completele outside
mainstream biology, few historical (and independent) threads of work that
recognised the importance of organisation and consequently of the systems
view. \citet{rashevsky_topology_1954} explained the need for organisation or
what he called the relational aspects of biology to complement usual approaches
that follow only the metric aspects of physical systems, which is the usual
abstraction in Physics where systems are viewed only by their quantifiable
properties (captured by numeric variables). \citet{rosen_relational_1958}
developed a theory, based on category theory, for the relational/organisational
aspects of biology and was a proponent of a systems view
\citep{rosen1991life}. \citet{varela_autopoiesis:_1974} developed the concept of
`autopoieisis' (self creation through organisation) as a necessary condition for
autonomous living entities. Following this, \citet{fontana_what_1994}
highlighted the importance of organisation as the definining characteristic of
living systems and also developed a theory of organisation as a basis for
biological understanding borrowing from computer science
\citep[$\lambda$-calculus in particular;][]{fontana_barrier_1996}. Finally, the
systems view entered the mainstream through the \emph{systems biology} movement
\citep{kitano2002systems} as a way to consolidate increasingly diverse
experimental molecular biology datasets.

% need for modelling
As the systems we are trying to understand become more complex, which
particularly true for the systems view, intuitive thinking and mental models of
the processes are not adequate tools to aid our understanding. 

%from simple systems biology models to multi-scale systems biology to return to
%the more fundamental question
Taking this to its extreme and recognising the importance of organisation in
understanding life the systems view means the need for multi-scale view and
multi-scale modelling
means multi-scale
modelling with explicity representation of processes at multiple levels of
organisation to understand the links between scales and how life or particular
phenomena in life emerge through this, which cannot be explained only as a
sum of their parts.

%small scale examples
in the samll scale we have very comprehensive attempts at making the links,
single cell Karr etc.
!!This is what we care about not the systems biology kitano, we care about the
bold digital organisms etc.!!

%plants
in the larger scale we have
Then plants

%ecology and crops
Plants are interesting because modelling has been happening also on the other
side at the higher scale. There is ground to unify here

This is where our work is placed buidling on this multiscale models (supra
multiscale) reaching to crop or ecology and evolution.

This further creates some problems social and technical, for example languages (\ref{})

Place our work in this thesis.
place here

In the rest of this chapter I will give an overview of the uses of the
multi-scale plant models beyond the organism

% Dichotomy between reductionism, if we take this to the extreme 
% the ones that advocated for a systems view are the ones that understood the
% importance of organisation
% organisation is not important, if we understand molecular biology we
% understand everything ->

%give examples
Boom systems biology

Models?

Then comprehensive models for small organisms


However, as we now have a good knowledge 

\section{Multi-scale models for understanding -- from plants to ecosystems}
introduce how and why questions
And then This will need some time to place the current work in other works in
plant ecology anad evolution

Nothings makes sense except in the light of population genetics

I think what needs justification is the need for putting the organism (the
individual in the population) There I can use the Doebeli etc.


\section{Multi-scale models for engineering -- from plants to crops}
This is perhaps easier? Need for multiscale, there are multiple papers
advocating for this, in fact there are whole movements

\section{Languages for multi-scale biology}
why modelling? What are the purposes of models?
The fundamental difference in the question asked in Biology means that 
Only say a few general things, there's a whole chapter about this next.

\section{Contributions}
bullet points and references to later chapters
This need to be refutable, like a contract for the thesis

\printbibliography[heading=bibintoc]
\end{document}

% I think for the conclusion we somehow need to link to the multi-language for
% each scale and then one for the organisation, not sure