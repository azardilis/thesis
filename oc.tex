Understanding the links between biological processes at multiple scales, from
genomes and molecular regulation to populations and evolution along with their
interactions with the environment, is a major challenge in understanding
life. Apart from understanding this is also becoming important in attempts to
engineer traits, for example in crops, starting from genetics or from genomes at
different environmental conditions (genotype x environment ->
trait). In this work we instead address the problem of controlling the
environment to achieve particular traits of individual plants (e.g biomass) or
populations of plants (climate control problem). While the automatic climate
control problem has been studied before for greenhouses, the focus was more on
minimising energy consumption and/or achieving certain predefined climate
regimes and rarely on optimising crop traits \citep{Chalabi1996,
udinktenCate1983, Challa_1990, Aaslyng2003}. More insulated and sophisticated
growth environments with more precise control are becoming available as part of
the urban and controlled-environment agriculture movement \citep{mok_strawberry_2014, despommier_farming_2013}. There are many
commercial examples (\href{https://aerofarms.com/}{Aerofarms},
\href{https://motorleaf.com/}{Motorleaf}) and others like Intelligent Growth
Solutions (\href{https://www.intelligentgrowthsolutions.com/}{IGS}) that are based on
novel technologies, for example, for reducing energy consumption and increasing
control over the growth space (lightning patent \citep{aykroyd_novel_2016},
automated growth tower patent \citep{aykroyd_automated_2018}). These more
sophisticated growth environment allow us to focus more on crop traits and even
more precisely at specific quality standards (size, uniformity) that crops need
to meet (see for example EU marketing standards on fruit and vegetables;
\citep{eu-543-2011}).

In order to address the above, we present a formulation of the climate control
problem for \textit{Arabidopsis thaliana} plants as an optimal control problem
\citep{kirk_optimal_2012} where the control variables are climate conditions
(temperature) and the performance criterion is a crop trait. It has already been
proposed that more mechanistic models are needed when linking between multiple
scales instead of the usually empirical models used in crop modelling
\citep{yin_role_2004, yin_modelling_2010}. Here we use the Framework Model (FMv1;
\citep{chew2014multiscale}) as a mathematical description of vegetative plant
growth of \textit{Arabidopsis thaliana} that provides mechanistic links from
molecular regulation all the way to whole-plant traits, like biomass. In
particular, our contributions in this work are:

\begin{itemize}
\item Formulation and solution of a \textit{direct problem} of offline climate
control where the control variable is the temperature of a growth chamber in
order to achieve a particular plant biomass at a particular time (predicted from
the FM). The solution to the problem uses standard gradient-based optimisation
techniques by discretising the growth period and assuming constant temperature
within each time interval.
\item Formulation and solution of an \textit{indirect problem} of offline
climate control. Here we assume that we cannot control temperature precisely in
the growth space, which leads to temperature inhomogeneities. We further assume
that we can control the position of plants (e.g. switch their positions) in a
linear array that has a temperature gradient. The control variable is the
position of the plants and the performance criterion is homogeneity in final
plant biomass (as predicted from the FM) for the population of plants in the
array. For the solution of the problem we use techniques from combinatorial
optimisation since the control variables are discrete.
\end{itemize}

\section{The model and main idea}
For this work we use the Framework Model (FM) as described in
\citep{chew2014multiscale}. For the next of this work we will use the FM as a
black box that translates a temperature signal to a biomass signal (growth)
over a time period. However since we are trying to control temperature to
indirectly control growth, it is instructive to look at two particular metabolic
processes that are highly temperature dependent.

\begin{figure}
\centering
\includegraphics[width=\linewidth]{figures/fmFig/agentFlows.eps}
\caption{Metabolic processes as flows in the Framework Model (FM). A The FM
  keeps track of carbon movement in different pools, one for each organ (leaves
  and root) and two whole-plant reserves of sucrose carbon (c) and starch carbon
  (s). Maintenance respiration and assimilation (intake of carbon) are
  explicitly temperature dependent (indicated by red dots) B Photosynthesis and
  maintenance rate trends for a range of temperature values.}
\label{fig:fm}
\end{figure}

The main output of the FM is the growth rate of the plant given environmental
inputs (CO2 level, temperature, light intensity, and temperature). All the
metabolic processes keep track of the main building block of new mass, carbon,
and are represented as flows between various pools representing the organs (root
and leaves only at the vegetative stage) and whole plant reserves of carbon as
sucrose (C) and starch (S) (Figure~\ref{fig:fm}A). The two metabolic processes
that are temperature dependent are (i) assimilation, the intake of carbon from
the environment represented as an in-flow into the central reservoir of sucrose
carbon (C) and (ii) maintenance respiration, the use of carbon for maintaining
the life-sustaining processes inside the plant represented as an out-flow from
the main sucrose carbon reservoir. Figure~\ref{fig:fm}B shows the rates of these
flows for a range of temperature while keeping the other environmental inputs
constant (light intensity,120 $\mu \mathrm{mol} \cdot m^{-2} \cdot s^{-1}$, CO2=420ppm).

While temperature directly affects only two of the processes it also indirectly
affects the other since all of them compete for the sucrose carbon in the main
reservoir. Therefore and not surprisingly temperature is a major determinant of
growth rate and final biomass. The \emph{main idea} of this work is to control
temperature and therefore indirectly control growth to achieve a particular
growth- related objective of interest using the FM as our ground truth to
evaluate the growth of an \textit{Arabidopsis thaliana} plant at different
temperature inputs.

In the FM the growing plants starts as seeds and wait for a period of time
(emergence period), that is a function of temperature, before they emerge and
start vegetative growth. In the following we only control temperature in the
vegetative growth phase and not in the emergence period. Therefore we assume
that seeds are kept in a constant temperature of 22\textdegree C.


\section{Direct problem}
Plants have strong and fast responses to environmental signals, like
temperature. Temperature affects metabolic processes like photosynthesis and
respiration so it is a major determinant of growth, which is an important crop
trait (see previous section). The main idea here is to control temperature and therefore indirectly
control growth to achieve a particular phenotype that is relevant for
crop/agricultural purposes. In particular we consider the problem of reaching a
specific biomass after a particular period of time by manipulating the
temperature in the growth space of a plant (e.g. growth chamber). This could be
relevant for growers that have specific requirements both in terms of time and
crop attributes. The optimal control problem then becomes to find an optimal
temperature function $T^*(t)$ over a period $[t_0, t_f]$ that minimises the
difference between the final biomass of the plant (as predicted by the FM) and a
target biomass $m_0$:
$$
J = m_{T^*}(t_f) - m_0
$$
where $m_{T*}(t)$ is the temperature dependent biomass trajectory predicted from
the FM. We further have some simple upper and lower bounds on the value of the
temperature function $T_l < T(t) < T_u$. Here we consider a single plant but this
is still relevant for crops (populations of plants) if we assume that the
temperature is homogeneous inside the growth space.

\begin{figure}
\centering
\includegraphics[scale=0.25]{figures/directPFig/ocProb}
\caption{Direct problem of climate control. For the \emph{direct problem} we try
  to find a temperature function or sequence of temperature values in the
  optimisation formulation in order for the final biomass value as predicted by
  the FM (using this temperature function as input) to reach a specific
  predefined value, $m_0$.}
\label{fig:directP}
\end{figure}

We can convert the optimal control into a standard non-linear optimisation
problem by discretising the time domain $[t_0, t_f]$ into $k$ intervals and
assuming a constant temperature value inside each interval \citep{kraft_converting_1985}.
\begin{definition}[biomass-only problem]
The \emph{biomass-only problem} is the problem of finding a sequence of
temperature values, $T_{1, k}=T_1, \dots, T_k$, that minimises the square of the
difference between the final biomass value of the plant at $t_f$,
$m_{T_{1, k}}(t_f)$, as predicted by the FM, for the sequence $T_{1, k}$.
\begin{align*}
& \argmin_{T_1, \dots T_k} \; (m_{T_{1, k}}(t_f) - m_0)^2 \; \; \text{subject to} \\
& T_l < T_{i} < T_u; \; \; \; i=1 \dots k
\end{align*}
\end{definition}
Note that we sometimes write $T_{i, k}$ for the sequence $T_1, \dots, T_k$.

An extension to this is where we also try to minimise the control effort.
\begin{definition}[biomass+control effort problem]
The \emph{biomass+control effort} problem is an extension to the
\emph{biomass-only problem} where the sought sequence of temperatures
$T_{1, k}=T_1, \dots, T_k$ minimises both the distance to the target biomass,
$m_0$, and the control effort defined as the average jump between successive
temperatures in the sequence, $\underline{T}$.
$$ \argmin_{T_1, \dots T_k} \; (m_{T_{1, k}}(t_f) - m_0)^2 \; + \; \frac{1}{k} \sum_{i=2}^{k} T_i - T_{(i-1)}$$
\end{definition}
The bound constraints are the simple lower/upper bounds as before. 

\subsection{Results}
In this section we solve the two formulations of the optimisation problems
(direct problem) using the \texttt{fmincon} function for constrained non-linear
optimisation from the Matlab optimisation toolbox for some choices of $m_0$,
$t_f$, and $k$. While we vary $m_0$ to explore different instances of the
problem, we generally leave $t_f$ and $k$ constant ($t_f=512h$, $k=4$). There
are methods that adapt $k$ that we could have used that are implemented in more
sophisticated tools (mesh refinement, AMIGO tool;
\citep{balsa-canto_amigo2_2016}). The final time is dictated by the bolting time
of the plant under the range of temperatures that we consider since the FM only
considers vegetative growth (before bolting).

We use a performance metric, $\mathrm{log}(\sqrt{L(T_{1, k}, t_f, m_0)} /
m_0)$, to assess particular optimisations for different problem instances
(specific $k$, $t_f$, and $m_0$) that states the final value of the biomass
objective function $L(T_{1, k}, t_f, m_0))=(m_{T_{1, k}}(t_f) -
m_0)^2$ as a percentage of the target biomass.

\begin{definition}[controllability]
A biomass-only or biomass+control effort problem instance (specific $m_0$,
$t_f$) is \emph{controllable} if there exists a $k$ and a particular sequence of
temperature $T_{1, k}=T_1, \dots, T_k$ such that the ratio of the loss function
to the target biomass, $m_0$, is less than a tolerance value, $\epsilon$.
$$
\frac{\sqrt{L(T_{1, k}, m_0, t_f)}}{m_0} < \epsilon
$$
\end{definition}
In the following we will assume that a specific sequence of temperatures that
satisfies our tolerance, $\epsilon$, exists if the particular optimisation algorithm we
use can converge to that sequence.

Figure~\ref{fig:directPRes} shows results of the optimisation procedure for both
the biomass-only and biomass+control effort problems defined in the previous
section for a range of target biomasses. For the biomass-only problem there is a
controllable range of biomasses $[0.03, 0.25]$ (Figure~\ref{fig:directPRes}A)
for a tolerance $\epsilon=0.1$ , $k=4$, and $t_f=512$h. The upper and lower bound
constraint for the temperature values are set to 10 \textdegree C and 30
\textdegree C respectively. The optimisation converges to different solutions on
different runs (Figure~\ref{fig:directPRes}B and C, D, E for details of
particular solutions for $m_0=0.05$, $m_0=0.1$, and $m_0=0.15$ respectively). It
is also interesting to explore if the optimisation increases the controllable
range compared to a naive exhaustive simulation of the FM with constant
temperature inputs in the constrained space [10 \textdegree C, 30 \textdegree
C]. Exhaustive simulation of the FM in the space [10 \textdegree C, 30
\textdegree C] with a $0.5$ \textdegree C step gives final (at $t_f=512$h)
biomasses between $0.0093$g at 10 \textdegree C and $0.1806$g at 17.5
\textdegree C while more than one temperature input can give the same (or very
close) final biomass (orange line, Figure~\ref{fig:directPRes}B).

\begin{figure}
\centering
\includegraphics[width=\linewidth]{figures/directResFig/res}
\caption{ Results for both biomass-only and biomass+control effort formulations
  of the direct problems over a range of target biomasses with $k=4$ and
  $t_f=512$. A The performance metric for the biomass-only problem formulation
  over a range of target biomasses from 0.01 to 0.4. The other environmental
  inputs are constant: CO2, 420ppm; light intensity,
  120$\mu\mathrm{mol} \cdot m^{-2} \cdot s^{-1}$; 12:12-h light/dark cycle. For
  the first 139 hours of simulations the plants are still seeds and are assumed
  to be kept in a constant temperature of 22\textdegree C. The controllable
  range is under the dotted line $[0.03, 0.25]$. B Comparison between the
  solutions obtained with the optimisation and a naive exhaustive simulation
  with constant temperatures over the interval [10 \textdegree C, 30 \textdegree
  C]. For the optimisation solutions per $m_0$ we give the two most commonly
  occurring temperatures in the solution space (10 runs of the optimisation per
  $m_0$). C, D, E Four example solutions (temperature sequences) and
  corresponding biomass time series (as obtained from FM simulation) for target
  biomasses $m_0=0.05$ (C), $m_0=0.1$ (D), and $m_0=0.15$ (E) in the
  controllable range. F Performance metric for the biomass+control effort
  problem formulation over a range of target biomasses from 0.01 to 0.4. The
  controllable region is $[0.03, 0.21]$ G Same as B but for the biomass+control
  effort problem formulation H, I, J Similar to C, D, E but for the
  biomass+control effort formulation.  }
\label{fig:directPRes}
\end{figure}

For the biomass+control effort problem we find a smaller controllable range
$[0.03, 0.21]$ compared to the biomass-only problem
(Figure~\ref{fig:directPRes}F) with the same parameters. The optimisation
procedure though in this case mostly converges to the same solution at different
runs and the solutions mostly keep the temperature constant
(Figure~\ref{fig:directPRes}F and H, I, J for details of particular solutions
for $m_0=0.05$, $m_0=0.1$, and $m_0=0.15$ respectively). Comparing the solutions
to the naive exhaustive search as before we can see the optimisation converges
to constant temperature solutions that are very close to the ones given by the
simulation with a similar controllable range. In cases where two constant
temperature inputs give the same final biomass the optimisation cannot
distinguish between the two and can converge to either (for example,
Figure~\ref{fig:directPRes}H and temperature distributions in G).

Finally, we can compare the results of the optimal strategy with a random
strategy (as a control) where the sequence of temperatures, $T_1, \dots, T_k$ is
picked at random (Figure~\ref{fig:compsAllDir}). For 100 simulation runs for
both optimal and random strategies the optimal strategy performs significantly
better for target biomasses $m_0=0.05$ and $m_0=0.1$.

\begin{figure}
\centering
\includegraphics[width=\linewidth]{figures/directResFig/compsAll}
\caption{
  Comparison between optimal and random strategies for both formulations of the
  direct problem. The comparison is for two target biomasses in the controllable
  range $m_0=0.05$ and $m_0=0.1$ over 100 runs of the optimisation 100 runs of
  the random strategy where temperatures are chosen at random for the 4
  intervals.
}
\label{fig:compsAllDir}
\end{figure}


\section{Growth space inhomogeneities experiment}
\label{sec:exp}
In the direct problem formulation and the entire premise of this work we have
the assumption that we can perfectly control the environmental conditions inside
the growth space such that all the growing plants can grow in perfectly
homogeneous conditions. Based on our experience of growing plants in academic
growth spaces, which are specifically designed to reduced experimental technical
variability, this assumptions is rarely true. To test this we conducted an
experiment in a growth room designed to test the difference in growth (biomass)
of two sets of \textit{Arabidopsis thaliana} plants grown in two different
locations on a shelf (Figure~\ref{fig:expRes}E) in a typical growth room. The
first group of plants is grown in the middle of the shelf and the second on the
side of the shelf.

\begin{figure}
\centering
\includegraphics[width=\linewidth]{figures/exp/expRes.eps}
\caption{
  Results of experiment designed to investigate growth inhomogeneities resulting
  from environmental inhomogeneities in a controlled growth room. A Mean biomass
  for the two groups (middle, side) for the two samples (23 and 27 days) and
  fitted exponential curves. B Full distribution for the two samples (23 and 27
  days) C Average temperature time series over the three sensors placed along
  the positions of the middle and side groups. D Average light time series
  averaged over the three sensors places along the positions of the middle and
  side groups E Experimental setup showing the positions of the plants and
  sensors used to get the environmental data in C, D.
}
\label{fig:expRes}
\end{figure}

The environmental conditions in terms of light and temperature are very
different between the two sets of plants (Figure~\ref{fig:expRes}C, D). We took
two samples at two different stages of vegetative development and before
bolting, one at 23 days after germination (plants M1/1[1-6], M2/2[1-6] for the
middle group and plants S1/1[1-6], S2/2[1-6] for the side group --
Figure~\ref{fig:expRes}) and one 27 days after germination (plants M1/2[1-6],
M2/1[1-6] for the middle group and plants S1/2[1-6], S2/1[1-6] for the side
group -- Figure~\ref{fig:expRes}). The differences in the environmental
conditions affected the growth and resulted in differences in biomass of the two
groups (Figure~\ref{fig:expRes}A, B). While the differences in biomass is not
statistically significant in the 23-day sample, in the 27-day sample the
difference in biomass between the two groups is significant. The environmental
differences are amplified during the exponential growth of the plants.


\section{Indirect problem}
In the direct problem formulation we assumed perfect control inside the growth
space of a crop such that the temperature is homogeneous and all the plants are
exposed to the same conditions. Despite progress in the technology of growth
chambers this assumption is rarely true even in academic growth environments
(see previous section, for example). This is problematic because it means growth
inhomogeneities along the population. We therefore turn into a related problem
where we try to reduce as much as possible temperature and therefore growth
inhomogeneities in a population of plants by controlling their positions in a
growth space during development. This, again, could be relevant for growers
where there are requirements for homogeneity or even in academic settings where
technical variability is expected to be minimal.

\begin{figure}
\centering
\includegraphics[width=\linewidth]{figures/indirectPFig/indirectOC.eps}
\caption{Indirect problem of climate control. In the \emph{indirect problem} we
  try to find a position function (map from time and plant id to position in a
  linear array) such that the final biomasses of the plants as predicted by the
  FM (using the position dependent temperature of the plants as input) to be as
  close to each other as possible (minimum Gini-index).}
\label{fig:indirectP}
\end{figure}

We assume that have $n$ plants and the growth space is a linear array of $n$
positions such that each plant occupies one position. We further assume that we
are given a global temperature perhaps coming from an optimisation procedure
(for example, see the direct problem before) but due to non-precise control this
gives rise to a temperature gradient along the array where the plants are
positioned. The optimal control problem becomes then to find an optimal position
function $P^*(i, t)$ that gives the position of plant $i$ at time $t$ over a
period $[t_0, t_f]$ such that some non-uniformity index over the final biomasses
of the $n$ plants is minimised:
$$
J = G(m_1(t_f), \dots, m_n(t_f))
$$
Here we write $m_i(t_f)$ for the biomass of $i$-th plant at the final time $t_f$
and we use the Gini index (denoted by $G$) as our non-uniformity metric.

Like the direct problem we can turn the above into an optimisation problem by
diving the time domain $[t_0, t_f]$ into $k$ intervals and assuming that the
positions of the plants are constant inside each interval. The problem then
becomes to find the a sequence of $n \cdot k$ values,
$$
P = \begin{bmatrix} 
    p_{11} & p_{12} & \dots \\
    \vdots & \ddots & \\
    p_{n1} &        & p_{nk} 
    \end{bmatrix}
$$
,where $p_{i, j}$, is the position of plant $j$ at interval $i$:
\begin{align*}
& \min_{P} \; G(m_1(t_f), \dots, m_n(t_f)) \; \text{subject to} \\
& 1 < p_{ij} < n \\
& p_{ij} \; \text{integers}
\end{align*}

The positioning of the plant indirectly determines the temperature function,
$T(t)$, over $[t_0, t_f]$ for the plants. Unlike the direct problem, however,
the decision variables of the optimisation problem are not continuous but rather
discrete. We therefore cannot use standard gradient-based optimisation
techniques. In the results in the following sections we use a technique from
combinatorial optimisation (see next section) that uses some heuristics to
search the space of all solutions while following our objective function from
above.

A constraint that we have overlooked in our formulation above is the non-overlap
constraint. Since only one plant can occupy each position at any interval we
have a further constraint that for any interval $j$ all the values in the
sequence $p_{1j}, \dots, p_{nj}$ should be unique. In order to not have to deal with
this extra constraint we make a change to the formulation of the problem such
that the solutions of the combinatorial optimisation procedure are guaranteed to
satisfy our non-overlap constraint without having the constraint explicitly in
the optimisation procedure. In particular suppose that we have a permutation
matrix, $\Pi(n)$, that lists all the possible permutation of $n$ numbers:

$$
\Pi = \begin{bmatrix} 
    1 & 2 & 3 & \dots n \\
    2 & 1 & 3 & \dots n \\
    \vdots &  & \vdots \\
    n &  n-1 & \dots  &  1 \\
    \end{bmatrix}
$$
We redefine the optimisation with the decision variables being indices into the
permutation matrix so for example if at interval $j$ one we get $\pi_j$ then we
have the positions of each plant, $i$, at that interval $P_{ij} = \Pi_{(\pi_j, i)}$

\begin{definition}[permutation-index problem]
The \emph{permutation-index problem} is a variant of the indirect problem of
climate control where we seek a sequence of values $\pi_1, \dots \pi_k$ that minimise the
inhomogeneity in the final biomasses (after $t_f$ hours, as predicted by the FM)
of $n$ plants.
\begin{align*}
& \min_{\pi_1, \dots \pi_k} \; G(m_1(t_f), \dots, m_n(t_f)) \; \text{subject to} \\
& 1 < \pi_i < n! \\
& \pi_i \; \text{integers}
\end{align*}
\end{definition}

The number of all possible permutations for $n$ numbers is given by $n!$. The
order of permutations in $\Pi$ is important here as we need to make sure that
solutions that are close together in the permutation index space are also close
together in the positions space. Here we assume we have that and moreover that
every permutation listed in $\Pi$ is one flip away from the permutation above
it. For the instances of the problems we dealt with this solution was
adequate. However, for larger instances the search space might become
prohibitively large ($n!$) in which case we can go back to the previous
formulation of the problem and introduce an overlap penalty in the objective
function.

\subsection{Results} 
In this section we explore solutions to the permutation-index problem defined in
the previous section using combinatorial optimisation techniques. In particular
we use a Matlab implementation (MEIGO tool; \citep{banga2014})of the Variable
Neighbourhood Search algorithm (VNS; \citep{mladenovic1997variable}).

\begin{figure}
\centering
\includegraphics[width=\linewidth]{figures/indirectResFig/indRes}
\caption{
  Comparison between optimal and random strategies for the indirect problem. A
Four example solutions returned by the optimisation procedure for four
intervals, five plants, and a growing space of five positions arranged
linearly. Each colour represents one plant and a line the movement of a plant
across the growing space over the four intervals. The temperature gradient
starts from 20\textdegree C in one end of the array and goes up to 24\textdegree
C on the other end. B Comparison of the objective function values (normalised
Gini-index of final biomasses of plants) over 100 runs of the optimal and random
strategies where the plants take random positions at each intervals. The
Gini-index for the static strategy is shown with a dotted line.
}
\label{fig:compsAllInDir}
\end{figure}

For the following experiment described in this section we assumed a growth space
of $5$ position with $5$ growing plants, one for each position, $k=4$ intervals
and $t_f=512$ hours. The temperature gradient goes from 20 \textdegree C at
position 1 on one end of the growth array to 24 \textdegree C at position $5$ at
the other end of the array. The positions in between are assumed have a
temperature linearly interpolated between the values at the two ends of the
array. We use a normalised performance metric $\tilde{G}=G/G_s$ where $G_s$ is
the value of the Gini-index for a position matrix where the positions of the
plants are unchanged over the growth period.

Unsurprisingly the optimisation procedure returns results where the plant
positions are shuffled over the time intervals (example solutions:
Figure~\ref{fig:compsAllInDir}A). The optimal strategy is significantly better
though than a naive strategy where the plants are randomly assigned positions at
each interval (Figure~\ref{fig:compsAllInDir}B). Both strategies are better over
100 runs than the static strategies where the plant positions are unchanged in
the growing period (static $G$ value indicated by dotted line,
Figure~\ref{fig:compsAllInDir}B).


\section{Discussion}
We present a formulation of the climate control problem for achieving particular
growth-related plant (or population of plants) attributes as an optimal control
problem and solution after transforming it to an optimisation problem. Optimal
control of climate in greenhouses has been studied for a long time but since
greenhouses are not insulated to weather conditions, greater effort (and
therefore energy) is needed for climate control. Optimal control studies have
therefore mainly been focused on reducing energy consumption \citep{fisher1997,
ramirezArias2012, delSagrado2016} with very few also taking into account plant
processes like photosynthesis for optimising growth as well as energy
consumption\citep{harun2015, Aaslyng2003}. Here we assume that our growth space
is more insulated so that more precise control is available at less effort. This
allows us to focus on crop traits and even at more precise quality control
standards like size and uniformity. We do, however, consider a control effort
quantity in the biomass+control effort problem formulation of the direct problem
that should be related to energy consumption. In the indirect problem we do not
have that even though switching the plant positions will require significant
energy even at sophisticated growth environments.  %related work


%, only one with some kind of crop %trait in mind (photosynthesis rate; and only
one for indoor %farming with LEDs %Even the one that considers photosynthesis
only does so to exploit regularities in the photosynthesis to minimise energy
consumption. For example, observation that temperature increase does not
increase photosynthesis rate at low light so no need to use energy to increase
temperature.  %No focus on energy but more on crop traits although we do
consider control effort (energy related) in the first problem.

%discussion of results
In order to address the climate control problem we start with a \emph{direct
problem} where we try to find an optimal input temperature signal to the FM so
that the output biomass signal has a particular value at some final time,
$t_f$. The formulation of the problem with an added penalty for control effort
along with exhaustive simulations with constant temperature inputs suggest that
a single constant temperature input is almost enough to achieve the same results
as the optimal strategy (Figure~\ref{fig:directPRes}A, B, F, G). There is very
little increase in the controllable range using the optimal strategy as opposed
to naive enumeration of the single temperature-final biomass map using
simulation of the FM (Figure~\ref{fig:directPRes}B, G).  %tomato result optimal
temperature, find paper As we have seen from experimental results though even if
we set a global optimal temperature (for example as given by a solution of an
optimisation problem representing the direct problem) we are not guaranteed that
all the plants will have the same conditions. Distances as little as 1m (same
shelf), even in spaces designed specifically to reduce climate inhomogeneities,
can lead to significant environmental and therefore growth differences among a
population of plants. We therefore turned to a formulation of an \emph{indirect
problem} of climate control where we accept the inhomogeneities and try to
minimise them (and therefore also growth inhomogeneities) by switching the
positions of plants along a space with a temperature gradient.

%limitations -- model
For the solutions of both problem statements, direct and indirect, we use the FM
as our ground truth to predict plant behaviour at different temperature input
profiles. The FM starts with biochemical models that have been validated only in
a narrow range of experimentally relevant temperatures so even in our seemingly
conservative temperature range ([10\textdegree C, 30 \textdegree C]) there might
be gaps in our understanding \citep{walker_temperature_2013}. This limits the
predictive power of the model across the range that we consider. Moreover there
might be other temperature effects on plant physiology that we do not
consider. The FM is deterministic but often there is variability in growth,
which might suggest an adaptive strategy in a possible implementation with a
continuous monitoring of the growing plants.

% limitations -- assumptions on growth space
%temperature switches, 

%conclusion + future
Our approach builds on ideas from traditional climate control in greenhouses and
the availability of more sophisticated growth spaces to suggest using optimal
control of climate for control of crop traits and quality standards, like size
and uniformity. Of course while the application of such methods is most
interesting for crop species that are practically relevant, here we chose the
model species \textit{Arabidopsis thaliana}. It will be interesting first to
test our offline in-silico optimisation experiments in an actual prototype
implementation for Arabidopsis plants given the limitations of the model-only
approach outlined above. Application of such techniques for more practically
relevant crops will also be very attractive and there is a number of models to
back such an application for commercial crops (potato models,
\citep{fleisher2017potato}; tomato model, \citep{heuvelink1999evaluation}; wheat
models, \citep{martre2015multimodel}). Finally, another possible application of,
for example, the indirect problem formulation might be in academic settings
where technical variability is expected to be minimal and there is an increase
in the use of automated phenotyping and growth platforms (PHENOPSIS
\citep{granier_phenopsis_2006}; platforms in the European Plant Phenotyping
Network, \url{https://eppn2020.plant-phenotyping.eu/}).


