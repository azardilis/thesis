\documentclass[phd]{infthesis}
\usepackage[utf8]{inputenc}
\usepackage[T1]{fontenc}
\usepackage[british]{babel}
\usepackage{microtype}
\usepackage[usenames,dvipsnames,svgnames,table]{xcolor}
\usepackage[english=british,autopunct=false]{csquotes}
\usepackage[natbib=true,style=authoryear-comp,maxbibnames=6]{biblatex}
\usepackage{graphicx}
\usepackage{textcomp}
\usepackage{wrapfig}
\usepackage{xfrac}
\usepackage{xspace}
\usepackage{mathcommon}
\usepackage[sc]{mathpazo}
\usepackage{hyperref}
\usepackage{expl3}
\usepackage{enumitem}
\usepackage{booktabs}
\usepackage{tabularx}
\usepackage{mathpartir}
\usepackage{fancyvrb}
\usepackage{inconsolata}
\usepackage{multirow}

\frenchspacing

% Bibliography
\addbibresource{thesis2.bib}
\bibliography{thesis2}

% Text
\newcommand{\ie}{i.e.\xspace}
\newcommand{\eg}{e.g.\xspace}

% Referencing
\newcommand{\chp}[1]{\S\ref{chp:#1}}
\newcommand{\sct}[1]{\S\ref{sec:#1}}
\newcommand{\ssec}[1]{\S\ref{subsec:#1}}
\newcommand{\eqn}[1]{Eq.~\ref{eq:#1}}
\newcommand{\eqns}[2]{Eq. \ref{eq:#1} and \ref{eq:#2}}
\newcommand{\lem}[1]{Lemma~\ref{lemma:#1}}
\newcommand{\lems}[2]{Lemmas \ref{lemma:#1} and \ref{lemma:#2}}
\newcommand{\thm}[1]{Th.~\ref{thm:#1}}
\newcommand{\fig}[1]{Fig.~\ref{fig:#1}}
\newcommand{\diagram}[1]{diagram~\ref{eq:#1}}
\newcommand{\app}[1]{Appendix~\ref{app:#1}}
\newcommand{\mcite}[1]{\textcolor{gray}{#1}} % missing cite
\newcommand{\defn}[1]{Def.~\ref{def:#1}}
\newcommand{\prop}[1]{Prop.~\ref{prop:#1}}

% Math
\renewcommand{\tuple}[1]{(#1)}
\DeclareMathOperator*{\expn}{exp}
\renewcommand*{\exp}[1]{e^{\,#1}} % \mathrm{e}^{#1}}
\renewcommand{\qedsymbol}{\ensuremath{\blacksquare}}
\newcommand{\partialto}{\rightharpoonup}
\newcommand{\id}{\vec{1}} % identity function
\newcommand{\mr}[1]{\mathrm{#1}}

\newcommand{\den}[1]{\llbracket #1 \rrbracket}
\newcommand{\m}[1]{\{\!| #1 |\!\}}
\newcommand{\M}[1]{\mathcal{#1}}
\newcommand{\MS}[0]{\mathrm{M}}
\newcommand{\SQ}[0]{\mathrm{S}}
\newcommand{\s}[1]{\underline{#1}}
\newcommand{\G}[0]{\Gamma}
\newcommand{\D}[0]{\Delta}
\newcommand{\mytt}{t\!t}
\newcommand{\myff}{f\!\!f}

\newcommand{\V}{\mathrm{V}}

\newcommand{\sel}{\mathrm{sel}}
\newcommand{\fold}{\mathrm{fold}}

\newcommand{\ms}{\mathrm{ms}}


\newtheorem{mydef}{Definition}
\def\dotminus{\mathbin{\ooalign{\hss\raise1ex\hbox{.}\hss\cr
  \mathsurround=0pt$-$}}}
\setlength{\tabcolsep}{8pt}
\renewcommand{\arraystretch}{1.0}

\newcommand{\match}{m}
\newcommand{\up}[1]{\uparrow\! #1}

\newcommand{\n}{\mathrm{n}}

% Other stuff
\newcommand{\maybe}[1]{\textcolor{gray}{#1}}
\newcommand{\todo}[1]{\textcolor{red}{TODO: #1}}

% rules
\newcommand{\ar}[2]{\mr{#1} \! = \! {#2}}

\setlength{\tabcolsep}{8pt}
\renewcommand{\arraystretch}{1.2}

\begin{document}
\chapter{Conclusions}
% summary
No complex system can be understood by extrapolation of the properties of its
elementary components. In order to understand a complex system, like an
organism, one needs to consider explanations at various levels of detail. This
is especially true in biological systems where entities (processes and so on)
and their organisation give rise to properties of processes at the next level of
detail and they in turn to the next until you get to the level of a living
entity, the organism. In order to understand the organism (and consequently
life) we therefore need to understand the links between the explanations at the
various levels of detail.

The importance of organisation between components has been stressed before and
it is in fact the fundamental concept of systems theory. While the focus in
Biology has been traditionally on single components the systems view has found
its way into mainstream biological thinking with the systems biology
movement. The stress on organisation and its constructive nature also leads to
questions about representation. The most widely used technical languages for the
representation of physical systems come from the dynamics world. Since the
questions there are usually different, abstracting a system to its properties
through variables and ignoring organisation is usually adequate. The
constructive organisation of biological systems though poses different
requirements although the value of abstraction still remains. We present
\emph{Chromar} a language that combine organisational aspects through the
explicit representation of discrete entities, agents, and non-organisational
aspects through abstract properties of these entities, attributes. The concept
of abstraction is there in extensions to the expression language used for the
dynamics of the attributes. Observables recognise the need for explanations at
different levels by allowing an explicit functional link between abstract
quantifiable properties and more explicit representations. Fluents, again, allow
abstraction via the description of system properties using time-dependent
deterministic value for describing parts of the system we do not wish to model
mechanistically.

In plant research the focus has always been on the model species
Arabidopsis. Functional-structural models have focused on the constructive
organisation of plant organs (development) as well as functional aspects like
metabolism. The Framework Model is inspired by these but adds more breadth by
explicitly including a genetic circuit and phenology models to time the
conceptual development towards flowering. The FM covers one part of plant
development but does not extend to reproduction, which means that it has limited
applicability in ecological studies. Ecological studies on the other hand
consider the full lifecycle but of an abstract version of plants where its
construction and physical development are absent. We presented \emph{FM-life}
an extension to the FM to the full lifecycle including reproduction and its
scale-up to the population level via a clustering approoach to tractable
simulate population of plants over multiple decades in different genotype x
environment scenarios.

Finally, we presented another more engineering oriented use of multi-scale plant
model as devices for optimising plant traits. Instead of the more traditional
genome engineering approach we focused on engineering the environment during
their development to achieve specific growth-related traits.

After this brief summary of the work presented in thid thesis, we next go
through and dicuss each aspect of our work in the context we have set above to
understand its limitations, implications that follow, and possible future work
that it suggests.



\section{Chromar and representations of multi-scale biology}
For the organisational parts of biology there is not much choice. We need
discrete objects relations and dynamics that add/create objects and
relate/unrelate objects.

Identify objects
Apart from these other modules can be identified

Shortcomings of Chromar:
in organisational aspects
Most basic structure + dynamics
No explicit relation although they can be encoded
Connections

Non-organisational aspects
Simple, again, but useful for a large range of application
Metric quantities and others

For the non-organisational parts the choice is greater since the
non-organisational parts are the abstractions that the modeller uses. Depending
on the system a different abstraction might be useful in understanding and
interpeting the process.

made the tie between non-organisational and organisational aspects
stronger. through fluents and observables
abstract vs non-abstract spaces
Make the variables more first-class

Biology is modular, the modules might be objects that we can see

Unities can be treated as undivided wholes or
as the interaction of their components (see Varela)

which suggests --->
--->


\subsection{Multi-scale models that combine}
What does the Chromar experience suggest?
Organisational aspects through Chromar or an object-based lang to capture either
things that we consider objects or abstract objects like processes (see
Barabasi)
The non-organisational aspects can be captured in any relevant abstraction
depending on the level of understanding Can be even treated as undivided wholes
if needed

Practically will need either mappings between formalisms or more pragmatically
something like mois cis\_interface


\subsection{Practical considerations}
None of these are important if people cannot use these things
Given the above library of models with 



\section{Organism-centred evolutionary ecology}
effect of one mechanism on context? This theme is also important for the
previous section, abstract then mechanisms can be added later. It might be
interesting to see if we understand the effects of a single mechanism on
higher-levels of organisation.
Not always at the mechanism level -- see Marr's levels

Since we're plant biologist and the plant is where we have mechanims information
we would like to see the effect on ecology and a plant centred

With phenology we kind of understand at hte first level of understanding, going
to the second level of mechanism or of how this is realised we need the organism


Experimental validation?


\section{Multi-scale organism models for engineering}
Finally,
for these I'm not sure we need the organism. The organism is there for the
fundamental questions
We can engineer the environemnt



















\printbibliography[heading=bibintoc]
\end{document}