\documentclass[phd]{infthesis}

\usepackage[utf8]{inputenc}
\usepackage[T1]{fontenc}
\usepackage[british]{babel}
\usepackage{microtype}
\usepackage[usenames,dvipsnames,svgnames,table]{xcolor}
\usepackage[english=british,autopunct=false]{csquotes}
\usepackage[natbib=true,style=authoryear-comp,maxbibnames=6]{biblatex}
\usepackage{graphicx}
\usepackage{textcomp}
\usepackage{wrapfig}
\usepackage{xfrac}
\usepackage{xspace}
\usepackage{mathcommon}
\usepackage[sc]{mathpazo}
\usepackage{hyperref}
\usepackage{expl3}
\usepackage{enumitem}

\frenchspacing

% Bibliography
\addbibresource{relWork.bib}
\bibliography{relWork}

% Text
\newcommand{\ie}{i.e.\xspace}
\newcommand{\eg}{e.g.\xspace}

% Referencing
\newcommand{\chp}[1]{\S\ref{chp:#1}}
\newcommand{\sct}[1]{\S\ref{sec:#1}}
\newcommand{\subsct}[1]{\S\ref{subsec:#1}}
\newcommand{\eqn}[1]{Eq.~\ref{eq:#1}}
\newcommand{\eqns}[2]{Eq. \ref{eq:#1} and \ref{eq:#2}}
\newcommand{\lem}[1]{Lemma~\ref{lemma:#1}}
\newcommand{\lems}[2]{Lemmas \ref{lemma:#1} and \ref{lemma:#2}}
\newcommand{\thm}[1]{Th.~\ref{thm:#1}}
\newcommand{\fig}[1]{Fig.~\ref{fig:#1}}
\newcommand{\diagram}[1]{diagram~\ref{eq:#1}}
\newcommand{\app}[1]{Appendix~\ref{app:#1}}
\newcommand{\mcite}[1]{\textcolor{gray}{#1}} % missing cite
\newcommand{\defn}[1]{Def.~\ref{def:#1}}
\newcommand{\prop}[1]{Prop.~\ref{prop:#1}}

% Math
\renewcommand{\tuple}[1]{\left(#1\right)}
\DeclareMathOperator*{\expn}{exp}
\renewcommand*{\exp}[1]{e^{\,#1}} % \mathrm{e}^{#1}}
\renewcommand{\qedsymbol}{\ensuremath{\blacksquare}}
\newcommand{\partialto}{\rightharpoonup}
\newcommand{\id}{\vec{1}} % identity function

% Other stuff
\newcommand{\maybe}[1]{\textcolor{gray}{#1}}
\newcommand{\todo}[1]{\textcolor{red}{TODO: #1}}

% Abstract
\begin{document}

\chapter{Related Work}
\label{chp:oc}

Formal models of physical systems serve two roles (i) documentation and
communication of our understanding (ii) formal analysis like simulation. The
most common language for describing the natural world is dynamical systems
theory. It has a long tradition in Physics starting from Newton, Poincare
etc. It relies on a conceptual abstraction where instead of describing the
objects that form the physical world, it describes quantifiable properties of
such objects. For example, when explaining the movement of planets one does not
talk about the moon but rather about the position of the moon. Then the
evolution of the system in time (or space) is described by (partial)
differential (or difference) equations that state the change in the numerical
values of the quantifiable properties that form the system as they interact with
each other. For example, one
might write the equation of position of a particle or a molecule diffusing in a
medium. This abstraction of avoiding dealing with the objects directly but
rather with some numerical variables representing their properties has been at
the core of the success of dynamical systems theory. A big ensemble of methods
have been developed for the analysis of such systems, which one gets for free if
they choose to describe a physical system of interest in these terms.

Dynamical systems languages have not been as successful in scenarios where the
structure of the  described system is not static. While one might be able to
find a quantifiable property of the system to describe, in cases where the
question posed is about the changing structure itself or about how it gives rise
to other dynamics, such an abstraction is not adequate. Consider for example
plant development and its effect on carbon intake for the entire
plant. While in some cases writing an equation for the size of the plant is
enough, a truly mechanistic understanding requires treating the developing structure that
gives rise to plant size explicitly. How would one write equations though for the size
of the leaves, for example, if new ones keep appearing?
In those case one might be better off with a language that allows the
description of the objects themselves, their interactions, and
organisation. Unsurprisingly such languages have been developed for
scientific fields that deal with parts of the natural world where structures are
dynamic and self-organising. A whole strand of work on languages of
increasing complexity exist for the description of biochemical systems starting
from simple molecular reactions. The field of plant development also has seen an
independent strand of work on languages with explicit description of objects.
Both these strands of work have seen a heavy interaction and inspiration %stronger than just inspired
from Computer Science where there exist many formalisms in the
theory of computation for the description of object behaviour and interaction.
% especially in concurrent computation but also in sequential

%we need names for the two kinds of languages

Here we are interested in understanding a plant or particular aspects of a plant
at the organism level starting from genomes and molecular mechanisms. In order
to understand the links between all relevant processes we need models with
explicit representation of processes at multiple levels. Apart from the
scientific challenges that this presents there are also technical and social
challenges among which is the choice of representation (language) for describing
our models. Since we are trying to recreate an organism in silico and especially
one where development is plastic and happens throughout its life history, we
need to be able to describe dynamics of discrete objects. At the same time we
also have the need for abstraction through properties since the complexity of
the task is large.

There are language in both the dynamics and structure space that combine aspects
of both worlds. The language we describe and that is a big part of this thesis
is situated in this space too. It comes from the biochemistry tradition and has
discrete objects at its core but at the same time also has attributes
to abstract away some internal structure of these objects. It further has
features for (i) linking the abstract (attributes) and the concrete
(observables) and (ii) explicitly using time, which is inspired, again, by the dynamics
world.

Since language design is as much art as it is science it can sometimes be hard
to properly situate a new language (Chromar) in the space of already existing languages for the
description of the natural world especially if one is only familiar with one
part of the world, perhaps the particular formalisms that are standard for their
disciplines. We therefore feel the need to include an overview of existing
languages (the ones we hinted to above) for Biology in order to be able to place
our work in the language design space. Since we have particular requirements for
structure representation, we will only deal with languages coming from the
discrete world or from the dynamics world that include some support for defining
discrete organisation as well.

Concretely, in this chapter we will do the following:
\begin{itemize}
\item An overview of languages coming from (i) the dicrete world with an emphasis on
  langauges from the biochemistry strand of work since Chromar is directly a
  product of these and (ii) the dynamics world that have some aspect of
  structural organisation. In order to place these languages in order to be able
  to late compare them we will place them all in a feature space. The features
  are related to their characteristics in regards to capturing complex
  organisation, dynamics but also meta-features like readability.
\item A comparison of representative languages from each family. While there are
  attempts for formal comparisons between languages, here the design space is so
  large that such an attempt is probably impossible. We instead will do our
  comparison informally through example where each example is meant to exemplify
  one of the interesting features from the overview.
\end{itemize}
% need more concrete statement of contributions
In the next chapter  we introduce Chromar through the same examples so we can
properly place it and its novel characteristics in the same feature space.
\cite{fontana_barrier_1996}

\section{Overview -- discrete objects}


\section{Overview -- dynamics going to structure}


\section{Comparison through examples}








\singlespace

%% Specify the bibliography file. Default is thesis.bib.


\printbibliography[heading=bibintoc]

%% ... that's all, folks!
\end{document}

%%% Local Variables:
%%% mode: latex
%%% TeX-master: t
%%% End:
