\documentclass[phd]{infthesis}

\usepackage[utf8]{inputenc}
\usepackage[T1]{fontenc}
\usepackage[british]{babel}
\usepackage{microtype}
\usepackage[usenames,dvipsnames,svgnames,table]{xcolor}
\usepackage[english=british,autopunct=false]{csquotes}
\usepackage[natbib=true,style=authoryear-comp,maxbibnames=6]{biblatex}
\usepackage{graphicx}
\usepackage{textcomp}
\usepackage{wrapfig}
\usepackage{xfrac}
\usepackage{xspace}
\usepackage{mathcommon}
\usepackage[sc]{mathpazo}
\usepackage{hyperref}
\usepackage{expl3}
\usepackage{enumitem}

\frenchspacing

% Bibliography
\addbibresource{relWork.bib}
\bibliography{relWork}

% Text
\newcommand{\ie}{i.e.\xspace}
\newcommand{\eg}{e.g.\xspace}

% Referencing
\newcommand{\chp}[1]{\S\ref{chp:#1}}
\newcommand{\sct}[1]{\S\ref{sec:#1}}
\newcommand{\subsct}[1]{\S\ref{subsec:#1}}
\newcommand{\eqn}[1]{Eq.~\ref{eq:#1}}
\newcommand{\eqns}[2]{Eq. \ref{eq:#1} and \ref{eq:#2}}
\newcommand{\lem}[1]{Lemma~\ref{lemma:#1}}
\newcommand{\lems}[2]{Lemmas \ref{lemma:#1} and \ref{lemma:#2}}
\newcommand{\thm}[1]{Th.~\ref{thm:#1}}
\newcommand{\fig}[1]{Fig.~\ref{fig:#1}}
\newcommand{\diagram}[1]{diagram~\ref{eq:#1}}
\newcommand{\app}[1]{Appendix~\ref{app:#1}}
\newcommand{\mcite}[1]{\textcolor{gray}{#1}} % missing cite
\newcommand{\defn}[1]{Def.~\ref{def:#1}}
\newcommand{\prop}[1]{Prop.~\ref{prop:#1}}

% Math
\renewcommand{\tuple}[1]{\left(#1\right)}
\DeclareMathOperator*{\expn}{exp}
\renewcommand*{\exp}[1]{e^{\,#1}} % \mathrm{e}^{#1}}
\renewcommand{\qedsymbol}{\ensuremath{\blacksquare}}
\newcommand{\partialto}{\rightharpoonup}
\newcommand{\id}{\vec{1}} % identity function

% Other stuff
\newcommand{\maybe}[1]{\textcolor{gray}{#1}}
\newcommand{\todo}[1]{\textcolor{red}{TODO: #1}}

% Abstract
\begin{document}

\chapter{Background and Related Work}
\label{chp:oc}
Formal models of physical systems serve two roles (i) documentation and
communication of our understanding and (ii) formal analysis (simulation). The
most common language for describing the natural world is dynamical systems
theory. It has a long tradition in Physics starting from Newton, Poincare
etc. It relies on a conceptual abstraction where instead of describing the
objects that form the physical world, it describes quantifiable properties of
such objects. For example, when explaining the movement of planets one does not
talk about the moon but rather about the position of the moon. Then the
evolution of the system in time (or space) is described by (partial)
differential (or difference) equations that state the change in the numerical
values of the quantifiable properties that form the system as they interact with
each other. For example, one
might write the equation of position of a particle or a molecule diffusing in a
medium. This abstraction of avoiding dealing with the objects directly but
rather with some numerical variables representing their properties has been at
the core of the success of dynamical systems theory. A big ensemble of methods
have been developed for the analysis of such systems, which one gets for free if
they choose to describe a physical system of interest in these terms.

Dynamical systems languages have not been as successful in scenarios where the
structure of the  described system is not static. While one might be able to
find a quantifiable property of the system to describe, in cases where the
question posed is about the changing structure itself or about how it gives rise
to other dynamics, such an abstraction is not adequate. Consider for example
plant development and its effect on carbon intake for the entire
plant. While in some cases writing an equation for the size of the plant is
enough, a truly mechanistic understanding requires treating the developing structure that
gives rise to plant size explicitly. How would one write equations for the size
of the leaves, for example, if new ones keep appearing?
In those cases one might be better off with a language that allows the
description of the objects themselves, their interactions, and
organisation. Unsurprisingly such languages have been developed for
scientific fields that deal with parts of the natural world where structures are
dynamic and self-organising. A whole strand of work on languages of
increasing complexity exist for the description of biochemical systems starting
from simple molecular reactions. The field of plant development also has seen an
independent strand of work on languages with explicit description of objects.
Both these strands of work have seen a heavy interaction and inspiration 
from Computer Science where there exist many formalisms in the
theory of computation for the description of object behaviour and interaction.
% especially in concurrent computation but also in sequential

%we need names for the two kinds of languages -- make the distinction so we can
%refer to them later
Here we are interested in understanding a plant or particular aspects of a plant
at the organism level starting from genomes and molecular mechanisms. In order
to understand the links between all relevant processes we need models with
explicit representation of processes at multiple levels. Apart from the
scientific challenges that this presents there are also technical and social
challenges, among which is the choice of representation (language) for
describing our models. Since we are trying to recreate an organism in silico and
especially one where development is plastic and happens throughout its life
history, we need to be able to describe dynamics of discrete objects. At the
same time we also have the need for abstraction through quantifiable properties
since the complexity of the task is large. 
% see Andrews email to Robert, Vincent after or during his sabbatical. I think
% there was something relevant there

There are languages in both the dynamics and structure space that combine aspects
of both worlds. The language we describe and that is a big part of this thesis
is situated in this space too. It comes from the biochemistry tradition and has
discrete objects at its core but at the same time also has attributes
to abstract away some internal structure of these objects. It further has
features for (i) linking the abstract (attributes) and the concrete
(observables) and (ii) explicitly using time, which is inspired, again, by the dynamics
world.

Since language design is as much art as it is science it can sometimes be hard
to properly situate a new language (Chromar) in the space of already existing
languages for the description of the natural world especially if one is only
familiar with one part of the world, perhaps the particular formalisms that are
standard for their disciplines. We therefore feel the need to include an
overview of existing languages (the ones we hinted to above) for Biology in
order to be able to place our work in the language design space. Since we have
particular requirements for structure representation, we will only deal with
languages coming from the discrete world or from the dynamics world that include
some support for defining discrete organisation as well.

Concretely, in this chapter we will do the following:
\begin{itemize}
\item An overview of languages coming from (i) the discrete world with an
emphasis on languages from the biochemistry strand of work since Chromar is a
direct product of these and (ii) the dynamics world that have some aspect of
structural organisation. In order to place these languages and later be able to
compare them we will place them all in a feature space. The features are related
to their characteristics in regards to capturing complex organisation, dynamics,
but also meta-features like readability.
\item A comparison of representative languages from each family. While there are
attempts for formal comparisons between languages
\citep{felleisen1991expressive}, here the design space is so large that such an
attempt is probably impossible. We instead will do our comparison informally
through example where each example is meant to exemplify one of the interesting
features from the overview.
\end{itemize}
% need more concrete statement of contributions
A good overview of the above dichotomy between dynamics and structure appears in
\citet{fontana1996barrier}.
In the next chapter  we introduce Chromar through the same examples so we can
properly place it and its novel characteristics in the same feature space.


\section{Examples}
In this section we will introduce two examples that we will throughout our late
overview of languages to illustrate limitations of formalisms and the motivation
for their extensions with new features.

\subsection{Root apical meristem and whole-plant effects}
Plant development happens at sites of meristematic activity. In the root this
site is at root cap and it is called Root Apical Meristem (RAM). In root
development this raises the question about the mechanism through which this gives
rise to the root architecture but also how the meristematic cells maintain their
position close to the root cap throughout development. The hormone auxin and its
distribution of concentration along the root architecture has been pinpointed as
a possible mechanism to explain both of these phenomena.

Here we will consider auxin dynamics in a growing 1-D array of root cells
proposed by \citep{mironova_plausible_2010} as a possible mechanism to explain
the observed distribution of auxin concentrations throughout the root. We will
further consider the effect of the resulting root architecture on above-ground
plant growth through its effect on water intake. We will only consider a very
abstract version of the shoot. This model is indicative of the kind of
comprehensive models that are the focus of this thesis. Here, for example, a
more detailed root model is placed in the context of the whole plant (more
abstract model) and the surrounding environment.

More precisely we will consider the following dynamics (Figure~\ref{fig:rootDev}):
\begin{itemize}
\item Auxin flow from the shoot
\item Auxin degradation
\item Auxin diffusion
\item Auxin active transport
\item Cell division and growth
\item Shoot growth
\end{itemize}


\subsection{Plant development in a field}
The example considers a very abstract view of plant development, but has
nevertheless enough details to demonstrate the main features of our
notation. Our model is inspired by the Framework Model (FM) of
\citet{chew2014multiscale}, a modular whole-plant model that connects
traditional plant biology representations of molecular processes with
representations of organ and whole-plant development processes. We will put this
model in a more ecologically relevant context (field).

The above-ground part of an Arabidopsis plant
architecture before flowering: a collection of leaves arranged in a circle. Each
leaf photosynthesises, creating the main currency, carbon; uses some carbon for
maintenance and some for growth; and transfers any remaining carbon to the other
leaves. The FM represents the Arabidopsis rosette (collection of leaves) with no
preference in the transfer, thus we have an all-to-all communication. Similarly
to the FM, in our model all the molecular processes in our model reside in a
central plant `cell' which allows us to keep the leaves as carbon sinks and
track their growth, while avoiding the per-leaf molecular processes and their
communication (see Figure~\ref{fig:fm}).

The processes that affect growth are as follows: we think of \textit{carbon
assimilation} per leaf as increasing the carbon concentration of the central
Cell depending on the photosynthesis level of a leaf (which will depend on its
size); we think of \textit{maintenance respiration} as the central Cell giving
some carbon to a leaf; and we think of \textit{growth respiration} as the
central Cell giving some carbon to a leaf and the leaf mass increasing. We will
also have \textit{new leaf creation}. There are interesting dynamics here such
as the interaction between growth and assimilation: the more we grow, the more
the leaves can photosynthesise, and the more carbon can go to the central Cell.

We assume that the plants are arranged in a two-dimensional field and that they
compete for light, which affects their growth. While we use Arabidopsis here,
such models are interesting for the interactions and competition between crops
and weeds in the field.




\section{Object-based languages}

A significant line of work in languages from the discrete tradition has stemmed
from the world of Biochemistry. In the following we make a further distinction
between \emph{rule-based} languages and \emph{process-calculi based} languages
that differ in what they consider as their main unit of description.
Here we will focus more on languages from the rule-based camp since Chromar follows that tradition. 


Note the historical progression of simple objects \textrightarrow added
structure \textright reaching to the dynamics world. Both camps follow similar
trends driven by modelling requirements. The important thing for us here is the
trend and the features not the exact formalisms so while we consider the
rule-based languages that also applies to the other camp, which we will only
note in passing in the following text.

For more comprehensive review see?

%A good overview of this path of increasing complexity starting from CRNs is in
%Mjolness

\subsection{Unstructured: Petri Nets}
The first main camp of languages in this line are the \emph{rule-based}
formalisms starting from notations for Chemical Reactions Networks (CRN)
represented in the language of Petri Nets (PN). At the description level, PNs
represent discrete objects with types (species) where the dynamics are given as
reactions that state how different species interact to change the number of
objects of the corresponding species in the state of the system. The state of
the system is an unstructured collection (multiset) of copies (molecules) of
each species.

\begin{center}
    \includegraphics{figures/pns.eps}
\end{center}

A \emph{multiset} $m$ over a set $A$ is a function from $A$ to $\mathbb{N}$ counting
the multiplicity of each element $a \in A$ in the multiset. There is submultiset
relation on multisets where for two multisets $m$ and $m'$, $m \preceq m'$ if for each
element $a \in A$ $m(a) \leq m'(a)$. We write $M[A]$ for the set of all multisets
over a set $A$. Given a set of species $\Sigma$, a \emph{reaction} is a structure $\rho
= l \xrightarrow{k} r$ where $k \in R$ and both the left-hand and right-hand
sides are multisets over the species. 

\begin{definition}
A \emph{Petri Net} is a pair $(\Sigma, R)$ of sets of species $\Sigma$ and reactions $R$.
\end{definition}

The state of the Petri Net is a multiset over $\Sigma$. A reaction in $\rho \in R$ can be
applied to a state $s$ if $l(\rho) \preceq s$ giving a new state $s' = s -
l(\rho) + r(\rho)$, which we write as $\rho \bullet s$.

Petri Nets can be given a stochastic intepretation as Continuous Time Markov
Chains (CTMCs). A CTMC is a triple $(S, Q, I)$ with a $S$ a set of states, $Q: S
\times S \rightarrow \mathbb{R}$ that gives the transition rate between any two states, and $I$
the initial state. For each reaction apart from the base rate given by $k$ we
need to know how many times it can be applied in a give state $s$ (how many
times its left-hand side appears in the state). The multiplicity of a multiset
$m$ in another multiset $m'$ is given by:
$$
\mu(m, m') = \prod_{a \in m}  \binom{m'(a)}{m(a)}
$$
Given a PN $(\Sigma, R)$ and an initial state we can get a CTMC using the following
translation:
\begin{align*}
  S & = M[\Sigma] \\
  Q(s, s') &= \sum {k(\rho) \cdot \mu(l(\rho), s) | \rho \in R, r \bullet s = s'}
\end{align*}


%nesting

\subsection{Structured}
Extensions to simple reactions add support for adding organisation to the unstructured
multisets of objects in PNs. Organisation can be thought of as explicitly adding relations
over the collections of objects. There are many relations that could be represented and
the ones in this thread are, unsurprisingly, inspired by cell biochemistry. The
first organisational principle is that of nesting inspired by the compartment
organisation inside a cell. Following this cell organisation, languages usually
distinguish two kinds of objects -- simple ones that cannot be nested and more
complex ones that can. P-systems is a language for membrane computing and while
it initially considered only simple objects enclosed in a static (potentially
multilevel) membrane structure, in later extensions membranes become first class
and are equipped with dynamics. SMMR also makes this distinction between simple
objects (species) and complex objects that can be nested (agents). The Calculus of
Wrapped Compartments also considers nesting but it also adds notational features
for expressing computations that happen on the membrane. A later extension adds
names to compartments and they can then be thought of as the agents of SMMR or the
membranes of P-systems. The dynamics of these systems reflect the interpretation
of the hierarchical relation over collections objects as the conceptual view of
compartmental organisation inside cells. Therefore sometimes the permitted operations are
directly inspired by this view while in other cases they seem more generic. In
all cases, however, the additional structure allows to express more stringent
conditions on the left-hand side of rules that select based on structure and
type (as possible in PNs).
\begin{center}
    \includegraphics{figures/nest.eps}
\end{center}
The nesting relation can also take other interpretations, for example as
part-of relations. At a static level this might be reasonable but some of the
allowed dynamics might not be sensible any more as they only make sense for a
particular cell inspired interpretation of the relation. 
Another organisational principle sometimes considered explicitly is that of
complexation inspired by protein complexes. Kappa considers the state of the
system as graphs, for example.

Other calculi have different combinations of the above features of relation
(organisation) representation, more than one relation

An interesting discussion on representing part-whole relations and the
distinction of implicit versus explicit representations in object-based
systems appears in \citet{}.

\subsection{Combining structure and dynamics}
Going beyond the object world, other extensions to these rule-based languages
try to combine traditional features of object-based formalisms with dynamical
features to represent more abstract quantifiable properties of the
objects.

\subsubsection*{Coloured Stochastic Multilevel Multiset Rewriting (CSMMR)}


\subsubsection*{Dynamical Grammars}


\subsubsection*{Coloured Petri Nets}
CSMMR is an extension to SMMR that adds parameters to agents. These
parameters can then take part in rules either passively by influencing rates
rules, conditions and so on but can also be actively changed in rules. ML-Rules,
like CSMMR, combine a nesting relation on objects with quantifiable
properties. There is also a parameterised Petri Net version called Coloured
Petri Nets that have been used in Biology before. It is interesting to note that
adding attributes allows one to implicitly represent relations based on
attribute selections on the left-hand side of rules.  Other formalisms have a
truly hybrid approach where the quantifiable properties of objects are defined
by differential equations as in dynamical systems theory, for example Dynamical
Grammars of \citep{} and Hybrid calculus of wrapped compartments of \citet{}.

Finally, further extensions consider these simple calculi as programming
languages and add advanced features that are geared towards the user. LBS and
Dynamical Grammars add parameterised modules, for example, to overcome one of
the main limitations in rule-base formalisms. LBS, again, also adds support for
mixing in expressions from a general purpose programming language.

In the comparison say something that parameters/properties/attributes/colours
are usually taken to be either passive parameters or a way to distinguish
between otherwise similar (or same type) objects.  However, when they are
abstractions for whole pathways or bigger things at the lower level, then one
might consider using higher level programming features for both the increased
expressiveness but also for readability.  Say something about readability in the
comparisons etc and how it relates to our requirements from models.

Richer behaviour = richer types = fits well with higher level programming features.
Introduce richer types



\subsection{Process-calculi based}
Another camp in this line of work are the \emph{process-calculi based} languages
inspired by formalisms such as the pi-calculus of
\cite{milner1999communicating}. The main metaphor is objects (usually molecules
in biochemistry) as processes and their interactions as process synchronisation,
which is the main interaction primitive in process calculi. Like the rule-based
camp these language have seen extensions starting from the simple stochastic
pi-calculus of \citet{}. The extensions are similarly inspired by the 

\subsection{Plant development}
Another line of work started in the plant development tradition.
They followed roughly the same evolution

Unlike the concurrency is not usually handled by mapping to continuous time
processes and using race conditions based on exponential distributions.


\section{Dynamics to structure}
\subsection{Simile}



\singlespace

%% Specify the bibliography file. Default is thesis.bib.


\printbibliography[heading=bibintoc]

%% ... that's all, folks!
\end{document}

%%% Local Variables:
%%% mode: latex
%%% TeX-master: t
%%% End:
