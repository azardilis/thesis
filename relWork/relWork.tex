\documentclass[phd]{infthesis}

\usepackage[utf8]{inputenc}
\usepackage[T1]{fontenc}
\usepackage[british]{babel}
\usepackage{microtype}
\usepackage[usenames,dvipsnames,svgnames,table]{xcolor}
\usepackage[english=british,autopunct=false]{csquotes}
\usepackage[natbib=true,style=authoryear-comp,maxbibnames=6]{biblatex}
\usepackage{graphicx}
\usepackage{textcomp}
\usepackage{wrapfig}
\usepackage{xfrac}
\usepackage{xspace}
\usepackage{mathcommon}
\usepackage[sc]{mathpazo}
\usepackage{hyperref}
\usepackage{expl3}
\usepackage{enumitem}

\frenchspacing

% Bibliography
\addbibresource{relWork.bib}
\bibliography{relWork}

% Text
\newcommand{\ie}{i.e.\xspace}
\newcommand{\eg}{e.g.\xspace}

% Referencing
\newcommand{\chp}[1]{\S\ref{chp:#1}}
\newcommand{\sct}[1]{\S\ref{sec:#1}}
\newcommand{\subsct}[1]{\S\ref{subsec:#1}}
\newcommand{\eqn}[1]{Eq.~\ref{eq:#1}}
\newcommand{\eqns}[2]{Eq. \ref{eq:#1} and \ref{eq:#2}}
\newcommand{\lem}[1]{Lemma~\ref{lemma:#1}}
\newcommand{\lems}[2]{Lemmas \ref{lemma:#1} and \ref{lemma:#2}}
\newcommand{\thm}[1]{Th.~\ref{thm:#1}}
\newcommand{\fig}[1]{Fig.~\ref{fig:#1}}
\newcommand{\diagram}[1]{diagram~\ref{eq:#1}}
\newcommand{\app}[1]{Appendix~\ref{app:#1}}
\newcommand{\mcite}[1]{\textcolor{gray}{#1}} % missing cite
\newcommand{\defn}[1]{Def.~\ref{def:#1}}
\newcommand{\prop}[1]{Prop.~\ref{prop:#1}}

% Math
\renewcommand{\tuple}[1]{\left(#1\right)}
\DeclareMathOperator*{\expn}{exp}
\renewcommand*{\exp}[1]{e^{\,#1}} % \mathrm{e}^{#1}}
\renewcommand{\qedsymbol}{\ensuremath{\blacksquare}}
\newcommand{\partialto}{\rightharpoonup}
\newcommand{\id}{\vec{1}} % identity function

% Other stuff
\newcommand{\maybe}[1]{\textcolor{gray}{#1}}
\newcommand{\todo}[1]{\textcolor{red}{TODO: #1}}

% Abstract
\begin{document}

\chapter{Related Work}
\label{chp:oc}
Formal models of physical systems serve two roles (i) documentation and
communication of our understanding and (ii) formal analysis (simulation). The
most common language for describing the natural world is dynamical systems
theory. It has a long tradition in Physics starting from Newton, Poincare
etc. It relies on a conceptual abstraction where instead of describing the
objects that form the physical world, it describes quantifiable properties of
such objects. For example, when explaining the movement of planets one does not
talk about the moon but rather about the position of the moon. Then the
evolution of the system in time (or space) is described by (partial)
differential (or difference) equations that state the change in the numerical
values of the quantifiable properties that form the system as they interact with
each other. For example, one
might write the equation of position of a particle or a molecule diffusing in a
medium. This abstraction of avoiding dealing with the objects directly but
rather with some numerical variables representing their properties has been at
the core of the success of dynamical systems theory. A big ensemble of methods
have been developed for the analysis of such systems, which one gets for free if
they choose to describe a physical system of interest in these terms.

Dynamical systems languages have not been as successful in scenarios where the
structure of the  described system is not static. While one might be able to
find a quantifiable property of the system to describe, in cases where the
question posed is about the changing structure itself or about how it gives rise
to other dynamics, such an abstraction is not adequate. Consider for example
plant development and its effect on carbon intake for the entire
plant. While in some cases writing an equation for the size of the plant is
enough, a truly mechanistic understanding requires treating the developing structure that
gives rise to plant size explicitly. How would one write equations for the size
of the leaves, for example, if new ones keep appearing?
In those cases one might be better off with a language that allows the
description of the objects themselves, their interactions, and
organisation. Unsurprisingly such languages have been developed for
scientific fields that deal with parts of the natural world where structures are
dynamic and self-organising. A whole strand of work on languages of
increasing complexity exist for the description of biochemical systems starting
from simple molecular reactions. The field of plant development also has seen an
independent strand of work on languages with explicit description of objects.
Both these strands of work have seen a heavy interaction and inspiration 
from Computer Science where there exist many formalisms in the
theory of computation for the description of object behaviour and interaction.
% especially in concurrent computation but also in sequential

%we need names for the two kinds of languages -- make the distinction so we can
%refer to them later
Here we are interested in understanding a plant or particular aspects of a plant
at the organism level starting from genomes and molecular mechanisms. In order
to understand the links between all relevant processes we need models with
explicit representation of processes at multiple levels. Apart from the
scientific challenges that this presents there are also technical and social
challenges, among which is the choice of representation (language) for
describing our models. Since we are trying to recreate an organism in silico and
especially one where development is plastic and happens throughout its life
history, we need to be able to describe dynamics of discrete objects. At the
same time we also have the need for abstraction through quantifiable properties
since the complexity of the task is large. 
% see Andrews email to Robert, Vincent after or during his sabbatical. I think
% there was something relevant there

There are languages in both the dynamics and structure space that combine aspects
of both worlds. The language we describe and that is a big part of this thesis
is situated in this space too. It comes from the biochemistry tradition and has
discrete objects at its core but at the same time also has attributes
to abstract away some internal structure of these objects. It further has
features for (i) linking the abstract (attributes) and the concrete
(observables) and (ii) explicitly using time, which is inspired, again, by the dynamics
world.

Since language design is as much art as it is science it can sometimes be hard
to properly situate a new language (Chromar) in the space of already existing languages for the
description of the natural world especially if one is only familiar with one
part of the world, perhaps the particular formalisms that are standard for their
disciplines. We therefore feel the need to include an overview of existing
languages (the ones we hinted to above) for Biology in order to be able to place
our work in the language design space. Since we have particular requirements for
structure representation, we will only deal with languages coming from the
discrete world or from the dynamics world that include some support for defining
discrete organisation as well.

Concretely, in this chapter we will do the following:
\begin{itemize}
\item An overview of languages coming from (i) the dicrete world with an
emphasis on languages from the biochemistry strand of work since Chromar is a
direct product of these and (ii) the dynamics world that have some aspect of
structural organisation. In order to place these languages and later be able to
compare them we will place them all in a feature space. The features are related
to their characteristics in regards to capturing complex organisation, dynamics,
but also meta-features like readability.
\item A comparison of representative languages from each family. While there are
attempts for formal comparisons between languages
\citep{felleisen1991expressive}, here the design space is so large that such an
attempt is probably impossible. We instead will do our comparison informally
through example where each example is meant to exemplify one of the interesting
features from the overview.
\end{itemize}
% need more concrete statement of contributions
In the next chapter  we introduce Chromar through the same examples so we can
properly place it and its novel characteristics in the same feature space.


\section{Overview -- discrete formalisms}

\subsection{Biochemistry}
A significant line of work in languages from the discrete tradition has stemmed
from the world of Biochemistry. The first main camp of languages in this line
are the \emph{rule-based} formalisms starting from notations for Chemical Reactions
Networks (CRN). At the description level, CRNs represent discrete objects with types
(species) where the dynamics are given as reactions that state how different
species interact to change the number of objects of the corresponding species in the state of the
system. The state of the system is an unstructured collection (multiset) of
copies (molecules) of each species. Since all the objects of a
species are equivalent, an alternative and equivalent formulation sees the state
of the system as a record of the number of objects of each species. For example
if we had species $A$ and $B$ the state could be $\{A \mapsto 3, B \mapsto 5 \}$ to denote
that there are three molecules of species $A$ and 5 molecules of species
$B$. Stochastic Petri Nets represent this view in a graphical way as bipartite graphs with
two types of nodes, place nodes that represent species and hold a marking
(number of molecules) and transition nodes that represent reactions that when
'fired' change the markings of the place nodes.

\begin{center}
    \includegraphics{figures/pns.eps}
\end{center}

%A good overview of this path of increasing complexity starting from CRNs is in Mjolness
Extensions to simple reactions add support for adding organisation to the unstructured
multisets of objects in CRNs. Organisation is represented as relations on the
collection of objects. There are many relations that could be represented and
the ones in this thread are, unsuprisingly, inspired by cell biochemistry.
P-systems and SMMR add nesting to represent hierarchical organisation to
the objects to represent, for example, the compartment organisation inside
cells. The dynamics (rules) can then select based on the additional structure,
for example can say if it is inside the other etc.

%add figure here


Adding attributes
Relations can also be represented in the dynamics.



Another camp in this line of work are the \emph{process-calculi based} languages
inspired by formalisms such as the pi-calculus of
\cite{milner1999communicating}. The main metaphor is molecules as processes

Like the rule-based
camp these language have seen an evolution from the simple stochastic
pi-calculus of Regev
Based on process calculi



\subsubsection*{Interpretations}
Interpretation as CTMCs
Given rates for each reaction, CRNs are usually
interpreted as Continuous Time Markov Chains (CTMCs) where the states are all
possible multisets over the species of the system. Stochastic Petri Nets that give
rates to transitions are equivalent to CRNs and are likewise interpreted as
CTMCs.

Interpretation as ODEs

Qualitative Intepretation

\subsection{Plant development}
Another line of work started in the plant development tradition.
They followed roughly the same evolution

Unlike the concurrency is not usually handled by mapping to continuous time
processes and using race conditions based on exponential distributions.

\subsection{Database and Knowledge representation}




\section{Overview -- dynamics going to structure}
This is harder


\section{Comparison through examples}








\singlespace

%% Specify the bibliography file. Default is thesis.bib.


\printbibliography[heading=bibintoc]

%% ... that's all, folks!
\end{document}

%%% Local Variables:
%%% mode: latex
%%% TeX-master: t
%%% End:
